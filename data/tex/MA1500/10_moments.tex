% !TEX root = main.tex
%======================================================================
\chapter{Moments}\label{chap:moments}
%======================================================================

%----------------------------------------------------------------------
\section{Transformations of random variables}
%----------------------------------------------------------------------
Let $(\Omega,\prob)$ be a finite probability space, let $X$ be a random variable on $\Omega$, and let $\{x_1,\ldots,x_n\}$ be its range.

\bigskip
For any function $g:\R\to\R$, the \emph{composition} of $g$ with $X$ is defined to be the function
\[\begin{array}{rlcl}
g(X): 	& \Omega & \to		& \R \\
		& \omega & \mapsto	& g\big[X(\omega)\big].
\end{array}\]

The function $g(X)$ is also a random variable on $\Omega$. Let $\{y_1,y_2,\ldots,y_m\}$ denote its range.

\bigskip
The PMF of $g(X)$ is completely determined by the PMF of $X$:
\[
\prob\big[g(X)=y_j\big] = \sum_{x_i\in B_j}\prob(X=x_i) 
\quad\text{where}\quad 
B_j  = \{x_i:g(x_i)=y_j\}
\]
Note that the sets $B_1,B_2,\ldots,B_m$ form a partition of $\{x_1,x_2,\ldots,x_n\}$.


\bigskip
To compute the expected value of the transformed variable $g(X)$, the following theorem shows that we need not compute its PMF explicitly. The result is sometimes called the \emph{law of the unconscious statistician}. 
% theorem
\begin{theorem}\label{thm:lus}
Let $f(x)$ denote the PMF of $X$. Then for any function $g:\R\to\R$, the expected value of $g(X)$ is
\[
\expe\big[g(X)\big] = \sum_{i=1}^n g(x_i)f(x_i)
\]
\end{theorem}

% proof
\begin{proof}
Let $\{y_1,\ldots,y_m\}$ be the range of $g(X)$, and let
\[
B_j  = \{x_i:g(x_i)=y_j\}.
\]
Then $\prob\big[g(X)=y_j\big] = \sum_{x_i\in B_j}\prob(X=x_i)$, so
\begin{align*}
\expe\big[g(X)\big]
	& = \sum_{j=1}^m y_j\,\prob\big[g(X)=y_j\big] \\
	& = \sum_{j=1}^m y_j\sum_{x_i\in B_j} \prob(X=x_i) \\
	& = \sum_{j=1}^m \sum_{x_i\in B_j} y_j\prob(X=x_i) \\
	& = \sum_{j=1}^m \sum_{x_i\in B_j} g(x_i)\prob(X=x_i) \text{\quad because $y_j=g(x_i)$ whenever $x_i\in B_j$,}\\
	& = \sum_{i=1}^n g(x_i)\prob(X=x_i),
\end{align*}
where the last equality follows because $\{B_1,B_2,\ldots,B_m\}$ is a partition of $\{x_1,\ldots,x_n\}$.
\qed
\end{proof}

% example: dice
\begin{example}
Let $X$ be a random variable with the following PMF:
\[\begin{array}{|c|cccc|}\hline
x 			& -2		& -1		& 1		& 3	\\ \hline
\prob(X=x)	& 1/4	& 1/8	& 1/4	& 3/8 \\ \hline
\end{array}\]
Find the expected value of $Y=X^2$.
\end{example}

\begin{solution}
The random variable $Y=X^2$ has the following PMF:
\[\begin{array}{|c|ccc|}\hline
y			& 1		& 4		& 9	\\ \hline
\prob(Y=y)	& 3/8	& 1/4	& 3/8	\\ \hline
\end{array}\]
The expected value of $Y$ is therefore
\[
\expe(Y) 
	= \sum_{y\in\{1,4,9\}} y\,\prob(Y=y) 
	= \left(1\times\frac{3}{8}\right) + \left(4\times\frac{1}{4}\right) + \left(9\times\frac{3}{8}\right) 
	= \frac{19}{4}.
\]
Alternatively,
\[
\expe(X^2)
	= \sum_{x\in\{-2,-1,1,3\}} x^2\,\prob(X=x) 
	= \left(4\times\frac{1}{4}\right) + \left(1\times\frac{1}{8}\right) + \left(1\times\frac{1}{4}\right) + \left(9\times\frac{3}{8}\right) 
	= \frac{19}{4}.
\]
\end{solution}


%----------------------------------------------------------------------
\section{Moments} 
%----------------------------------------------------------------------

% definition: moments
\begin{definition}
Let $X:\Omega\to\R$ be a random variable, and let $k\in\{0,1,2,\ldots\}$ be a non-negative integer.
\ben
\it $\expe(X^k)$ is called the \emph{$k$th moment of $X$ about the origin}, and denoted by $\mu'_k$.
In particular,
	\bit
	\it $\mu'_0 = \expe(X^0) = 1$,
	\it $\mu'_1 = \expe(X)$ is called the \emph{mean} of $X$. This is usually denoted by $\mu$.
	\it $\mu'_2 = \expe(X^2)$ is called the \emph{mean-square} of $X$.
	\eit
\it $\expe\big[(X-\mu)^k\big]$ is called the \emph{$k$th moment of $X$ about the mean}, and denoted by $\mu_k$.
In particular,
	\bit
	\it $\mu_0 = 1$,
	\it $\mu_1 = 0$,
	\it $\mu_2 = \expe\big[(X-\mu)^2\big]$ is called the \emph{variance} of $X$. This is usually denoted by $\sigma^2$.
	\eit
\een
\end{definition}

\begin{remark}
\bit
\it Moments about the origin are also called \emph{raw moments}. 
\it Moments about the mean are also called \emph{central moments}. 
%\it The positive square root of the variance is called called the \emph{standard deviation} of $X$
%\it Standard deviation is usually denoted by $\sigma$, and the variance by $\sigma^2$.
\eit
\end{remark}

%----------------------------------------------------------------------
\section{Variance}
%----------------------------------------------------------------------
Let $X$ be a random variable with range $\{x_1,x_2,\ldots,x_n\}$, let $f(x_i) = \prob(X=x_i)$ denote its PMF.

The expected value and variance of $X$ are 
\[
\expe(X) = \sum_{i=1}^{n}x_i f(x_i) \equiv \mu.
\qquad\text{and}\qquad 
\var(X) = \sum_{i=1}^{n}(x_i-\mu)^2 f(x_i) \equiv \sigma^2.
\]

\bit
\it The expected value $\expe(X)$ represents the \emph{centre} or \emph{location} of a distribution..
\it The variance $\var(X)$ quantifies its \emph{spread} or \emph{dispersion} around the expected value.
\eit

Expectation is a linear operator (Theorem~\ref{thm:prop_expe}): 
\[
\expe(aX+bY) = a\expe(X) + b\expe(Y)\quad\text{for all}\quad a,b\in\R.
\]

% theorem: properties of variance
In contrast, variance does \emph{not} have the linearity property:
\begin{theorem}\label{thm:properties_of_variance}
If $a,b\in\R$, then $\var(aX+b) = a^2\var(X)$.
\end{theorem}
% proof
\begin{proof}
By the linearity of expectation, $\expe(aX+b)=a\expe(X)+b$ and
\begin{align*}
\var(aX+b)	& = \expe\left(\big[(aX+b) - \expe(aX+b)\big]^2\right) \\
			& = \expe\left(\big[aX - \expe(aX)\big]^2\right) \\
			& = \expe\left(a^2\big[X - \expe(X)\big]^2\right) \\
			& = a^2\expe\left(\big[X - \expe(X)\big]^2\right) \\
			& = a^2\var(X)
\end{align*}
\end{proof}

\begin{remark}
Note that the variance operator is \emph{translation invariant}: 
\[
\var(X+b)=\var(X)\quad\text{for all}\quad b\in\R.
\]
\end{remark}

% remark: variance formula
Central moments can be expressed in terms of raw moments. In particular, we have the following expression for variance, which is useful when performing computations.
\begin{lemma}
$\var(X) = \expe(X^2) - \expe(X)^2$.
\end{lemma}

\begin{proof}
Let $\{x_1,x_2,\ldots,x_n\}$ be the range of $X$, and let $\mu=\expe(X)$. Taking $g(x) = (x-\mu)^2$ in Theorem~\ref{thm:lus},
\begin{align*}
\var(X) = \expe\big[(X-\mu)^2\big]
	& = \sum_{i=1}^n (x_i-\mu)^2 f(x_i) \\
	& = \sum_{i=1}^n (x_i^2 - 2x_i\mu + \mu^2) f(x_i) \\
	& = \sum_{i=1}^n x_i^2 f(x_i) - 2\mu\sum_{i=1}^n x_i f(x_i) + \mu^2\sum_{i=1}^n f(x_i) \\
	& = \expe(X^2) - \mu^2
\end{align*}
where the last equality follows because $\sum_{i=1}^n x_i f(x_i)=\mu$ and $\sum_{i=1}^n f(x_i) = 1$.
\end{proof}

% example
\begin{example}
Let $X$ be a random variable taking values in the range $\{1,2,3,4,5,6\}$, and let $f(x)$ denote its PMF. Find the variance of $X$ when
\ben
\it $f(x) = 1/6$ for all $x\in\{1,2,3,4,5,6\}$;
\it $f(x) = 1/4$ for $x\in\{3,4\}$ and $f(x) = 1/8$ for $x\in\{1,2,5,6\}$.
\een
\end{example}

\begin{solution}
\bit
\it $\expe(X) 	= \sum_{x=1}^6 x f(x)	= \frac{1}{6}(1+2+3+4+5+6)		= \frac{7}{2}$.
\it $\expe(X^2) 	= \sum_{x=1}^6 x^2 f(x)	= \frac{1}{6}(1+4+9+16+25+36) 	= \frac{91}{6}$.
\it $\var(X)		= \expe(X^2)-\expe(X)^2	= \frac{91}{6}-\frac{49}{4}		= \frac{35}{12} = 2.92$.
\it[] 
\it $\expe(X)	= \sum_{x=1}^6 x f(x) 	= \frac{1}{8}(1+2+5+6) + \frac{1}{4}(3+4) = \frac{7}{2}$.
\it $\expe(X^2)	= \sum_{x=1}^6 x^2 f(x) = \frac{1}{8}(1+4+25+36) + \frac{1}{4}(9+16) = \frac{58}{4}$.
\it $\var(X)		= \expe(X^2)-\expe(X)^2	= \frac{58}{4} - \frac{49}{4} = \frac{9}{4} = 2.25$.
\eit
\end{solution}

%----------------------------------------------------------------------
\section{Location, scale and shape*}
%----------------------------------------------------------------------
When trying to describe a distribution, it is natural to look for its \emph{location}, \emph{scale} (size), and \emph{shape}.

\bigskip
% mean and variance
For any random variable $X:\Omega\to\R$,
\ben
\it the first moment of $X$ about the origin ($\mu$)  describes its location, 
\it the second moment of $X$ about the mean ($\sigma^2$) describes its scale, and
\it the higher moments of $X$ describe the shape of its distribution.
\een

\paragraph{Location}
To locate $X$, we look for a point $\mu\in\R$ such that the expected squared deviation $\expe\big[(X-\mu)^2\big]$ around this point is minimum. By the linearity of expectation,
\[
\expe\big[(X-\mu)^2\big] = \expe(X^2 - 2\mu X + \mu^2) = \expe(X^2) - 2\mu\expe(X) + \mu^2
\]
To find the value of $\mu$ that minimises the expected squared deviation, we can differentiate the right-hand side with respect to $\mu$ and set the resulting expression to zero. This yields $\mu=\expe(X)$.
\bit
\it The location of a distribution is described by its \emph{expected value}, $\mu$.
\eit

\paragraph{Scale}
The size of $X$ should not depend on its location. Thus we consider the \emph{centred} variable $Y=X-\mu$, which has the property that $\expe(Y)=0$. The expected squared deviation of $X$ around $\mu$ is equal to its \emph{variance}, $\sigma^2 = \expe\big[(X-\mu)^2\big]$. 
\bit
\it The scale of a distribution is quantified by its \emph{standard deviation}, $\sigma$.
\eit

\paragraph{Shape}
The shape of a distribution should not depend on its location nor its scale. Thus we consider the so-called \emph{standardised} variable,
\[
Z = \frac{X-\mu}{\sigma}.
\]
\vspace*{-4ex}
\begin{lemma}
$\expe(Z)=0$ and $\var(Z)=1$.
\end{lemma}
\begin{proof}
\bit
\it By the linearity of expectation, $\displaystyle\expe(Z)=\expe\left(\frac{X-\mu}{\sigma}\right)=\frac{1}{\sigma}\big[\expe(X)-\mu\big] = 0$.
\it By the properties of variance, $\displaystyle\var(Z)=\var\left(\frac{X-\mu}{\sigma}\right)=\frac{1}{\sigma^2}\var(X) = 1$.
\eit
\end{proof}
%----------------------------------------------------------------------
\section{Skewness and kurtosis$^{*}$}
%----------------------------------------------------------------------
Higher moments of the standardised variable $\displaystyle Z=\frac{X-\mu}{\sigma}$ quantify the \emph{shape} of a distribution.
% definition
\begin{definition}
The \emph{skewness} of a random variable $X$ is 
\[
\gamma_1 = \expe\left[\left(\frac{X-\mu}{\sigma}\right)^3\right] = \frac{\mu_{3}}{\sigma^3}
\]
where $\mu_3 = \expe\big[(X-\mu)^3]$ is the third central moment of $X$. 
\end{definition}

Skewness is a measure of \emph{asymmetry}:
\bit
\it Negative skew ($\gamma_1 < 0$). Long tail on the left, mass concentrated on the right.
\it Positive skew ($\gamma_1 > 0$). Long tail on the right, mass concentrated on the left.
\eit

%% example: skewness of binomial
%\begin{example}\label{ex:highermoments}
%The skewness of the binomial distribution with parameters $n$ and $p$ isIf $X\sim\text{Binomial}(n,p)$, the skewness of $X$ is given by
%\[
%\gamma_1 = \frac{1-2p}{\sqrt{np(1-p)}} = \begin{cases}
%	< 0 	& \text{if } p > \frac{1}{2} \\
%	= 0 	& \text{if } p = \frac{1}{2} \\
%	> 0 	& \text{if } p < \frac{1}{2} \\
%\end{cases}	
%\]
%\end{example}

% definition: kurtosis
\begin{definition}
The \emph{kurtosis} of a random variable $X$ is 
\[
\gamma_2 = \expe\left[\left(\frac{X-\mu}{\sigma}\right)^4\right] = \frac{\mu_{4}}{\sigma^4}
\]
\end{definition}
where $\mu_4 = \expe\big[(X-\mu)^4]$ is the fourth central moment of $X$.

\bigskip
The word \textit{kurtosis} comes from the Greek \textbf{\textit{kurtos- `bulge'}}.
\bit
\it Large kurtosis (leptokurtic: $\gamma_2 > 3$). Tall and narrow peak, with heavy tails.
\it Small kurtosis (platykurtic: $\gamma_2 < 3$). Low and wide peak, with thin tails.
\eit

%\begin{remark}
%If $X\sim N(\mu,\sigma^2)$ then
%$\displaystyle
%\expe\left[\left(\frac{X-\mu}{\sigma}\right)^4\right] = 3.
%$
%\bit
%\it The kurtosis of the normal distribution is zero.
%\it Kurtosis quantifies "peakiness" relative to that of the normal distribution.
%\eit
%\end{remark}
%%

% example (cont) 
%\begin{examplecont}{\ref{ex:highermoments}}
%If $X\sim\text{Binomial}(n,p)$, the kurtosis of $X$ is given by
%\[
%\gamma_2 = \frac{1-6p(1-p)}{np(1-p)} = \begin{cases}
%	< 0 				& \text{if } \left|p -\frac{1}{2}\right| < \frac{1}{2\sqrt{3}} \\
%	> 0 				& \text{if } \left|p -\frac{1}{2}\right| > \frac{1}{2\sqrt{3}} \\
%\end{cases}
%\]
%\bit
%\it The excess kurtosis is minimum at $p=\frac{1}{2}$, where it takes the value $\gamma_2 = -\frac{2}{n}$.
%\it Note that $\gamma_2\to 0$ as $n\to\infty$, so the excess kurtosis of the binomial distribution approaches that of the normal distribution as $n\to\infty$. 
%\it The \emph{de Moivre - Laplace theorem} shows that the binomial distribution (when scaled appropriately) converges to  the normal distribution as $n\to\infty$.
%\eit
%\end{examplecont}



%----------------------------------------------------------------------
\section{Exercises}
% !TEX root = main.tex
% ex10_moments.tex
\begin{exercise}
\begin{questions}
%----------------------------------------
\question
Let $X$ be a random variable with the following PMF, where $c$ is a constant:
\[
\begin{array}{|c|ccc|}\hline
x       		& -2		& 0		& 2		\\ \hline
\prob(X=x)	& c/6	& c/2 & c/3 	\\ \hline
\end{array}
\]
\begin{parts}
\part
Find the value of $c$.
\begin{answer}
\[
\frac{c}{6} + \frac{c}{2} + \frac{c}{3} = 1 \quad\Rightarrow\quad c=1.
\]
\end{answer}
\part Show that $\expe(X)=1/3$ and $\var(X)=17/9$.
\begin{answer}
The expected value and variance of $X$ are
\begin{align*}
\expe(X) 	& = \left(-2 \times\frac{1}{6}\right) + \left(0 \times\frac{1}{2}\right) + \left(2 \times\frac{1}{3}\right) = \frac{1}{3}, \\
\expe(X^2) 	& = \left(4 \times\frac{1}{6}\right) + \left(0 \times\frac{1}{2}\right) + \left(4 \times\frac{1}{3}\right) = 2, \\
\var(X)		& = \expe(X^2) - \expe(X)^2 = 2 - \frac{1}{9} = \frac{17}{9}.
\end{align*}
\end{answer}

\part Find $\expe(Y)$ and $\var(Y)$ when $Y=3X+4$.
\begin{answer}
By the properties of expectation and variance,
\begin{align*}
\expe(Y)		& = \expe(3X+4) = 3\expe(X)+4 = 5, \\
\var(Y)		& =  \var(3X+4) = 9\var(X) = 17.
\end{align*}
\end{answer}

\end{parts}

%----------------------------------------
\question
Let $X$ be a random variable with the following PMF, where $c$ is a constant:
\[
\begin{array}{|c|ccc|}\hline
x       		& 1		& 2		& 3		\\ \hline
\prob(X=x)	& c/8	& c/8 & c/4 	\\ \hline
\end{array}
\]
\begin{parts}
\part
Find the value of $c$.
\begin{answer}
\[
\frac{c}{8} + \frac{c}{8} + \frac{c}{4} = 1 \quad\Rightarrow\quad c=2.
\]
\end{answer}
\part
Find $\expe(X)$ and $\var(X)$.
\begin{answer}
\begin{align*}
\expe(X) 	& = \left(1 \times\frac{1}{4}\right) + \left(2 \times\frac{1}{4}\right) + \left(3 \times\frac{1}{2}\right) = \frac{9}{4}, \\
\expe(X^2) 	& = \left(1 \times\frac{1}{4}\right) + \left(4 \times\frac{1}{4}\right) + \left(9 \times\frac{1}{2}\right) = \frac{23}{4}, \\
\var(X)		& = \expe(X^2) - \expe(X)^2 = \frac{23}{4} - \frac{81}{16} = \frac{11}{16}.
\end{align*}
\end{answer}
\part
Let $Y=2X+3$. Show that $\expe(Y)=15/2$ and $\var(Y)=11/4$.
\begin{answer}
\begin{align*}
\expe(Y)		& = \expe(2X+3) = 2\expe(X)+3 = 15/2, \\
\var(Y)		& =  \var(2X+3) = 4\var(X) = 11/4.
\end{align*}
\end{answer}
\end{parts}

%----------------------------------------
\question
Let $X$ be a random variable with the following PMF, where $c$ is a constant:
\[
f(x) = \begin{cases}
	\frac{x}{8}	& \text{for }\ x=1,2, \\
	c 				& \text{for }\ x=3,4, \\
	\frac{1}{8}	& \text{for }\ x=5, \\
	0				& \text{otherwise.}
\end{cases}	
\]
\begin{parts}

\part Show that $c = 1/4$.
\begin{answer}
Because $f(x)$ is a PMF we need that $\sum_{x=1}^5 \prob(X=x) = 1$, so
\[
\frac{1}{8}+\frac{2}{8}+c+c+\frac{1}{8} = 1 \quad \Rightarrow \quad c = \frac{2}{8} = \frac{1}{4}.
\]
\end{answer}

\part Find the mean and variance of $X$.
\begin{answer}
\begin{align*}
\expe(X) 
			& = \sum_{x=1}^5 xf(x) \\
			& = \left(1\times\frac{1}{8}\right) + \left(2\times\frac{2}{8}\right) + \left(3\times\frac{2}{8}\right) + 
					\left(4\times\frac{2}{8}\right) + \left(5\times\frac{1}{8}\right) \\
			& = \frac{1+4+6+8+5}{8} = \frac{24}{8} = 3. \\
\expe(X^2)	
			& = \sum_{x=1}^5 x^2 f(x) \\
			& = \left(1\times\frac{1}{8}\right) + \left(4\times\frac{2}{8}\right) + \left(9\times\frac{2}{8}\right) + 
					\left(16\times\frac{2}{8}\right) + \left(25\times\frac{1}{8}\right) \\
			& = \frac{1+8+18+32+25}{8} = \frac{84}{8} = \frac{21}{2}. \\
\end{align*}
Hence, $\var(X) = \expe(X^2)-\expe(X)^2 = \frac{21}{2} - 9 = \frac{3}{2}$.
\end{answer}

\part Find the mean and variance of $Y=2X-1$.
\begin{answer}
Using the properties of expectation and variance,
\begin{align*}
\expe(Y)		& = \expe(2X-1) = 2\expe(X)-1 = 5, \\
\var(Y) 		& = \var(2X-1) = 4\var(X) = 6.
\end{align*}

Alternatively, 
\begin{align*}
\expe(Y) 
			& = \sum_{x=1}^5 (2x-1)f(x) \\
			& = \left(1\times\frac{1}{8}\right) + \left(3\times\frac{2}{8}\right) + \left(5\times\frac{2}{8}\right) + 
					\left(7\times\frac{2}{8}\right) + \left(9\times\frac{1}{8}\right) \\
			& = \frac{1+6+10+14+9}{8} = \frac{40}{8} = 5. \\
\end{align*}
Similarly,
\begin{align*}
\expe(Y^2)	
			& = \sum_{x=1}^5 (2x-1)^2f(x) \\
			& = \left(1\times\frac{1}{8}\right) + \left(9\times\frac{2}{8}\right) + \left(25\times\frac{2}{8}\right) + 
					\left(49\times\frac{2}{8}\right) + \left(81\times\frac{1}{8}\right) \\
			& = \frac{1+18+50+98+81}{8} = \frac{248}{8} = 31. \\
\end{align*}
Hence, $\var(Y) = \expe(Y^2)-\expe(Y)^2 = 31 - 25 = 6$.
\end{answer}
\end{parts}

%----------------------------------------
% mean & variance
% RND p116q9
\question
Let $X$ be a random variable with mean $\mu$ and variance $\sigma^2$. If $c$ is any real number, show that 
\[
\expe\big[(X-c)^2\big] = \sigma^2 + (\mu-c)^2
\]
and deduce that $\expe\big[(X-c)^2\big]$ is minimised when $c=\mu$.
\begin{answer}
\begin{align*}
\expe\big((X-c)^2\big)
	& = \expe\Big[\big[(X-\mu)+(\mu-c)\big]^2\Big] \\
	& = \expe\big[(X-\mu)^2 + 2(X-\mu)(\mu-c) + (\mu-c)^2\big] \\
	& = \expe\big[(X-\mu)^2\big] + 2(\mu-c)\expe(X-\mu) + (\mu-c)^2 \qquad\text{(by linearity)}\\
	& = \expe\big[(X-\mu)^2\big] + (\mu-c)^2\quad \text{because $\expe(X-\mu) = 0$.}
\end{align*}

Since $(\mu-c)^2\geq 0$, it follows that $\expe\big[(X-c)^2\big] \geq \expe\big[(X-\mu)^2\big]$, with equality if $c=\mu$. 

This shows that $\expe\big[(X-c)^2\big]$ is minimised when $c=\mu$.
\end{answer}


\end{questions}
\end{exercise}
%======================================================================
\endinput
%======================================================================

%----------------------------------------------------------------------

%======================================================================
\endinput
%======================================================================
