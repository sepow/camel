% !TEX root = main.tex
% ex11_uniform_bernoulli_binomial.tex
\begin{exercise}
\begin{questions}
%----------------------------------------
% RND p97q5
\question
A pair of fair dice is rolled six times. What is the probability of getting a total of seven
\ben
\it twice,
\it at least once,
\it more than three times?
\een
\begin{answer}
Let $X$ be the number of times that a total of 7 is obtained. 
\bit
\it Then $X\sim\text{Binomial}(n,p)$ with $n=6$ and $p=\frac{6}{36}=\frac{1}{6}$.
\eit
\ben
\it $\prob(X=2) 
	= \binom{6}{2}\left(\frac{1}{6}\right)^2 \left(\frac{5}{6}\right)^4 
	= 0.2009$.
\it $\prob(X\geq 1) 
	= 1 - \prob(X=0) 
	= 1 - \left(\frac{5}{6}\right)^6 
	= 0.6651$.
\it $\prob(X>3)
	= \prob(X=4) + \prob(X=5) +\prob(X=6)
	= 30\left(\frac{1}{6}\right)^4 \left(\frac{5}{6}\right)^2
		+ 6\left(\frac{1}{6}\right)^5 \left(\frac{5}{6}\right) 
		+ \left(\frac{1}{6}\right)^6 
	= 0.0087$.
\een
\end{answer}

%----------------------------------------
% RND p97q4 (modified)
\question
A biased coin, for which the probability of getting a head is $1/4$, is tossed $10$ times. What are the probabilities of observing
\ben
\it exactly two heads,
\it fewer than two heads,
\it more than two heads,
\it at most two heads, 
\it at least two heads?
\een
\begin{answer}
Let $X$ be the number of heads obtained. Then $X\sim\text{Binomial}(10,0.15)$. 
\ben
\it $\prob(X=2) 
	= \binom{10}{2}\left(\frac{1}{4}\right)^2 \left(\frac{3}{4}\right)^8 
	= 0.2816$.
\it $\prob(X\leq 1) 
	= \prob(X=0) + \prob(X=1) 
	= \left(\frac{3}{4}\right)^{10} + 10\left(\frac{1}{4}\right)^1 \left(\frac{3}{4}\right)^9 
	= 0.2441$.
\it $\prob(X>2)
	= 1 = \prob(X\leq 2)
	= 1 - \prob(X=0) + \prob(X=1) +\prob(X=2)
	= 0.4744$.
\it $\prob(X\leq 2)
	= \prob(X=0) + \prob(X=1) +\prob(X=2)
	= 0.5260$.
\it $\prob(X\geq 2)
	= 1 = \prob(X<2)
	= 1 - \prob(X=0) + \prob(X=1)
	= 0.7560$.
\een
\end{answer}
%----------------------------------------

%----------------------------------------
% binomial
% RND p98q6a
\question
The probability that a production line produces a faulty item is $0.1$. Are you more likely to find at most one faulty item in a sample of $10$ items, or at most two faulty items in a sample of $20$ items?

\begin{answer}
\bit
\it Let $X$ be the number of faulty items in a sample of size $10$. Then $X\sim\text{Binomial}(10,0.1)$.
\it Let $Y$ be the number of faulty items in a sample of size $20$. Then $Y\sim\text{Binomial}(20,0.1)$.
\eit
\begin{align*}
\prob(\text{at most one faulty item in 10})
	& = \prob(X=0) + \prob(X=1) \\
	& = (0.9)^{10} + 10(0.1)(0.9)^9 = 0.7361. \\
\prob(\text{at most two faulty items in 20})
	& = \prob(Y=0) + \prob(Y=1) + \prob(Y=2) \\
	& = (0.9)^{20} + 20(0.1)(0.9)^9 + 190(0.1)^2(0.9)^{18} = 0.6769.
\end{align*}
Thus we are more likely to observe at most one faulty item in a sample of $10$ items.
\end{answer}

%----------------------------------------
\question % GS 2.7.7
Airlines find that customers who reserve a seat fail to turn up with probability $0.1$. To avoid having empty seats, EasyJet always sell 10 tickets for their 9-seater aeroplane, while Ryanair always sell 20 tickets for their 18-seater aeroplane. Which of the two airlines is most often overbooked?
\begin{answer}
Let $X$ and $Y$ denote the (random) number of people on an EasyJet and Ryanair flight respectively. Then $X\sim\text{Binomial}(10,0.9)$ and $Y\sim\text{Binomial}(20,0.9)$, so
\begin{align*}
\prob(X=k)	& = \binom{10}{k}\left(\frac{9}{10}\right)^k\left(1-\frac{9}{10}\right)^{10-k} \\
\prob(Y=k)	& = \binom{20}{k}\left(\frac{9}{10}\right)^k\left(1-\frac{9}{10}\right)^{20-k}
\end{align*}
Thus
\begin{align*}
\prob(\text{EasyJet flight is overbooked}) 
	& = \prob(X=10) = \left(\frac{9}{10}\right)^{10} = 0.3487 \\
\prob(\text{Ryanair flight is overbooked}) 
	& = \prob(Y=19)+\prob(Y=20) \\
	& = 20\left(\frac{9}{10}\right)^{19}\left(\frac{1}{10}\right) + \left(\frac{9}{10}\right)^{20} = 0.3917
\end{align*}	  
so Ryanair is overbooked more often than EasyJet.
\end{answer}

%----------------------------------------
% binomial (backwards)
% RND p99q8
\question
A random variable $X$ has binomial distribution with mean $1.5$ and variance $1.275$. Find the probability that $X$ is at most $2$.

\begin{answer}
If $X\sim\text{Binomial}(n,p)$ then $\expe(X)=np$ and $\var(X)=np(1-p)$. Solving the equations $np=1.5$ and $np(1-p)=1.275$ in terms of $n$ and $p$, we obtain $n=10$ and $p=0.15$. Thus
\begin{align*}
\prob(X\leq 2) = \sum_{k=0}^2 \prob(X=k)
	& = \sum_{k=0}^2 \binom{10}{k}(0.15)^k (0.85)^{10-k} \\
	& = (0.85)^{10} + 10(0.15)(0.85)^9 + 45(0.15)^2(0.85)^8 \\
	& = 0.8202.
\end{align*}
\end{answer}

%----------------------------------------
% mean & variance
% GS 3.11.14
\question
Let $X_1,X_2,\ldots,X_n$ be independent random variables, with $X_i\sim\text{Bernoulli}(p_i)$ for $i=1,2,\ldots,n$. Show that the mean and variance of their sum $X=X_1+X_2+\ldots+X_n$ are given by
\[
\expe(X) = \displaystyle\sum_{i=1}^n p_i \quad\text{and}\quad \var(X) = \displaystyle\sum_{i=1}^n p_i(1-p_i) \quad\text{respectively.}
\]
\begin{answer}
Since $X_i\sim\text{Bernoulli}(p_i)$ we have $\expe(X_i)=p_i$ and $\var(X_i)=p_i(1-p_i)$. 

By the linearity of expectation,
\begin{align*}
\expe(X)	& = \expe\left(\sum_{i=1}^n X_i\right) = \sum_{i=1}^n \expe(X_i) = \sum_{i=1}^n p_i, \\
\intertext{and because the $X_i$ are independent,}
\var(X)	& = \var\left(\sum_{i=1}^n X_i\right) = \sum_{i=1}^n \var(X_i) = \sum_{i=1}^n p_i(1-p_i),
\end{align*}
as required.
\end{answer}

%----------------------------------------
% binomial (function)
% RND p99q10
\question
Suppose that $n$ independent Bernoulli trials are carried out, each having probability of success $p$. Let the number of successes and failures obtained in these trials be denoted by $X$ and $Y$ respectively. Find the PMF of $Z=X-Y$, and show that $\expe(Z)=n(2p-1)$. 
\par
[Hint: use the fact that $Y=n-X$.]
\begin{answer}
Since $Y=n-X$ we have that $Z = X-(n-X) = 2X-n$, and since $X$ takes values in the set $\{0,1,2,\ldots,n\}$, it follows that $Z$ takes values in the set $\{-n,-n+2,-n+4,\ldots,n-4,n-2,n\}$. 

Thus the PMF of $Z$ is
\begin{align*}
\prob(Z=k) 
	& = \prob(2X-n=k) \\
	& = \prob\left(X = \frac{1}{2}(n+k)\right) \\
	& = \begin{cases}
	\displaystyle\binom{n}{\frac{1}{2}(n+k)} p^{\frac{1}{2}(n-k)}(1-p)^{\frac{1}{2}(n-k)}	&\text{for}\ \ k\in \{-n,-n+2,\ldots,n-2,n\} \\[2ex]
	0	& \text{otherwise.}
\end{cases}	
\end{align*}
The expected value of $Z$ can be easily computed using the linearity of expectation:
\[
\expe(Z) = \expe(2X-n) = 2\expe(X)-n = 2np - n = n(2p-1),
\]	
as required. 
\end{answer}

\end{questions}
\end{exercise}
%======================================================================
\endinput
%======================================================================
