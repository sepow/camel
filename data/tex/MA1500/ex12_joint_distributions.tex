% !TEX root = main.tex
% ex12_joint_distributions.tex
\begin{exercise}
\begin{questions}
%----------------------------------------
\question
A fair coin is tossed twice. Let $X$ be the number of heads, and let $Y$ be the indicator variable of the event $\{X=2\}$. Find the joint PMF of $X$ and $Y$.
\begin{answer}
The possible outcomes, along with the associated values taken by $X$ and $Y$, are shown in the following table:
\[
\begin{array}{|c|cccc|}\hline
\omega		& TT		& TH		& HT 	& HH		\\ \hline
X(\omega)	& 0		& 1 		& 1		& 2 		\\ 
Y(\omega)	& 0		& 0		& 0		& 1		\\ \hline
\end{array}
\]
Hence the joint PMF of $X$ and $Y$ is as follows:
\[
\begin{array}{|cc|ccc|} \hline
	&		& 		& x 	& 		\\
	& 		& 0 		& 1		& 2 		\\ \hline
y	& 0		& 1/4	& 1/2	& 0		\\ 
	& 1		& 0 		& 0		& 1/4	\\ \hline
\end{array}
\]
\end{answer}


%----------------------------------------
\question % GS 2.5.2
Let $X$ be a Bernoulli random variable with parameter $p$. %Then $\prob(X=0)=1-p$ and $\prob(X=1)=p$.
\begin{parts}
%--------------------
\part Let $Y=1-X$. Find the joint PMF of $X$ and $Y$.
\begin{answer}
\[
f_{X,Y}(x,y) = \left\{\begin{array}{ll}
	p	& \text{if } (x,y) = (1,0) \\
	1-p	& \text{if } (x,y) = (0,1) \\
	0	& \text{otherwise.}
\end{array}\right.
\]
\end{answer}
%--------------------
\part Let $Z=X(1-X)$. Find the joint PMF of $X$ and $Z$.
\begin{answer}
\[
f_{X,Z}(x,z) = \begin{cases}
	p	& \text{if } (x,z) = (1,0) \\
	1-p	& \text{if } (x,z) = (0,0) \\
	0	& \text{otherwise.}
\end{cases}
\]
\end{answer}
%--------------------
\end{parts}

%%----------------------------------------
%\question % GS 2.7.7
%Airlines find that each passenger who reserves a seat fails to turn up with probability $0.1$, independently of other passengers. To avoid empty seats, EasyJet always sell 10 tickets for their 9-seater aeroplane, while Ryanair always sell 20 tickets for their 18-seater aeroplane. Which of the two airlines is most often overbooked?
%\begin{answer}
%Let $X$ and $Y$ denote the (random) number of people on an EasyJet and Ryanair flight respectively. Then $X\sim\text{Binomial}(10,0.9)$ and $Y\sim\text{Binomial}(20,0.9)$, so
%\begin{align*}
%\prob(X=k)	& = \binom{10}{k}\left(\frac{9}{10}\right)^k\left(1-\frac{9}{10}\right)^{10-k} \\
%\prob(Y=k)	& = \binom{20}{k}\left(\frac{9}{10}\right)^k\left(1-\frac{9}{10}\right)^{20-k}
%\end{align*}
%Thus
%\begin{align*}
%\prob(\text{EasyJet flight is overbooked}) 
%	& = \prob(X=10) = \left(\frac{9}{10}\right)^{10} = 0.3487 \\
%\prob(\text{Ryanair flight is overbooked}) 
%	& = \prob(Y=19)+\prob(Y=20) \\
%	& = 20\left(\frac{9}{10}\right)^{19}\left(\frac{1}{10}\right) + \left(\frac{9}{10}\right)^{20} = 0.3917
%\end{align*}	  
%so Ryanair is overbooked more often than EasyJet.
%\end{answer}

%%==========================================================================
%\question
%Let $X$ and $Y$ be two random variables with joint PMF given by the following table:
%\[
%\begin{array}{|cc|cccc|}\hline
%    &       &       & \multicolumn{2}{c}{y} &   \\
%    &       & 0     & 1     & 2     & 3     \\ \hline
%    & 0     & 0     & 3/56  & 6/56  & 1/56  \\
%\raisebox{-1.5ex}{$x$}   & 1     & 3/56  & 18/56 & 9/56  & 0  \\
%    & 2     & 6/56  & 9/56  & 0     & 0  \\
%    & 3     & 1/56  & 0     & 0     & 0  \\ \hline
%\end{array}
%\]
% 
%\begin{parts}
%%--------------------
%\part Find the conditional PMF of $X$ given $Y=0$.
%\begin{answer}
%The conditional PMF $\prob(X=x\,|\,Y=0)$ is as follows:
%\[
%\begin{array}{c|cccc}
%x          		& 0     & 1     & 2     & 3     \\ \hline
%f_{X|Y}(x|0)		& 0     & 3/10  & 6/10  & 1/10  \\
%\end{array}
%\]
%\end{answer}
%%--------------------
%\part Find the conditional PMF of $Y$ given $X=1$.
%\begin{answer}
%The conditional PMF $\prob(Y=y\,|\,X=1)$ is as follows:
%\[
%\begin{array}{c|cccc}
%y            	& 0     & 1     & 2     & 3     \\ \hline
%f_{Y|X}(y|1)		& 1/10  & 6/10  & 3/10  & 0     \\
%\end{array}
%\]
%\end{answer}
%%--------------------
%\end{parts}
%


%==========================================================================
\question
Let $X$ and $Y$ be two independent random variables with PMFs
%\begin{center}
%\begin{minipage}{\linewidth}
%\begin{minipage}{0.48\linewidth}
\[
\begin{array}{c|cc}
x     & 1     & 2   \\ \hline
f_X(x)  & 1/3   & 2/3 \\
\end{array}
%\]
%\end{minipage}
\text{\qquad and \qquad}
%\begin{minipage}{0.48\linewidth}
%\[
\begin{array}{c|ccc}
y     & -1    & 0     & 1     \\ \hline
f_Y(y)  & 1/4   & 1/2   & 1/4   \\
\end{array}
\text{\qquad respectively.}
\]
%\end{minipage}
%\end{minipage}
%\end{center}

\begin{parts}
%--------------------
\part Compute the joint PMF of $X$ and $Y$.
\begin{answer}
Since $X$ and $Y$ are independent we have that $f_{X,Y}(x,y)=f_X(x)f_Y(y)$, which yields the following joint PMF.
\[
\begin{array}{|cc|ccc|c|}\hline
    &       &       & y     &       &     \\
    &       & -1    & 0     & 1     & f_X(x)    \\ \hline
\raisebox{-1.0ex}{$x$}   & 1     & 1/12  & 1/6   & 1/12  & 1/3       \\
    & 2     & 1/6   & 1/3   & 1/6   & 2/3       \\ \hline
    & f_Y(y)& 1/4   & 1/2   & 1/4   & 1         \\ \hline
\end{array}
\]
\end{answer}
%--------------------
\part Compute the joint PMF of the random variables $U=1/X$ and $V=Y^2$. 
\begin{answer}
Let $f_{U,V}(u,v)$ denote the joint PMF of $U$ and $V$. Clearly, $U$ takes the values $0.5$ and $1$, while $V$ takes the values $0$ and $1$. We compute (for example)
\[
f_{U,V}(0.5,1) = f_{U,V}(2,-1) + f_{U,V}(2,1) = 1/6 + 1/6 = 1/3
\]
to get the joint PMF $f_{U,V}$ shown in the following table:
\[
\begin{array}{|cc|cc|c|} \hline
    &       & \multicolumn{2}{c|}{v} &   \\
    &       & 0     & 1     & f_U(u)    \\ \hline
\raisebox{-1.0ex}{$u$}   & 1/2   & 1/3   & 1/3   & 2/3       \\
    & 2     & 1/6   & 1/6   & 1/3       \\ \hline
    & f_V(v)& 1/2   & 1/2  & 1          \\ \hline
\end{array}
\]
\end{answer}
%--------------------
\part Show that $U$ and $V$ are independent.
\begin{answer}
The marginal PMFs $f_U(u)$ and $f_V(v)$ are computed by summing the rows and columns of the joint PMF table. From here, we see that $U$ and $V$ are independent because $f_{U,V}(u,v) = f_U(u)f_V(v)$ for every pair of values $(u,v)$.
\end{answer}
%--------------------
\end{parts}

%==========================================================================
\question
The random variables $X$ and $Y$ have the joint PMF
\[
f(x,y) = \left\{\begin{array}{ll}
	c|x+y| 	& \text{if}\quad x,y\in\{-2,-1,0,1,2\} \\
	0		& \text{otherwise,}
\end{array}\right.	
\]
where $c$ is a constant.
\begin{parts}
%--------------------
\part 
Find the value of $c$.
\begin{answer}
First we tabulate the values of $|x+y|$:
\[
\begin{array}{|cc|ccccc|}\hline
	&		&		&		& y		&		&		\\
	&		& -2		& -1		& 0		& 1		& 2 		\\ \hline
	& -2		& 4 		& 3		& 2 		& 1		& 0		\\
	& -1		& 3		& 2		& 1		& 0		& 1		\\
x	& 0		& 2		& 1		& 0		& 1		& 2		\\
	& 1		& 1		& 0		& 1		& 2		& 3		\\
	& 2		& 0		& 1		& 2		& 3		& 4		\\ \hline
\end{array}\]
Because the probabilities must sum to $1$, it follows that $c=1/40$. 
\par
Hence the joint PMF of $X$ and $Y$, along with their marginal PMFs, are as follows:
\[
\begin{array}{|cc|ccccc|c|}\hline
	&		&		&		& y		&		&		&			\\
	&		& -2		& -1		& 0		& 1		& 2 		&			\\ \hline
	& -2		& 4/40	& 3/40	& 2/40	& 1/40	& 0		&	10/40 	\\
	& -1		& 3/40	& 2/40	& 1/40	& 0		& 1/40	&	 7/40 	\\
x	& 0		& 2/40	& 1/40	& 0		& 1/40	& 2/40	&	 6/40 	\\
	& 1		& 1/40	& 0		& 1/40	& 2/40	& 3/40	&	 7/40 	\\
	& 2		& 0		& 1/40	& 2/40	& 3/40	& 4/40	&	10/40 	\\ \hline
	&		& 10/40	& 7/40	& 6/40	& 7/40	& 4/40	& \\ \hline
\end{array}
\]
\end{answer}
%--------------------
\part 
Find $\prob(X=0,Y=-2)$.
\begin{answer}
$\prob(X = 0 \text{ and } Y = -2) = f(0, -2) = 2/40 = 1/20$.
\end{answer}
%--------------------
\part 
Find $\prob(X=2)$.
\begin{answer}
$\prob(X=2) = f_X(2) = 10/40 = 1/4$
\end{answer}
%--------------------
\part 
Find $\prob(|X-Y|\leq 1)$.
\begin{answer}
\begin{align*}
\prob(|X-Y|\leq 1) 
	& = \prob(-1\leq X-Y\leq 1) \\
	& = \prob(X-Y = -1,0\text{ or } 1) \\
	& = \prob\big((X = Y - 1)\,{\cup}\,(X = Y)\,{\cup}\,(X = Y + 1)\big) \\
	& = 8/40 + 12/40 + 8/40 \\
	& = 7/10
\end{align*}
\end{answer}
%--------------------
\end{parts}

\end{questions}
\end{exercise}
%======================================================================
\endinput
%======================================================================
