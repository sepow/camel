% !TEX root = main.tex

\begin{exercise}
\begin{questions}
%----------------------------------------
% cond prob
\question
Let $A$ and $B$ be events such that $\prob(A)=0.4$, $\prob(B)=0.5$ and $\prob(A\cup B)=0.8$.\par
Compute the following probabilities:
\begin{parts}
\part $\prob(A\cap B)$.
\begin{answer}
$\prob(A\cap B) = \prob(A) + \prob(B) - \prob(A\cup B) = 0.4 + 0.5 - 0.8 = 0.1$.
\end{answer}
\part $\prob(A\cup B^c)$.
\begin{answer}
$\prob(A\cup B^c) = 1 - \prob(B\setminus A) = 1 - \big[\prob(B)-\prob(A\cap B)\big] = 1 - 0.4 = 0.6$.
\end{answer}
\part $\prob(A\,|\,B)$.
\begin{answer}
$\prob(A|B) = \prob(A\cap B)/\prob(B) = 0.1/0.5 = 0.2$.
\end{answer}
\part $\prob(A\,|\,A\cup B)$.
\begin{answer}
$\prob(A|A\cup B) 	= \prob(A)/\prob(A\cup B) = 0.4/0.8 = 0.5$.
\end{answer}
\end{parts}

%----------------------------------------
% cond prob
\question
Let $A$, $B$ and $C$ be events such that $\prob(A)=0.7$, $\prob(B)=0.6$, $\prob(C)=0.5$, $\prob(A\cap B)=0.4$, $\prob(A\cap C)=0.3$, $\prob(B\cap C)=0.2$ and $\prob(A\cap B\cap C)=0.1$. \par
Compute the following probabilities:
\begin{parts}
%--------------------
\part $\prob(A\cup B)$.
\begin{answer}
$\prob(A\cup B) 
	= \prob(A) + \prob(B) - \prob(A\cap B) 
	= 0.7 + 0.6 - 0.4 
	= 0.9$.
\end{answer}
%----------
\part $\prob(A|B)$.
\begin{answer}
$\displaystyle\prob(A|B) 
	= \frac{\prob(A\cap B)}{\prob(B)} 
	= \frac{0.4}{0.6} 
	= \frac{2}{3}$.
\end{answer}
%----------
\part $\prob(A\,|\,A\cup B)$.
\begin{answer}
$\prob(A|A\cup B) 
	= \displaystyle\frac{\prob[A\cap (A\cup B)]}{\prob(A\cup B)} 
	= \displaystyle\frac{\prob(A)}{\prob(A\cup B)} 
	= \frac{0.7}{0.9} 
	= \frac{7}{9}$.
\end{answer}
%--------------------
\part $\prob(A\cup B\cup C)$.
\begin{answer}
By the inclusion-exclusion principle,
\begin{align*}
\prob(A\cup B\cup C)
	& = [\prob(A) + \prob(B) + \prob(C)] - [\prob(A\cap B) + \prob(A\cap C) + \prob(B\cap C)] + \prob(A\cap B\cap C) \\
	& = (0.7 + 0.6 + 0.5) - (0.4 + 0.3 + 0.2) + 0.1 = 1.
\end{align*}
\end{answer}
%----------
\part $\prob(A^{c}\cap B^{c}\cap C)$.
\begin{answer}
Because $A^{c}\cap B^{c}\cap C = (A\cup B)^{c}\cap C$, the sets $A\cup B$ and $A^{c}\cap B^{c}\cap C$ form a partition of $A\cup B\cup C$. Hence $\prob(A^{c}\cap B^{c}\cap C) = \prob(A\cup B\cup C) - \prob(A\cup B) = 1.0 - 0.9 = 0.1$
\end{answer}
%----------
\part $\prob(A^{c}\cap B^{c}\cap C|A\cup B)$.
\begin{answer}
Because $A^{c}$ and $A$ are disjoint, $\prob(A^{c}\cap B^{c}\cap C|A\cup B) = 0$.
\end{answer}
\end{parts}

%----------------------------------------
% cond
\question
A student has three opportunities to pass an exam. The probability of failing the first attempt is 0.6; the probability of failing the second attempt, given that they have failed the first is 0.75, and the probability of failing the third attempt, given that they have failed the first and second is 0.4.
\begin{parts}
\part What is the probability that the student eventually passes the exam.
\begin{answer}
Let $F_i$ denote the event that the student fails at the $i$th
attempt, so that 
\[
\prob(F_1)=0.6,\quad \prob(F_2|F_1)=0.75\quad\mbox{and}\quad \prob(F_3|F_1\cap F_2)=0.4
\]
The probability that the student fails all three attempts is
\[
\prob(F_1\cap F_2\cap F_3) = \prob(F_1)\prob(F_2|F_1)\prob(F_3|F_1\cap F_2) = 0.6\times 0.75\times 0.4 = 0.18
\]
Hence, the probability that the student eventually passes is $1 - 0.18 = 0.82$.
%\paragraph{The chain rule:} The probability that events $A$ and $B$ both occur is $\prob(A\cap B) = \prob(B|A)\prob(A)$: this is sometimes called the \emph{chain rule}. For three events $A$, $B$ and $C$, the probability that all three occur is
%\begin{align*}
%\prob(A\cap B\cap C)
%	& = \prob\big((A\cap B)\cap C\big) \\
%	& = \prob(C|A\cap B)\prob(A\cap B) \\
%	& = \prob(C|A\cap B)\prob(B|A)\prob(A) \\
%\end{align*}
%Similarly, the probability that $A$, $B$, $C$ and $D$ all occur is
%\begin{align*}
%\prob(A\cap B\cap C\cap D) 
%	& = \prob\big((A\cap B\cap C)\cap D\big)	\\
%	& = \prob(D|A\cap B\cap C)\prob(A\cap B\cap C) \\
%	& = \prob(D|A\cap B\cap C)\prob(C|A\cap B)\prob(B|A)\prob(A) \\
%\end{align*}
%and so on.
\end{answer}
\part What are the respective probabilities of passing at the first, second and third attempts.
\begin{answer}
The probability that the student passes on the first attempt is $1-\prob(F_1) = 1-0.6 = 0.4$.
\par
The probability that the student takes a second test is $\prob(F_1)=0.6$. If the second test is taken, the (conditional) probability that the student passes it is $1-\prob(F_2|F_1) = 1-0.75 =
0.25$. Hence, the probability that the student passes on the second attempt is $0.6\times 0.25 = 0.15$.
\par
Similarly, the probability that the student takes the third test is $\prob(F_1\cap F_2) = \prob(F_1)\prob(F_2|F_1) = 0.6\times 0.75 = 0.45$. If the third test is taken, the (conditional) probability that the student passes it is $1-\prob(F_3|F_1\cap F_2)= 1-0.4 = 0.6$. Hence, the probability that the student passes on the third attempt is $0.45\times 0.6 = 0.27$. (The probability of eventually passing is $0.4 + 0.15 + 0.27 = 0.82$, which agrees with the answer to part (a).)
\end{answer}
\end{parts}

%----------------------------------------
% bayes
\question
An insurance company divides its customers into three categories: $60$\% of customers are classed as low-risk, $30$\% as moderate-risk and $10$\% as high-risk. The probabilities that low-risk customers, moderate-risk customers and high-risk customers make a claim in any given year are $0.01$, $0.1$ and $0.5$ respectively. Given that a customer makes a claim this year, what is the probability that the customer is in the high-risk category?

\begin{answer}
Let $L$ denote the event that the customer is low-risk, $M$ the event that the customer is moderate-risk, and $H$ the event that the customer is high-risk:
\[
\prob(L) = 0.6,\quad \prob(M) = 0.3,\quad \prob(H) = 0.1.
\]
Let $C$ be the event that the customer makes a claim this year:
\[
\prob(C\,|\,L) = 0.01,\quad \prob(C\,|\,M) = 0.1,\quad \prob(C\,|\,H) = 0.5.
\]
We need to find $\prob(H\,|\,C)$:
\[
\prob(H\,|\,C) = \frac{\prob(H\cap C)}{\prob(C)} = \frac{\prob(C\,|\,H)\prob(H)}{\prob(C)}.
\]
The events $\{L,M,H\}$ form a partition of sample space (the set of all customers), so by the law of total probability,
\[
\prob(C) = \prob(C\,|\,L)\prob(L) + \prob(C\,|\,M)\prob(M) + \prob(C\,|\,H)\prob(H),
\]
and hence
\[
\prob(H\,|\,C) = \frac{\prob(C\,|\,H)\prob(H)}{\prob(C\,|\,L)\prob(L) + \prob(C\,|\,M)\prob(M) + \prob(C\,|\,H)\prob(H)},
\]
which is Bayes' theorem. The probability that a customer is in the high-risk category, given that the customer makes a claim this year, is 
\[
\prob(H\,|\,C) 
	= \frac{(0.5\times 0.1)}{(0.01\times 0.6) + (0.1\times 0.3) + (0.5\times 0.1)}
	= 0.5814\quad\text{(approx.).}
\]
\end{answer}

%----------------------------------------
% bayes
\question
A horse has three opportunities to clear a fence. The probability that it fails at the first attempt is $0.4$. The probability that it fails at the second attempt, given that it has failed at the first attempt, is $0.3$. The probability that it fails at the third attempt, given that it has failed at the first and second attempts, is $0.8$.
\begin{parts}
\part What is the probability that the horse eventually clears the fence?
\part What are the respective probabilities that the horse clears the fence at the first, second and third attempts.
\end{parts}

\begin{answer}
Let $A$, $B$ and $C$ denote the events that the horse fails at the first, second and third attempts, respectively.
\[
\prob(A)=0.4,\quad \prob(B|A)=0.3,\quad \prob(C|A\cap B)=0.8.
\]
\ben
\it
The probability that the horse fails all three attempts is
\[
\prob(A\cap B\cap C) = \prob(A)\prob(B|A)\prob(C|A\cap B) = 0.4\times 0.3\times 0.8 = 0.096,
\]
so the probability that the horse eventually succeeds is $1 - 0.096 = 0.904$.
\it
\bit
\it 
The probability that the horse succeeds at the first attempt is $1-\prob(A) = 0.6$.
\it
The probability that the horse has a second attempt is $\prob(A)=0.4$. In this case the (conditional) probability of success is $1-\prob(B|A) = 1-0.3 = 0.7$. Hence the probability of success on the second attempt is $0.4\times 0.7 = 0.28$.
\it 
The probability that the horse has a third attempt is $\prob(A\cap B) = \prob(B|A)\prob(A) = 0.3\times 0.4 = 0.12$. In this case the (conditional) probability of success is $1-\prob(C|A\cap B)= 1-0.8 = 0.2$. Hence the probability of success on the third attempt is $0.12\times 0.2 = 0.024$.
\eit
Check: the probability of success on either the first, second or third attempt is $0.6 + 0.28 + 0.024 = 0.904$, which agrees with our answer to part (i).
\een
\end{answer}

%----------------------------------------
\end{questions}
\end{exercise}

%======================================================================
\endinput
%======================================================================
