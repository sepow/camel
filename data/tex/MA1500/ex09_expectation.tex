% !TEX root = main.tex
% ex09_expectation.tex
\begin{exercise}
\begin{questions}
%----------------------------------------
%----------------------------------------
%----------------------------------------
% simple rv
% RND p88q4
\question
Two fair dice are rolled independently. Let $X$ denote the larger of the two scores. Find the PMF of $X$, and its expected value $\expe(X)$.

\begin{answer}
The PMF of $X$ is as follows:
\[
\begin{array}{|c|cccccc|}\hline
k       & 1     & 2      & 3		& 4 		& 5 		& 6 \\ \hline
p_k		& 1/36  & 3/36	& 5/36	& 7/36	& 9/36	& 11/36  \\ \hline
\end{array}
\]
The expected value of $X$ is therefore
\[
\expe(X) = \left(1 \times\frac{1}{36}\right) + \left(2 \times\frac{3}{36}\right) + \left(3 \times\frac{5}{36}\right) + \left(4 \times\frac{7}{36}\right) + \left(5 \times\frac{9}{36}\right) + \left(6 \times\frac{11}{36}\right)  = \frac{161}{36}
\]
\end{answer}



%----------------------------------------
% lottery
% RND p90q9
\question
A thousand tickets are sold in a lottery, in which there is one prize of $\pounds 200$, four prizes of $\pounds 50$ and ten prizes of $\pounds 10$. If a ticket costs $\pounds 1$, calculate the expected net gain from one ticket. 

\begin{answer}
Let $X$ be the net gain from one ticket. The PMF of $X$ is as follows:
\[
\begin{array}{|c|cccc|} \hline
k       & 199     & 49      	& 9			& -1 		\\ \hline
p(k)		& 1/1000  & 4/1000	& 10/1000	& 985/1000	\\ \hline
\end{array}
\]
The expected value of $X$ is therefore
\[
\expe(X) = \left(199 \times\frac{1}{1000}\right) + \left(49 \times\frac{4}{1000}\right) + \left(9 \times\frac{10}{1000}\right) + \left(-1 \times\frac{985}{1000}\right) = - \frac{500}{1000} = -\frac{1}{2}
\]
so we expect to lose 50p on each ticket we buy.
\end{answer}

%----------------------------------------
% simple rv
% RND p100q12
\question
A gambler chooses a number between $1$ and $6$. Three fair dice are rolled, and if the gambler's number appears $k$ times ($k=1,2,3$), then she wins $\pounds k$, but if her number fails to appear she loses $\pounds 1$. What is the gambler's expected winnings?

\begin{answer}
Let $W$ denote the gambler's winnings and let $X$ be the number of times that the chosen number appears. Then $X\sim\text{Binomial}(3,\frac{1}{6})$, and
\begin{align*}
\prob(W=-1)	& = \prob(X=0) = 	\left(\frac{5}{6}\right)^3							= \frac{125}{216}, \\
\prob(W= 1)	& = \prob(X=1) = 	3\left(\frac{1}{6}\right)\left(\frac{5}{6}\right)^2 	= \frac{75}{216}, \\
\prob(W= 2)	& = \prob(X=2) = 	3\left(\frac{1}{6}\right)^2\left(\frac{5}{6}\right)   	= \frac{15}{216}, \\
\prob(W= 3)	& = \prob(X=3) = 	\left(\frac{1}{6}\right)^3 = \frac{1}{216}. \\
\end{align*}
The expected value of $W$ is therefore
\[
\expe(W) = \left(-1 \times\frac{125}{216}\right) + \left(1 \times\frac{75}{216}\right) + \left(2 \times\frac{15}{216}\right) + \left(3 \times\frac{1}{216}\right) = -\frac{17}{216}.
\]
The gambler can therefore expect to lose nearly 8p per game.
\end{answer}


%----------------------------------------
% expectation
% RND p92q13
\question
Let $X$ be a random variable with the following PMF, where $0\leq\alpha\leq1/2$ is a constant:
\[
f(x) = \begin{cases}
	\alpha		& \text{if}\ \ x=1, \\
	1-2\alpha	& \text{if}\ \ x=2, \\
	\alpha		& \text{if}\ \ x=3, \\
	0			& \text{otherwise,}
\end{cases}
\]
\begin{parts}
\part Find $\expe(X)$ and $\displaystyle\expe\left(\frac{1}{X}\right)$.
\begin{answer}
\begin{align*}
\expe(X) 
	& = \sum_{k=1}^3 kf(k) = 1\alpha + 2(1-2\alpha) + 3\alpha = 2. \\
\expe\left(\frac{1}{X}\right)
	& = \sum_{k=1}^3 \frac{1}{k} f(k) = 1\alpha + \frac{1}{2}(1-2\alpha) + \frac{1}{3}\alpha = \frac{\alpha}{3} + \frac{1}{2}.
\end{align*}
\end{answer}
\part Find a condition on $\alpha$ under which $\expe\left(\displaystyle\frac{1}{X}\right) = \displaystyle\frac{1}{\expe(X)}$.
\begin{answer}
If $\displaystyle\expe\left(\frac{1}{X}\right) = \frac{1}{\expe(X)}$ we must have that $\frac{\alpha}{3} + \frac{1}{2} = \frac{1}{2}$. 
\par
Hence the required condition is that $\alpha=0$. 
\end{answer}
\part Use your result to show that $\displaystyle\expe\left(\frac{1}{X}\right)\neq \displaystyle\frac{1}{\expe(X)}$ in general.
\begin{answer}
The above argument shows that $\displaystyle\expe\left(\frac{1}{X}\right) \neq \displaystyle\frac{1}{\expe(X)}$ whenever $\alpha\neq 0$.
\end{answer}
\end{parts}

\end{questions}
\end{exercise}
%======================================================================
\endinput
%======================================================================
