% !TEX root = main.tex
%======================================================================
\chapter{Covariance and Correlation}\label{chap:covar}
%======================================================================

Let $X$ and $Y$ be simple random variables. Let $\{x_1,x_2,\ldots,x_m\}$ be the range of $X$, and let $\{y_1,y_2,\ldots,y_n\}$ be the range of $Y$.
Recall that $X$ and $Y$ together are described by their joint PMF,
\[
f(x,y) = \prob(X=x,Y=y).
\]
where $f(x,y)\neq 0$ only if $x\in\{x_1,x_2,\ldots,x_m\}$ and $y\in\{y_1,y_2,\ldots,y_n\}$.

%----------------------------------------------------------------------
\section{Product moments}
%----------------------------------------------------------------------
%Let $(\Omega,\prob)$ be a finite probability space.
%\bit
%\it A simple random variable $X:\Omega\to\R$ is described by its PMF:
%\[
%f(x) = \prob(X=x).
%\]
%\it A pair of simple random variables $X,Y:\Omega\to\R$ are described by their joint PMF:
%\[
%f(x,y) = \prob(X=x,Y=y).
%\]
%\eit
Let $g:\R^2\to\R$ be any function of two variables:
\[
\begin{array}{cccc}
g:		& \R^2	& \to	& \R \\
		& (x,y)	& \mapsto	& g(x,y)
\end{array}
\]
$X$ and $Y$ can be combined to create a new random variable $g(X,Y)$, defined by
\[
\begin{array}{cccc}
g(X,Y):		& \Omega	& \to		& \R \\[0.5ex]
			& \omega 	& \mapsto	& g\big[X(\omega),Y(\omega)\big].
\end{array}
\]
As in Theorem~\ref{thm:lus}, the expected value of $g(X,Y)$ can be computed without having to compute its distribution explicitly:  
%% lemma
%\begin{theorem}\label{thm:lus2}
%The expected value of $g(X,Y)$ can be written as
\[
\expe\big[g(X,Y)\big] = \sum_{i=1}^m\sum_{j=1}^n g(x_i,y_j)f(x_i,y_j).
\]
%\end{theorem}
%\proofomitted
%
%\break % <<

% corollary
The special case $g(x,y)=xy$ yields the following definition. %in Theorem~\ref{lem:lus2} leads to the following:

\begin{definition}
$\expe(XY)$ is called the \emph{product moment} of $X$ and $Y$.
\end{definition}

To compute $\expe(XY)$, we can either compute the PMF of $XY$ explicitly, or use the joint PMF of $X$ and $Y$ as above:
\[
\expe(XY) = \sum_{i=1}^m\sum_{j=1}^n x_i\,y_j\,f(x_i,y_j).
\]
%where the sum is taken over all pairs $(x,y)$ for which $f(x,y)>0$.

%\bit
%\it The product moment $\expe(XY)$ quantifies the \emph{dependence} between $X$ and $Y$.
%\eit
%
% lemma: independent implies uncorrelated
\begin{lemma}\label{lem:independent_implies_uncorrelated}
If $X$ and $Y$ are independent then $\expe(XY)=\expe(X)\expe(Y)$.
\end{lemma}

% proof
\begin{proof}
%$X$ and $Y$ are independent, so $f(x,y)=f_X(x)f_Y(y)$ for all $x,y\in\R$.
%Hence
%\begin{align*}
%\expe(XY) = \sum_{x,y} xy\, f(x,y)
%	& = \sum_{x,y} xy\, f_X(x)f_Y(y) \\
%	& = \left(\sum_x x\, f_X(x)\right)\left(\sum_y y\, f_Y(y)\right) \\
%	& = \expe(X)\expe(Y).
%\end{align*}
%
%Here is a more explcit proof, in which we specify the possible values taken by $X$ and $Y$.
%\bit
%\it Let $\{x_1,\ldots,x_m\}$ denote the range of $X$.
%\it Let $\{y_1,\ldots,y_n\}$ denote the range of $Y$. 
%\eit
%Then
\begin{align*}
\expe(XY) 
	& = \sum_{i=1}^m\sum_{j=1}^n x_i y_j \, f(x_i,y_j) \\
	& = \sum_{i=1}^m\sum_{j=1}^n x_i y_j \, f_X(x_i)f_Y(y_j) \qquad\text{by independence.}\\
	& = \left(\sum_{i=1}^m x_i \, f_X(x_i)\right)\left(\sum_{j=1}^n y_j \, f_Y(y_j)\right) \\
	& = \expe(X)\expe(Y).
\end{align*}
\end{proof}

%----------------------------------------------------------------------
\section{Correlation}
%----------------------------------------------------------------------

% definition: uncorrelated
\begin{definition} 
Two random variables $X$ and $Y$ are said to be \emph{correlated} if $\expe(XY)\neq\expe(X)\expe(Y)$, otherwise they are said to be \emph{uncorrelated}.
\end{definition}

% theorem: properties of variance
\begin{theorem}\label{thm:additivity_of_variance}
If $X$ and $Y$ are uncorrelated then $\var(X+Y) = \var(X) + \var(Y)$.
\end{theorem}

% proof
\begin{proof}
If $X$ and $Y$ are uncorrelated, we have $\expe(XY)=\expe(X)\expe(Y)$, so
\begin{align*}
\var(X+Y)	& = \expe\left(\big[X+Y - \expe(X+Y)\big]^2\right) \\
			& = \expe\left(\big[X-\expe(X)\big]^2 + 2\big[XY-\expe(X)\expe(Y)\big] + \big[Y-\expe(Y)\big]^2\right) \\
			& = \var(X) + 2\left[\expe(XY)-\expe(X)\expe(Y)\right] + \var(Y) \\
			& = \var(X) + \var(Y).
\end{align*}
\vspace*{-3ex}
\end{proof}
\begin{remark}
If $X$ and $Y$ are independent then they are uncorrelated, but the converse is not necessarily true.
\end{remark}

%Independent variables are uncorrelated, but uncorrelated variables are not necessarily independent.
%% example
%\begin{example}
%Let $X$ and $Y$ be two Bernoulli random variables with parameter $\frac{1}{2}$. Show that the random variables $U=X+Y$ and $V=|X-Y|$ are uncorrelated but not independent.
%\end{example}
%
%\begin{solution}
%The PDFs of $U$, $V$ and $UV$ are shown in the following table:
%\[
%\begin{array}{|l|ccc|}\hline
%k				& 0		& 1		& 2	\\ \hline
%\prob(U=k)		& 1/4	& 1/2	& 1/4	\\ \hline
%\prob(V=k)		& 1/2	& 1/2	& \\ \hline
%\prob(UV=k) 		& 1/2	& 1/2	& \\ \hline
%\end{array}
%%\qquad\qquad
%%\begin{array}{|cc|cc|c|}\hline
%%X & Y & U & V & UV \\ \hline
%%0 & 0 & 0 & 0 & 0 \\
%%0 & 1 & 1 & 1 & 1 \\
%%1 & 0 & 1 & 1 & 1 \\
%%1 & 1 & 2 & 0 & 0 \\ \hline
%%\end{array}
%\]
%
%%The probability mass functions are:
%%\[\begin{array}{|c|ccc|}\hline
%%u			& 0		& 1		& 2	\\ \hline
%%P(U=u)		& 1/4	& 1/2	& 1/4	\\ \hline \hline
%%v			& 0		& 1		& \\ \hline			
%%P(V=v)		& 1/2	& 1/2	& \\ \hline \hline
%%w			& 0		& 1		& \\ \hline			
%%P(UV=w) 		& 1/2	& 1/2	& \\ \hline
%%\end{array}\]
%%
%\bit
%\it $\expe(U) = \sum_k k\prob(U=k) = \left(0\times 1/4\right) + \left(1\times 1/2\right) + \left(1\times 1/4\right) = 1$.
%\it $\expe(V) = \sum_k k\,\prob(V=k) = \left(0\times 1/2\right) + \left(1\times 1/2\right) = 1/2$.
%\it $\expe(UV) = \sum_k k\,\prob(UV=k) = \left(0\times 1/2\right) + \left(1\times 1/2\right) = 1/2$.
%\eit
%Hence $U$ and $V$ are uncorrelated, because $\expe(UV)=\expe(U)\expe(V)$. However, 
%\bit
%\it $\prob(U=0,V=0) = 1/4$, but 
%\it $\prob(U=0)\prob(V=0) = 1/4 \times 1/2 = 1/8$.
%\eit
%Hence $U$ and $V$ are not independent, because $\prob(U=0,V=0)\neq\prob(U=0)\prob(V=0)$.
%\end{solution}

%----------------------------------------------------------------------
\section{Covariance}
%----------------------------------------------------------------------
The covariance of $X$ and $Y$ is the product moment of the centred variables $X-\expe(X)$ and $Y-\expe(Y)$.

% definition: covariance 
\begin{definition}
The \emph{covariance} of $X$ and $Y$ is 
\begin{align*}
\cov(X,Y) 
	& = \expe\left[(X-\expe(X))(Y-\expe(Y))\right]  \\
	& = \expe(XY) - \expe(X)\expe(Y)
\end{align*}
\end{definition}

\begin{remark}
\ben
\it Variance is a special case of covariance: $\var(X) = \cov(X,X)$.
\it $X$ and $Y$ are uncorrelated if and only if $\cov(X,Y)=0$. 
\een
\end{remark}

% tedious example (continued)
\begin{example}
Let $X$ and $Y$ be random variables with joint PMF shown in the following table. 
%\small
%\[\begin{array}{|c|c|c|c|}\hline
%	& Y=2	& Y=3	& Y=4	\\ \hline
%X=1	& 1/12	& 1/6	& 0 		\\ \hline
%X=2	& 1/6	& 0		& 1/3 	\\ \hline
%X=3	& 1/12	& 1/6  	& 0		\\ \hline
%\end{array}\]
%\normalsize
\[
\begin{array}{|cc|ccc|}\hline
	&       &       & y     &   \\
    &       & 2     & 3     & 4 \\ \hline
	& 1		& 1/12	& 1/6	& 0 		\\ 
x	& 2		& 1/6	& 0		& 1/3 	\\ 
	& 3		& 1/12	& 1/6  	& 0		\\ \hline
\end{array}
\]
Find the covariance of $X$ and $Y$.
%\[\begin{array}{|cc|ccc|}\hline
%	&	& 		& Y(\Omega) 	&			\\ 
%	&	& 2		& 3		& 4		\\\hline
%	& 1	& 1/12	& 1/6	& 0 		\\
%X(\Omega)	& 2	& 1/6	& 0		& 1/3 	\\
%	& 3	& 1/12	& 1/6  	& 0		\\ \hline
%\end{array}\]
\end{example}

\begin{solution}
To find $\expe(X)$ and $\expe(Y)$, we need the marginal PMFs of $X$ and $Y$:
\[\begin{array}{|c|ccc|}\hline
x		& 1		& 2		& 3		\\ \hline
P(X=x)	& 1/4	& 1/2	& 1/4	\\ \hline
\end{array}
\qquad
\begin{array}{|c|ccc|}\hline
y		& 2		& 3		& 4		\\ \hline
P(Y=y)	& 1/3	& 1/3	& 1/3	\\ \hline
\end{array}\]
\bit
\it $\expe(X) = \left(1\times 1/4\right) + \left(2\times 1/2\right) + \left(3\times 1/4\right) = 2$.
\it $\expe(Y) = \left(2\times 1/3\right) + \left(3\times 1/3\right) + \left(4\times 1/3\right) = 3$.
\eit

To find $\expe(XY)$, we compute the PMF of the random variable $Z=XY$:
\[\begin{array}{|c|cccccc|}\hline
z		& 2		& 3		& 4		& 6		& 8		& 9	\\ \hline
P(XY=z)	& 1/12	& 1/6	& 1/6	& 1/12	& 1/3	& 1/6 \\ \hline
\end{array}\]
The expectation of $XY$ is therefore equal to
\[
\expe(XY) = \left(2\times\frac{1}{12}\right) + \left(2\times\frac{1}{6}\right) + \left(4\times\frac{1}{6}\right) + \left(6\times\frac{1}{12}\right) + \left(8\times\frac{1}{3}\right) + \left(9\times\frac{1}{6}\right) = 6.
\]
Hence the covariance of $X$ and $Y$ is
\begin{align*}
\cov(X,Y)
	=\expe(XY)-\expe(X)\expe(Y)
	= 6 - (2\times 3)
	= 0.
\end{align*}
Thus $X$ and $Y$ are uncorrelated (although they are not independent).
\end{solution}

%----------------------------------------------------------------------
\section{The correlation coefficient}
%----------------------------------------------------------------------
%The correlation coefficient of $X$ and $Y$ is the product moment of the standardised variables $\displaystyle\frac{X-\expe(X)}{\var(X)}$ and $\displaystyle\frac{Y-\expe(Y)}{\var(Y)}$.
%The correlation coefficient of $X$ and $Y$ is the product moment of the standardised variables $[X-\expe(X)]/\var(X)$ and $[Y-\expe(Y)]/\var(Y)$.


The correlation coefficient of $X$ and $Y$ is the product moment of the standardised variables 
\[
\displaystyle\frac{X-\expe(X)}{\sqrt{\var(X)}} \qquad\text{and}\qquad \displaystyle\frac{Y-\expe(Y)}{\sqrt{\var(Y)}}.
\]

\begin{definition}
The \emph{correlation coefficient} of $X$ and $Y$ is 
\[
\rho(X,Y) = \frac{\cov(X,Y)}{\sqrt{\var(X)\cdot\var(Y)}}
\]
\end{definition}

\begin{remark}
\bit
\it We will show that $|\rho(X,Y)|\leq 1$ for any two random variables $X$ and $Y$.
\it $\rho(X,Y)$ provides a \emph{standardised} measure of (linear) dependence between $X$ and $Y$.
\it Note that $X$ and $Y$ are uncorrelated if and only if $\rho(X,Y)=0$. 
\eit
\end{remark}
%% theorem
%\begin{theorem}\label{thm:correlation}
%For any two random variables $X$ and $Y$, 
%\[
%|\rho(X,Y)|\leq 1,
%\]
%with equality if and only if $Y=aX$ for some $a\in\R$.
%\end{theorem}
%
% theorem
\begin{theorem}[Cauchy-Schwarz inequality]\label{thm:cauchyschwarz}
For any two random variables $X$ and $Y$, 
\[
\expe(XY)^2\leq\expe(X^2)\expe(Y^2)
\]
with equality if and only if $Y=aX$ for some $a\in\R$.

\end{theorem}

\begin{proof}
Let $a\in\R$ consider the random variable $Z=aX-Y$. 
\par
By the properties of expectation,
%\begin{align*}
%Z^2\geq 0 \text{ for all } a\in\R
%	& \Rightarrow \expe(Z^2)\geq 0 \text{ for all } a\in\R \\
%	& \Rightarrow \expe(a^2X^2 - 2aXY + Y^2) \geq 0 \text{ for all } a\in\R \\
%	& \Rightarrow a^2\expe(X^2) - 2a\expe(XY) + \expe(Y^2) \geq 0 \text{ for all } a\in\R.
%\end{align*}
\begin{align*}
Z^2\geq 0 
	& \Rightarrow \expe(Z^2)\geq 0  \\
	& \Rightarrow \expe(a^2X^2 - 2aXY + Y^2) \geq 0 \\
	& \Rightarrow a^2\expe(X^2) - 2a\expe(XY) + \expe(Y^2) \geq 0.
\end{align*}

\bigskip
\bit
\it Let $h(a)=a^2\expe(X^2) - 2a\expe(XY) + \expe(Y^2)$. This is a quadratic expression in $a$.
\eit
Since $h(a)\geq 0$ for all $a\in\R$, the roots of the quadratic equation $h(a)=0$, given by
\[
a = \frac{\expe(XY)\pm\sqrt{\expe(XY)^2-\expe(X^2)\expe(Y^2)}}{\expe(X^2)}
\]
are either both complex (discriminant is negative) or co-incide (discriminant is zero). 

%\bigskip
\bit
\it Hence $\expe(XY)^2-\expe(X^2)\expe(Y^2) \leq 0$, or equivalently, $\expe(XY)^2\leq \expe(X^2)\expe(Y^2)$.
\eit

%\bigskip
Finally, the discriminant is zero if and only if $Z=0$.
\bit
\it Hence $\expe(XY)^2=\expe(X^2)\expe(Y^2)$ if and only if $Y=aX$ for some $a\in\R$.\qed
\eit
\end{proof}

% theorem
\begin{theorem}\label{thm:correlation}
For any two random variables $X$ and $Y$, 
\[
|\rho(X,Y)|\leq 1,
\]
with equality if and only if $Y=aX$ for some $a\in\R$.
\end{theorem}

\begin{proof}
By the Cauchy-Schwarz inequality,
\begin{align*}
\cov(X,Y)^2
	& = \expe\Big(\big[X-\expe(X)\big]\big[Y-\expe(Y)\big]\Big)^2 \\
	& \leq \expe\Big(\big[X-\expe(X)\big]^2\big)\expe\Big(\big[Y-\expe(Y)\big]^2\Big) \\
	& = \var(X)\var(Y),
\end{align*}
with equality if and only if $Y=aX$ for some $a\in\R$. Hence,
\[
|\cov(X,Y)| \leq \sqrt{\var(X)\var(Y)},
\text{ which implies that }
|\rho(X,Y)|	\leq 1,
\]
with equality if and only if $Y=aX$ for some $a\in\R$. 
\end{proof}


%----------------------------------------------------------------------
%\newpage
\section{Exercises}
% !TEX root = main.tex
% ex14_covariance.tex
\begin{exercise}
\begin{questions}
%----------------------------------------
\question
Let $X$ and $Y$ be two Bernoulli random variables with parameter $\frac{1}{2}$. Show that the random variables $U=X+Y$ and $V=|X-Y|$ are uncorrelated but not independent.
\begin{answer}
The PDFs of $U$, $V$ and $UV$ are shown in the following table:
\[
\begin{array}{|l|ccc|}\hline
k				& 0		& 1		& 2	\\ \hline
\prob(U=k)		& 1/4	& 1/2	& 1/4	\\ \hline
\prob(V=k)		& 1/2	& 1/2	& \\ \hline
\prob(UV=k) 		& 1/2	& 1/2	& \\ \hline
\end{array}
\]
\bit
\it $\expe(U) = \sum_k k\prob(U=k) = \left(0\times 1/4\right) + \left(1\times 1/2\right) + \left(1\times 1/4\right) = 1$.
\it $\expe(V) = \sum_k k\,\prob(V=k) = \left(0\times 1/2\right) + \left(1\times 1/2\right) = 1/2$.
\it $\expe(UV) = \sum_k k\,\prob(UV=k) = \left(0\times 1/2\right) + \left(1\times 1/2\right) = 1/2$.
\eit
Hence $U$ and $V$ are uncorrelated, because $\expe(UV)=\expe(U)\expe(V)$. However, 
\bit
\it $\prob(U=0,V=0) = 1/4$, but 
\it $\prob(U=0)\prob(V=0) = 1/4 \times 1/2 = 1/8$.
\eit
Hence $U$ and $V$ are not independent, because $\prob(U=0,V=0)\neq\prob(U=0)\prob(V=0)$.
\end{answer}

%%---------------------------------------------
%\question
%Two random variables $X$ and $Y$ have joint PMF shown in the following table:
%\[
%\begin{array}{|cc|ccc|}\hline
%	&       &       & y     &   \\
%    &       & 0     & 1     & 2 \\ \hline
%	& 0     & 1/24  & 3/24  & 2/24 \\
%x   & 1     & 2/24  & 4/24  & 6/24 \\
%	& 2     & 1/24  & 1/24  & 4/24 \\ \hline
%\end{array}
%\]
%Find the covariance and correlation coefficient of $X$ and $Y$
%\begin{answer}
%To compute $\cov(X,Y)$, we need $\expe(XY)$, $\expe(X)$ and $\expe(Y)$. The marginal PMFs of $X$ and $Y$ are given by
%\[
%\begin{array}{c|cccc}
%x       & 0     & 1      & 2 \\ \hline
%p_X(x)  & 6/24  & 12/24  & 6/24  
%\end{array}
%\]
%and
%\[
%\begin{array}{c|cccc}
%y       & 0     & 1     & 2 \\ \hline
%p_Y(y)  & 4/24  & 8/24  & 12/24  
%\end{array}
%\]
%so
%\begin{align*}
%\expe(X) & = (0\times 6/24) + (1\times 12/24) + (2\times  6/24) = 1 \\
%\expe(Y) & = (0\times 4/24) + (1\times  8/24) + (2\times 12/24) = 32/24
%\end{align*}
%
%To compute $\expe(XY)$, we need the PMF of $Z=XY$:
%\[
%\begin{array}{c|cccc}
%z           & 0     & 1     & 2    & 4 \\ \hline
%p_{XY}(z)   & 9/24  & 4/24  & 7/24 & 4/24 \\
%\end{array}
%\]
%From here,
%\begin{align*}
%\expe(XY) 	& = (0\times 9/24) + (1\times 4/24) + (2\times 7/24) + (4\times 4/24) = 34/24 \\
%\intertext{and therefore}
%\cov(X,Y) 	& = \expe(XY)-\expe(X)\expe(Y) = 34/24 - 32/24 = 1/12
%\end{align*}
%
%To compute $\rho(X,Y)$, we also need $\var(X)$ and $\var(Y)$:
%\begin{align*}
%\expe(X^2) & = (0\times 6/24) + (1\times 12/24) + (4\times  6/24) = 36/24 \\
%\expe(Y^2) & = (0\times 4/24) + (1\times  8/24) + (4\times 12/24) = 56/24
%\intertext{so}
%\var(X)	& = \expe(X^2) - \expe(X)^2 = 36/24 - 1 = 1/2 \\
%\var(Y) & = \expe(Y^2) - \expe(Y)^2 = 56/24 - (32/24)^2	= 5/9
%\intertext{and therefore}
%\rho(X,Y) & = \frac{\cov(X,Y)}{\sqrt{\var(X)}\sqrt{\var(Y)}} = 0.158
%\end{align*}
%\end{answer}

\end{questions}
\end{exercise}
%======================================================================
\endinput
%======================================================================

%----------------------------------------------------------------------

%======================================================================
\endinput
%======================================================================
