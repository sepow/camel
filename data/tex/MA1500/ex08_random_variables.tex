% !TEX root = main.tex
% ex08_random_variables.tex
\begin{exercise}
\begin{questions}
%----------------------------------------
%----------------------------------------
\question
Let $I_A$ denote the indicator function of an event $A$. Show that

\begin{parts}
%--------------------
\part $I_{A^c} = 1 - I_A$.
\begin{answer}
To show that two random variables $X$ and $Y$ are equal, we must show that $X(\omega)=Y(\omega)$ for all $\omega\in\Omega$.
\bit
\it For all $\omega\in\Omega$, $I_{A^c}(\omega)=0$ if and only if $I_A(\omega)=1$.
\eit
\end{answer}
%--------------------
\part $I_{A\cap B} = I_A I_B$.
\begin{answer}
For all $\omega\in\Omega$, $I_{A\cap B}(\omega)=1$ if and only if both $I_A(\omega)=1$ and $I_B(\omega)=1$.
\end{answer}
%--------------------
\part $I_{A\cup B} = I_A + I_B - I_{A\cap B}$.
\begin{answer}
For all $\omega\in\Omega$, 
\[
\begin{array}{lll}
1 - I_{A\cup B}(\omega)
 & = I_{(A\cup B)^{c}}(\omega)								& \text{ by (a)} \\
 & = I_{A^{c}\cap B^{c}}(\omega)							& \text{ by De Morgan's laws} \\
 & = I_{A^{c}}(\omega)I_{B^{c}}(\omega)						& \text{ by (b)} \\
 & = (1 - I_A(\omega))(1 - I_B(\omega))						& \text{ by (a)} \\
 & = 1 - I_A(\omega) - I_B(\omega) + I_A(\omega)I_B(\omega)	&			\\
 & = 1 - I_A(\omega) - I_B(\omega) + I_{A\cap B}(\omega) 	& \text{ by (b)}
\end{array}
\]
\end{answer}
\end{parts}

%----------------------------------------
\question
Let $\Omega$ be the sample space of some random experiment, $\mathcal{F}$ be a field of events over $\Omega$, and let $A\in\mathcal{F}$. Show that the indicator variable $I_A:\Omega\to\R$ of event $A$, defined by
\[
I(\omega) =
  \begin{cases}
   1 & \text{if } \omega\in A, \\
   0 & \text{if } \omega\notin A,
  \end{cases}
\]
is indeed a random variable on $(\Omega,\mathcal{F})$.
\begin{answer}
\[
\{\omega:I_A(\omega) \leq x\} = \begin{cases}
	\emptyset	& \text{ for } x < 0, \\
	A^c			& \text{ for } 0 \leq x < 1, \\
	\Omega		& \text{ for } x \geq 1.
	\end{cases}
\]
Because $\mathcal{F}$ is a field of sets, $\emptyset$, $A^c$ and $\Omega$ are all elements of $\mathcal{F}$. Thus all sets of the form $\{\omega:I_A(\omega) \leq x\}$ are contained in $\mathcal{F}$, so $I_A$ is a random variable.

For any $B\in\mathcal{B}$,
\bit
\it if $1\in B$, then $\{\omega:X(\omega)\in B\} = A$, which is contained in $\mathcal{F}$;
\it if $1\notin B$, then $\{\omega:X(\omega)\in B\} = \emptyset$, which is also contained in $\mathcal{F}$.
\eit
\end{answer}

%----------------------------------------
\question
Let $X_1$ and $X_2$ be the numbers obtained in two independent throws of a fair die. Find the PMF of each of the following random variables:
\begin{parts}
%--------------------
\part $X_1$,
\begin{answer}
$\prob(X_1=k) = 1/6$ for $k=1,\ldots,6$.
\end{answer}

%--------------------
\part $Y = 7 - X_1$,
\begin{answer}
$\prob(Y=k) = 1/6$ for $k=1,\ldots,6$.
\end{answer}

%--------------------
\part $U = \max(X_1,X_2)$,
\begin{answer}
Let $U=\max\{X_1,X_2\}$. Then since $\{X_1\leq k\}$ and $\{X_2\leq k\}$ are independent events,
\begin{align*}
\prob(U\leq k) 
	& = P(X_1\leq k \text{ and } X_2\leq k) \\
    & = P(X_1\leq k)P(X_2\leq k) \\
    & = (k/6)\cdot (k/6) = k^2/36
\end{align*}
Thus,
\[
\prob(U=k)  
	= P(U\leq k)-P(U\leq k-1)
    = \frac{(k^2-(k-1)^2)}{36} 
    = \frac{(2k-1)}{36}.
\]    
\end{answer}

%--------------------
\part $V = X_1-X_2$.
\begin{answer}
The values of $V=X_1-X_2$ at each point of the sample space $\Omega=\{(i,j):1\leq i,j\leq 6\}$ are
\[\begin{array}{|cc|cccccc|}\hline
	& 	& & & j & & & \\
	& 	& 1	 & 2  & 3  & 4  & 5  & 6 \\\hline
	& 1 	& 0  & 1  & 2  & 3  & 4  & 5 \\
	& 2 	& -1 & 0  & 1  & 2  & 3  & 4 \\
i	& 3 	& -2 & -1 & 0  & 1  & 2  & 3 \\
	& 4 	& -3 & -2 & -1 & 0  & 1  & 2 \\
	& 5 	& -4 & -3 & -2 & -1 & 0  & 1 \\
	& 6 	& -5 & -4 & -3 & -2 & -1 & 0 \\ \hline
\end{array}\]
The required probabilities are obtained by counting the number of outcomes that give the same value of $V=X_1-X_2$:
\small
\[\begin{array}{r|rrrrrrrrrrr}
v         & -5  	& -4	 	& -3 	& -2		& -1		& 0		& 1		& 2		& 3		& 4		& 5		\\ \hline
\prob(V=v)    & 1/36	& 2/36	& 3/36	& 4/36	& 5/36	& 6/36	& 5/36	& 4/36	& 3/36	& 2/36	& 1/36 	\\
\end{array}\]
\normalsize
\end{answer}

%--------------------
\part $W = |X_1-X_2|$.
\begin{answer}
\[
\begin{array}{c|cccccc}
w			& 0		& 1		& 2 		& 3 		& 4 		& 5 		\\ \hline
\prob(W=w)	& 6/36	& 10/36	& 8/36	& 6/36	& 4/36	& 2/36 
\end{array}
\]
\end{answer}
%--------------------
\end{parts}

%\newpage

%----------------------------------------
\question
An urn contains 7 red balls and 3 blue balls. If 5 balls are selected at random without replacement, find the PMF of the number of red balls selected.
\begin{answer}
Let $X$ be the number of red balls selected. The PMF of $X$ is
\[
\begin{array}{c|cccc}
x		& 2 		& 3 		& 4 		& 5 \\ \hline
f(x)	& 1/12	& 5/12 	& 5/12 	& 1/12 
\end{array}
\]
For example, 
\[
f(3) = \prob(X=3) = \frac{\binom{7}{3}\binom{3}{2}}{\binom{10}{5}} = \frac{5}{12}.
\]
This \emph{Hypergeometric distribution}: suppose $n$ objects are chosen from a collection of $N$ objects, of which exactly $K$ are of a particular kind. Choosing an object of this kind is called a \emph{success}. The probability of observing exacly $k$ such objects in $n$ draws \emph{without replacement} from the initial population of $N$ objects, is
\[
f(x) = \frac{\binom{K}{x}\binom{N-K}{n-x}}{\binom{N}{n}}.
\]
Note that $f(x)$ is non-zero only if $\max(0,n-N+K)\leq x\leq \min(K,n)$.
\end{answer}


%----------------------------------------
\question
Let $X$ be a random variable with the following PMF:
\[\begin{array}{c|cccc}
x		& -2 	& 0 		& 1 		& 4 \\ \hline
f(x)	& 0.4	& 0.1 	& 0.3 	& 0.2 
\end{array}\]
Sketch the PMF and CDF of $X$.
\begin{answer}
The CDF of $X$ is as follows:
\[
F(x) = \left\{\begin{array}{ll}
0  	&  x < -2, \\
0.4	& -2 \leq x < 0, \\
0.5	&  0 \leq x < 1, \\
0.8	&  1 \leq x < 4, \\
1	&  x \geq 4.
\end{array}\right.
\]
\end{answer}


%----------------------------------------
% simple rv
% RND p78
\question
Let $X$ be a random variable with PMF:
\[
f(x) = \begin{cases}
	c		& \text{if}\ \ x=0, \\
	3c		& \text{if}\ \ x=1, \\
	6c		& \text{if}\ \ x=2, \\
	0			& \text{otherwise.}
\end{cases}
\]
where $c$ is an unknown constant.

\begin{parts}
%--------------------
\part Find the value of $c$. 
\begin{answer}
%\newpage
Since $\sum_x f(x)=1$ we see that $10c=1$, so $c=1/10$.
\end{answer}
%--------------------
\part Find the probabilities $\prob(X<2)$, $\prob(X\leq 2)$ and $\prob(X>1)$.
\begin{answer}
\bit
\it $\prob(X < 2) = \prob(X=0)+\prob(X=1) =  \frac{4}{10}$.
\it $\prob(X\leq 2) = \prob(X=0)+\prob(X=1)+\prob(X=2) = 1$.
\it $\prob(X>1) = \prob(X=2) = \frac{6}{10}$.
\eit
\end{answer}
%--------------------
\part What is the smallest value of $k$ for which $\prob(X\leq k) > 0.25$?
\begin{answer}
Since $\prob(X\leq 1) = \frac{4}{10}$ and $\prob(X < 1) = \frac{1}{10}$, $k=1$ is the smalest value for which $\prob(X\leq k)\geq \frac{1}{4}$.
\end{answer}
%--------------------
\end{parts}


%----------------------------------------
% binomial (easy)
% RND p97q5
\question
A pair of fair dice is rolled six times. What is the probability of getting a total of seven
\ben
\it twice,
\it at least once,
\it more than three times?
\een
\begin{answer}
Let $X$ be the number of times that a total of 7 is obtained. 
\bit
\it Then $X\sim\text{Binomial}(n,p)$ with $n=6$ and $p=\frac{6}{36}=\frac{1}{6}$.
\eit
\ben
\it $\prob(X=2) 
	= \binom{6}{2}\left(\frac{1}{6}\right)^2 \left(\frac{5}{6}\right)^4 
	= 0.2009$.
\it $\prob(X\geq 1) 
	= 1 - \prob(X=0) 
	= 1 - \left(\frac{5}{6}\right)^6 
	= 0.6651$.
\it $\prob(X>3)
	= \prob(X=4) + \prob(X=5) +\prob(X=6)
	= 30\left(\frac{1}{6}\right)^4 \left(\frac{5}{6}\right)^2
		+ 6\left(\frac{1}{6}\right)^5 \left(\frac{5}{6}\right) 
		+ \left(\frac{1}{6}\right)^6 
	= 0.0087$.
\een
\end{answer}

\end{questions}
\end{exercise}
%======================================================================
\endinput
%======================================================================
