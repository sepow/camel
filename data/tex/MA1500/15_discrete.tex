% !TEX root = main.tex
%======================================================================
\chapter{Discrete Probability}\label{chap:discrete}
%======================================================================

%----------------------------------------------------------------------
%\section{Finite probability spaces (recap)}
%----------------------------------------------------------------------

So far, we have considered random experiments that have only a finite number of outcomes.
\bit
\it Let $\Omega = \{\omega_1,\omega_2,\ldots,\omega_n\}$ be a finite sample space.
%\it The power set of $\Omega$, denoted by $\mathcal{P}(\Omega)$, contains all possible events, and is also a finite set.
\it The power set of $\Omega$, which contains all possible events, is also a finite set.
\eit
To define a probability measure on $\mathcal{P}(\Omega)$, we first defined a \emph{probability mass function} on $\Omega$:
\[
\begin{array}{rccl}
	p :	& \Omega		& \to		& [0,1] \\
		& \omega_k		& \mapsto	& p(\omega_k)
\end{array}
\]
%with the property that $\displaystyle\sum_{\omega\in\Omega} p(\omega) = 1$. 
with the property that $\displaystyle\sum_{k=1}^n p(\omega_k) = 1$. 
This was extended to a probability measure on $\mathcal{P}(\Omega)$ by defining the probability of an event to be the sum of the probabilities of the outcomes it contains:
\[
\begin{array}{rccl}
	\prob:	& \mathcal{P}(\Omega)	& \to		& [0,1] \\[1ex]
			& A						& \mapsto	& \displaystyle\sum_{\omega\in A} p(\omega)
\end{array}
\]

%The pair $(\Omega,\prob)$ was called a \emph{finite probability space}, with the implicit understanding that the underlying field of events is the power set of $\Omega$. 

A theory of probability on countably infinite sample spaces can be developed in much the same way, the main difference being that finite sums are replaced by infinite sums.


%----------------------------------------------------------------------
\section{Cardinality}
%----------------------------------------------------------------------
\begin{definition}
Let $A$ be a set, let $\N$ denote the set of natural numbers, and let $\R$ denote the set of real numbers.
\ben
\it If there exists a bijection $\phi:A\to\N$, we say that $A$ is \emph{countably infinite}.
\it If there exists a bijection $\phi:A\to\R$, we say that $A$ is \emph{uncountable}.
\een
A set is called \emph{countable} if it is either finite or countably infinite.
\end{definition}

\begin{example}
Consider a random experiment in which a coin is tossed repeatedly until the first head occurs. There is a countably infinite set of possible sequences:
\[
\Omega = \{H,TH,TTH,TTTH,TTTTH,\ldots\}
\]
%Let $\omega_k$ be the outcome consisting of $k-1$ tails followed by a head (so $\omega_1 \equiv H$, $\omega_2 \equiv HT$, $\omega_3 \equiv HTT$, and so on).

\bit
\it Let $\omega_k$ be the outcome consisting of $k-1$ tails followed by a head
\it Let $0<t<1$ be the probability that the coin shows heads. 
\eit

If we assume that the trials are independent, the probability mass function for this experiment is
\[
p(\omega_k) = (1-t)^{k-1}t\text{\qquad for\quad} k\in\{1,2,3,\ldots\}.
\]
The sum of these probabilities is equal to one:
\[
\sum_{k=1}^{\infty} p(\omega_k) = t\sum_{k=1}^{\infty} (1-t)^{k-1}  = t\sum_{k=0}^{\infty} (1-t)^k  = \frac{t}{1-(1-t)} = 1.
\]
Here, we have used the formula for the sum of a geometric progression with $a = t$ and $r = (1-t)$:
\[
a + ar + ar^2 + \ldots = \frac{a}{1-r} \text{\qquad provided $|r|<1$},
\]
%with $a = p$ and $r = (1-p)$.

Let $A$ be the event that the coin is tossed an even number of times: $A = \{\omega_2,\omega_4,\omega_6,\ldots \}$.
%\[
%A = \{\omega_2,\omega_4,\omega_6,\ldots \}.
%\]
%This event corresponds to the sequences $TH, TTTH, TTTTTH,\ldots$.
%The probability that the coin is tossed an even number of times is 

The probability of this event is:
\[
\prob(A) 
	= \sum_{\omega\in A} p(\omega) 
	= \sum_{j=1}^{\infty} p(\omega_{2j}) 
	= \sum_{j=0}^{\infty} t(1-t)^{2j+1}
	= t(1-t)\sum_{j=0}^{\infty} (1-t)^{2j}
	= \frac{t(1-t)}{1-(1-t)^2}
	= \frac{1-t}{2-t}.
\]
where we have again used the formula for the sum of a geometric progression, this time with $a = t(1-t)$ and $r = (1-t)^2$.
%\bit
%\it Note that if $p=1/2$ then $\prob(A) = 1/3$.
%\eit
\end{example}


%----------------------------------------------------------------------
\section{Discrete probability spaces}
%----------------------------------------------------------------------

% definition: discrete probability spaces 
\begin{definition}
Let $\Omega$ be a countable sample space, and let $\mathcal{P}(\Omega)$ denote its power set.
\ben
\it % 1
A \emph{probability mass function} on $\Omega$ is a function
\[
\begin{array}{ccccc}
p 	& :	& \Omega		& \to 		& [0,1] \\
	&	& \omega		& \mapsto	& p(\omega),
\end{array}
\]
with the property that $\displaystyle\sum_{\omega\in\Omega} p(\omega) = 1$.
\it % 2
A \emph{probability measure} on $\Omega$ is a function 
\[
\begin{array}{ccccl}
\prob 	& :	& \mathcal{P}(\Omega)	& \to 		& [0,1] \\[1ex]
		&	& A						& \mapsto	& \displaystyle\sum_{\omega\in A} p(\omega)
\end{array}
\]
where $p(\omega)$ is a probability mass function on $\Omega$.
\it % 3
The pair $(\Omega,\prob)$ is called a \emph{discrete probability space} on $\Omega$.
\een
\end{definition}

\begin{remark}
Finite sets are countable, so finite probability spaces are also discrete probabilty spaces.
\end{remark}

%----------------------------------------------------------------------
\section{Discrete random variables}
%----------------------------------------------------------------------

% defin: simple & discrete random variables
\begin{definition}
Let $(\Omega,\mathcal{F},\prob)$ be an arbitrary probability space. A random variable $X:\Omega\to\R$ is called
\ben
\it \emph{simple} if it takes only a finite number of distinct values, and 
\it \emph{discrete} if it takes at most a countably infinite number of distinct values.
\een
\end{definition}

% remark
\begin{remark}
\bit
\it Finite sets are countable, so simple random variables are also discrete random variables.
\it Any random variable on a discrete probability space must be a discrete random variable.
\eit
\end{remark}

% defn: PMF
\begin{definition}
%Let $(\Omega,\prob)$ be a discrete probability space. 
The \emph{probability mass function} (PMF) of a discrete random variable $X$ is the function
\[
\begin{array}{cccl}
f:	& \mathbb{R}	& \longrightarrow	& [0,1] \\
	& x 			& \mapsto			& \prob(X = x).
\end{array}
\]
\end{definition}

\begin{remark}
\ben
\it $f(x)=0$ for all but a countable number of values $x\in\R$.
\it If $\{x_1,x_2,x_3,\ldots\}$ is the range of $X$, then $\sum_{k=1}^{\infty} f(x_k) = 1$. Can you prove this?
\een
\end{remark}


%----------------------------------------------------------------------
\section{Convergent Series}
%----------------------------------------------------------------------
Let $a_1,a_1,a_2,\ldots$ be an infinite sequence of real numbers. The associated \emph{series} is the (ordered) sum of its terms:
\[
\sum_{k=0}^{\infty} a_k = a_0 + a_1 + a_2 + \ldots
\]

\begin{definition}
A series $\displaystyle\sum_{k=0}^{\infty} a_k$ is said to \emph{converge} if $\displaystyle\sum_{k=0}^{\infty} a_k < \infty$, otherwise it is said to \emph{diverge}.
\end{definition}

%% defn: convergent series
%\begin{definition}
%\bit
%\it A series $\sum_{k=0}^{\infty} a_k$ is said to \emph{converge} if the sequence of partial sums $\sum_{k=0}^{n}a_k$ converges to a finite limit as $n\to\infty$. 
%\it Otherwise the series is said to \emph{diverge}.
%\eit
%\end{definition}
%
%\begin{remark}
%Loosely speaking, 
%\bit
%\it a series converges if $\sum_{k=0}^{\infty} a_k < \infty$, and %diverges  if $\sum_{k=0}^{\infty} a_k = \infty$.
%\it a series diverges  if $\sum_{k=0}^{\infty} a_k = \infty$.
%\eit
%\end{remark}

\begin{example}[Convergent series]\label{ex:convergent_series}
\ben
\it 
$\displaystyle\sum_{k=0}^{\infty} ar^k 					\ =\ a + ar + ar^2 + ar^3 + \ldots = \frac{a}{1-r} \text{\quad provided $|r|<1$}.$
\it
$\displaystyle\sum_{k=1}^{\infty} \frac{1}{k^2}			\ =\ 1 + \frac{1}{2} + \frac{1}{3} + \frac{1}{4} + \ldots = \frac{\pi^2}{6}.$
\it
$\displaystyle\sum_{k=1}^{\infty} \frac{1}{k^4}			\ =\ 1 + \frac{1}{2^4} + \frac{1}{3^4} + \frac{1}{4^4} + \ldots = \frac{\pi^4}{90}.$
\it
$\displaystyle\sum_{k=0}^{\infty} \frac{1}{2^k}			\ =\ 1 + \frac{1}{2} + \frac{1}{4} + \frac{1}{8} + \ldots = 2.$
\it
$\displaystyle\sum_{k=0}^{\infty} \frac{1}{k!}				\ =\ 1 + 1 + \frac{1}{2} + \frac{1}{6} + \frac{1}{24} + \ldots = e.$
\it
$\displaystyle\sum_{k=0}^{\infty} \frac{\lambda^k}{k!} 	\ =\ 1 + \lambda + \frac{\lambda^2}{2} + \frac{\lambda^3}{6} + \frac{\lambda^4}{24} + \ldots = e^{\lambda} \text{\quad for all $\lambda\in\R$}.$
\een
%\normalsize
\end{example}

Not all series are convergent.
\begin{example}[Divergent series]
\mbox{}\vspace*{-4ex}
\ben
\it $\displaystyle\sum_{k=1}^{\infty} k 					\ =\ 1 + 2 + 3 + 4 + \ldots$
\it $\displaystyle\sum_{k=1}^{\infty} \frac{1}{k} 		\ =\ 1 + \frac{1}{2} + \frac{1}{3} + \frac{1}{4} + \ldots$ \qquad (This is called the \emph{harmonic series}.)
%\it $\displaystyle\sum_{p\text{ prime}} \frac{1}{p} 	\ =\ \frac{1}{2} + \frac{1}{3} + \frac{1}{5} + \frac{1}{7} +  \ldots$
\een 

%\begin{align*}
%\sum_{k=1}^{\infty} k		 		& = 1 + 2 + 3 + 4 + \ldots \\
%\sum_{k=1}^{\infty} \frac{1}{k} 		& = 1 + \frac{1}{2} + \frac{1}{3} + \frac{1}{4} + \ldots (\text{This is called the \emph{harmonic series}}.) \\
%\sum_{p\text{ prime}} \frac{1}{p} 	& = \frac{1}{2} + \frac{1}{3} + \frac{1}{5} + \frac{1}{7} +  \ldots 
%\end{align*}
\end{example}

Let $a_0,a_1,a_2,\ldots$ be an infinite sequence of non-negative real numbers. If the associated series $\sum_{k=0}^{\infty} a_k$ converges, the sequence can be used to define a PMF on the the non-negative integers $\{0,1,2,\ldots\}$ as follows:
\[
f(k) = \frac{a_k}{\sum_{k=0}^{\infty} a_k} \qquad\text{for $k=0,1,2,\ldots$.}
\]
Note that $\sum_{k=0}^{\infty} f(k) = 1$, so $f$ is indeed a PMF. For the series listed in Example~\ref{ex:convergent_series},
\bit
\it the first series yields the \emph{geometric distribution} (next lecture)
\it the last series yields the \emph{Poisson distribution} (next lecture).
\eit
%\end{remark}
%

%----------------------------------------------------------------------
\section{Expectation}
%----------------------------------------------------------------------
The expectation of a discrete random variable is computed in much the same way that the expectation of a simple random variable is computed, the only difference again being that finite sums are replaced by infinite sums.

\begin{definition}
Let $X$ be a random variable on a discrete probability space $(\Omega,\prob)$. 
The \emph{expectation} of $X$ is defined to be
\[
\expe(X) = \sum_{\omega\in\Omega} X(\omega)p(\omega).
\]
\end{definition}

We also have the following analogues of Theorem~\ref{thm:expe_pmf} and Theorem~\ref{thm:lus}, which were proved for simple random variables. The proofs for the discrete case are similar (see exercises).
% theorem
\begin{theorem}
Let $X$ be a discrete random variable, let $\{x_1,x_2,\ldots\}$ be the range of $X$, and let $f(x)$ be its PMF. Then
\[
\expe(X) = \sum_{i=1}^{\infty} x_i\,f(x_i),
\]
and for any function $g:\R\to\R$,
\[
\expe\big[g(X)\big] = \sum_{i=1}^{\infty} g(x_i)f(x_i).
\]
\end{theorem}


% proof
%\begin{proof}
%See Exercises.
%To find $\prob\big(g(X)=y\big)$, we sum the probabilities $\prob(X=x)$ over all $x$ for which $g(x)=y$.
%\begin{align*}
%\expe\left(g(X)\right)
%	& = \sum_y y\,\prob\left(g(X)=y\right) \\
%	& = \sum_y y\sum_{\{x:g(x)=y\}}\prob(X=x) \\
%	& = \sum_y \sum_{\{x:g(x)=y\}}y\,\prob(X=x) \\
%	& = \sum_x g(x)\prob(X=x)
%\end{align*}
%\end{proof}

%%----------------------------------------------------------------------
%\section{Examples}
%%----------------------------------------------------------------------
%
\begin{example}
Let $X$ be a discrete random variable, taking values in the set $\{1,2,3,\ldots\}$ with probabilities
\[
\prob(X=k) = \frac{1}{2^k}
\]
Show that $\expe(X)=2$.
\end{example}

\begin{solution}
\begin{align*}
\expe(X) = \sum_{k=1}^{\infty} k \prob(X=k)
	& = \sum_{k=1}^\infty k \frac{1}{2^k} \\
	& = \sum_{j=0}^\infty (j+1)\frac{1}{2^{j+1}} 		\qquad \qquad\text{(where we have set $j=k-1$)} \\
	\intertext{\newpage}
	& = \frac{1}{2}\left(\sum_{j=0}^\infty j \frac{1}{2^j} + \sum_{j=0}^\infty \frac{1}{2^j}\right) \\
	& = \frac{1}{2}\left(\sum_{j=1}^\infty j \frac{1}{2^j} + 2\right) \\
	& = \frac{1}{2}\expe(X) + 1.
\end{align*}
so $\expe(X)=2$, as required.
\end{solution}

%% example
%\begin{example}
%A coin has probability $p > 0$ of showing heads. The coin is tossed repeatedly until the first head occurs. Let $X$ be the number of times that the coin is tossed.
%\bit
%\it $X$ takes values in the set $\{1,2,3,4,5,\ldots\}$.
%\it If we assume that the trials are independent, the PMF of $X$ is
%\[
%\prob(X=k) = (1-p)^{k-1}p\text{\qquad for\quad} k\in\{1,2,3,\ldots\}.
%\]
%\it The sum of these probabilities is equal to one:
%\[
%\sum_{k=1}^{\infty} \prob(X=k) = p\sum_{k=1}^{\infty} (1-p)^{k-1}  = \frac{p}{1-(1-p)} = 1.
%\]
%where we have used the formula for the sum of a geometric progression:% with $a = p$ and $r = (1-p)$:
%\[
%a + ar + ar^2 + \ldots = \frac{a}{1-r} \text{\qquad provided $|r|<1$},
%\]
%with $a = p$ and $r = (1-p)$.
%%\it
%%$X$ is said to have \emph{geometric distribution} with parameter $p$.
%
%\eit
%
%\end{example}

%% example
%\begin{example}
%Let $X$ be a discrete random variable, taking values in the set $\{1,2,\ldots\}$ with probabilities
%\[
%\prob(X=k) = \frac{90}{\pi^4 k^4}
%\]
%
%This is indeed a PDF because 
%%$\displaystyle\sum_{k=0}^{\infty} \prob(X=k) = \frac{90}{\pi^4}\sum_{k=1}^{\infty}\frac{1}{k^4}$
%%and
%$\displaystyle\sum_{k=1}^{\infty} \frac{1}{k^4} = \frac{\pi^4}{90}$.
%
%
%The expected value of $X$ is
%\begin{align*}
%\expe(X) 
%	= \sum_{k=1}^{\infty} k \prob(X=k)
%	= \frac{90}{\pi^4}\sum_{k=0}^{\infty}\frac{1}{k^3} 
%%	= \zeta(3)/\zeta(4)
%	\approx 1.1106.
%\end{align*}
%
%\bit
%\it This is indeed a PDF on the set $\{0,1,2,\ldots\}$ because
%\[
%\sum_{k=0}^{\infty} \prob(X=k) = \frac{90}{\pi^4}\sum_{k=1}^{\infty}\frac{1}{k^4}
%\text{\qquad and\qquad}
%\sum_{k=0}^{\infty} \frac{1}{k^4} = \frac{\pi^4}{90}.
%\]
%\it The expected value of $X$ is
%\begin{align*}
%\expe(X) 
%	= \sum_{k=0}^{\infty} k \prob(X=k)
%	= \frac{90}{\pi^4}\sum_{k=0}^{\infty}\frac{1}{k^3} 
%%	= \zeta(3)/\zeta(4)
%	\approx 1.1106.
%\end{align*}
%\eit
%\end{example}
%\begin{remark}
%The \emph{Riemann zeta function} is the function
%$
%\zeta(s) = \displaystyle\sum_{k=1}^{\infty}\frac{1}{k^s}
%$
%where $s\in\mathbb{R}$.
%\bit
%\it $\zeta(s)$ diverges for all $s\leq 1$
%\it $\zeta(s)$ converges for all $s > 1$. 
%\it In particular, $\zeta(2)=\pi^2/6$, $\zeta(3)\approx 1.2021$ and $\zeta(4)=\pi^4/90$.
%\eit
%\end{remark}
%
% example
\begin{example}
Let $X$ be a discrete random variable, taking values in the range $\{1,2,\ldots\}$ with PMF given by
\[
f(k) = \frac{6}{\pi^2 k^2}.
\]
Show that this is indeed a PMF, and compute the expected value of $X$.
\end{example}

\begin{solution}
This is a PDF because $\displaystyle\sum_{k=1}^{\infty} \frac{1}{k^2} = \frac{\pi^2}{6}$.
The expected value of $X$ is
\begin{align*}
\expe(X) 
	= \sum_{k=1}^{\infty} k f(k)
	= \frac{6}{\pi^2}\sum_{k=0}^{\infty}\frac{1}{k}.
\end{align*}
The series $\displaystyle \sum_{k=0}^{\infty}\frac{1}{k}$ diverges, so $X$ has infinite expectation!
\end{solution}


%----------------------------------------------------------------------
\section{Exercises}
% !TEX root = main.tex
% ex12_discrete.tex
\begin{exercise}
\begin{questions}
%----------------------------------------
\question
Consider the function
\[
p(k) = \left\{\begin{array}{ll}
	\displaystyle\frac{c}{k(k+1)} & \text{for } k=1,2,3,\ldots, \\[2ex]
	0				& \text{otherwise}.
\end{array}\right.	
\]
Find the value of $c$ for which this function is a probability mass function on the positive integers.% $k\in\{1,2,3,\ldots\}$.
\par
[\slshape Hint. Use partial fractions and the so-called ``method of differences".\normalfont]
\begin{answer}
For $p$ to be a PMF, we need that $\displaystyle\sum_{k=1}^{\infty} p(k) = 1$:
\begin{align*}
\sum_{k=1}^{\infty} p(k) 
	& = c\sum_{k=1}^{\infty}\frac{1}{k(k+1)} \\
	& = c\sum_{k=1}^{\infty}\left(\frac{1}{k}-\frac{1}{k+1}\right) \\
	& = c\left[\left(1-\frac{1}{2}\right) + \left(\frac{1}{2}-\frac{1}{3}\right) + \left(\frac{1}{3}-\frac{1}{4}\right) + \ldots\right] \\
	& = c.
\end{align*}
Thus $c=1$.
\end{answer}

%----------------------------------------
\question
Let $X:\Omega\to\R$ be a discrete random variable, let $\{x_1,x_2,\ldots\}$ be the range of $X$, let $f(x)$ be the PMF of $X$, and let $g:\R\to\R$ be any function.
\begin{parts}
\part Show that $\displaystyle \expe(X) = \sum_{i=1}^{\infty} x_i\,f(x_i)$.
\begin{answer}
We prove the result only for the case where $\Omega$ is a countable sample space (the uncountable case involves hard integration).
To this end, let $p$ denote the associated probability mass function on $\Omega$, and consider the sets $A_i=\{\omega: X(\omega) = x_i\}$ for $i=1,2,\ldots$. By construction, the sets $A_1,A_2,\ldots$ form a partition of $\Omega$, and $f(x_i) = \prob(A_i)$, so
\begin{align*}
\expe(X)
	= \sum_{\omega\in\Omega} X(\omega) p(\omega) 
	& = \sum_{i=1}^{\infty}\sum_{\omega\in A_i} X(\omega) p(\omega) \\
	& = \sum_{i=1}^{\infty}x_i \sum_{\omega\in A_i} p(\omega) \qquad\text{because $X(\omega)=x_i$ for all $\omega\in A_i$,} \\
	& = \sum_{i=1}^{\infty}x_i \prob(A_i) 
	= \sum_{i=1}^{\infty}x_i f(x_i).
\end{align*}
\end{answer}
\part Show that $\displaystyle \expe\big[g(X)\big] = \sum_{i=1}^{\infty} g(x_i)f(x_i)$.
\begin{answer}
Again, we prove the result only for when $\Omega$ is countable. As in part (a),
\begin{align*}
\expe\big[g(X)\big]
	= \sum_{\omega\in\Omega} g\big[X(\omega)\big] p(\omega) 
	& = \sum_{i=1}^{\infty}\sum_{\omega\in A_i} g\big[X(\omega)\big] p(\omega) \\
	& = \sum_{i=1}^{\infty}g(x_i) \sum_{\omega\in A_i} p(\omega) \qquad\text{because $X(\omega)=x_i$ for all $\omega\in A_i$,} \\
	& = \sum_{i=1}^{\infty}x_i \prob(A_i)
	= \sum_{i=1}^{\infty}x_i f(x_i). 
\end{align*}
\end{answer}
\end{parts}
%%----------------------------------------
%\question
%The \emph{zeta distribution} with (real-valued) parameter $s>1$ has probability mass function
%\[
%p(k) = \frac{1}{k^s\zeta(s)} \quad\text{for}\quad k=1,2,3,\ldots
%\]
%where $\zeta(\cdot)$ is the \emph{zeta function}, defined by
%\[
%\zeta(s) = \sum_{k=1}^{\infty} \frac{1}{k^s} = \frac{1}{1^s} + \frac{1}{2^s} + \frac{1}{3^s} + \ldots \qquad(s\in\R),
%\]
%which has the property that $\zeta(s)<\infty$ if and only if $s>1$. Let $X$ have zeta distribution with parameter $s>1$. 
%\ben
%\it Show that $\expe(X)$ is infinite whenever $s\leq 2$.
%\it Show that $\var(X)$ is infinite whenever $s\leq 3$.
%\een
%\begin{answer}
%\bit
%\it The expected value of $X$ is
%\[
%\expe(X) = \sum_{k=0}^{\infty} k p(k) = \frac{1}{\zeta(s)}\sum_k \frac{1}{k^{s-1}} = \frac{\zeta(s-1)}{\zeta(s)}
%\]
%which can only be finite for $s>2$. Thus $\expe(X)$ is infinite for all $s\leq 2$.
%\it The variance of $X$ can be written as $\var(X)=\expe(X^2)-\expe(X)^2$. The expected value $\expe(X^2)$ is given by
%\[
%\expe(X^2) = \sum_{k=0}^{\infty} k^2 p(k) = \frac{1}{\zeta(s)}\sum_k \frac{1}{k^{s-2}} = \frac{\zeta(s-2)}{\zeta(s)}
%\]
%which can only be finite for $s>3$. Thus $\var(X)$ is infinite for all $s\leq 3$.
%\eit
%\end{answer}
\end{questions}
\end{exercise}

%======================================================================
\endinput
%======================================================================

%----------------------------------------------------------------------

%======================================================================
\endinput
%======================================================================
