% !TEX root = main.tex
%======================================================================
\chapter{Conditional Distributions}\label{chap:cond-dist}
%======================================================================

Let $X$ and $Y$ be simple random variables, let $\{x_1,x_2,\ldots,x_m\}$ be the range of $X$, and let $\{y_1,y_2,\ldots,y_n\}$ be the range of $Y$.

%----------------------------------------------------------------------
\section{Conditional distributions}
%----------------------------------------------------------------------
Let $A$ and $B$ be two events. Recall that the conditional probability of $A$ given $B$ is
\[
\prob(A|B) = \frac{\prob(A\cap B)}{\prob(B)}
\]
This idea extends to random variables. 


% defn: conditional cdf/pmf
\begin{definition}
Let $x\in\R$ be a fixed number, and suppose that $\prob(X=x)>0$.
\ben
\it The \emph{conditional CDF} of $Y$ given $X=x$ is the function
\[
\begin{array}{cccc}
F_{Y|X}	:	& \R	& \to		& [0,1] \\[1ex]
			& y & \mapsto	& \prob(Y\leq y \,|\, X=x).
\end{array}
\]
\it The \emph{conditional PMF} of $Y$ given $X=x$ is the function
\[
\begin{array}{cccc}
f_{Y|X}	:	& \R		& \to		& [0,1] \\[1ex]
			& y 		& \mapsto	& \prob(Y=y \,|\, X=x).
\end{array}
\]
\een
\end{definition}

%The conditional probability that event $\{Y=y\}$ occurs given that event $\{X=x\}$ occurs is:
%\[
%\prob(Y=y\,|\,X=x) = \displaystyle\frac{\prob(X=x,Y=y)}{\prob(X=x)}.
%\]

% lemma 
\begin{lemma}\label{lem:cond-pmf}
The conditional PMF of $Y$ given $X=x$ can be written as 
\[
f_{Y|X}(y|x) = \displaystyle \frac{f_{X,Y}(x,y)}{f_X(x)}.
\]
where $f_{X,Y}$ is the joint PMF of $X$ and $Y$, and $f_X$ is the marginal PMF of $X$.
\end{lemma}

\begin{proof}
\[
f_{Y|X}(y|x) = \prob(Y=y\,|\,X=x) = \frac{\prob(X=x,Y=y)}{\prob(X=x)} = \frac{f_{X,Y}(x,y)}{f_X(x)}.
\]
\end{proof}

\begin{remark}[Independence]
Recall that $X$ and $Y$ are independent if and only if 
\[
f_{X,Y}(x,y)=f_X(x)f_Y(y) \quad\text{for all}\quad x,y\in\R.
\]
Hence by Lemma~\ref{lem:cond-pmf}, $X$ and $Y$ are independent if and only if 
\[
f_{Y|X}(y|x) = f_Y(y) \quad\text{for all}\quad x,y\in\R.
\]
where $f_Y$ is the marginal PMF of $Y$. Thus if $X$ and $Y$ are independent, the value taken by $X$ does not affect the distribution of $Y$.
\end{remark}

%The conditional probability that event $\{Y=y\}$ occurs given that event $\{X=x\}$ occurs is:
%\[
%\prob(Y=y\,|\,X=x) = \displaystyle\frac{\prob(X=x,Y=y)}{\prob(X=x)}.
%\]
%\bit
%\it Let $f_{X,Y}(x,y)$ be the joint PMF of $X$ and $Y$.
%\it Let $f_X(x)$ and $f_Y(y)$ be the marginal PMFs of $X$ and $Y$ respectively.
%\it The conditional PMF of $Y$ given $X=x$ can be written as 
%\[
%f_{Y|X}(y|x) = \displaystyle \frac{p(x,y)}{f_X(x)}.
%\]
%\it $X$ and $Y$ are independent if and only if $f_{X,Y}(x,y)=f_X(x)f_Y(y)$ for all $x,y\in\R$.
%\it Hence $X$ and $Y$ are independent if and only if $f_{Y|X}(y|x) = f_Y(y)$ for all $x,y\in\R$.
%\eit

% tedious example
\begin{example}\label{ex:cond-dist-tedious}
Let $X$ and $Y$ be random variables with joint PMF shown in the following table. 
\[
\begin{array}{|cc|ccc|}\hline
	&       &       & y     &   \\
    &       & 2     & 3     & 4 \\ \hline
	& 1		& 1/12	& 1/6	& 0 		\\ 
x	& 2		& 1/6	& 0		& 1/3 	\\ 
	& 3		& 1/12	& 1/6  	& 0		\\ \hline
\end{array}
\]
%\[\begin{array}{|c|c|c|c|}\hline
%	& Y=2	& Y=3	& Y=4	\\ \hline
%X=1	& 1/12	& 1/6	& 0 		\\ \hline
%X=2	& 1/6	& 0		& 1/3 	\\ \hline
%X=3	& 1/12	& 1/6  	& 0		\\ \hline
%\end{array}\]
Find the conditional distribution of $Y$ given that (i) $X=1$, (ii) $X=2$, (iii) $X=3$.
\end{example}

\begin{solution}
 The conditional distributions are obtained by re-scaling the rows of the table.
%\[\begin{array}{|cc|c|c|c|}\hline
%				&			& Y=2	& Y=3	& Y=4	\\ \hline 
%f_{Y|X}(y\,|\,0)	=& \prob(Y=y\,|\,X=0)	& 1/3	& 2/3	& 0		\\ \hline
%f_{Y|X}(y\,|\,1)	=& \prob(Y=y\,|\,X=1)	& 1/3	& 0		& 2/3	\\ \hline
%f_{Y|X}(y\,|\,2)	=& \prob(Y=y\,|\,X=2)	& 1/3	& 2/3	& 0		\\ \hline
%\end{array}\]
\[\begin{array}{|c|c|c|c|}\hline
					& Y=2	& Y=3	& Y=4	\\ \hline 
f_{Y|X}(y\,|\,0)	& 1/3	& 2/3	& 0		\\ \hline
f_{Y|X}(y\,|\,1)	& 1/3	& 0		& 2/3	\\ \hline
f_{Y|X}(y\,|\,2)	& 1/3	& 2/3	& 0		\\ \hline
\end{array}\]
\end{solution}

%----------------------------------------------------------------------
\section{Conditional expectation}
%----------------------------------------------------------------------
Let $x$ be a value such that $\prob(X=x)>0$. 
\begin{definition}
\ben
\it The \emph{conditional expectation of $Y$ given $X=x$} is a number,
\[
\expe(Y|X=x) 
	= \sum_{j=1}^n y_j\,f_{Y|X}(y_j|x).
%	= \sum_y y\,\prob(Y=y|X=x).
\]
\it The \emph{conditional expectation of $Y$ given $X$} is a random variable,
\[\begin{array}{llll}
\expe(Y|X):	& \Omega 	& \to 		& \R \\
			& \omega		& \mapsto 	& \expe\big(Y|X=X(\omega)\big).
\end{array}\]
\een
\end{definition}

% remark
\begin{remark}
Let $g(x) = \expe(Y|X=x)$. The distribution of the random variable $g(X)=\expe(Y|X)$ depends only on the distribution of $X$, and by Theorem~\ref{thm:lus}, 
%\par
%Then $g(X) = \expe(Y|X)$ is a transformation of $X$, and by Theorem~\ref{thm:lus}, 
%Let $\psi(x) = \expe(Y|X=x)$. Then $\expe(Y|X)=\psi(X)$ is a random variable, whose distribution depends only on the distribution of $X$, and whose expected value is
%The distribution of $\expe(Y|X)$ depends only on the distribution of $X$, and its expected value is given by
%\it By Theorem~\ref{thm:lus}, 
\[
\expe\big[\expe(Y|X)\big] \equiv \expe\big[g(X)\big] = \sum_{i=1}^m g(x_i) f_X(x_i) \equiv \sum_{i=1}^m \expe(Y|X=x_i) f_X(x_i)
\]
where $f_X$ is the marginal distribution of $X$.
%\eit
\end{remark}

%----------------------------------------------------------------------
\section{Law of total expectation}
%----------------------------------------------------------------------

% theorem
\begin{theorem}[The law of total expectation]\label{thm:law_of_total_expectation}
For any two simple random variables $X$ and $Y$,
\[
\expe\big[\expe(Y|X)\big] = \expe(Y).
\]
\end{theorem}
\begin{proof}
%Let $\psi(x) = \expe(Y|X=x)$.
\begin{align*}
\expe\big[\expe(Y|X)\big] 
	= \sum_{i=1}^m \expe(Y|X=x_i) f_X(x_i)
	& = \sum_{i=1}^m \left(\sum_{j=1}^n y_j f_{Y|X}(y_j|x_i)\right) f_X(x_i) \\
	& = \sum_{j=1}^n y_j \left(\sum_{i=1}^m f_{Y|X}(y_j|x_i)f_X(x_i)\right) \\
	& = \sum_{j=1}^n y_j \left(\sum_{i=1}^m f_{X,Y}(x_i,y_j)\right)
	= \sum_{j=1}^n y_j f_Y(y_j)
	= \expe(Y).
\end{align*}
\qed
\end{proof}

\begin{remark}
The law of total expectation provides a useful way of computing $\expe(Y)$:
\[
\expe(Y) = \sum_{i=1}^m \expe(Y|X=x_i)\prob(X=x_i).
\]
This is analogous to the \emph{partition theorem}.
\end{remark}

% tedious example (continued)
\begin{example}
Consider the random variables $X$ and $Y$ of Example~\ref{ex:cond-dist-tedious}.
\ben
\it Find the conditional expectation $Y$ given that (i) $X=1$, (ii) $X=2$ and (iii) $X=3$.
\it Find the distribution of $\expe(Y|X)$, and verify that the law of total expectation holds.
\een
\end{example}

\begin{solution}
The conditional expectation of $Y$ given that $X=1$, $X=2$ and $X=3$ are, respectively,
\bit
\it $\expe(Y|X=1) = \left(2\times\frac{1}{3}\right) + \left(3\times\frac{2}{3}\right) + \left(4\times 0\right) = \frac{8}{3}$,
\it $\expe(Y|X=2) = \left(2\times\frac{1}{3}\right) + \left(3\times 0\right) + \left(4\times\frac{2}{3}\right) = \frac{10}{3}$,
\it $\expe(Y|X=3) = \left(2\times\frac{1}{3}\right) + \left(3\times\frac{2}{3}\right) + \left(4\times 0\right) = \frac{8}{3}$.
\eit

The marginal distribution of $X$, along with the associated values of $\expe(Y|X=x)$, are shown in the following table.
\[\begin{array}{|c|ccc|}\hline
x				& 1		& 2		& 3		\\ \hline
\prob(X=x)		& 1/4	& 1/2	& 1/4	\\ \hline \hline
\expe(Y|X=x)		& 8/3	& 10/3	& 8/3	\\ \hline
\end{array}\]
Hence the distribution of $\expe(Y|X)$ is
\[\begin{array}{|c|cc|}\hline
z							& 8/3   & 10/3 \\ \hline
\prob\big(\expe(Y|X)=z\big)	& 1/2   & 1/2  \\ \hline
\end{array}\]
and the expected value of $\expe(Y|X)$ is therefore 
\[
\expe\big[\expe(Y\,|\,X)\big] = \left(\frac{8}{3}\times\frac{1}{2}\right) + \left(\frac{10}{3}\times\frac{1}{2}\right) = 3.
\]
This agrees with the value of $\expe(Y)$ computed from the marginal distribution of $Y$, which verifies that the law of total expectation holds in this case. 
\end{solution}


%----------------------------------------------------------------------
\section{Exercises}
% !TEX root = main.tex
% ex15_conditional_distributions.tex
\begin{exercise}
\begin{questions}
%----------------------------------------
\question
Two random variables $X$ and $Y$ have joint PMF shown in the following table:
\[
\begin{array}{|cc|ccc|}\hline
	&       &       & y     &   \\
    &       & 0     & 1     & 2 \\ \hline
	& 0     & 1/24  & 3/24  & 2/24 \\
x   & 1     & 2/24  & 4/24  & 6/24 \\
	& 2     & 1/24  & 1/24  & 4/24 \\ \hline
\end{array}
\]
\begin{parts}
\part
Find the covariance and correlation coefficient of $X$ and $Y$
\begin{answer}
To compute $\cov(X,Y)$, we need $\expe(XY)$, $\expe(X)$ and $\expe(Y)$. The marginal PMFs of $X$ and $Y$ are given by
\[
\begin{array}{c|cccc}
x       & 0     & 1      & 2 \\ \hline
f_X(x)  & 6/24  & 12/24  & 6/24  
\end{array}
\]
and
\[
\begin{array}{c|cccc}
y       & 0     & 1     & 2 \\ \hline
f_Y(y)  & 4/24  & 8/24  & 12/24  
\end{array}
\]
so
\begin{align*}
\expe(X) & = (0\times 6/24) + (1\times 12/24) + (2\times  6/24) = 1 \\
\expe(Y) & = (0\times 4/24) + (1\times  8/24) + (2\times 12/24) = 32/24
\end{align*}

To compute $\expe(XY)$, we need the PMF of $Z=XY$:
\[
\begin{array}{c|cccc}
z           & 0     & 1     & 2    & 4 \\ \hline
f_{XY}(z)   & 9/24  & 4/24  & 7/24 & 4/24 \\
\end{array}
\]
From here,
\begin{align*}
\expe(XY) 	& = (0\times 9/24) + (1\times 4/24) + (2\times 7/24) + (4\times 4/24) = 34/24 \\
\intertext{and therefore}
\cov(X,Y) 	& = \expe(XY)-\expe(X)\expe(Y) = 34/24 - 32/24 = 1/12
\end{align*}

To compute $\rho(X,Y)$, we also need $\var(X)$ and $\var(Y)$:
\begin{align*}
\expe(X^2) & = (0\times 6/24) + (1\times 12/24) + (4\times  6/24) = 36/24 \\
\expe(Y^2) & = (0\times 4/24) + (1\times  8/24) + (4\times 12/24) = 56/24
\intertext{so}
\var(X)	& = \expe(X^2) - \expe(X)^2 = 36/24 - 1 = 1/2 \\
\var(Y) & = \expe(Y^2) - \expe(Y)^2 = 56/24 - (32/24)^2	= 5/9
\intertext{and therefore}
\rho(X,Y) & = \frac{\cov(X,Y)}{\sqrt{\var(X)}\sqrt{\var(Y)}} = 0.158
\end{align*}
\end{answer}

\part
Find conditional expectation of $X$ given that $Y=1$.
\begin{answer}
The conditional PMF of $X$ given $Y=1$ is
\[
\begin{array}{c|cccc}
x         	& 0    & 1    & 2 \\ \hline
P(X=x|Y=1)	& 3/8  & 4/8  & 1/8 \\
\end{array}
\]
The conditional expectation of $X$ given $Y=1$ is
\[
\expe(X|Y=1)
	= \sum_x x P(X=x|Y=1) 
	= \left(0\times\frac{3}{8}\right) + \left(1\times\frac{4}{8}\right) + \left(2\times\frac{1}{8}\right) 
	= \frac{3}{4}
\]
Compare this to the unconditional expectation, $\expe(X)=1$
\end{answer}
\end{parts}

%--------------------

%---------------------------------------------
\question
Two random variables $X$ and $Y$ have joint PMF shown in the following table:
\[
\begin{array}{|cc|ccc|}\hline
	&       &       	& y     &   		\\
	&       & -1    	& 0     	& 1		\\ \hline
	& 0     & 2/28	& 2/28	& 3/28 	\\
x 	& 1     & 1/28	& 2/28  	& 4/28 	\\
	& 2     & 1/28	& 4/28  	& 9/28 	\\ \hline
\end{array}
\]

\begin{parts}
%--------------------
\part
Find the marginal distributions of $X$ and $Y$.
\begin{answer}
The marginal distributions are
\[
\begin{array}{|c|ccc|}\hline
x       		& 0     & 1     & 2 \\ \hline
f_X(x)  		& 7/28  & 7/28  & 14/28 \\ \hline
\end{array}
\]
and
\[
\begin{array}{|c|ccc|}\hline
y       		& 0     & 1     & 2 \\ \hline
f_Y(y)  		& 4/28  & 8/28  & 16/28  \\ \hline
\end{array} 
\]
\end{answer}
%--------------------
\part
Are $X$ and $Y$ are independent? Justify your answer.
\begin{answer}
$X$ and $Y$ are not independent, because (for example)
\[
\prob(X=0,Y=-1) = \frac{2}{28}\quad\text{and}\quad \prob(X=0)\prob(Y=-1) = \frac{7}{28}\times\frac{4}{28} = \frac{1}{28}
\]
so $\prob(X=0,Y=-1) \neq \prob(X=0)\prob(Y=-1)$.
\end{answer}
%--------------------
\part
Find the expected values of $X$ and $Y$.
\begin{answer}
The expected values are
\begin{align*}
\expe(X) 
	& = \left(0\times\frac{7}{28}\right) + \left(1\times\frac{7}{28}\right) + \left(2\times\frac{14}{28}\right) = \frac{35}{28} = \frac{5}{4}. \\
\expe(Y) 
	& = \left(-1\times\frac{4}{28}\right) + \left(0\times\frac{8}{28}\right) + \left(1\times\frac{16}{28}\right) = \frac{12}{28} = \frac{3}{7}. \\
\end{align*}
\end{answer}
%--------------------
\part
Find the covariance of $X$ and $Y$.
\begin{answer}
The distribution of $Z=XY$ is 
\[
\begin{array}{|c|ccccc|}\hline
z       		& -2     & -1   & 0		& 1		& 2 		\\ \hline
\prob(XY=z)	& 1/28  & 1/28  & 13/28 	& 4/28	& 9/28	\\ \hline
\end{array}
\]
The product moment $\expe(XY)$ is therefore
\small
\[
\expe(XY) = \left(-2\times\frac{1}{28}\right) + \left(-1\times\frac{1}{28}\right) + \left(0\times\frac{13}{28}\right) + \left(1\times\frac{4}{28}\right) + \left(2\times\frac{9}{28}\right) = \frac{19}{28}
\]
\normalsize
so 
\[
\cov(X,Y) = \expe(XY)-\expe(X)\expe(Y) = \frac{19}{28} - \left(\frac{5}{4}\times\frac{3}{7}\right) = \frac{4}{28} = \frac{1}{7}.
\]
\end{answer}
%--------------------
\part
Find the conditional expectation of $Y$ given that 
\begin{subparts}
\subpart $X=0$,
\subpart $X=1$,
\subpart $X=2$.
\end{subparts}
\begin{answer}
The conditional distributions are obtained by re-scaling the rows of the table:
\[
\begin{array}{|c|ccc|}\hline
y       					& -1		& 0		& 1 		\\ \hline
\prob(Y=y\,|\,X=0)		& 2/7	& 2/7	& 3/7  	\\ \hline
\prob(Y=y\,|\,X=1)		& 1/7	& 2/7	& 4/7  	\\ \hline
\prob(Y=y\,|\,X=2)		& 1/14	& 4/14	& 9/14 	\\ \hline
\end{array}
\]
The conditional expectations are therefore
\begin{align*}
\expe(Y\,|\,X=0) 
	& = \left(-1\times \frac{2}{7}\right) + \left(0\times\frac{2}{7}\right) + \left(1\times\frac{3}{7}\right) = \frac{1}{7} \\
\expe(Y\,|\,X=1) 
	& = \left(-1\times \frac{1}{7}\right) + \left(0\times\frac{2}{7}\right) + \left(1\times\frac{4}{7}\right) = \frac{3}{7} \\
\expe(Y\,|\,X=2) 
	& = \left(-1\times \frac{1}{14}\right)+ \left(0\times\frac{4}{14}\right)+ \left(1\times\frac{9}{14}\right) = \frac{4}{7}
\end{align*}	
\end{answer}
%--------------------
\part
Find the distribution of the conditional expectation $\expe(Y\,|\,X)$.
\begin{answer}
The distribution of the conditional expectation $g(X)=\expe(Y\,|\,X)$ is determined by the marginal distribution of $X$:
\[
\begin{array}{|c|ccc|}\hline
x       				& 0     & 1     & 2 \\ \hline
z=g(x) 			& 1/7   & 3/7   & 4/7 \\ \hline
\prob(g(X)=z)		& 1/4   & 1/4   & 1/2  \\ \hline
\end{array}
\]
\end{answer}
%--------------------
\part
Check that the law of total expectation holds in this case.
\begin{answer}
The expected value of $g(X)$ is 
\[
\expe(\big(\expe(Y\,|\,X)\big) 
	= \left(\frac{1}{7}\times \frac{1}{4}\right) + \left(\frac{3}{7}\times\frac{1}{4}\right) + \left(\frac{4}{7}\times\frac{1}{2}\right) 
	= \frac{3}{7}
	= \expe(Y)
\]
as required.
\end{answer}
%--------------------
\end{parts}

%---------------------------------------------
\question
A fair six-sided die is thrown once. Let $X$ be the score shown on the die, and let $A$ be the event that $X$ is an even number. Find the conditional PMF of $X$ given that $A$ occurs, and use this to find the conditional expectation of $X$ given that $A$ occurs.
\begin{answer}
The conditional probability that $X=x$ given that event $A$ occurs is
\[
f_{X|I_A}(x|1) = \frac{\prob(X=x\text{ and $X$ is even})}{\prob(X\text{ is even})}.
\]
Hence the conditional PMF of $X$ given that $A$ occurs is 
\[
\begin{array}{|r|cccccc|}\hline
x       			& 1     & 2     & 3     & 4     & 5     & 6    \\ \hline
f_{X|I_A}(x|1)		& 0     & 1/3   & 0     & 1/3   & 0     & 1/3  \\ \hline
\end{array}
\]
The conditional expectation of $X$ given $I_A=1$ is therefore
\[
\expe(X|I_A=1) = \sum_{x=1}^6 x \, f_{X|I_A}(x|I_A=1) = \left(2\times\frac{1}{3}\right) + \left(4\times\frac{1}{3}\right) + \left(6\times\frac{1}{3}\right) = 4.
\]
\end{answer}


%---------------------------------------------
\question
Two fair coins are tossed. Let $X_1$ and $X_2$ be random variables, with $X_1=1$ if the first coin lands on heads and $X_1=-1$ if it lands on tails, and $X_2=1$ if the second coin lands on heads and $X_2=-1$ if it lands on tails. Now consider the random variables
\[
X = \frac{X_1+X_2}{2}\qquad\text{and}\qquad Y = \frac{X_1-X_2}{2}.
\]
\begin{parts}
%--------------------
\part Compute the correlation coefficient of $X$ and $Y$.
\begin{answer}
The joint PMF of $X$ and $Y$ is 
\[
\begin{array}{|cc|ccc|}\hline
	&		&		& y		&		\\
	&		& -1		& 0		& 1		\\	\hline
	& -1		& 0		& 1/4	& 0		\\	
x	& 0		& 1/4	& 0		& 1/4	\\	
	& 1		& 0		& 1/4	& 0		\\ \hline
\end{array}
\]
The covariance of $X$ and $Y$ is
\begin{align*}
\cov(X,Y)
	& = \expe(XY)-\expe(X)\expe(Y) \\
	& = \expe\left(\frac{(X_1+X_2)(X_1-X_2)}{4}\right) - \expe\left(\frac{X_1+X_2}{2}\right)\expe\left(\frac{X_1-X_2}{2}\right) \\
	& = \frac{1}{4}\expe(X_1^2 - X_2^2) - \frac{1}{4}\expe(X_1+X_2)\expe(X_1-X_2) \\
	& = \frac{1}{4}\big(\expe(X_1^2) - \expe(X_2^2)\big) - \frac{1}{4}\expe(X_1+X_2)\big(\expe(X_1)-\expe(X_2)\big) \\
	& = 0
\end{align*}
Thus $\rho(X,Y) = 0$. Note that $X$ and $Y$ are not independent (e.g.\ if $X=1$ then $Y=0$).
\end{answer}
%--------------------
\part Compute the conditional PMF and conditional expectation of $Y$ given that 
\begin{subparts}
\subpart $X=-1$, 
\subpart $X=0$,
\subpart $X=1$.
\end{subparts}
\begin{answer}
Given that $X=-1$, the conditional probabilities are
\[
\prob(Y=-1|X=-1) = 0,\quad \prob(Y=0|X=-1) = 1,\quad \prob(Y=1|X=-1) = 0.
\]
Given that $X=0$, the conditional probabilities are
\[
\prob(Y=-1|X=0) = 1/2,\quad \prob(Y=0|X=0) = 0,\quad \prob(Y=1|X=0) = 1/2.
\]
Given that $X=1$, the conditional probabilities are 
\[
\prob(Y=-1|X=1) = 0,\quad \prob(Y=0|X=1) = 1,\quad \prob(Y=1|X=0) = 0.
\]
%\prob(Y=-1|X=0)  = 1/2	& \prob(Y=0|X=0)  = 0	& \prob(Y=1|X=0)  = 1/2 	\\
%\prob(Y=-1|X=1)  = 0		& \prob(Y=0|X=1)  = 1	& \prob(Y=1|X=1)  = 0 	\\ \hline
%\end{array}


%\[
%\begin{array}{|lll|}\hline
%\prob(Y=-1|X=-1) = 0		& \prob(Y=0|X=-1) = 1	& \prob(Y=1|X=-1) = 0 	\\
%\prob(Y=-1|X=0)  = 1/2	& \prob(Y=0|X=0)  = 0	& \prob(Y=1|X=0)  = 1/2 	\\
%\prob(Y=-1|X=1)  = 0		& \prob(Y=0|X=1)  = 1	& \prob(Y=1|X=1)  = 0 	\\ \hline
%\end{array}
%\end{array}
%\]
In each case, the conditional expectation $\expe(Y|X=x)$ is zero:
\begin{align*}
\expe(Y|X=-1)	& = (-1\times 0) + (0\times 1) + (1\times 0) = 0, \\
\expe(Y|X=0)	& = (-1\times 1/2) + (0\times 0) + (1\times 1/2) = 0, \\
\expe(Y|X=1)	& = (-1\times 0) + (0\times 1) + (1\times 0) = 0.
\end{align*}
\end{answer}
%--------------------
\part Verify that the law of total expectation holds in this case.
\begin{answer}
The law of total expectation is verified in this case, because $\expe(Y) = 0$ and
\begin{align*}
\expe\big(\expe(Y|X)\big) 
	& = \sum_x \expe(Y|X=x)P(X=x) \\
	& = (0\times 1/4) + (0\times 1/2) + (0\times 1/4) \\
	& = 0,
\end{align*}
so $\expe(Y) = \expe\big(\expe(Y|X)\big)$.
\end{answer}
%--------------------
\end{parts}



\end{questions}
\end{exercise}
%======================================================================
\endinput
%======================================================================

%----------------------------------------------------------------------

%======================================================================
\endinput
%======================================================================
