% !TEX root = main.tex
%----------------------------------------------------------------------
\chapter{Moment Generating Functions}\label{chap:mgfs}
%----------------------------------------------------------------------
%----------------------------------------------------------------------
\section{Moment generating functions}
%----------------------------------------------------------------------

\bit
\it PGFs are defined only for discrete random variables taking non-negative integer values. 
\it MGFs are defined for any random variable.
\eit

% definition
\begin{definition}
The \emph{moment generating function} (MGF) of a random variable $X$ is a function $M:\R\to [0,\infty]$ given by
\[
M(t) = \expe(e^{tX}).
\] 
\end{definition}

% remarks
\begin{remark}
\ben
\it $e^{tX}$ is non-negative, so its expectation is well-defined, and $\expe(e^{tX})\geq 0$.
\it
For a discrete random variable $X$ taking non-negative integer values, 
\[
M(t) = \expe(e^{tX}) = \expe\big[(e^t)^X\big] = G(e^t),
\]
where $G$ is the PGF of $X$.
\it 
MGFs are related to \emph{Laplace transforms}.
\een
\end{remark}

% example: discrete (easy)
\begin{example}
The MGFs of some notable discrete distributions can be computed as follows:
\[
\begin{array}{llll}
X\sim\text{Bernoulli}(p):\quad	& G(t) = 1 - p + pt\quad	&\quad\Rightarrow\quad & M(t) =  1 - p + pe^t		\\[2ex]
X\sim\text{Binomial}(n,p):		& G(t) = (1-p+pt)^n			&\quad\Rightarrow\quad & M(t) = (1 - p + pe^t)^n	\\[2ex]
X\sim\text{Poisson}(\lambda):	& G(t) = e^{\lambda(t-1)}	&\quad\Rightarrow\quad & M(t) = e^{\lambda(e^t-1)}	\\ 
\end{array}
\]
\end{example}

MGFs have properties similar to those of PGFs:
% theorem
\begin{theorem}
\ben
\it If $X$ and $Y$ are independent, then $M_{X+Y}(t) = M_X(t)M_Y(t)$.
\it If $Y = a + bX$, then $M_Y(t) = e^{at} M_X(bt)$
\een
\end{theorem}

% proof
\begin{proof}
\ben
\it By independence, 
\[
M_{X+Y}(t) = \expe(e^{t(X+Y)}) = \expe(e^{tX}e^{tY}) = \expe(e^{tX})\expe(e^{tY}) = M_X(t)M_Y(t).
\]
\it For $Y=a+bX$, 
\[
M_Y(t) = \expe\big(e^{t(a+bX)}\big) = e^{at}\expe\big(e^{btX}\big) = e^{at}M_X(bt).
\]
\een
\end{proof}

% theorem
\begin{theorem}\label{taylor:mgf}
Let $M(t)$ be the MGF of the random variable $X$. If $M(t)$ converges on an open interval $(-R,R)$ centred at the origin, then 
\[
M(t) = \sum_{k=0}^{\infty} \frac{\expe(X^k)}{k!} t^k
\]
\end{theorem}

% proof
\begin{proof}
%Within its radius of convergence, $M(t)$ is equal to its Taylor expansion:
Using the series expansion of $e^{tX}$ and the linearity of expectation,
\[
M(t) 
	= \expe(e^{tX}) 
	= \expe\left(\sum_{k=0}^{\infty} \frac{(tX)^k}{k!}\right)
	= \sum_{k=0}^{\infty} \frac{\expe(X^k)}{k!} t^k
\]
\end{proof}

\begin{corollary}
Let $X$ be a random variable, and let $M(t)$ denote its MGF. Then %$\expe(X) = M'(0)$, and more generally,
\[
\expe(X^n) = M^{(n)}(0),
\]
where $M^{(n)}(0)$ is the $n$th derivative of $M(t)$ evaluated at $t=0$. In particular,
\[
M(0) = 1,\quad M'(0) = \expe(X),\quad M''(0) = \expe(X^2),\quad \text{and so on.}
\]
%Thus $M(t)$ is the \emph{exponential generating function} of the moment sequence $\mu_0,\mu_1,\mu_2,\ldots$, so
%\[
%M^{(n)}(0) = \mu_n \quad\text{for all } n
%\]
%where $M^{(n)}(0)$ denotes the $n$th derivative of $M(t)$ evaluated at $t=0$:
%\[
%M^{(n)}(0) = \left.\frac{d^{n}M}{dt^{m}}\right|_{t=0}.
%\]
%In particular, $M(0)=1$, $M'(0)=\expe(X)$ and $M''(0)-M'(0)^2=\var(X)$.
\end{corollary}



% example: exponential
\begin{example}[Exponential distribution]
Let $X\sim\text{Exponential}(\lambda)$ where $\lambda>0$ is a rate parameter. 
\ben
\it Show that the MGF of $X$ is given by $M(t)=\displaystyle\frac{\lambda}{\lambda-t}$.
\it Use $M(t)$ to find the mean and variance of $X$.
\een
\end{example}

\begin{solution}
\ben
\it % << (i)
The PDF of $X$ is $f(x)=\lambda e^{-\lambda x}$ for $x>0$ (and zero otherwise), so
\begin{align*}
M(t) 
	= \int_0^\infty e^{tx} f(x)\,dx 
	= \lambda \int_0^\infty e^{tx} e^{-\lambda x}\,dx 
	& = \lambda \int_0^\infty e^{-(\lambda-t)x}\,dx \\
	& = \frac{-\lambda}{\lambda-t} \left[e^{-(\lambda-t)x}\right]_0^\infty
	= \frac{\lambda}{\lambda-t}.
\end{align*}
\it % << (ii)
\bit
\it $M'(t) = \displaystyle \frac{\lambda}{(\lambda-t)^2}$, which yields $\mu_1 = M'(0)  = \displaystyle\frac{1}{\lambda}$.
\it $M''(t) = \displaystyle \frac{2\lambda}{(\lambda-t)^3}$ which yields $\mu_2 = M''(0)  = \displaystyle\frac{2}{\lambda^2}$.
\it Thus $\sigma^2 = \mu_2 - \mu_1^2 = \displaystyle\frac{1}{\lambda^2}$.
\eit
\een
\end{solution}
%-------------------------

%-------------------------
% example
\begin{example}[Gamma distribution]
The PDF of the $\text{Gamma}(k,\theta)$ distribution is given by
\[
f(x) = \left\{\begin{array}{ll}
	\displaystyle\frac{x^{k-1}e^{-x}}{\Gamma(k)} & x>0, \\
	0 & \text{othewise.}
\end{array}\right.
\]
Show that the MGF of the $\text{Gamma}(k,\theta)$ distribution is given by
\[
M(t) = \frac{1}{(1-\theta t)^k}
\]
\end{example}

\begin{solution}
Let $X\sim\text{Gamma}(k,1)$. Then
% PDF of $X$ is 
%$
%f(x) = \left\{\begin{array}{ll}
%	\displaystyle\frac{x^{k-1}e^{-x}}{\Gamma(k)} & x>0, \\
%	0 & \text{othewise.}
%\end{array}\right.
%$
%
%Its MGF can be computed as follows:
\begin{align*}
M(t) = \expe(e^{tX})
	= \int_0^\infty e^{tx} f(x)\,dx 
	& = \frac{1}{\Gamma(k)}\int_0^\infty x^{k-1}e^{-(1-t)x}\,dx \\
	& = \frac{1}{\Gamma(k)}\frac{1}{(1-t)^k}\int_0^\infty \big[(1-t)x\big]^{k-1}e^{-(1-t)x}(1-t)\,dx \\
	& = \frac{1}{\Gamma(k)}\frac{1}{(1-t)^k}\int_0^\infty y^{k-1} e^{-y}\,dy \\
	& = \frac{1}{(1-t)^k}.
\end{align*}
The scaled variable $Y=\theta X$ has distribution $Y\sim\text{Gamma}(k,\theta)$, so by the properties of MGFs, 
\[
M_Y(t) = M_X(\theta t) = \frac{1}{(1-\theta t)^k}.
\]
\end{solution}
%-------------------------

%-------------------------
% example
\begin{example}[Normal distribution]
By first considering the MGF of the $N(0,1)$ distribution, show that the MGF of the $N(\mu,\sigma^2)$ distribution is given by
\[
M(t) = \exp\left(\mu t + \frac{1}{2}\sigma^2t^2\right).
\]
\end{example}

\begin{solution}
Let $Z\sim N(0,1)$ be a standard normal variable; this has PDF
\[
f(z) = \frac{1}{\sqrt{2\pi}}e^{-\frac{1}{2}z^2}\qquad (z\in\R).
\]
Thus
\[
M(t) = \expe(e^{tZ})
	 = \int_{-\infty}^{\infty} e^{tz} f(z)\,dz 
	 = \frac{1}{\sqrt{2\pi}}\int_{-\infty}^{\infty} e^{tz-\frac{1}{2}z^2}\,dz 
\]
The exponent $tz-\frac{1}{2}z^2$ can be written as $\frac{1}{2}t^2 -\frac{1}{2}(z-t)^2$, so
\begin{align*}
M(t)
	& = \frac{1}{\sqrt{2\pi}}\int_{-\infty}^{\infty} e^{\frac{1}{2}t^2} e^{-\frac{1}{2}(z-t)^2}\,dz \\
	& = e^{\frac{1}{2}t^2} \left(\frac{1}{\sqrt{2\pi}}\int_{-\infty}^{\infty} e^{-\frac{1}{2}(z-t)^2}\,dz\right) \\
	& = e^{\frac{1}{2}t^2}.
\end{align*}

Let $X=\mu+\sigma Z$. Then $X\sim N(\mu,\sigma^2)$ and by the properties of MGFs,
\[
M_X(t) = e^{\mu t}M_Z(\sigma t) = e^{\mu t}e^{\frac{1}{2}\sigma^2t^2}= e^{\mu t + \frac{1}{2}\sigma^2t^2},
\]
as required.
\end{solution}

%----------------------------------------------------------------------
\section{Characteristic functions}
%----------------------------------------------------------------------

MGFs are useful, but the expectations that define them may not always be finite. Characteristic functions do not suffer this disadvantage.

% definition
\begin{definition}
The \emph{characteristic function} of a random variable $X$ is a function $\phi:\R\to\C$ given by 
\[
\phi(t) = \expe(e^{itX}) \qquad\text{where}\qquad i=\sqrt{-1}.
\]
\end{definition}

% fourier
\begin{remark}
\bit
%\it $\phi(t) = \expe(\cos tX)+ i\expe(\sin tX)$.
\it If $M(t)$ is the MGF of $X$, its characteristic function is given by $\phi(t) = M(it)$.
\it $\phi:\R\to\C$ exists for all $t\in\R$.
\it Characteristic functions are related to \emph{Fourier transforms}.
\eit
\end{remark}


%-------------------------
% theorem
\begin{theorem}%[Properties of characteristic functions]\label{thm:properties_cf}
\ben
%\it $\phi(0)=1$ and $|\phi(t)|\leq 1$ for all $t$.
\it If $X$ and $Y$ are independent, then $\phi_{X+Y}(t) = \phi_X(t)\phi_Y(t)$.
\it If $Y = a + bX$, then $\phi_Y(t) = e^{-iat}\phi_X(bt)$
\een
\end{theorem}
\proofomitted

%% proof
%\begin{proof}
%\ben
%\it % << (i)
%$\phi(0)=\expe(e^0)=\expe(1) = 1$, and 
%$\displaystyle |\phi(t)| = \left|\int e^{itx}\,dF(x)\right| \leq \int |e^{itx}|\,dF(x) = 1$.
%\it % << (ii)
%By independence, $\phi_{X+Y}(t) = \expe(e^{it(X+Y)}) = \expe(e^{itX}e^{itY})$, and since 
%\[
%e^{itX} = \cos(tX) + i\sin(tX)\quad\text{and}\quad e^{itY} = \cos(tY) + i\sin(tY),
%\]
%we see that 
%\[
%e^{itX}e^{itY} = \big[\cos(tX)\cos(tY) - \sin(tX)\sin(tY)\big] + i\big[\cos(tX)\sin(tY) + \sin(tX)\cos(tY)\big]
%\]
%Taking the expectation of both sides, and using the fact that $X$ and $Y$ are independent, 
%\begin{align*}
%\phi_{X+Y}(t) 
%	& = \expe\big(\cos(tX)+i\sin(tX)\big)\expe\big(\cos(tY)+i\sin(tY)\big) \\
%	& = \expe(e^{itX})\expe(e^{itY}) \\
%	& = \phi_X(t)\phi_Y(t).
%\end{align*}
%\it % << (ii)
%$\phi_Y(t) 
%	= \expe\big(e^{it(a+bX)}\big) 
%	= \expe\big(e^{iat}e^{i(bt)X}\big) 
%	= e^{iat}\expe(e^{i(bt)X}) 
%	= e^{iat}\phi_X(bt)$.
%\een
%\end{proof}

%----------------------------------------------------------------------
\subsection{The inversion theorem} 
%----------------------------------------------------------------------
The \emph{inversion theorem} asserts that a random variable is entirely specified by its characteristic function, meaning that $X$ and $Y$ have the same characteristic function if and only if they have the same distribution. We state the inversion theorem only for continuous distributions:
% theorem
\begin{theorem}[Fourier inversion theorem]
If $X$ is continuous with density function $f$ and characteristic function $\phi$, then
\[
f(x) = \frac{1}{2\pi}\int_{-\infty}^{\infty} e^{-itx}\phi(t)\,dt 
\] 
at every point $x$ at which $f$ is differentiable.
\end{theorem}
\proofomitted

%% corollary
%\begin{corollary}
%Two random variables $X$ and $Y$ have the same characteristic function if and only if they have the same distribution function.
%\end{corollary}

%----------------------------------------------------------------------
\subsection{The continuity theorem} 
%----------------------------------------------------------------------
For a sequence of random variables $X_1,X_2,\ldots$, the \emph{continuity theorem} asserts that if the cumulative distribution functions $F_1,F_2,\ldots$ of the sequence approaches some limiting distribution $F$, then the charactaristic functions $\phi_1,\phi_2,\ldots$ of the sequence approaches the characteristic function of $F$.

% convergence in distribution
\begin{definition}
A sequence of distribution functions $F_1,F_2,\ldots$ is said to \emph{converge} to the distribution function $F$, denoted by $F_n\to F$, if $F_n(x)\to F(x)$ as $n\to\infty$ at each point $x$ at which $F$ is continuous.
\end{definition}

% theorem: continuity
\begin{theorem}[Continuity theorem]
Let $F_1,F_2,\ldots$ and $F$ be distribution functions, and let $\phi_1,\phi_2,\ldots$ and $\phi$ denote the corresponding characteristic functions. 
\ben
\it If $F_n\to F$ then $\phi_n(t)\to\phi(t)$ for all $t$.
\it If $\phi_n(t)\to\phi(t)$ then $F_n\to F$ provided $\phi(t)$ exists and is continuous at $t=0$.
\een
\end{theorem}
\proofomitted

%\begin{remark}
%The inversion and continuity theorems are needed to prove the central limit theorem (Lecture~\ref{chap:clt}).
%\end{remark}

%----------------------------------------------------------------------
\section{Exercises}
% !TEX root = main.tex
%----------------------------------------------------------------------
\begin{exercise}
\begin{questions}
%----------------------------------------
%----------------------------------------
\question
Let $X$ be a discrete random variable, taking values in the set $\{-3, -2, -1, 0, 1, 2, 3\}$ with uniform probability, and let $M(t)$ denote the MGF of $X$.
\ben
\it % << (i)
Show that $M(t) = \frac{1}{7}(e^{-3t} +e^{-2t} +e^{-t} +1 +e^{t} +e^{2t} +e^{3t})$.
\it % << (ii)
Use $M(t)$ to compute the mean and variance of $X$.
\een

\begin{answer}
\ben
\it % << (i)
$
M(t) 
	= \expe(e^{Xt}) = \sum_k e^{tk} p(k)
	= \frac{1}{7}(e^{-3t} +e^{-2t} +e^{-t} +1+e^{t} +e^{2t} +e^{3t})
$
\it % << (ii)
Using the power series expansion $e^x=\displaystyle\sum_{k=0}^{\infty}\frac{x^k}{k!}$, the MGF of $X$ can be written as
\begin{align*}
M(t) 
& = \frac{1}{7}\left[  \left(1 - 3t + \frac{9t^{2}}{2} +\ldots\right)
					+ \left(1 - 2t + \frac{4t^{2}}{2} +\ldots\right)
					+ \left(1 - t  + \frac{ t^{2}}{2} +\ldots\right)\right. \\
& \qquad\qquad\left.		+ 1 
					+ \left(1 + t  + \frac{ t^{2}}{2} +\ldots\right)
					+ \left(1 + 2t + \frac{4t^{2}}{2} +\ldots\right)
					+ \left(1 + 3t + \frac{9t^{2}}{2} +\ldots\right)\right] \\
& = 1 + \frac{4t^{2}}{2} + \ldots
\end{align*}
\bit
\it The mean is the coefficient of $t$, so $\mu = 0$.
\it The second non-central moment is the coefficient of $\frac{1}{2}t^{2}$, so $\mu'_2 = 4$.
\it The variance is therefore $\sigma^2 = \mu'_2 - \mu^2 = 4$.
\eit
A more direct method is to use the fact that $\mu = M'_X(0)$ and $\mu'_2 = M''_X(0)$. Since
\begin{align*}
M'(t)	& = \frac{1}{7} (-3e^{-3t} -2e^{-2t} -e^{-t} +e^{t} +2e^{2t} +3e^{3t}), \\
M''(t)	& = \frac{1}{7} (9e^{-3t} +4e^{-2t} +e^{-t} +e^{t} +4e^{2t} +9e^{3t})
\intertext{it follows that}
M'(0)	& = \frac{1}{7} (-3 -2 -2 + 1 + 2 + 3) = 0, \\
M''(0)	& = \frac{1}{7} (9 +4 +1 +1 +4 +9) = 4.
\end{align*}
\een
\end{answer}


%==========================================================================
\question
A continuous random variable $X$ has MGF given by $M(t)=\exp(t^2+3t)$. Find the distribution of $X$. 

\begin{answer}
\[
M_{X}(t) = \exp(t^{2} + 3t) = \exp(3t + \frac{1}{2}2t^2)).
\]
The MGF is of the form $\exp(\mu t + \frac{1}{2}\sigma^2 t^2)$, which is the MGF of a normal distribution. The mean $\mu$ is 3 and the variance $\sigma^2$ is 2, so $X\sim N(3, 2)$. Note that under reasonable conditions, a random variable can be uniquely identified by its MGF: only normally distributed random variables have MGFs of the form $\exp(\mu t + \frac{1}{2}\sigma^2 t^2)$. 
\end{answer}


%==========================================================================
\question
Let $X$ be a discrete random variable with probability mass function
\[
\prob(X=k) = \begin{cases}
	q^k p		& k=0,1,2,\ldots \\
	0			& \text{otherwise},
\end{cases}	
\]
where $0<p<1$ and $q=1-p$. 
\ben
\it % << (i)
Show that the MGF of $X$ is given by $M(t) = \displaystyle\frac{p}{1-qe^t}$ for $t < -\log q$.
\it % << (ii)
Find the PGF of $X$.
\it % << (iii)
Use the PGF of $X$ to find the PMF of $Y = X + 1$.
\it % << (iv)
Use $M(t)$ to find the mean and variance of $X$.
\een

\begin{answer}
\ben
\it % << (i)
The MGF of $X$ is
\begin{align*}
M(t) 
	= \expe(e^{Xt}) 
	= \sum\limits_{k=0}^{\infty }e^{kt} q^k p 
	= p\sum\limits_{k=0}^{\infty }\big[qe^{t}\big]^k 
	= \frac{p}{1-qe^{t}}
\end{align*}
provided that $|qe^{t}|<1$, or equivalently that $t<-\log q$.
\it % << (ii)
Since $M(t) = G(e^t)$, the PGF is
\[
G(t) = M(\log t) = \frac{p}{1-qt} .
\]
Note that $G(0)=0$ and $G(1)=1$.
\it % << (iii)
To derive the PMF of $Y=X+1$, we have
\[
G_Y(t) = \expe(t^Y) = \expe(t^{X+1}) = \expe(t^X.t) = t\expe(t^X) = t G_(t)
\]
so by part (i),
\[
G_Y(t) = \frac{pt}{1-qt} = pt(1 + qt + q^2t^2 + \ldots
\]
Comparing the coefficients in this expression with those of $G_Y(t)$ expressed as a power series,
\[
G_Y(t) = \sum_{k=0}^{\infty} \prob(Y=k) t^k = \prob(Y=0) + \prob(Y=1)t + \prob(Y=2)t^2 + \ldots
\]
we see that 
\[\begin{array}{lll}
\prob(Y = 0) & = \text{constant term} 			& = 0 \\
\prob(Y = 1) & = \text{coefficient of $t$} 		& = p \\
\prob(Y = 2) & = \text{coefficient of $t^2$}		& = p(1 - p) \\
\prob(Y = 3) & = \text{coefficient of $t^3$}		& = p(1 - p)^2 \text{ etc.}
\end{array}\]
As we might expect, $\prob(Y = k) = \prob(X = k - 1)$ for $k = 0,1,\ldots$.

\it % << (iiv)
\[
M'(t) = \frac{d}{dt}\left(\frac{p}{1-qe^t}\right) = \frac{pqe^t}{(1-qe^t)^2}
\]
so
\[
\mu = M'(0) = \frac{pq}{(1-q)^2} = \frac{q}{p} = \frac{1-p}{p}.
\]
Similarly,
\[
M''_{X}(t) 
	= \frac{d}{dt}\left(\frac{pqe^t}{(1-qe^t)^2}\right) 
	= \frac{pqe^t(1+qe^t)}{(1-qe^t)^3}
\]
so
\[
\mu'_2 = M''_{X}(0) = \frac{q(1+q)}{p^2} = \frac{(1-p)(2-p)}{p}
\]
and hence
\[
\var(X) = \mu'_2 - \mu^2 = \frac{q}{p^2} = \frac{1-p}{p^2}.
\]
\een
\end{answer}



%==========================================================================
\question
Let $M(t)$ denote the MGF of the normal distribution $\text{N}(0,\sigma^2)$. By exanding $M(t)$ as a power series in $t$, show that the moments $\mu_k$ of the $\text{N}(0,\sigma^2)$ distribution are zero if $k$ is odd, and equal to 
\[
\mu_{2m} = \frac{\sigma^{2m}(2m)!}{2^m m!} \qquad\text{if $k=2m$ is even.}
\]

\begin{answer}
The MGF of the $N(\mu,\sigma^2)$ distribution is $\exp\left(\mu t + \frac{1}{2}\sigma^2 t^2 \right)$, so 
\[
M(t) = \exp\left(\frac{\sigma^2 t^2}{2}\right).
\]
Expand this as a power series in $t$:
\[
M(t) = 1 + \frac{\sigma^2 t^2}{2} + \frac{\sigma^4 t^4}{4\times 2!} + \frac{\sigma^6 t^6}{8\times 3!} 
			+ \ldots + \frac{\sigma^{2m}t^{2m}}{2^m\times m} + \ldots
\]
The $k$th moment $\mu_k$ is the coefficient of $t^{k}/k!$ in this expansion. In particular, the skewness $\gamma_1=\mu_3/\sigma^3$ is zero, and the excess kurtosis is
\[
\gamma_2 = \frac{\mu_4}{\sigma^4} -3 = \frac{\sigma^4 4!}{4\times 2!}\cdot\frac{1}{\sigma^4} - 3 = 0.
\]
\end{answer}



%==========================================================================
\question
Let $X\sim\text{Exponential}(\theta)$ where $\theta$ is a scale parameter.
%Then $X$ has probability density function given by
%\[
%p(x) = \begin{cases}
%\frac{1}{\theta} e^{-x/\theta} 	&  x > 0, \\
%0  										& \text{otherwise.}
%\end{cases}
%\]
%
\ben
\it Show that the MGF of $X$ is $M(t) = \displaystyle\frac{1}{1-\theta t}$.
\it By expanding this expression as a power series in $t$, find the first four non-central moments of $X$.
\it Find the skewness $\gamma_1$ and the excess kurtosis $\gamma_2$ of $X$.
\een

\begin{answer}
\ben
\it % (a)
\begin{align*}
M(t) = \expe(e^{Xt})
	& = \int e^{xt} \frac{1}{\theta}e^{-x/\theta}\,dx \\
	& = \frac{1}{\theta}\int_0^\infty e^{-x(1-\theta t)/\theta}\,dx \\
	& = -\frac{1}{1-\theta t}\int_0^\infty \frac{d}{dx} e^{-x(1-\theta t)/\theta}\,dx \\
	& = -\frac{1}{1-\theta t}\left[e^{-x(1-\theta t)/\theta}\right]_0^\infty \\
	& = \frac{1}{1-\theta t}.
\end{align*}
\it % (b)
Using the binomial expansion,
\[
M(t)	= (1-\theta t)^{-1}
		= 1 + \theta t + \theta^2 t^2 + \theta ^3 t^3 + \theta ^4 t^4 + \ldots
\]
The non-central moment $\mu_k$ is the coefficient of $\frac{t^k}{k!}$ in this power series expansion, so $\mu = \theta$, $\mu_2 = 2\theta^2$, $\mu_3 = 6\theta^3$ and $\mu_4 = 24\theta^4$.
\it % (c)
The skewness and excess kurtosis are
\begin{align*}
\gamma_1 	& = \frac{\mu_3}{\sigma^3} = \frac{2\theta^3}{\theta^3} = 2, \quad\text{and} \\
\gamma_2 	& = \frac{\mu_4}{\sigma^4} - 3 = \frac{9\theta^4}{\theta^4} -3 = 6,
\end{align*}
respectively.
\een
\end{answer}


%==========================================================================
\question
Let $X_1,X_2,\ldots$ be independent and identically distributed random variables, with each $X_i\sim\text{N}(\mu,\sigma^2)$.
% Let $M(t)$ denote the common MGF of the $X_i$. 
\ben
\it Find the MGF of the random variable $\displaystyle\bar{X}=\frac{1}{n}\sum_{i=1}^{n} X_i$.
\it Show that $\bar{X}$ has a normal distribution, and find its mean and variance.           
\een
\begin{answer}
Let $M(t)$ denote the common MGF of the random variables $X_i$:
\[
M(t) = \exp\left(\mu t+\frac{1}{2}\sigma^2 t^2\right)
\]
Moment generating functions have the following properties:
\bit
\it $M_{X+Y}(t)=M_X(t)M_Y(t)$
\it If $Y=a+bX$ then $M_Y(t) = e^{at}M_X(bt)$.
\eit
In view of these, the MGF of $\bar{X}$ is
\begin{align*}
M_{\bar{X}}(t)
	& = M_{\frac{1}{n}(X_1+X_2+\ldots+X_n)}(t) \\ 
	& = M_{X_1+X_2+\ldots+X_n}\left(\frac{t}{n}\right) \\ 
	& = \left[M\left(\frac{t}{n}\right)\right]^n 
	 = \exp\left[\frac{\mu t}{n} + \frac{\sigma^2 t^2}{2n^2}\right]^n \\
	& = \exp\left[\mu t + \frac{1}{2}\left(\frac{\sigma^2}{n}\right) t^2\right]
\end{align*}
which is the MGF of a normal distribution with mean $\mu$ and variance $\sigma^2/n$.
\end{answer}



%==========================================================================
\question
%Let $X$ be a continuous random variable having gamma distribution with shape parameter $k>0$ and scale parameter $\theta>0$, denoted by $X\sim\text{Gamma}(k,\theta)$. The probability density function of $X$ is
% \[
%p(x) = \begin{cases}
%	\displaystyle\frac{1}{\theta^k\text{Gamma}ma(k)}x^{k-1}e^{-x/\theta}	& \text{for}\ \ x > 0, \\[2ex]
%	0														& \text{otherwise,}
%\end{cases}	
%\]
Let $X_1\sim\text{Gamma}(k_1,\theta)$ and $X_2\sim\text{Gamma}(k_2,\theta)$ be independent random variables. Use the MGFs of $X_1$ and $X_2$ to find the distribution of the random variable $Y=X_1+X_2$.


\begin{answer}
The $MGFs$ of $X_1$ and $X_2$ are:
\[
M_{X_1}(t) = \frac{1}{(1-\theta t)^{k_1}}
\quad\text{and}\quad
M_{X_2}(t) = \frac{1}{(1-\theta t)^{k_2}}.
\]
Since $X_1$ and $X_2$ are independent, 
\[
M_{X_1+X_2}(t) 
=	M_{X_1}(t)M_{X_2}(t) 
= \frac{1}{(1-\theta t)^{k_1+k_2}}
\]
This is the MGF of a $\text{Gamma}(k_1+k_2,\theta)$ random variable.
\end{answer}

% GS 5.9.3
%==========================================================================
\question
A coin has probability $p$ of showing heads. The coin is tossed repeatedly until exactly $k$ heads occur. Let $N$ be the number of times the coin is tossed. Using the continuity theoram for characteristic functions, show that the distribution of the random variable $X=2pN$ converges to a gamma distribution as $p\to 0$.

\begin{answer}
Let $N=Z_1+Z_2+\ldots+Z_k$ with $Z_i\sim\text{Geometric}(p)$ independent. Let $\phi_Z(t)$ denote the common characteristic function of the $Z_i$:
\[
\phi_Z(t) = \expe(e^{itZ}) = \frac{pe^{it}}{1-qe^{it}} \qquad\text{where $Z\sim\text{Geometric}(p)$ and $q=1-p$.}
\]
By the properties of characteristic functions, the characteristic function of $N$ is
\[
\phi_N(t) = \big[\phi_Z(t)\big]^k = \left[\frac{pe^{it}}{1-qe^{it}}\right]^k
\]
and the characteristic function of $X=2pN$ is
\begin{align*}
\phi_X(t) = \phi_N(2pt) 
	= \left[\frac{pe^{2pit}}{1-(1-p)e^{2pit}}\right] 
	= \left[\frac{1 + 2pit + o(1)}{1 - 2it + o(1)}\right]
	\rightarrow \frac{1}{1 - 2it} \qquad\text{as}\quad p\to 0,
\end{align*}	
where $o(1)$ respresnts a quantity that tends to zero as $p\to 0$. This is the characteristic function of the $\text{Gamma}(k,\frac{1}{2})$ distribution. The result then follows by the continuity theorem for characteristic functions.
\end{answer}


% GS 5.12.23(a)
%==========================================================================
\question
Let $X$ and $Y$ be independent and identically distributed random variables, with means equal to $0$ and variances equal to $1$. Let $\phi(t)$ denote their common characteristic function, and suppose that the random variables $X+Y$ and $X-Y$ are independent. Show that $\phi(2t)=\phi(t)^3\phi(-t)$, and hence deduce that $X$ and $Y$ must be independent standard normal variables.

\begin{answer}
Let $U=X+Y$ and $V=X-Y$. Since $U$ and $V$ are independent, we have $\phi_{U+V}(t)=\phi_U\phi_V$, or equivalently $\phi_{2X}=\phi_{X+Y}\phi_{X-Y}$. Thus, since $\phi_{2X}(t)=\phi(2t)$,$\phi_{X+Y}=\phi(t)^2$ and $\phi{X-Y}(t) = \phi(t)\phi(-t)$, it follows that
\[
\phi(2t) = \phi(t)^3\phi(-t)
\]
To show that $X,Y\sim\text{N}(0,1)$, it is sufficient to show that $\phi(t) = e^{-\frac{1}{2}t^2}$ (by the inversion theorem).

It can be shown that characteristic functions are symmetric: $\phi(t)=\phi(-t)$ for all $t$. 

Thus we have $\phi(2t)=\phi(t)^4$, and hence
\[
\phi(t) 
	= \phi\left(\frac{1}{2}t\right)^4 
	= \phi\left(\frac{1}{4}t\right)^{16} 
	= \ldots 
	= \phi\left(\frac{1}{2^n}t\right)^{2^{2^n}} \qquad\text{for $n=0,1,2,\ldots$}.
\]
Hence,	
\[
\phi(t) 
	= \left[1 - \frac{1}{2}\left(\frac{t}{2^n}\right)^2 + \ldots\right]^{2^{2^n}} 
	\to e^{-\frac{1}{2}t^2} \qquad\text{as $n\to\infty$.}
\]
\end{answer}


% GS 5.12.46/47
%==========================================================================
%\question
%A biased coin shows heads with probability $p$. 
%\ben
%\it % << (i)
%The coin is tossed repeately. Let $W_n$ be the number of tosses until the first run of $n$ consecutive heads, and let $G_n(t)=\expe(t^{W_n})$ denote its probability generating function. Show that 
%\[
%G_n(t) = \frac{ptG_{n-1}(t)}{1-qtG_{n-1}(t)} = \frac{(1-pt) p^n t^n}{1-t + q p^n t^{n+1}}
%\]
%\it % << (ii)
%In a sequence of exactly $n$ tosses, let $L_n$ be the length of the longest run of heads. For $r\geq 1$, show that
%\[
%1 + \sum_{n=1}^{\infty}t^n\prob(L_n < r) = \frac{1-p^r t^r}{1 - t + q p^r t^{r+1}}.
%\]
%\een
%
%\begin{answer}
%\ben
%\it % << (i)
%We have that
%\[
%W_n = \begin{cases}
%	W_{n-1} + 1					& \text{with probability $p$,} \\
%	W_{n-1} + 1	+\widetilde{W}_n & \text{with probability $q$,} \\
%\end{cases}
%\]
%where $\widetilde{W}_n$ is independent of $W_{n-1}$ has the same distribution as $W_n$. Hence
%\[
%G_n(t) = ptG_{n-1}(t) + qtG_{n-1}(t)G_n(t).
%\]
%The recurrence relation can be solved by induction, using the fact that $G_0(t)=1$. 
%\it % << (ii)
%Let $W_r$ be the number of tosses until we first observe $r$ consecutive heads. 
%\par
%Then $\prob(L_n < r)=\prob(W_r > n)$, and hence
%\[
%1 + \sum_{n=1}^{\infty} t^n \prob(L_r < r) = \sum_{n=0}^{\infty} t^n \prob(W > n) = \frac{1-\expe(t^{W_r})}{1-t},
%\]
%where $\expe(t^{W_r}) = G_r(t)$ is as defined in part (i).
%\een
%\end{answer}


%% GS 5.8.1 (i - iii)
%%==========================================================================
%\question
%If $\phi$ is a characteristic function, show that $\bar{\phi}$, $\phi^2$ and $|\phi|^2$ are also characteristic functions.
%\begin{answer}
%\ben
%\it % << (i)
%$\overline{\phi_X}(t) = \overline{\expe(e^{itX})} = \expe(e^{-itX}) = \phi_{-X}(t)$.
%\it % << (ii)
%Let $X_1$ and $X_2$ are independent random variables with common characteristic function $\phi$. Then \[\phi_{X_1+X_2}(t)=\phi_{X_1}(t)\phi_{X_2}(t) = \phi(t)^2.\]
%\it % << (iii)
%Similarly, \[\phi_{X_1-X_2}(t)=\phi_{X_1}(t)\phi_{-X_2}(t) =\phi_{X_1}(t)\,\overline{\phi_{X_2}(t)} = |\phi(t)|^2.\]
%\een
%\end{answer}
%
%
%% GS 5.8.8
%%==========================================================================
%\question
%Let $X$ be exponentially distributed with (rate) parameter $\lambda$. Show by elementary integration that its characteristic function is $\displaystyle\phi(t)=\frac{\lambda}{\lambda-it}$.
%\begin{answer}
%By definition, $\expe(e^{itX}) = \expe(\cos(tX)) + i\expe(\sin(tX))$. 
%
%Integrating by parts, we get
%\begin{align*}
%\expe(\cos(tX)) & = \int_{0}^{\infty}\cos(tX)\lambda e^{-\lambda x}\,dx = \frac{\lambda^2}{\lambda^2+t^2}, \\
%\expe(\sin(tX)) & = \int_{0}^{\infty}\sin(tX)\lambda e^{-\lambda x}\,dx = \frac{\lambda t}{\lambda^2+t^2}. \\
%\end{align*}
%Thus
%\[
%\expe(e^{itX}) = \frac{\lambda^2 + i\lambda t}{\lambda^2+t^2} = \frac{\lambda}{\lambda - it}.
%\]
%\end{answer}
%----------------------------------------
\end{questions}
\end{exercise}
%----------------------------------------------------------------------

%======================================================================
\endinput
%======================================================================

%----------------------------------------------------------------------

%======================================================================
\endinput
%======================================================================
