% !TEX root = main.tex
%----------------------------------------------------------------------
\begin{exercise}
\begin{questions}
%----------------------------------------
%==========================================================================
\question
Let $X\sim\text{Binomial}(m,p)$ and $Y\sim\text{Binomial}(n,p)$. Show that $X+Y\sim\text{Binomial}(m+n,p)$,
\begin{answer}
The PGFs of $X$ and $Y$ are
\[
G_X(t) = (1-p+pt)^m
\quad\text{and}\quad
G_Y(t) = (1-p+pt)^n
\]
Using the properties of PGFs,
\[
G_{X+Y}(t) = G_X(t)G_Y(t) = (1-p+pt)^m (1-p+pt)^n = (1-p+pt)^{m+n},
\]
which we recognise as the PGF of the $\text{Binomial}(m+n,p)$ distribution.
\end{answer}

%==========================================================================
\question
Show that a discrete distribution on the non-negative integers is uniquely determined by its PGF, in the sense that if two such random variables $X$ and $Y$ have PGFs $G_X(t)$ and $G_Y(t)$ respectively, then $G_X(t)=G_Y(t)$ if and only if $\prob(X=k)=\prob(Y=k)$ for all $k=0,1,2,\ldots$. 
\begin{answer}
The PGFs of $X$ and $Y$ are
\[
G_X(t) = \sum_{k=1}^{\infty}\prob(X=k)t^k
\quad\text{and}\quad
G_Y(t) = \sum_{k=1}^{\infty}\prob(Y=k)t^k
\]
Clearly, if $\prob(X=k)=\prob(Y=k)$ for all $k=0,1,2,\ldots$, then $G_X(t)=G_Y(t)$. Conversely, $G_X(1)=1$ implies that its power series expansion (about the origin) is unique, and likewise for $G_Y$. Thus if $G_X=G_Y$, their power series must have identical coefficients, so $\prob(X=k)=\prob(Y=k)$ for all $k=0,1,2,\ldots$, as required
\end{answer}

%==========================================================================
\question
The PGF of a random variable is given by $G(t)=1/(2-t)$.
%\[
%G(t) = \frac{1}{2-t}.
%\]
What is its PMF?
\begin{answer}
To find the PMF, we need to express $G(t)$ as a power series:
\[
G_X(t) 
	= \frac{1}{2-t} 
	= \frac{1}{2}\left(1 - \frac{s}{2}\right)^{-1}
	= \frac{1}{2}\left(1 + \frac{t}{2} + \left(\frac{t}{2}\right)^2 + \ldots\right)
	- \sum_{k=0}^{\infty} \left(\frac{1}{2}\right)^{k+1} t^k
\]
Thus the PMF of $X$
\[
\prob(X=k) = \left(\frac{1}{2}\right)^{k+1} \quad\text{for } k=0,1,2,\ldots.
\]

\end{answer}

%% GS 5.2.3(a)
%%==========================================================================
%\question
%Let $X\sim\text{Poisson}(Y)$, where $Y\sim\text{Poisson}(\lambda)$. Show that $G_{X+Y}(t) = \exp\big[\lambda(te^{t-1}-1)\big]$.
%
%\begin{answer}
%Let $Y=y$ be fixed. Then $X\sim\text{Poisson}(y)$, so $\expe(t^X) =  \expe(e^{y(t-1)})$ so
%\[
%\expe(t^{X+Y}|Y=y) = \expe(t^{X+y}) = t^y\expe(t^X) = t^y e^{y(t-1)}.
%\]
%Thus (by the law of total expectation),
%\begin{align*}
%G_{X+Y} = \expe(t^{X+Y}) 
%	= \expe\big(\expe(t^{X+Y}|Y)\big) 
%	= \expe\big(t^Y e^{Y(t-1)}\big) 
%	= \expe\big((te^{t-1})^Y\big) 
%	= G_Y(te^{t-1}) 
%%	= \exp\big(\lambda(te^{t-1}-1)\big)
%	= e^{\lambda(te^{t-1}-1)}
%\end{align*}
%\end{answer}
%

% GS 5.2.4
%==========================================================================
\question
Let $X\sim\text{Binomial}(n,p)$. Using the PGF of $X$, show that 
\[
\expe\left(\frac{1}{1+X}\right) = \frac{1-(1-p)^{n+1}}{(n+1)p}.
\]

\begin{answer}
Let $G(t)$ be the PGF of $X$. Then $G(t)=\expe(t^X) = (q+pt)^n$ where $q=1-p$. 

Now
\[
\int_0^1 t^x\,dt = \left[\frac{t^{1+x}}{1+x}\right]_0^1 = \frac{1}{1+x},
\]
so
\[
\expe\left(\frac{1}{1+X}\right) 
	= \expe\left(\int_0^1 t^X \,dt\right)
	= \int_0^1 \expe(t^X)\,dt 
	= \int_0^1 (q+pt)^n\,dt
	= \frac{1-q^{n+1}}{(n+1)p}
\]
\end{answer}

%% GS 5.1.6
%%==========================================================================
%\question
%Let $U\sim\text{Uniform}(0,1)$ be a continuous variable, and let $X\sim\text{Binomial}(n,U)$. Use probability generating functions to show that $X$ has the discrete uniform distribution on $\{0,1,\ldots, n\}$.
%\par [Hint. First find $\expe(t^X|U=u)$, then use the law of total expectation.]
%
%\begin{answer}
%The PGF of a discrete random variable $Y$ distributed uniformly on $\{0,1,2,\ldots,n\}$ is
%\[
%\expe(t^Y) 
%	= \sum_{k=0}^n t^k \prob(Y=k) = \frac{1}{n+1}\sum_{k=0}^n t^k = \frac{1}{n+1}\left(\frac{1-t^{n+1}}{1-t}\right)
%\]
%For $U=u$ fixed, $X\sim\text{Binomial}(n,u)$ so $\expe(t^X|U=u) = (1 - u + ut)^n$. Taking the expectation of this as $U$ varies uniformly over $(0,1)$, the PGF of $X$ is
%\begin{align*}
%\expe(t^X) = \expe\big[\expe(t^X|U)\big]
%	& = \int_{0}^{1} (1 - u + ut)^n \,du \\
%	& = \frac{1}{n+1}\int_0^1 \frac{d}{du}\left[\frac{\big(1-u(1-t)\big)^{n-1}}{1-s}\right]\,du \\
%	& = \frac{1}{n+1}\left[\frac{\big(1-u(1-t)\big)^{n-1}}{1-t}\right]_0^1 \\
%	& = \frac{1}{n+1}\left(\frac{1-t^{n+1}}{1-t}\right)
%\end{align*}
%$X$ and $Y$ have the same distribution, so $X$ is distributed uniformly on $\{0,1,\ldots, n\}$
%\end{answer}


% GS 5.1.2
%%==========================================================================
%\question
%Let $X$ be a discrete random variable taking non-negative integer values, let $G(t)$ denote its probability generating function, and let $a_n=\prob(X>n)$ denote the so-called \emph{tail} probabilities of $X$. 
%\ben
%\it % << (i)
%Show that the generating function of the sequence $\{a_0,a_1,\ldots\}$ is $H(t) = \displaystyle\frac{1-G(t)}{1-t}$.
%\it % << (ii)
%Show that $\expe(X)=H(1)$ and $\var(X)=2H'(1) + H(1) - H(1)^2$.
%\een
%
%\begin{answer}
%\ben
%\it % << (i)
%%Let $I_{\{X>n\}}$ be the indicator function of the event $\{X>n\}$. Then
%\begin{align*}
%H(t) = \sum_{n=0}^{\infty}t^n\prob(X>n)
%	= \expe\left(\sum_{n=0}^{\infty}t^n I(X>n)\right) 
%	= \expe\left(\sum_{n=0}^{X-1}t^n \right) 
%	= \expe\left(\frac{1-t^X}{1-t}\right) 
%	= \frac{1-G(t)}{1-t}
%\end{align*}	
%\it % << (ii)
%Using L'H\^{o}pital's rule,
%\[
%H(1) = \lim_{t\uparrow 1} H(t)
%= \lim_{t\uparrow 1} \left[\frac{1-G(t)}{1-t}\right]
%= \lim_{t\uparrow 1} \left[\frac{G'(t)}{1}\right]
%= G'(1)
%= \expe(X)
%\]
%Similarly, since $\displaystyle H'(t) = \frac{1-G(t)-(1-t)G'(t)}{(1-t)^2}$ we have
%\begin{align*}
%H'(1) = \lim_{t\uparrow 1} H(t)
%	& = \lim_{t\uparrow 1} \left[\frac{1-G(t)-(1-t)G'(t)}{(1-t)^2} \right] \\
%	& = \lim_{t\uparrow 1} \left[\frac{G''(t)}{2}\right] 
%	= \frac{1}{2}G''(t)
%	= \frac{1}{2}\left[\var(X) - G'(1) + G'(1)^2\right]
%\end{align*}
%so $\var(X)=2H'(1) + H(1) - H(1)^2$.
%\een
%\end{answer}

%----------------------------------------
\end{questions}
\end{exercise}
%----------------------------------------------------------------------

%======================================================================
\endinput
%======================================================================
