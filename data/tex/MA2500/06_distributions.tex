% !TEX root = main.tex
%----------------------------------------------------------------------
\chapter{Distributions}\label{chap:distributions}
%----------------------------------------------------------------------
%----------------------------------------------------------------------
\section{Probability on the real line}
%----------------------------------------------------------------------
Let $X:\Omega\to\R$ be a random variable, and recall the probability measure on $(\R, \mathcal{B})$,  defined by
\[
\prob_X(B) = \prob(X\in B) = \prob\big(\big\{\omega:X(\omega)\in B\}\big),
\]
where $\mathcal{B}$ is the Borel $\sigma$-field over $\R$.

% definition
\begin{definition}
\ben
\it
The \emph{distribution} of $X$ is the probability measure $\prob_X(B) = \prob(X\in B)$.
\it 
The \emph{cumulative distribution function} (CDF) of $X$ is the function $F(x) = \prob(X\leq x)$.
\it 
The \emph{survival function} (SF) of $X$ is the function $S(t) = \prob(X > t)$.
\een
\end{definition}

\begin{remark}
The survival function is also called the \emph{complementary} distribution function.
If $X$ represents the \emph{lifetime} of some random system, then $S(t)=\prob(X>t)$ is the probability that the system survives beyond time $t$. In this context, $F(t) = 1 - S(t)$ is called the \emph{lifetime distribution function}.
\end{remark}


%----------------------------------------------------------------------
\section{Cumulative distribution functions (CDFs)}
%----------------------------------------------------------------------
Proposition~\ref{prop:rv_alt} states that $X:\Omega\to\R$ is a random variable if and only if the sets $\{X\leq x\}$ are \emph{events} over $\Omega$:
\[
\{X\leq x\} = \big\{\omega:X(\omega)\leq x\big\}\in\mathcal{F} \text{\quad for all\quad} x\in\R.
\]

It can be shown that the probability measure
\[
\prob_X(B) = \prob(X\in B) = \prob\big(\big\{\omega:X(\omega)\in B\}\big),
\]
is uniquely defined by the values it takes on the events $\{X\leq x\}$ for $x\in\R$. Consequently, the distribution of a random variable is uniquely determined by its \emph{cumulative distribution function} (CDF):

%$\prob_X$ is uniquely determined by the \emph{distribution function} of $X$.

%\eit
%\[
%\{X\leq x\} = \{\omega: X(\omega)\leq x\} \text{\quad for\quad} x\in\R.
%\]


% definition: cdf
\begin{definition}
The \emph{cumulative distribution function} (CDF) of a random variable $X:\Omega\to\R$ is the function
\[
\begin{array}{cccl}
F:	& \mathbb{R}	& \longrightarrow	& [0,1] \\
	& x 			& \mapsto			& \prob(X\leq x).
\end{array}
\]
\end{definition}

%\bit
%$\prob_X(B) = \prob(X\in B)$ is uniquely defined by the values it takes on the set of events 
%$$
%\{X\leq x\} = \{\omega: X(\omega(\leq x\} \text{\qquad} x\in\R
%$$ 
%\eit
%
%
%\begin{corollary}
%$\prob_X$ is uniquely determined by the \emph{distribution function} of $X$.
%\end{corollary} 

%\begin{theorem}
%Let $X:\Omega\to\R$, be a random variable on $(\Omega,\mathcal{F},\prob)$ and define the set function
%\[
%\begin{array}{rccl}
%	\prob_F:	& \mathcal{B}(\R)	& \to 		& [0,1] \\
%				& B				& \mapsto 	& \prob(X\in B).
%\end{array}
%\]
%Then $\prob_X$ is a probability measure on $(\R,\mathcal{B}(\R))$.
%\end{theorem}

%\newpage

% theorem
\begin{theorem}\label{thm:pmeas_F}
Let $F:\R\to[0,1]$ be a CDF. Then there is a unique probability measure $\prob_F:\mathcal{B}\to[0,1]$ on the real line with the property that
\begin{align*}
\prob_F\big((a,b]\big) = F(b) - F(a)
\end{align*}
for every such half-open interval $(a,b]\in\mathcal{B}$.
\end{theorem}
\proofomitted

\bit
\it The triple $(\R,\mathcal{B},\prob_F)$ is sometimes called the \emph{probability space induced by $F$}.
\eit

\begin{remark}
Compare the probability measure $\prob_F$ of the interval $(a,b]\subset\R$ to the usual measure of its \emph{length}:
\bit
\it Length: $\mathbb{L}\big((a,b]\big) = b - a$
\it Probability measure:  $\prob_F\big((a,b]\big) = F(b)-F(a)$.
\eit
%Note that if a random variable $X:\Omega\to\R$ has distribution function $F$, then
%\[
%\prob_F\big((a,b]\big) = \prob\big(\big\{\omega: a < X(\omega) \leq b\big\}\big)
%\]
Thus $\mathbb{P}_F\big( (a,b]\big)$ quantifies the ``amount of probability'' in any given interval $(a,b]$.
\end{remark}

%----------------------------------------------------------------------
%\newpage
\section{Properties of CDFs}
%----------------------------------------------------------------------

% theorem: properties cdf
\begin{theorem}\label{thm:properties_cdf}
A cumulative distribution function $F:\R\to[0,1]$ has the following properties:
\ben
\it if $x < y$ then $F(x) \leq F(y)$,
\it $F(x)\to 0$ as $x\to-\infty$,
\it $F(x)\to 1$ as $x\to+\infty$, and
\it $F(x+h)\to F(x)$ as $h\downarrow 0$ (right continuity).
\een
\end{theorem}

% proof
\begin{proof}
\ben
%----------------------------------------
\it % (i): increasing
To show that $F$ is increasing, let $x < y$ and consider the events 
\[\begin{array}{lll}
A	& = \{X\leq x\}	& = \{\omega: X(\omega)\leq x\}, \\
B	& = \{X\leq y\}	& = \{\omega: X(\omega)\leq y\}.
\end{array}\]
By construction, $F(x)=\prob(A)$ and $F(y)=\prob(B)$ so by the monotonicity of probability measures (Theorem~\ref{thm:properties_of_probability_measures}), 
\[
x< y \Rightarrow A\subseteq B \Rightarrow \prob(A) \leq \prob(B) \Rightarrow F(x) \leq F(y).
\]

%----------------------------------------
\it % (ii): F(x) -> 0 as x -> -\infty
To show that $F(x)\to 0$ as $x\to-\infty$, let
\[
B_n = \{X\leq -n\} = \{\omega:X(\omega)\in (-\infty,-n]\} \text{\quad for } n=1,2,\ldots
\]
so that $F(-n)= \prob(X\leq -n) = \prob(B_n)$. 

The sequence $B_1,B_2,\ldots$ is decreasing ($B_{n+1}\subseteq B_n$), with 
\[
\bigcap_{n=1}^{\infty} B_n = \emptyset,
\]
because for any $x$, there exists an $n$ such that $x\notin (-\infty,-n]$.
\par
By the continuity of probability measures (Theorem~\ref{thm:continuity_of_probability_measures}),
\[
\lim_{n\to\infty}F(-n) = \lim_{n\to\infty}\prob(B_n) = \prob\left(\bigcap_{n=1}^n B_n\right) = \prob(\emptyset) = 0,
\]
and because $F(x)$ is an increasing function,
\[
\lim_{n\to\infty}F(-n)=0 \text{\quad} \Leftrightarrow \text{\quad} \lim_{x\to-\infty}F(x)=0.
\]
%----------------------------------------
\it % (iii): F(x) -> 1 as x -> \infty
To show that $F(x)\to 1$ as $x\to\infty$, let
\[
A_n = \{X\leq n\} =  \{\omega:X(\omega)\in (-\infty,n]\}\text{\quad for } n=1,2,\ldots,
\]
so that $F(n)= \prob(X\leq n) = \prob(A_n)$.

The sequence $A_1,A_2,\ldots$ is increasing ($A_n\subseteq A_{n+1}$), with 
\[
\bigcup_n A_{n=1}^{\infty} = \Omega,
\]
 because for any $x$, there exists an $n$ such that $x\in (-\infty,n]$. 
 \par
 By the continuity of probability measures,
\[
\lim_{n\to\infty}F(n) = \lim_{n\to\infty}\prob(A_n) = \prob\left(\bigcup_{n=1}^{\infty} A_n\right) = \prob(\Omega) = 1,
\]
and because $F(x)$ is an increasing function,
\[
\lim_{n\to\infty}F(n)=1 \text{\quad} \Leftrightarrow \text{\quad} \lim_{x\to\infty}F(x)=1.
\]
%----------------------------------------
\it % (iv): right continuity
To show that $F(x)$ is right-continuous, let
\[
B_n=\left\{\omega: X(\omega)\in\left(-\infty,x+\frac{1}{n}\right]\right\}
\]
%so that $F\left(x+\frac{1}{n}\right) = \prob\left(X\leq x+\frac{1}{n}\right) = \prob(B_n)$.
so that $F\left(x+1/n\right) = \prob\left(X\leq x+1/n\right) = \prob(B_n)$.

The sequence $B_1,B_2,\ldots$ is decreasing ($B_{n+1}\subseteq B_n$), with 
\[
\bigcap_{n=1}^{\infty} B_n=(-\infty,x],
\]
so
\[
\prob\left(\bigcap_{n=1}^\infty B_n\right) 
	= \prob\big(\{\omega: X(\omega)\in (-\infty,x]\}\big) 
	= \prob\big(\{\omega: X(\omega)\leq x\}\big) 
	= F(x).
\]
By the continuity of probability measures,
\[
F(x) = \prob\left(\bigcap_{n=1}^{\infty} B_n\right) = \lim_{n\to\infty} \prob(B_n) = \lim_{n\to\infty}F\left(x+\frac{1}{n}\right).
\]
which concludes the proof.
\een
\end{proof}

% theorem: 
\begin{theorem}\label{thm:characterization_cdf}
Let $F:\R\to[0,1]$ be a function with properties (i)-(iv) of Theorem~\ref{thm:properties_cdf}. Then $F$ is a cumulative distribution function.
\end{theorem}
\proofomitted

% remark
\begin{remark}
The last two theorems make no explicit reference to random variables:
\bit
\it many different random variables can have the same distribution function;
\it a distribution function can represent many different random variables.
\eit
\end{remark}




%----------------------------------------------------------------------
%\newpage
\section{Discrete distributions and PMFs}
%----------------------------------------------------------------------
The \emph{range} of a random variable $X:\Omega\to\R$ is the set of all possible values it can take: 
\[
\text{Range}(X) = \{x\in\R : X(\omega)=x \text{ for some } \omega\in\Omega\}.
\]

% defn: discrete rv & pmf
\begin{definition}
\bit
\it $X:\Omega\to\R$ is called a \emph{discrete random variable} if its range is a countable subset of $\R$. 
\it A discrete random variable is described by its \emph{probability mass function} (PMF),
\[
\begin{array}{rccl}
	f:	& \R		& \to 		& [0,1] \\
		& k		& \mapsto 	& \prob (X=k),
\end{array}
\]
which must have the property that $\sum_k f(k) = 1$.
\it A probability mass function defines a \emph{discrete probability measure} on $\R$,
\[
\begin{array}{rccl}
	\prob_X:	& \mathcal{B}	& \to 		& [0,1] \\
			& B					& \mapsto 	& \displaystyle\sum_{k\in B} \prob(X=k),
\end{array}
\]
\it The triple $(\R,\mathcal{B},\prob_X)$ is called a \emph{discrete probability space} over $\R$.
\eit

\end{definition}

%----------------------------------------------------------------------
\section{Continuous distributions and PDFs}
%----------------------------------------------------------------------
% definition: continuous distributions
\begin{definition}\label{def:abs_cts_cdf}
\bit
\it
A cumulative distribution function $F:\R\to[0,1]$ is said to be \emph{absolutely continuous} if there exists an integrable function $f:\mathbb{R}\to [0,\infty)$ such that 
\[
F(x) = \int_{-\infty}^x f(t)\,dt \text{\quad for all\quad} x\in\R.
\]
\it
The function $f:\mathbb{R}\to [0,\infty)$ is called the \emph{probability density function} (PDF) of $F$.
\it 
The triple $(\R,\mathcal{B},\prob_F)$ is called a \emph{continuous probability space} over $\R$.
\eit
\end{definition}

% definition: continuous random variables
\begin{definition}
A \emph{continuous random variable} is one whose distribution function is absolutely continuous.
\end{definition}

If $X:\Omega\to\R$ is a continuous random variable, then
\bit
\it $f(x) = F'(x)$ for all $x\in\R$.
\it Probabilities correspond to areas under the curve $f(x)$: 
\[
\prob_X\big(\,(a,b]\,\big) = \prob(a< X\leq b) = F(b) - F(a) = \int_a^b f(x)\,dx.
\]
%\it The following heuristic interpretation is often useful:
%\[
%\prob\big[X\in (x,x+dx)\big] \approx f(x)\,dx,\qquad x\in\R.
%\]
\it Note that $\prob(X=x)=0$ for all $x\in\R$.
\eit

% remark: careful now
\begin{remark}
The continuity of a random variable $X:\Omega\to\R$ refers to the continuity of its distribution function, and \emph{not} to the continuity (or otherwise) of itself as a function on $\Omega$.
\end{remark}

%----------------------------------------------------------------------
\section{Exercises}
% !TEX root = main.tex
%----------------------------------------------------------------------
\begin{exercise}
\begin{questions}
%----------------------------------------
%----------------------------------------
% GS 2.1.4
\question
Let $F$ and $G$ be CDFs, and let $0<\lambda<1$ be a constant. Show that $H = \lambda F + (1-\lambda)G$ is also a CDF.
\begin{answer}
Let $H(x) = \lambda F(x) + (1-\lambda)G(x)$. It is easy to show that $H$ has the following properties:
\bit
\it if $x < y$ then $H(x) \leq H(y)$,
\it $H(x)\to 0$ as $x\to-\infty$,
\it $H(x)\to 1$ as $x\to+\infty$, and
\it $H(x+\epsilon)\to H(x)$ as $\epsilon\downarrow 0$.
\eit
Thus $H$ is a distribution function.
\end{answer}

%----------------------------------------
%%GS 2.7.9
%\question
%Let $X$ be a random variable, and let $F$ denote its CDF. Find the CDFs of the following random variables in terms of $F$:
%\begin{parts}
%\part $X^{+} = \max\{0,X\}$.
%\begin{answer}
%TODO
%\end{answer}
%\part $X^{-} = -\min\{0,X\}$.
%\begin{answer}
%TODO
%\end{answer}
%\part $|X| = X^{+} + X^{-}$.
%\begin{answer}
%TODO
%\end{answer}
%\part $-X$.
%\begin{answer}
%TODO
%\end{answer}
%----------------------------------------
\question
Let $X_1$ and $X_2$ be the numbers observed in two independent rolls of a fair die. Find the PMF of each of the following random variables:
\begin{parts}
%%--------------------
%\part $X_1$,
%\begin{answer}
%$P(X_1=k) = 1/6$ for $k=1,\ldots,6$.
%\end{answer}
%--------------------
\part $Y = 7 - X_1$,
\begin{answer}
$P(Y=k) = 1/6$ for $k=1,\ldots,6$.
\end{answer}
%--------------------
\part $U = \max(X_1,X_2)$,
\begin{answer}
Let $U=\max\{X_1,X_2\}$. Then since $\{X_1\leq k\}$ and $\{X_2\leq k\}$ are independent events,
\begin{align*}
P(U\leq k) 
	& = P(X_1\leq k \text{ and } X_2\leq k) \\
    & = P(X_1\leq k)P(X_2\leq k) \\
    & = (k/6)\cdot (k/6) = k^2/36
\end{align*}
Thus,
\[
P(U=k)  
	= P(U\leq k)-P(U\leq k-1)
    = \frac{(k^2-(k-1)^2)}{36} 
    = \frac{(2k-1)}{36}
\]    
\end{answer}
%--------------------
\part $V = X_1-X_2$.
\begin{answer}
The values of $V=X_1-X_2$ at each point of the sample space $\Omega=\{(i,j):1\leq i,j\leq 6\}$ are
\[\begin{array}{|cc|cccccc|}\hline
	& 	& & & j & & & \\
	& 	& 1	 & 2  & 3  & 4  & 5  & 6 \\\hline
	& 1 	& 0  & 1  & 2  & 3  & 4  & 5 \\
	& 2 	& -1 & 0  & 1  & 2  & 3  & 4 \\
i	& 3 	& -2 & -1 & 0  & 1  & 2  & 3 \\
	& 4 	& -3 & -2 & -1 & 0  & 1  & 2 \\
	& 5 	& -4 & -3 & -2 & -1 & 0  & 1 \\
	& 6 	& -5 & -4 & -3 & -2 & -1 & 0 \\ \hline
\end{array}\]
The required probabilities are obtained by counting the number of outcomes that give the same value of $V=X_1-X_2$:
\small
\[\begin{array}{r|rrrrrrrrrrr}
v         & -5  	& -4	 	& -3 	& -2		& -1		& 0		& 1		& 2		& 3		& 4		& 5		\\ \hline
P(V=v)    & 1/36	& 2/36	& 3/36	& 4/36	& 5/36	& 6/36	& 5/36	& 4/36	& 3/36	& 2/36	& 1/36 	\\
\end{array}\]
\normalsize
\end{answer}
%--------------------
\part $W = |X_1-X_2|$.
\begin{answer}
\[
\begin{array}{c|cccccc}
w		& 0		& 1		& 2 		& 3 		& 4 		& 5 		\\ \hline
P(W=w)	& 6/36	& 10/36	& 8/36	& 6/36	& 4/36	& 2/36 
\end{array}
\]
\end{answer}
%--------------------
\end{parts}


%----------------------------------------
\question
The PDF of a continuous random variable $X$ is given by
$
f(x) = \left\{\begin{array}{ll}
	cx^2 	& 1\leq x\leq 2,  \\
	0		& \text{otherwise.}
\end{array}\right.	
$
\begin{parts}
%--------------------
\part Find the value of the constant $c$, and sketch the PDF of $X$.
\begin{answer}
 The PDF must integrate to 1:
\[
\int_{-\infty}^{\infty}f(x)\,dx
	= \int_{1}^{2} cx^{2}\,dx 
	= \left[\frac{cx^{3}}{3} \right]_{1}^{2} 
	= \frac{7c}{3}
	 = 1
\]
so $c=3/7$. (The sketch is a quadratic curve between $x=1$ and $x=2$.)
\end{answer}
%--------------------
\part Find the value of $P(X > 3/2)$.
\begin{answer}
\[
P(X > 3/2) 	= \int_{3/2}^{2}\frac{3x^{2}}{7}\,dx 
			= \left[\frac{x^{3}}{7}\right]_{3/2}^{2} 
			= \frac{37}{56}
\]
\end{answer}
%--------------------
\part Find the CDF of $X$.
\begin{answer}
For $1\leq x\leq 2$,
\[
F(x) 	= \int_{-\infty}^{x} f(x)\,dx
		= \int_{1}^{x}\frac{3x^{2}}{7}\,dx
		= \left[ \frac{x^{3} }{7} \right] _{1}^{x} 
		= \frac{x^{3}-1}{7}
\]
so the CDF of $X$ is
\[
F(x) = \left\{\begin{array}{ll}
	0				& x < 1 \\
	\frac{1}{7}(x^{3}-1)		& 1\leq x < 2 \\
	1				& x \geq 2
\end{array}\right.	
\]	
\end{answer}
%--------------------
\end{parts}

%----------------------------------------
% GS 2.3.5(a)
\question
The PDF of a continuous random variable $X$ is given by
$
f(x) = \begin{cases}
	cx^{-d}	& \text{for } x > 1, \\
	0		& \text{otherwise.}
\end{cases}
$
\begin{parts}
%--------------------
\part Find the range of values of $d$ for which $f(x)$ is a probability density function.
\begin{answer}
The function $f(x)=cx^{-d}$ is only integrable if $d>1$, in which case
\[
\int_{-\infty}^\infty f(x)\,dx = \int_1^\infty \frac{c}{x^d}\,dx = \left[\frac{-c}{(d-1)x^{d-1}}\right]_1^{\infty} = \frac{c}{d-1}
\]
\end{answer}
%--------------------
\part If $f(x)$ is a density function, find the value of $c$, and the corresponding CDF.
\begin{answer}
If $f(x)$ is a probability density function, we require that $\int_{-\infty}^\infty f(x)\,dx = 1$, so we must have that $c = d-1$. The corresponding distribution function is
\[
F(x) = \int_{-\infty}^x f(u)\,du  
	= \int_1^\infty \frac{d-1}{u^d}\,du 
	= \left[\frac{-1}{x^{d-1}}\right]_1^x
	= 1 - \frac{1}{x^{d-1}}
\]
for $x>1$, and zero otherwise.
\end{answer}
%--------------------
\end{parts}

%----------------------------------------
% GS 2.3.5(b)
\question
Let $\displaystyle f(x) = \frac{ce^x}{(1+e^x)^2}$ be a PDF, where $c$ is a constant. Find the value of $c$, and the corresponding CDF.
\begin{answer}
By inspection, $f(x) = F'(x)$ where $F(x) = \frac{ce^x}{1+e^x}$. Writing this as $F(x) = \frac{c}{e^{-x}+1}$ it is easy to see that $F(x)\to c$ as $x\to\infty$, so we must have that $c=1$.
\end{answer}

%----------------------------------------
% GS 2.2.3
\question
Let $X_1,X_2,\ldots$ be independent and identically distributed observations, and let $F$ denote their common CDF. If $F$ is unknown, describe and justify a way of estimating $F$, based on the observations. [Hint: consider the indicator variables of the events $\{X_j\leq x\}$.]
\begin{answer}
Let $X$ be a random variable with same CDF, and let $I_j(x)$ the indicator variable of the event $\{X_j\leq x\}$. Then
\[
\prob(X\leq x) \approx \frac{1}{n}\sum_{j=1}^{n} I_j(x).
\]
The RHS yields the proportion of observations that are at most equal to $x$.
\end{answer}

%----------------------------------------
\end{questions}
\end{exercise}
%----------------------------------------------------------------------

%======================================================================
\endinput
%======================================================================

%----------------------------------------------------------------------

%======================================================================
\endinput
%======================================================================
