% !TEX root = main.tex
%----------------------------------------------------------------------
\chapter{Joint Distributions}\label{chap:joint}
%----------------------------------------------------------------------
%----------------------------------------------------------------------
\section{Joint distributions}
%----------------------------------------------------------------------

% defn: joint distributions
\begin{definition}
Let $X,Y:\Omega\to\R$ be random variables on the probability space $(\Omega,\mathcal{F},\prob)$.
\ben
\it % (i)
The \emph{joint distribution} of $X$ and $Y$ is the function
\[
\begin{array}{cccc}
\prob_{X,Y}:		& \mathcal{B}^2	& \to		& [0,1] \\
				& (A,B)			& \mapsto	& \prob(X\in A,Y\in B).
\end{array}
\]
\it The \emph{joint CDF} of $X$ and $Y$ is 
\[
\begin{array}{cccc}
F_{X,Y}:		& \R^2	& \to		& [0,1] \\
			& (x,y)	& \mapsto	& \prob(X\leq x,Y\leq y).
\end{array}
\]
\it 
The \emph{marginal CDF} of $X$ is the function
\[
\begin{array}{cccc}
F_X:	& \mathbb{R} 	& \to		& [0,1] \\[1ex]
	& x				& \mapsto	& \prob(X\leq x),
\end{array}
\]
and the marginal CDF of $Y$ is
\[
\begin{array}{cccc}
F_Y:	& \mathbb{R}		& \to		& [0,1] \\[1ex]
	& y				& \mapsto	& \prob(Y\leq y).
\end{array}
\]

%\it The \emph{marginal CDFs} of $X$ and $Y$ are the functions $F_X(x) = \prob(X\leq x)$ and $F_Y(y) = \prob(Y\leq y)$ respectively.
%\[
%F_X(x) = \prob(X\leq x) 
%\text{\quad and\quad}
%F_Y(y)	= \prob(Y\leq y)
%\text{\quad respectively.}
%\]
%\begin{align*}
%F_X(x)	& = \prob(X\leq x), \text{ and} \\
%F_Y(y)	& = \prob(Y\leq y).
%\end{align*}
%respectively.
\een
\end{definition}


%% defn: joint distribution function
%\begin{definition}
%Let $X,Y:\Omega\to\R$ be random variables on the probability space $(\Omega,\mathcal{F},\prob)$.
%\ben
%\it The \emph{joint cumulative distribution function} of $X$ and $Y$ is defined by
%\[
%\begin{array}{cccl}
%F_{X,Y}:	& \R^2	& \longrightarrow	& [0,1] \\
%	& (x,y) 	& \mapsto			& \prob(X\leq x, Y\leq y)
%\end{array}
%\]
%i.e.
%\[
%F_{X,Y}(x,y) = \prob(X\leq x, Y\leq y) = \prob\big(\{\omega: X(\omega)\leq x, Y(\omega)\leq y\}\big)
%\]
%
%\it The \emph{marginal distribution functions} of $X$ and $Y$ are
%\begin{align*}
%F_X(x)	& = \prob(X\leq x) = \prob(\{\omega\in\Omega : X(\omega)\leq x\}), \text{ and} \\
%F_Y(y)	& = \prob(Y\leq y) = \prob(\{\omega\in\Omega : Y(\omega)\leq y\})
%\end{align*}
%respectively.
%\een
%\end{definition}

%\begin{remark}
%For clarity, we sometimes write $F_{X,Y}$ to denote the joint distribution of $X$ and $Y$.
%\end{remark}

%----------------------------------------------------------------------
\section{Properties of Joint CDFs}
%----------------------------------------------------------------------

% theorem: properties of joint cdfs
\begin{theorem}\label{thm:properties_jointcdf}
Let $F:\R^2\to [0,1]$ be a joint CDF.
\ben
\it Limiting behaviour:
\[\begin{array}{lll}
\lim_{x\to-\infty}F(x,y)=0, &
\lim_{y\to-\infty}F(x,y)=0, &
\lim_{\substack{x\to-\infty\\y\to-\infty}}F(x,y)=0, \\
\lim_{x\to+\infty}F(x,y)=F_Y(y), & 
\lim_{y\to+\infty}F(x,y)=F_X(x), &
\lim_{\substack{x\to+\infty\\y\to+\infty}} F(x,y)=1.
\end{array}\]

\it 	Monotonicity: 
\[
F(x,y)\leq F(x+u,y+v)\quad\text{for all $u,v\geq 0$.}
\]

\it Inclusion-exclusion:
\[
\prob\big(a< X\leq b,\ c < Y\leq d\big) = F(b,d) - F(a,d) - F(b,c) + F(a,c).
\]

\it 	Upper continuity: 
\[
F(x+u,y+v) \longrightarrow F(x,y) \text{\quad as\quad} u\downarrow 0\text{ and }v\downarrow 0,
\]
where $u\downarrow 0$ means that $u$ converges to zero through positive values (a.k.a. ``from above'').
\een
\end{theorem}

\proofomitted

%\begin{proof}
%\ben
%\it % <<<< (i)
%Let $B_n	= \{\omega: X(\omega)\leq -n, Y(\omega)\leq y\}.$ Then $B_1,B_2,\ldots$ is a decreasing sequence of events, with 
%\[
%\bigcaf_{n=1}^{\infty}B_n=\emptyset.
%\]
%Hence by the continuity of probability measures,
%\[
%\lim_{n\to\infty}F(-n,y) = \lim_{n\to\infty}\prob(B_n) 
%	= \prob\left(\bigcaf_{n=1}^{\infty} B_n\right) = \prob(\emptyset) = 0.
%\]
%so $F(-n,y)\to \prob(\emptyset) = 0$ as $n\to\infty$.
%\par
%Similarly, we can show that $F(x,-n)\to 0$ as $n\to\infty$ using an identical argument applied to the sets $B_n = \{\omega: X(\omega)\leq x, Y(\omega)\leq -n\}.$ Furthermore,
%\[
%\lim_{\substack{x\to-\infty\\y\to-\infty}} F(x,y)
%	= \lim_{x\to-\infty}\left(\lim_{y\to-\infty} F(x,y)\right)
%	= \lim_{x\to-\infty} 0
%	= 0.
%\]% <<< (id)
%Let $A_n = \{\omega: X(\omega)\leq n, Y(\omega)\leq y\}.$ Then $A_1,A_2,\ldots$ is an increasing sequence of events, with 
%\[
%\bigcuf_{n=1}^{\infty}A_n=\{\omega:Y(\omega)\leq y\}.
%\]
%Hence by the continuity of probability measures,
%\[
%F_Y(y) = \prob(Y\leq y) = \prob\big(\cuf_{n=1}^{\infty}A_n\big) = \lim_{n\to\infty} \prob(A_n) = \lim_{n\to\infty}F(n,y).
%\]
%% <<<< (ie)
%Using an identical argument applied to the sets $A_n = \{\omega: X(\omega)\leq x, Y(\omega)\leq n\},$
%\[
%F_X(x) = \lim_{n\to\infty}F(x,n),
%\]
%and moreover,
%\[
%\lim_{\substack{x\to\infty\\y\to\infty}} F(x,y)
%	= \lim_{x\to\infty}\left(\lim_{y\to\infty} F(x,y)\right)
%	= \lim_{x\to\infty} F_X(x)
%	= 1.
%\]
%
%
%
%\it % <<<< (iii): monotonicity
%Monotonicity. Let $u,v>0$ and consider the sets
%\[
%A = \{X\leq x, Y\leq y\} \text{\quad and\quad} B = \{X\leq x+u, Y\leq y+v\}.
%\]
%
%Since $A\subseteq B$, by the monotonicity property of probability measures we have
%\[
%F(x,y) = \prob(A) \leq \prob(B) = F(x+u,y+v).
%\] 
%
%\it % <<<< (iv): inclusion-exclusion
%Inclusion-exclusion. Let $a< b$ and $c< d$, and consider the events
%%\begin{align*}
%%\prob\big(a< X\leq b,\ c < Y\leq d\big)
%%	& = \prob(X\leq b,Y\leq d) - \prob(X\leq a,Y\leq d) \\
%%	& \text{\qquad\qquad} - \prob(X\leq b,Y\leq c) + \prob(X\leq a,Y\leq c) \\
%%	& = F(b,d)-F(a,d)-F(b,c)+F(a,c).
%%\end{align*}
%\[
%A = \{X\leq b,\ c < Y\leq d\} \text{\quad and\quad} B = \{a < X\leq b,\ Y\leq d\}.
%\]
%%so that $\prob(A) = F(b,d) - F(b,c)$ and $\prob(B) = F(b,d) - F(a,d)$, and
%Then
%\[
%\begin{array}{rlrl}
%\prob(A) & = F(b,d) - F(b,c),\quad  & \prob(A\cup B) & = F(b,d) - F(a,c), \\
%\prob(B) & = F(b,d) - F(a,d), 		& 
%\end{array}
%\]
%%\bit
%%\it $\prob(A\cup B) = F(b,d)$ and 
%%%\it $\prob(A) = F(b,d) - F(b,c)$,
%%%\it $\prob(B) = F(b,d) - F(a,d)$, and
%%\prob(A\cap B) = F(b,d) - F(a,c)$.
%%\eit
%Hence
%\begin{align*}
%\prob\big(a< X\leq b,\ c < Y\leq d\big)
%	& = \prob(A\cap B) \\
%	& = \prob(A) + \prob(B) - \prob(A\cup B) \\
%	& = F(b,d)-F(a,d)-F(b,c)+F(a,c).
%\end{align*}
%% picture
%%\picbox[Inclusion-Exclusion]{1cm}
%\it % <<<< (iv): continuity from above
%Continuity from above. Consider the events
%\[
%B_n = \left\{(X,Y)\in \left(-\infty,x+\frac{1}{n}\right]\times \left(-\infty,y+\frac{1}{n}\right] \right\}.
%\]
%By construction, 
%\[
%\prob(B_n) 
%	= F_{X,Y}\left(x+\frac{1}{n},y+\frac{1}{n}\right)
%	= \prob\left(X\leq x+\frac{1}{n}, Y\leq y+\frac{1}{n}\right).
%\]
%$B_1,B_2,\ldots$ is a decreasing sequence of events, with 
%\[
%\bigcaf_{n=1}^{\infty} B_n = \{X\leq x,Y\leq y\}
%\]
%Hence, by the continuity of probability measures,
%\[
%\lim_{n\to\infty}F_{X,Y}\left(x+\frac{1}{n},y+\frac{1}{n}\right)
%	= \lim_{n\to\infty} \prob(B_n)
%	= \prob\left(\bigcaf_{n=1}^{\infty} B_n\right)
%	= F_{X,Y}(x,y).
%\]
%\een
%\end{proof}

%----------------------------------------------------------------------
\section{Independent random variables}
%----------------------------------------------------------------------
Recall that two events $A$ and $B$ are called \emph{independent} if $\prob(A\cap B)=\prob(A)\prob(B)$.

% definition
\begin{definition}
Two random variables $X,Y:\Omega\to\R$ defined on a probability space $(\Omega,\mathcal{F},\prob)$ are said to be \emph{independent} if the events 
\begin{align*}
\{X\leq x\} & \equiv \{\omega\,:\, X(\omega)\leq x\} \\
\{Y\leq y\} & \equiv \{\omega\,:\, Y(\omega)\leq y\}
\end{align*}
are independent for all $x,y\in\R$.
\end{definition}

%Thus two random variables are independent if and only if 
%\[
%\prob(X\leq x, Y\leq y) = \prob(X\leq x)\prob(Y\leq y) \text{\quad for all\quad} x,y\in\R.
%\]

%

% theorem: indepencence <=> joint df = product of marginal dfs
The following lemma is easily proved.
% theorem
\begin{lemma}\label{lem:product_marginal_cdfs}
Let $X$ and $Y$ be random variables with joint CDF $F_{X,Y}(x,y)$ and marginal CDFs $F_X(x)$ and $F_Y(y)$ respectively. Then $X$ and $Y$ are independent if and only if
\[
F_{X,Y}(x,y) = F_X(x)F_Y(y) \text{\quad for all\quad} x,y\in\R.
\]
\end{lemma}

%----------------------------------------------------------------------

\section{Identically distributed random variables}
%----------------------------------------------------------------------
% definition
\begin{definition}
Two random variables $X,Y:\Omega\to\R$ defined on a probability space $(\Omega,\mathcal{F},\prob)$ are said to be \emph{identically distributed} if $\prob_X = \prob_Y$, or equivalently $F_X = F_Y$.
\end{definition}

Thus $X$ and $Y$ are identically distributed if and only if 
\bit
\it $\prob_X(B) \equiv \prob(X\in B) = \prob(Y\in B) \equiv \prob_Y(B)$ for all $B\in\mathcal{B}$, or equivalently
\it $F_X(t) \equiv \prob(X\leq t) = \prob(Y\leq t) \equiv F_Y(t)$ for all $t\in\R$.
\eit

%%----------------------------------------------------------------------
%
%\section{Conditional distributions}
%%----------------------------------------------------------------------
%Let $X,Y:\Omega\to\R$ be random variables on a probability space $(\Omega,\mathcal{F},\prob)$.
%
%%% defn: joint probability distribution
%%\begin{definition}
%%\ben
%%\it % (i)
%%The \emph{joint probability distribution} of $X$ and $Y$ is the set function
%%\[
%%\begin{array}{cccl}
%%\prob_{X,Y}:		& \mathcal{F}^2	& \to		& [0,1] \\
%%				& (A,B)			& \mapsto	& \prob(X\in A, Y\in B),
%%\end{array}
%%\]
%%\it 
%%The \emph{joint cumulative distribution function} of $X$ and $Y$ is the function  
%%\[
%%\begin{array}{cccl}
%%F_{X,Y}:		& \R^2	& \to		& [0,1] \\
%%			& (x,y)			& \mapsto	& \prob(X\leq x,Y\leq y).
%%\end{array}
%%\]
%%\een
%%\end{definition}
%%
%%
%
%% defn: conditional distributions
%\begin{definition}
%\ben
%\it % (i)
%The \emph{conditional distribution} of $Y$ given $X$ is the set function
%\[
%\begin{array}{cccl}
%\prob_{Y|X}:		& \mathcal{B}^2	& \to		& [0,1] \\
%				& (A,B)			& \mapsto	& \prob(Y\in B\,|\, X\in A).
%\end{array}
%\]
%\it The \emph{conditional cumulative distribution function} of $Y$ given $X$ is the function
%\[
%\begin{array}{cccl}
%F_{Y|X}:		& \R^2	& \to		& [0,1] \\
%			& (x,y)	& \mapsto	& \prob(Y\leq y\,|\, X\leq x).
%\end{array}
%\]
%\een
%\end{definition}
%
%
%
%% proposition: quotient
%\begin{proposition} 
%The conditional CDF of $Y$ given $X$ satisfies
%\[
%F_{Y|X}(x,y) = \frac{F_{X,Y}(x,y)}{F_X(x)},
%\]
%where $F_X$ is the marginal CDF of $X$, and $F_{X,Y}$ is the joint CDF of $X$ and $Y$.
%
%\vspace*{2ex}
%\begin{proof}
%\[
%F_{Y|X}(x,y) 
%	= \prob(Y\leq y\,|\,X\leq x)
%	= \frac{\prob(\{Y\leq y\}\cap\{X\leq x\})}{\prob(X\leq x)}
%	= \frac{F_{X,Y}(x,y)}{F_X(x)}.
%\]
%\end{proof}
%\end{proposition}
%

%----------------------------------------------------------------------

\section{Jointly discrete distributions}
%----------------------------------------------------------------------
% definition: discrete random vectors
\begin{definition}
\ben
\it Two random variables $X$ and $Y$ are called \emph{jointly discrete} if the random vector $(X,Y)$ only takes values in a countable subset of $\R^2$. 
\it Two jointly discrete random variables $X$ and $Y$ are described by their \emph{joint PMF}: 
\[
\begin{array}{cccl}
f_{X,Y}:		& \R^2	& \to		& [0,1] \\
			& (x,y)	& \mapsto	& \prob(X=x,Y=y).
\end{array}
\]
%\it Their \emph{marginal PMFs} are $f_X(x)=\prob(X=x)$ and $f_Y(y)=\prob(Y=y)$ respectively.
\it The \emph{marginal PMF of $X$} is the function $f_X(x)=\prob(X=x)$.
\it The \emph{marginal PMF of $Y$} is the function $f_Y(y)=\prob(Y=y)$.
\een
\end{definition}

% example: dice
\begin{example}\label{ex:joint:dice}
A fair die is rolled once. Let $\omega$ denote the outcome, and consider the random variables
\[
X(\omega) = \left\{\begin{array}{cl}
	1 & \text{ if $\omega$ is odd}, \\
	2 & \text{ if $\omega$ is even},
\end{array}\right. 
\text{\quad and\quad}
Y(\omega) = \left\{\begin{array}{cl}
	1 & \text{ if $\omega\leq 3$}, \\
	2 & \text{ if $\omega\geq 4$}.
\end{array}\right.
\]
Find the joint PMF of $X$ and $Y$.
\end{example}

\begin{solution}
\begin{center}
\begin{tabular}{c|cccccc}
$\omega$ 	& $1$ & $2$ & $3$ & $4$ & $5$ & $6$ \\ \hline
$X(\omega)$ 	& $1$ & $2$ & $1$ & $2$ & $1$ & $2$ \\
$Y(\omega)$ 	& $1$ & $1$ & $1$ & $2$ & $2$ & $2$
\end{tabular}
\end{center}

The joint PMF of $X$ and $Y$ is shown in the following table
\begin{center}
\begin{tabular}{c|cc}
		& Y=1 	& Y=2	\\ \hline
X=1		& 1/3	& 1/6	\\ 
X=2		& 1/6 	& 1/3	
\end{tabular}
\end{center}
The marginal PMFs of $X$ and $Y$ are recovered by summing across the rows and columns.
\end{solution}



%% conditional CDF and PMF
%\begin{definition}
%Let $X$ and $Y$ be two jointly discrete random variables, and let $x$ be such that $\prob(X=x)>0$.
%\ben
%\it The \emph{conditional CDF} of $Y$ given $X=x$ is 
%\[
%F_{Y|X}(y|x) = \prob(Y\leq y\,|\,X=x).
%\]
%\it The \emph{conditional PMF} of $Y$ given $X=x$ is 
%\[
%f_{Y|X}(y|x) = \prob(Y=y\,|\,X=x).
%\]
%\een
%\end{definition}
%
%% lemma
%\begin{lemma}
%The conditional PMF of $Y$ given $X=x$ satisfies
%\[
%f_{Y|X}(y|x) = \frac{f_{X,Y}(x,y)}{f_X(x)}.
%\]
%\end{lemma}
%
%% proof
%\begin{proof}
%\[
%f_{Y|X}(y|x) 
%	= \prob(Y=y\,|\,X=x) 
%	= \frac{\prob(\{Y=y\}\cap\{X=x\})}{\prob(X=x)}
%	= \frac{f_{X,Y}(x,y)}{f_X(x)}.
%\]
%\end{proof}
%
%% theorem: law of total probability
%\begin{theorem}[The law of total probability]
%If $X$ and $Y$ are jointly discrete random variables, then $\displaystyle f_Y(y) = \sum_x f_{Y|X}(y|x) f_X(x).$
%\end{theorem}
%\begin{proof}
%\[
%\sum_x f_{Y|X}(y|x) f_X(x) 
%	= \sum_x \left(\frac{f_{X,Y}(x,y)}{f_X(x)}\right) f_X(x)
%	= \sum_x f_{X,Y}(x,y)
%	= f_Y(y).
%\]
%\end{proof}
%

%%----------------------------------------------------------------------
%%
%\subsection{Independence}
%%----------------------------------------------------------------------
%% lemma: independence for discrete RVs
%\begin{lemma}[Independence]
%Two jointly discrete random variables $X$ and $Y$ are independent if and only if 
%\[
%f_{Y|X}(y|x) = f_Y(y)\quad\text{for all}\quad x,y\in\R.
%\]
%\end{lemma}
%
%% proof
%\begin{proof}
%If $X$ and $Y$ are independent then for all $x,y\in\R$,
%\[
%\prob(Y=y\,|\,X=x) 
%	= \frac{\prob(X=x,Y=y)}{\prob(X=x)} 
%	= \frac{\prob(X=x)\prob(Y=y)}{\prob(X=x)} 
%	= \prob(Y=y).
%\]
%\end{proof}

%%----------------------------------------------------------------------
%%
%\subsection{Independence}
%%----------------------------------------------------------------------
%% lemma: independence for discrete RVs
The following lemma is easily proved.
\begin{lemma}
Two jointly discrete random variables $X$ and $Y$ are independent if and only if 
\[
f_{X,Y}(x,y) =f_X(x) f_Y(y)\quad\text{for all}\quad x,y\in\R.
\]
\end{lemma}
%
%% proof
%\begin{proof}
%If $X$ and $Y$ are independent then for all $x,y\in\R$,
%\[
%\prob(Y=y\,|\,X=x) 
%	= \frac{\prob(X=x,Y=y)}{\prob(X=x)} 
%	= \frac{\prob(X=x)\prob(Y=y)}{\prob(X=x)} 
%	= \prob(Y=y).
%\]
%\end{proof}

%----------------------------------------------------------------------
\section{Jointly continuous distributions}
%----------------------------------------------------------------------
\begin{definition}
\ben
\it
Two random variables $X$ and $Y$ are called \emph{jointly continuous} if their joint CDF can be written as
\[
F(x,y) = \int_{-\infty}^x\int_{-\infty}^y f(u,v)\,du\,dv \text{\qquad} x,y\in\R
\]
for some integrable function $f_{X,Y}:\R^2\to[0,\infty)$ called the \emph{joint PDF} of $X$ and $Y$.
\it
The \emph{marginal PDFs} of $X$ and $Y$ are defined by $f_X(x)=F'_X(x)$ and $f_Y(y)=F'_Y(y)$ respectively, where $F_X(x)$ and $F_Y(y)$ are the marginal CDFs of $X$ and $Y$ respectively.
\een
\end{definition}

% example: darts
\begin{example}
A dart is thrown at a circular dartboard of radius $\rho$. The point at which the dart hits the board determines a distance $R$ from the centre, and an angle $\Theta$ with (say) the upward vertical. Assume that the dart does in fact hit the board, and that regions of equal area are equally likely to be hit. Show that $R$ and $\Theta$ are jointly continuous random variables.
\end{example}
\begin{solution}
\[
\begin{array}{lll}
\prob(R\leq r) 			& = \displaystyle \frac{\pi r^2}{\pi\rho^2} = \frac{r^2}{\rho^2} 
						& \text{ for } 0\leq r\leq \rho. \\[2ex]
\prob(\Theta\leq \theta)	& = \displaystyle \frac{\theta}{2\pi} 									
						& \text{ for } 0\leq \theta\leq 2\pi.
\end{array}
\]
Since regions of equal area are equally likely to be hit, the joint distribution function of $(R,\Theta)$ satisfies
\[
F_{R,\Theta}(r,\theta) = \prob(R\leq r,\Theta\leq\theta) = \prob(R\leq r)\prob(\Theta\leq\theta) = \frac{\theta r^2}{2\pi\rho^2}
\]
Thus we have
\[
F_{R,\Theta}(r,\theta) = \int_0^r\int_0^\theta f(u,v)\,du\,dv
\]
where
\[
f(u,v) =\frac{u}{\pi\rho^2}.
\]
Hence $R$ and $\Theta$ are jointly continuous random variables.
\end{solution}



% conditional CDF and PDF
%Let $X$ and $Y$ be jointly continuous random variables.
%\bit
%\it Suppose we observe that $X$ takes the value $x\in\R$.
%\it Since $\prob(X=x)=0$, we cannot condition on the event $\{X=x\}$.
%\eit

%% definition: conditional density function
%\begin{definition}\label{def:conditional_density}
%Let $X$ and $Y$ be jointly continuous random variables. The \emph{conditional density function} of $Y$ given $X=x$ is 
%\[
%f_{Y|X}(y|x) = \frac{f_{X,Y}(x,y)}{f_X(x)}
%\]
%\end{definition}
%
%%
%
%% theorem: LTP
%\begin{theorem}[The law of total probability]
%If $X$ and $Y$ are jointly continuous random variables,
%\[
%f_Y(y) = \int f_{Y|X}(y|x) f_X(x)\,dx
%\]
%\end{theorem}
%
%% proof
%\begin{proof}
%\[
%\int f_{Y|X}(y|x) f_X(x)\,dx 
%	= \int \left(\frac{f_{X,Y}(x,y)}{f_X(x)}\right) f_X(x)\,dx
%	= \int f_{X,Y}(x,y) f_X(x)\,dx
%	= f_Y(y)
%\]
%\end{proof}

%%----------------------------------------------------------------------
%
%\subsection{Support}
%%----------------------------------------------------------------------
%% defn
%\begin{definition}
%\ben
%\it % closed set
%A set $A\subset\R$ is said to be \emph{closed} if it contains all its limit points, i.e. if $x_n\to x$ as $n\to\infty$, where each $x_n\in A$, then $x\in A$.
%\it % support of function
%The \emph{support} of a function $h:\R\to\R$, denoted by $\supp(h)$, is the smallest closed set for which $h(x)=0$ for all $x\notin\supp(h)$.
%\it % support of rv
%The \emph{support} of a random variable $X:\Omega\to\R$ is defined to be the support of its PMF (discrete case) or PDF (continuous case).
%\it % support of pair
%The \emph{support} of a pair of random variable $X,Y:\Omega\to\R$ is defied to be the support of their joint PMF (discrete case) or joint PDF (continuous case).
%\een
%\end{definition}
%
%% remark
%\begin{remark}
%When we refer to the support of a random variable, we are referring to the support of its PMF (discrete case) or PDF (continuous case), \emph{not} to the support of its CDF, not the support of itself as a function over $\Omega$.
%\end{remark}

%----------------------------------------------------------------------

\subsection{Independence}
%----------------------------------------------------------------------
% lemma: independence for CRVs
\begin{lemma}
Two jointly continuous random variables $X$ and $Y$ are independent if and only if 
\ben
\it $f_{X,Y}(x,y) = f_X(x)f_Y(y)$ for all $x,y\in\R$, and
\it the support of $f_{X,Y}$ is a rectangular region in $\R^2$.
\een
\end{lemma}

% proof
\begin{proof}
Since $X$ and $Y$ are jointly continuous, 
\begin{align*}
F_{X,Y}(x,y)		& = \int_{-\infty}^x\int_{-\infty}^y f_{X,Y}(u,v)\,du\,dv \\
F_X(x)F_Y(y)		& = \left(\int_{-\infty}^x f_X(u)\,du\right)\left(\int_{-\infty}^y f_Y(v)\,dv\right)
\end{align*}
\bit
\it By independence, $F_{X,Y}(x,y)=F_X(x)F_Y(y)$, so the two integrals are equal.
\it Differentiating both sides with respect to $x$ and $y$, we get $f_{X,Y}(x,y)=f_X(x)f_Y(y)$.
\eit
\end{proof}

% remark: rectangles
\begin{remark}
If the value taken by $X$ affects the range of values taken by $Y$, then clearly $Y$ dependes on $X$. Hence if $X$ and $Y$ are independent, we need that $\supp(f_{X,Y})$ can be expressed as the Cartesian product of two sets in $\R$:
\[
\supp(f_{X,Y}) = \supp(f_X)\times \supp(f_Y) \text{\qquad where\quad} \supp(f_X),\supp(f_Y)\subseteq\R.
\]
For example:
\bit
\it The unit square is fine:	$\supp(f_{X,Y})=\{(x,y)\,:\, 0\leq x,y\leq 1\} = [0,1]\times[0,1]$.
\it The unit disc is not:	$\supp(f_{X,Y})=\{(x,y)\,:\,x^2+y^2\leq 1\}$.
\eit
\end{remark}


%----------------------------------------------------------------------

%----------------------------------------------------------------------
\begin{example}
Two jointly continuous random variables $X$ and $Y$ have joint PDF
\[
f_{X,Y}(x,y) = 
\left\{\begin{array}{ll}
	c(1-x)y	& 0\leq y \leq x \leq 1, \\
	0		& \text{otherwise.}
\end{array}\right.
\]
\ben
\it Show that $c=24$.
\it Find the marginal PDFs of $X$ and $Y$.
%\it Find the conditional density function of $Y$ given $X=x$. 
\een
\end{example}

\begin{solution}
\ben

\it % << (a)
\bit
\it For fixed $x\in[0,1]$, we must have $y\in[0,x]$.
\it For fixed $y\in[0,1]$, we must have $x\in[y,1]$.
\eit
\[
\int_{-\infty}^{\infty}\int_{-\infty}^{\infty}f_{X,Y}(x,y)\,dx\,dy
	= \int_0^1\int_0^x c(1-x)y\,dydx 
%	= c\int_0^1\left[\frac{(1-x)y^2}{2}\right]_0^x\,dx
	= \frac{c}{2}\int_0^1 x^2-x^3\,dx
%	= \frac{c}{2}\left[\frac{x^3}{3}-\frac{x^4}{4}\right]_0^1
	= \frac{c}{2}\left(\frac{1}{3}-\frac{1}{4}\right)
	= \frac{c}{24}.
\]
By the law of total probability this must equal $1$, so $c=24$.

\it % << (b)
The marginal density functions are:
\begin{align*}
f_X(x) 	
	& = 24\int_0^x (1-x)y\,dy 
	= 24(1-x)\left[\frac{y^2}{2}\right]_0^x
	= \begin{cases}
		12x^2(1-x) & 0\leq x\leq 1, \\
		0			& \text{otherwise.}
		\end{cases} \\
f_Y(y) 	
	& = 24\int_y^1 (1-x)y\,dx 
	= 24y\left[x - \frac{x^2}{2}\right]_y^1
	= \begin{cases}
		12y(1-y)^2 & 0\leq y\leq 1, \\
		0			& \text{otherwise.}
		\end{cases}
\end{align*}

%\it % << (c)
%The conditional density function of $Y$ given $X=x$ is
%\[
%f_{Y|X}(y\,|\,x) 	
%	= \frac{f_{X,Y}(x,y)}{f_X(x)} 
%%	= \frac{24(1-x)y}{12(1-x)x^2} 	
%	= \left\{\begin{array}{ll} 
%		\displaystyle\frac{24(1-x)y}{12(1-x)x^2} = \displaystyle\frac{2y}{x^2} 	& \text{if }y\leq x\leq 1, \\[2ex]
%	 	0 							& \text{otherwise.}
%	 \end{array}\right.
%\]	 
\een
\end{solution}

%----------------------------------------------------------------------

%----------------------------------------------------------------------
% example
\begin{example}
The continuous random variables $X$ and $Y$ have joint PDF
\[
f_{X,Y}(x,y) = 
\left\{\begin{array}{ll}
	cxy^2	& 0< x,y< 1 \\
	0		& \text{otherwise}
\end{array}\right.
\]
where $c$ is a constant.
\ben
\it Show that $c=6$.
\it Find the marginal PDFs of $X$ and $Y$. Are $X$ and $Y$ independent?
%\it Find the conditional PDF of $X$ given $Y=y$.
%\it Find the conditional PDF of $Y$ given $X=x$.
\it Show that $\prob(X+Y\geq 1)=9/10$.
\een

\end{example}
\begin{solution}

\ben
\it % << (a)
By the law of total probability, we must have that 
\[
\int_{-\infty}^{\infty}\int_{-\infty}^{\infty}f_{X,Y}(x,y)\,dx\,dy = \int_0^1\int_0^1 cxy^2\,dx\,dy = 1.
\]
Evaluating the integral, 
\[
\int_0^1\int_0^1 cxy^2\,dx\,dy 
	= c\int_0^1 \left[\frac{x^2y^2}{2}\right]_0^1\,dy
	= c\int_0^1 \frac{y^2}{2}\,dy
	= c\left[\frac{y^3}{6}\right]_0^1
	= \frac{c}{6}
\]
Thus we have $c=6$, so $f_{X,Y}(x,y)=6xy^2$ over $0\leq x,y\leq 1$ (and zero otherwise).

\it % << (b)
The marginal densities are as follows:
\[ 
f_X(x) 	= \int_0^1 f_{X,Y}(x,y)\,dy = \int_0^1 6xy^2\,dy = 6x\left[\frac{y^3}{3}\right]_0^1 		
	= \left\{\begin{array}{ll} 
		2x	& 0< x< 1, \\ 
		0 	& \text{otherwise.}
	\end{array}\right.
\]
\[
f_Y(y) 	= \int_0^1 f_{X,Y}(x,y)\,dx = \int_0^1 6xy^2\,dx = 6y^2\left[\frac{x^2}{2}\right]_0^1	
	= \left\{\begin{array}{ll} 
		3y^2		& 0< y< 1, \\ 
		0 		& \text{otherwise.}
	\end{array}\right.
\]
$X$ and $Y$ are independent because the support of $f_{X,Y}(x,y)$ is a rectangular region, and
\[
f_{X,Y}(x,y) = 6xy^2 = (2x)(3y^2) = f_X(x)f_Y(y) \text{\quad for all}\quad x,y\in\R.
\]

%\it % << (c)
%The conditional density of $X$ given $Y=y$ is
%\[
%f_{X|Y}(x|y) = \frac{f_{X,Y}(x,y)}{f_Y(y)} = \frac{6xy^2}{3y^2} = \left\{\begin{array}{ll} 
%	2x 	& 0< x< 1 \\ 
%	0 	& \text{ otherwise.}
%\end{array}\right.
%\]
%\it % << (c)
%The conditional density of $Y$ given $X=x$ is
%\[
%f_{Y|X}(y|x) = \frac{f_{X,Y}(x,y)}{f_X(x)} = \frac{6xy^2}{2x} 	= \left\{\begin{array}{ll} 
%	3y^2 	& 0< y< 1 \\ 
%	0 		& \text{ otherwise.}
%\end{array}\right.
%\]

\it % << (d)
The event $\{X+Y\geq 1\}$ is the same as $\{X\geq 1-Y\}$. For a fixed value $Y=y$,
\begin{align*}
\prob(X+y\geq 1)
	& = \prob (X\geq 1-y)  
	= \int_{1-y}^1 f_X(x)\,dx 
	= \int_{1-y}^1 2x\,dx 
%	& = \left[x^2\right]_{1-y}^1 \\
	= 1-(1-y)^2.
\end{align*}
Hence,
\begin{align*}
P(X+Y\geq 1) 
	& = \int_0^1 \prob(X+y\geq 1)f_Y(y)\,dy \\
	& = \int_0^1 3y^2\big(1-(1-y)^2\big)\,dy 
	= \int_0^1 6y^3 - 3y^4\,dy 
	= \left[\frac{6y^4}{4}-\frac{3y^5}{5}\right]_0^1 
	= \frac{9}{10}.
\end{align*}

\een
\end{solution}

%----------------------------------------------------------------------
\section{Exercises}
% !TEX root = main.tex
%----------------------------------------------------------------------
\begin{exercise}
\begin{questions}
%----------------------------------------
%----------------------------------------
% GS 2.5.2
\question
Let $X$ be a Bernoulli random variable with parameter $p$.% $\prob(X=0)=1-p$ and $\prob(X=1)=p$.
\begin{parts}
\part Let $Y=1-X$. Find the joint PMF of $X$ and $Y$.
\begin{answer}
\[
f_{X,Y}(x,y) = \begin{cases}
	p	& \text{if } (x,y) = (1,0) \\
	1-p	& \text{if } (x,y) = (0,1) \\
	0	& \text{otherwise.}
\end{cases}
\]
\end{answer}
\part Let $Y=1-X$ and $Z=XY$. Find the joint PMF of $X$ and $Z$.
\begin{answer}
\[
f_{X,Z}(x,z) = \begin{cases}
	1-p	& \text{if } (x,z) = (0,0) \\
	p	& \text{if } (x,z) = (1,0) \\
	0	& \text{otherwise.}
\end{cases}
\]
\end{answer}
\end{parts}


%%----------------------------------------
%% GS 2.5.3
%\question
%The random variables $X$ and $Y$ have the joint distribution function
%\[
%F(x,y) = \begin{cases}
%	(1-e^{-x})\left(\frac{1}{2} + \frac{1}{\pi}\tan^{-1} y\right)	& \text{if } x \geq 0, y\in\R\\
%	0															& \text{otherwise.}
%\end{cases}	
%\]
%Show that $X$ and $Y$ are jointly continuous.
%\begin{answer}
%Differentiating $F(x,y)$ with respect to $x$ and $y$, we obtain
%\[
%f(x,y) = \frac{\partial^2}{\partial x\partial y}F(x,y) = \frac{e^{-x}}{\pi(1+y^2)} \qquad \text{for }x\geq 0,\ y\in\R
%\]
%where we have used the fact that $\frac{d}{dy}(\tan^{-1} y) =  \frac{1}{1+y^2}$.
%%
%Thus there exists a function $f(x,y)$ such that
%\[
%F(x,y) = \int_{-\infty}^x \int_{-\infty}^y f(u,v)\,du\,dv
%\]
%so $X$ and $Y$ are jointly continuous.
%\end{answer}

%%----------------------------------------
%\question
%Let $X$ and $Y$ be two jointly discrete random variables with joint PMF given by the following table:
%\[
%\begin{array}{|cc|cccc|}\hline
%    &       &       & \multicolumn{2}{c}{y} &   \\
%    &       & 0     & 1     & 2     & 3     \\ \hline
%    & 0     & 0     & 3/56  & 6/56  & 1/56  \\
%\raisebox{-1.5ex}{$x$}   & 1     & 3/56  & 18/56 & 9/56  & 0  \\
%    & 2     & 6/56  & 9/56  & 0     & 0  \\
%    & 3     & 1/56  & 0     & 0     & 0  \\ \hline
%\end{array}
%\]
%
%\begin{parts}
%\part Find the conditional PDF of $X$ given $Y=0$.
%\begin{answer}
%\[
%\begin{array}{c|cccc}
%x          		& 0     & 1     & 2     & 3     \\ \hline
%f_{X|Y=0}(x)		& 0     & 3/10  & 6/10  & 1/10  \\
%\end{array}
%\]
%\end{answer}
%
%\part Find the conditional PMF of $Y$ given $X=1$.
%\begin{answer}
%\[
%\begin{array}{c|cccc}
%y            	& 0     & 1     & 2     & 3     \\ \hline
%f_{Y|X=1}(y)		& 1/10  & 6/10  & 3/10  & 0     \\
%\end{array}
%\]
%\end{answer}
%\end{parts}

%----------------------------------------
\question
Let $X$ and $Y$ be two independent discrete random variables with the following PMFs:
\begin{center}
\begin{minipage}{\linewidth}
\begin{minipage}{0.4\linewidth}
\[
\begin{array}{|c|cc|} \hline
x     & 1     & 2   \\ \hline
f_X(x)  & 1/3   & 2/3 \\ \hline
\end{array}
\]
\end{minipage}
%\text{\qquad\qquad}
\begin{minipage}{0.4\linewidth}
\[
\begin{array}{|c|ccc|} \hline
y     & -1    & 0     & 1     \\ \hline
f_Y(y)  & 1/4   & 1/2   & 1/4   \\ \hline
\end{array}
\]
\end{minipage}
\end{minipage}
\end{center}

\begin{parts}
%--------------------
\part Compute the joint PMF of $X$ and $Y$.
\begin{answer}
Since $X$ and $Y$ are independent we have that $f_{X,Y}(x,y)=f_X(x)f_Y(y)$, which yields the following joint distribution.
\[
\begin{array}{|cc|ccc|c|}\hline
    &       &       & y     &       &     \\
    &       & -1    & 0     & 1     & f_X(x)    \\ \hline
\raisebox{-1.0ex}{$x$}   & 1     & 1/12  & 1/6   & 1/12  & 1/3       \\
    & 2     & 1/6   & 1/3   & 1/6   & 2/3       \\ \hline
    & f_Y(y)& 1/4   & 1/2   & 1/4   & 1         \\ \hline
\end{array}
\]
\end{answer}
%--------------------
\part Compute the joint PMF of the random variables $U=1/X$ and $V=Y^2$.
\begin{answer}
Let $f_{U,V}$ denote the joint PMF of $U$ and $V$. Clearly, $U$ takes the values $0.5$ and $1$, while $V$ takes the values $0$ and $1$. We compute (for example)
\[
f_{U,V}(0.5,1) = f_{X,Y}(2,-1) + f_{X,Y}(2,1) = 1/6 + 1/6 = 1/3.
\]
Thus we obtain joint PMF shown in the following table:
\[
\begin{array}{|cc|cc|c|} \hline
    &       & \multicolumn{2}{c|}{v} &   \\
    &       & 0     & 1     & f_U(u)    \\ \hline
\raisebox{-1.0ex}{$u$}   & 1/2   & 1/3   & 1/3   & 2/3       \\
    & 2     & 1/6   & 1/6   & 1/3       \\ \hline
    & f_V(v)& 1/2   & 1/2  & 1          \\ \hline
\end{array}
\]
\end{answer}
%--------------------
\part Show that $U$ and $V$ are independent.
\begin{answer}
The marginal PMFs $f_U(u)$ and $f_V(v)$ are computed by summing the rows and columns of the joint PMF table. Thus we see that $U$ and $V$ are independent because $f_{U,V}(u,v) = f_U(u)f_V(v)$ for every pair of values $(u,v)$.
\end{answer}
%--------------------
\end{parts}


%----------------------------------------
\question
Two discrete random variables $X$ and $Y$ have the following joint PMF:
\[
f_{X,Y}(x,y) = \left\{\begin{array}{ll}
	c|x+y| 	& \text{for}\quad x,y\in\{-2,-1,0,1,2\}, \\
	0		& \text{otherwise,}
\end{array}\right.	
\]
where $c$ is a constant.
\begin{parts}
%--------------------
\part Show that $c=1/40$.
\begin{answer}
First we tabulate the values of $|x+y|$:
\begin{center}
\begin{tabular}{|cc|ccccc|}\hline
	&		&		&		& $y$	&		&		\\
	&		& -2		& -1		& 0		& 1		& 2 		\\ \hline
	& -2		& 4 		& 3		& 2 		& 1		& 0		\\
	& -1		& 3		& 2		& 1		& 0		& 1		\\
$x$	& 0		& 2		& 1		& 0		& 1		& 2		\\
	& 1		& 1		& 0		& 1		& 2		& 3		\\
	& 2		& 0		& 1		& 2		& 3		& 4		\\ \hline
\end{tabular}
\end{center}	
Because the probabilities must sum to $1$, it follows that $c=1/40$. 
\end{answer}

%--------------------
\part Find $\prob(X=0,Y=-2)$.
\begin{answer}
The joint PMF of $X$ and $Y$, along with their marginal distributions, are shown in the following table,
\begin{center}
\begin{tabular}{|cc|ccccc|c|}\hline
	&		&		&		& $y$	&		&		&			\\
	&		& -2		& -1		& 0		& 1		& 2 		&			\\ \hline
	& -2		& 4/40	& 3/40	& 2/40	& 1/40	& 0		&	10/40 	\\
	& -1		& 3/40	& 2/40	& 1/40	& 0		& 1/40	&	 7/40 	\\
$x$	& 0		& 2/40	& 1/40	& 0		& 1/40	& 2/40	&	 6/40 	\\
	& 1		& 1/40	& 0		& 1/40	& 2/40	& 3/40	&	 7/40 	\\
	& 2		& 0		& 1/40	& 2/40	& 3/40	& 4/40	&	10/40 	\\ \hline
	&		& 10/40	& 7/40	& 6/40	& 7/40	& 4/40	& \\ \hline
\end{tabular}
\end{center}	
Hence 
\[
\prob(X =0,Y=-2) = f_{X,Y}(0,-2) = 2/40 = 1/20.
\]
\end{answer}
%--------------------
\part Find $\prob(X=2)$.
\begin{answer}
$\prob(X=2) = f_X(2) = 10/40 = 1/4$
\end{answer}
%--------------------
\part Find $\prob(|X-Y|\leq 1)$.
\begin{answer}
\begin{align*}
\prob(|X-Y|\leq 1) 
	& = \prob(-1\leq X-Y\leq 1) \\
	& = \prob(X-Y = -1,0\text{ or } 1) \\
	& = \prob\big((X = Y - 1)\,{\cup}\,(X = Y)\,{\cup}\,(X = Y + 1)\big) \\
	& = 8/40 + 12/40 + 8/40 \\
	& = 7/10.
\end{align*}
\end{answer}
%--------------------

\end{parts}


%----------------------------------------
\question
Two continuous random variables $X$ and $Y$ have the joint PDF
\[
f_{X,Y}(x,y) = \left\{\begin{array}{ll}
	2x 	& \text{if}\quad 0\leq x,y\leq 1 \\
	0		& \text{otherwise.}
\end{array}\right.	
\]

\begin{parts}
%--------------------
\part Find the conditional distribution of $Y$ given that $X=x$.
\begin{answer}
The marginal PDF of $X$ is 
\[
f_{X}(x) 
	= \int_{y}f(x,y)\,dy 
	= \int_{0}^{1}2x\,dy 
	= \big[2xy\big]_{0}^{1} 
	= 2x \quad\text{for } 0\leq x\leq 1.
\]
The conditional PDF of $Y$ given $X = x$ is defined as 
\[
f_{Y|X}(y|x) 
	= \frac{f(x,y)}{f_{X}(x)} 
	= \frac{2x}{2x} 
	= 1 \quad\text{for } 0\leq y\leq 1
\]
Thus the conditional PDF of $Y$ given $X = x$ is $f_{Y|X}(y|x) = 1$ for $0\leq y\leq 1$, and because this is independent of $x$, the random variables $X$ and $Y$ are independent.

The marginal PDF of $Y$ is 
\[
f_{Y}(y) = \int_{0}^{1}2x\,dx = \big[x^{2}\big]_{0}^{1} = 1 \quad\text{for } 0\leq y\leq 1,
\]
and the conditional PDF of $X$ given $Y = y$ is
\[
f_{X|Y}(x|y) = 2x \quad\text{for } 0\leq x\leq 1
\]
\end{answer}
%--------------------
\part Find $\prob(Y\leq 0.5 | X=0.5)$ and $\prob(Y\leq 0.5 | X=0.75)$.
\begin{answer}
Because $X$ and $Y$ are independent ,
\[
\prob(Y\leq 0.5 | X=0.5)
	= \prob(Y\leq 0.5 | X=0.75)
	= \int_{0}^{0.5}1\,dy  
	= 0.5
\]	
In general, when X and Y are not independent, the probability that $Y\leq 0.5$ would depend on the value taken by $X$.	
\end{answer}

%--------------------
\part Find the marginal distribution of $Y$ and hence find $\prob(Y\leq 0.5)$.
\begin{answer}
\[
\prob(Y {\leq} 0.5) = \int_{0}^{0.5}1\,dy = \big[y\big] _{0}^{0.5} = 0.5
\]
\end{answer}
%--------------------
\end{parts}

%----------------------------------------
\question
Two continuous random variables $X$ and $Y$ have the following joint PDF:
\[
f_{X,Y}(x,y) = \left\{\begin{array}{ll}
	c(x^2+y)		& \text{when $-1\leq x\leq 1$ and $0\leq y \leq 1-x^2$}, \\
	0			& \text{otherwise.}
\end{array}\right.	
\]
where $c$ is a constant.

\begin{parts}
%--------------------
\part Show that $c=5/4$.
\begin{answer}
The joint PDF is non-zero over the region between the curve $y=1-x^2$ and the $x$-axis. 
%\par\resizebox{0.5\linewidth}{!}{\includegraphics{ex11q4a}}\par
\bit
\it For $x\in[-1,1]$ fixed, we integrate $y$ over the range $0\leq y\leq 1-x^2$.
\it For $y\in[0,1]$ fixed, we integrate $x$ over the range $-\sqrt{1-y}\leq x\leq +\sqrt{1-y}$.
\eit
To find $c$, the joint PDF must integrate to 1. 
\begin{align*}
\int\int f(x,y)\,dx\,dy = 1
	& \Rightarrow \int_{x=-1}^{1}\int_{y=0}^{1-x^{2} }c(x^{2} +y)\,dx\,dy = 1 \\ 
	& \Rightarrow \int_{-1}^{1}\left[ c\left( x^{2} y+\frac{y^{2} }{2} \right) \right]  _{0}^{1-x^{2} }\,dx = 1 \\
	& \Rightarrow \int_{-1}^{1}c\left( x^{2} (1-x^{2} )+\frac{(1-x^{2} )^{2} }{2} \right)\,dx = 1 \\
	& \Rightarrow \int_{-1}^{1}c\left( x^{2} -x^{4} +\frac{1}{2} -x^{2} +\frac{x^{4} }{2} \right)\,dx=1 \\
	& \Rightarrow \int_{-1}^{1}\frac{c}{2}  (1-x^{4} )\,dx = 1 \\
	& \Rightarrow \left[ \frac{c}{2} \left( x-\frac{x^{5} }{5} \right) \right] _{-1}^{1} = 1 \\
	& \Rightarrow \frac{c}{2} \left( 1-\frac{1}{5} +1-\frac{1}{5} \right) = 1 \\
\end{align*}	
Thus $c = 5/4$.
\end{answer}

%--------------------
\part Find $\prob(0\leq X\leq 0.5)$.
\begin{answer}
To calculate $\prob(0\leq X\leq 1/2)$, we first compute the marginal PDF of X.
\begin{align*}
f_{X}(x)
	& = \int_{0}^{1-x^{2} }\frac{5}{4}  (x^{2} +y)\,dy \\
	& = \left[ \frac{5}{4} \left( x^{2} y+\frac{y^{2} }{2} \right) \right] _{0}^{1-x^{2} } \\
	& = \frac{5}{8} (1-x^{4})
	\quad\text{for }-1\leq x\leq 1.
\end{align*}
Then we integrate this over $x$ in the range $[0,1/2]$:
\begin{align*}
\prob(0 {\leq} X {\leq} 1/2)
	& = \int_{0}^{1/2}\frac{5}{8}  (1-x^{4} )\,dx \\
	& = \left[\frac{5}{8} \left( x-\frac{x^{5} }{5}\right)\right]_{0}^{1/2} = \frac{79}{256}
\end{align*}
\end{answer}

\part Find $\prob(Y\leq X+1)$.
%--------------------
\begin{answer}
$Y\leq X+1$ is satisfied for pairs $(X,Y)$ in the region between the curves $y=1+x$ and the $x$-axis when $-1\leq x\leq 0$, and between the curve $y=1-x^2$ and the $x$-axis when $0\leq x\leq 1$. 

\bit
\it For fixed $x\in[-1,0]$, we must integrate $y$ over the range $0\leq y\leq 1+x$.
\it For fixed $x\in[0,+1]$, we must integrate $y$ over the range $0\leq y\leq 1-x^2$.
\eit
%\par\resizebox{0.5\linewidth}{!}{\includegraphics{ex11q4b}}\par
\fbox{\parbox{\linewidth}{\hfill It helps to draw a plot of the curves $y=1+x$ and $y=1-x^2$ \hfill\mbox{}}}

\begin{align*}
\prob(Y\leq X + 1)
	&  = \int_{-1}^{0}\int_{0}^{1+x} f(x,y)\,dy\,dx   +\int_{0}^{1}\int_{0}^{1-x^{2} }f(x,y)\,dy\,dx \\
	&  = \int_{-1}^{0}\int_{0}^{1+x}\frac{5}{4} (x^{2} +y)\,dy\,dx   +\int_{0}^{1}\int_{0}^{1-x^{2} }\frac{5}{4}   (x^{2} +y)\,dy\,dx \\
	& =  \int_{-1}^{0}\left[ \frac{5}{4} \left( x^{2} y+\frac{y^{2} }{2} \right) \right] _{0}^{1+x}  dx+\int_{0}^{1}\left[ \frac{5}{4} \left( x^{2} y+\frac{y^{2} }{2} \right) \right]  _{0}^{1-x^{2} } dx \\
	& = \int_{-1}^{0}\frac{5}{4} \left( x^{2} (1+x)+\frac{(1+x)^{2} }{2} \right)  dx+\int_{0}^{1}\frac{5}{4} \left( x^{2} (1-x^{2} )+\frac{(1-x^{2} )^{2} }{2} \right)  dx \\
	& = \int_{-1}^{0}\frac{5}{4} \left( x^{2} +x^{3} +\frac{1}{2} +x+\frac{x^{2} }{2} \right)  dx+\int_{0}^{1}\frac{5}{4} \left( x^{2} -x^{4} +\frac{1}{2} -x^{2} +\frac{x^{4} }{2} \right)  dx \\
	& = \left[ \frac{5}{4} \left( \frac{x}{2} +\frac{x^{2} }{2} +\frac{x^{3} }{2} +\frac{x^{4} }{4} \right) \right] _{-1}^{0} +\left[ \frac{5}{4} \left( \frac{x}{2} -\frac{x^{5} }{10} \right) \right] _{0}^{1}  \\
	& = \frac{13}{16}
\end{align*}	
\end{answer}

\part Find $\prob(Y=X^2)$.
%--------------------
\begin{answer}
$\prob(Y=X^2)=0$. This is because the region in the $(x, y)$ plane over which we integrate is a curve, $y = x^{2}$ (which has zero area). If we integrate first with respect to $y$, the range of integration is $x^{2}$ to $x^{2}$, giving zero for the definite integral. This is analagous to the calculation of $P(X = k)$ for a continuous univariate random variable (in which case the answer is again 0).\end{answer}

\end{parts}


%----------------------------------------
\end{questions}
\end{exercise}
%----------------------------------------------------------------------

%======================================================================
\endinput
%======================================================================

%----------------------------------------------------------------------

%======================================================================
\endinput
%======================================================================
