% !TEX root = main.tex
%----------------------------------------------------------------------
\begin{exercise}
\begin{questions}
%----------------------------------------
% GS p100
\question
Let $X$ and $Y$ have standard bivariate normal distribution, with joint PDF given by
%\[
%f(x,y)
%	= \frac{1}{2\pi\sigma_1\sigma_2\sqrt{1-\rho^2}}
%		\exp\left(-\frac{1}{2(1-\rho^2)}\left[\left(\frac{x-\mu_1}{\sigma_1}\right)^2 
%			-2\rho\left(\frac{x-\mu_1}{\sigma_1}\right)\left(\frac{y-\mu_2}{\sigma_2}\right) 
%				+\left(\frac{y-\mu_2}{\sigma_2}\right)^2 \right]\right)
%\]
\[
f(x,y) = \frac{1}{2\pi\sqrt{1-\rho^2}}\exp\left(-\frac{1}{2(1-\rho^2)}(x^2 - 2\rho xy + y^2)\right)
\]
where $\rho$ is a constant satisfying $-1 < \rho < 1$. 
\begin{parts}
\part
Check that $f(x,y)$ is indeed a joint PDF, by verifying that $f(x,y)\geq 0$ and $\displaystyle\int_{-\infty}^{\infty}\int_{-\infty}^{\infty} f(x,y)\,dxdy = 1$.
\begin{answer}
TODO
\end{answer}
\part
Check that $\cov(X,Y) = \displaystyle\int_{-\infty}^{\infty}\int_{-\infty}^{\infty} xy f(x,y)\,dxdy = \rho$.
\begin{answer}
TODO
\end{answer}
\part
Show that if $X$ and $Y$ are uncorrelated, then they are independent.
\begin{answer}
TODO
\end{answer}
\end{parts}
%----------------------------------------
\question
% GS p110
Let $X$ and $Y$ have standard bivariate normal distribution. Find the conditional distribution of $Y$ given $X=x$, and hence show that $\expe(Y|X) = \rho X$.
\begin{answer}
The conditional distribution of $Y$ given $X=x$ is $N(\rho x, 1-\rho^2)$.
\end{answer}
%----------------------------------------
% GS 4.7.5
\question
Let $X$ and $Y$ have standard bivariate normal distribution. Show that $X$ and $\displaystyle Z=\frac{Y-\rho X}{\sqrt{1-\rho^2}}$ are independent standard normal random variables.
\begin{answer}
TODO
\end{answer}
%----------------------------------------
% GS 4.7.6
\question
Let $X$ and $Y$ have standard bivariate normal distribution, and let $Z=\max\{X,Y\}$. Show that $\expe(Z)=\sqrt{(1-\rho)/\pi}$ and $\expe(Z^2)=1$.
\begin{answer}
TODO
\end{answer}
%----------------------------------------
\question
Let $U,V\sim N(0,1)$. Show that the random variables $X = U+V$ and $Y = U-V$ are independent.

\begin{answer}
\bit
\it The transformation is $g(u,v) = (u+v,u-v)$. 
\it To compute the inverse transformation, consider $x=u+v$ and $y=u-v$. 
\it Solving these, we obtain $u = \frac{1}{2}(x+y)$ and $v = \frac{1}{2}(x-y)$.
\it Thus the inverse transformation is $(u,v) = g^{-1}(x,y) = \left(\frac{1}{2}(x+y),\frac{1}{2}(x-y)\right)$
\eit

The Jacobian determinant of $g^{-1}(x,y)$ is
\[
J = 
\begin{vmatrix}
\frac{\partial u}{\partial x}  & \frac{\partial u}{\partial y}  \\
\frac{\partial v}{\partial x}  & \frac{\partial v}{\partial y} 
\end{vmatrix}
=
\begin{vmatrix}[r]
\frac{1}{2}   &  \frac{1}{2} \\
\frac{1}{2}  & -\frac{1}{2}
\end{vmatrix}
=
-\frac{1}{2}
\]
The joint PDF of $U$ and $V$ is 
\[
f(u,v) = \frac{1}{2\pi \sqrt{1-\rho ^{2}}} \exp\left( -\frac{u^2 + v^2 - 2\rho uv}{2(1-\rho^2)} \right)
\]

Now,
\begin{align*}
u^2 + v^2 - 2\rho uv
	& = \left(\frac{1}{2}(x+y)\right)^2 +\left(\frac{1}{2}(x-y)\right)^2 - 2\rho\left(\frac{1}{2}(x+y)\right)\left(\frac{1}{2}(x-y)\right) \\
	& = \frac{1}{2}x^2(1-\rho ) + \frac{1}{2}y^2(1+\rho )
\end{align*}	

The joint PDF of $X$ and $Y$ is therefore
\begin{align*}
f(x,y) 
	& = \frac{1}{2} \frac{1}{2\pi\sqrt{1-\rho ^{2}}}\exp\left( -\frac{x^{2} (1-\rho )}{4(1-\rho ^{2} )} -\frac{y^{2} (1+\rho )}{4(1-\rho ^{2})}\right) \\
	& = \frac{1}{\sqrt{4\pi(1+\rho)}}\exp\left(-\frac{x^{2}}{4(1+\rho)}\right) \times \frac{1}{\sqrt{4\pi(1-\rho)}}\exp\left(-\frac{y^{2}}{4(1-\rho )}\right) \\
\end{align*}
This is the product of the PDF of a $N\big(0,2(1+\rho)\big)$ variable and the PDF of a $N\big(0,2(1-\rho)\big)$ variable. Thus $X$ and $Y$ are independent. 
\end{answer}

\question
Let $X$ and $Y$ have bivariate normal distribution with means $\mu_1$ and $\mu_2$, variances $\sigma_1^2$ and $\sigma_2^2$, and correlation $\rho$. Show that the conditional distribution of $Y$ given $X=x$ is
\[
N\left(\mu_2 + \rho\left(\frac{\sigma_2}{\sigma_1}\right)(x-\mu_1), \sigma_2^2(1-\rho^2)\right).
\]
\begin{answer}
TODO
\end{answer}
%----------------------------------------
\question
\begin{parts}
\part % << (i)
Let $X$ and $Y$ be jointly continuous random variables, and let $f_{X,Y}$ be their joint PDF. Show that the PDF of the random variable $X+Y$ can be written as
\[
f_{X+Y}(t) 
	= \int_{-\infty}^{\infty} f_{X,Y}(x,t-x)\,dx
	= \int_{-\infty}^{\infty} f_{X,Y}(t-y,y)\,dy.
\]
\begin{answer}
Let $A = \{(x,y):x+y\leq z\}\subset\R^2$. Then
\[
\prob(X+Y\leq z)
	= \iint_A f(x,y)\,dxdy
	= \int_{x=-\infty}^{\infty} \int_{y=-\infty}^{z-x}f_{X,Y}(x,y)\,dy\,dx
\]
We change the variable of integration (in the inner integral), making the substitution $y=t-x$:
\begin{align*}
F_{X+Y}(z) = \prob(X+Y\leq z)
	& = \int_{x=-\infty}^{\infty} \int_{t=-\infty}^{z}f_{X,Y}(x,t-x)\,dt\,dx \\
	& = \int_{t=-\infty}^{z} \int_{x=-\infty}^{\infty}f_{X,Y}(x,t-x)\,dx\,dt
\end{align*}
where the final equality follows by reversing the order of integration. Thus the PDF of $X+Y$ is 
\[
f_{X+Y}(t) = \int_{-\infty}^{\infty} f_{X,Y}(x,t-x)\,dx \qquad\text{as required.}
\]
\end{answer}
\part % << (ii)
Hence, or otherwise, show that if $U,V\sim N(0,1)$ are independent, then $U+V\sim N(0,2)$. (This is a special case of Theorem~\ref{thm:sum_of_normal_variables}.)
\begin{answer}
By part (a), if two random variables $X$ and $Y$ are independent, the PDF of $X+Y$ is the \emph{convolution} of the marginal PDFs:
\[
f_{X+Y}(t) 
	= \int_{-\infty}^{\infty} f_X(x)f_Y(t-x)\,dx
	= \int_{-\infty}^{\infty} f_X(t-y)f_Y(y)\,dy.
\]
$U$ and $V$ are independent, so their joint PDF is
\[
f(u,v)=f_U(u)f_V(v) = \frac{1}{2\pi}\exp\left(-\frac{1}{2}(u^2 + v^2)\right) \qquad  u,v\in\R.
\]
Let $W=U+V$. Then because $U$ and $V$ are independent, 
\begin{align*}
f_W(w) 
	& = \int_{-\infty}^{\infty} f_U(u)f_V(w-u)\,du \\
	& = \frac{1}{2\pi} \int_{-\infty}^{\infty} \exp\left[-\frac{1}{2}\left(u^2 + (w-u)^2\right)\right]\,du \\
	& = \frac{1}{2\pi} e^{-\frac{1}{4}w^2} \int_{-\infty}^{\infty} \exp\left[-\left(u-\frac{w}{2}\right)^2\right]\,du
\end{align*}
We change the variable of integration, by making the substitution $t = \displaystyle\sqrt{2}\left(u-\frac{w}{2}\right)$:
\[
f_W(w) 
	= \frac{1}{2\sqrt{\pi}} e^{-\frac{1}{4}w^2} \int_{-\infty}^{\infty} \frac{1}{\sqrt{2\pi}}e^{-\frac{1}{2}w^2}\,dv
	= \frac{1}{2\sqrt{\pi}} e^{-\frac{w^2}{4}},
\]
which is the PDF of the $N(0,2)$ distribution
\end{answer}
\end{parts}


%----------------------------------------
\end{questions}
\end{exercise}
%----------------------------------------------------------------------

%======================================================================
\endinput
%======================================================================
