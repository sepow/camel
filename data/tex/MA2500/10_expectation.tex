% !TEX root = main.tex
%----------------------------------------------------------------------
\chapter{Expectation}\label{chap:expectationI}
%----------------------------------------------------------------------
%----------------------------------------------------------------------
%\section{Introduction}
%----------------------------------------------------------------------
Expectation is to random variables what probability is to events.
\bit
\it Random events are \emph{sets}, and are studied using \emph{set algebra}. 
\it Random variables are \emph{functions}, and are studied using \emph{mathematical analysis}.
\eit

\bigskip
% disc & continuous
%Let $X:\Omega\to\R$ be a random variable. 
Elementary probability theory provides the following computational formulae for the expectation of a random variable $X:\Omega\to\R$.
\ben
\it If $\Omega$ is a finite sample space with probability mass function $p:\Omega\to\R$, the expectation of $X$ is
\[
\expe(X) = \sum_{\omega\in\Omega} X(\omega) p(\omega).
\]
\it If $X$ is a discrete random variable with PMF $f(x)$ and range $\{x_1,x_2,\ldots\}$, the expectation of $X$ is
\[
\expe(X) = \sum_{i=1}^{\infty} x_i f(x_i).
\]
\it if $X$ is a continuous random variable with PDF $f(x)$, the expectation of $X$ is
\[
\expe(X) = \int_{-\infty}^{\infty} x f(x)\,dx.
\]
\een

\bit
%\it These are computational formulae, not formal definitions of expectation.
\it The convergence of such sums and integrals is not guaranteed under all circumstances.
\it If $X$ takes only non-negative values, we can accept that $\expe(X)=\infty$.
\it If $X$ can take both positive and negative values, we need that $\expe(X)<\infty$.
\eit

%----------------------------------------------------------------------
\section{Indicator variables}\label{sec:expe_indicator}
%----------------------------------------------------------------------
Consider the indicator variable of an event $A$,
\[
I_{A}(\omega) = \begin{cases} 
	1 & \text{if}\ \omega\in A, \\
	0 & \text{if}\ \omega\notin A.
\end{cases}	
\]
To be consistent with the probability measure $\prob(A)$, the only reasonable definition of expectation for $I_A$ is the following:

\begin{definition}[Expectation of indicator variables]
The expectation of an indicator variable $I_A:\Omega\to\R$ is defined by
\[
\expe(I_A) = \prob(A)
\]
\end{definition}

%----------------------------------------------------------------------
\section{Simple random variables}\label{sec:expe_simple}
%----------------------------------------------------------------------

% definition: simple r.v.s
%\begin{definition}
%%A random variable $X:\Omega\to\R$ is called a \emph{simple random variable} if it can take only finitely many values.
%A random variable $X:\Omega\to\R$ is called \emph{simple} if it can take only finitely many values.
%\end{definition}

Let $X$ be a simple random variable, let $\{x_1,x_2,\ldots,x_n$ denote its range, and let $\{A_1,A_2,\ldots,A_n\}$ be a partition of $\Omega$ such that $X(\omega)=x_i$ for all $\omega\in A_i$. Then $X$ can be expressed as a finite linear combination of indicator variables,

\[
X(\omega) = \sum_{i=1}^n x_i I_{A_i}(\omega)\qquad\text{where}\quad I_{A_i}(\omega) = \begin{cases} 
	1 & \text{if}\ \omega\in A_i, \\
	0 & \text{if}\ \omega\notin A_i.
\end{cases}	
\]

% definition: expectation of simple r.v.s
\begin{definition}[Expectation of simple random variables]
The expectation of a simple random variable $\displaystyle X = \sum_{i=1}^n x_i I_{A_i}$ is defined by
\[
\expe(X) = \sum_{i=1}^n x_i \prob(A_i).
\]
\end{definition}

\begin{remark}
%$\expe(X)$ is well-defined because all representations of $X$ give the same value for $\expe(X)$.
It can be shown all representations of $X$ as finite linear combinations of inidcator variables yield the same value for $\expe(X)$, so the expectation of a simple random variable is well-defined.
\end{remark}


% example: simple r.v.
\begin{example}\label{ex:expe}
A fair coin is tossed three times. Let $X:\Omega\to\R$ be the random variable
\[
X(\omega) = \begin{cases}
1	& \text{if}\ \ \omega\in A_1=\{TTT\} \\
2	& \text{if}\ \ \omega\in A_2=\{TTH,THT,HTT\} \\
3	& \text{if}\ \ \omega\in A_3=\{THH,HTH,HHT\} \\
4	& \text{if}\ \ \omega\in A_4=\{HHH\}
\end{cases}
\]
Compute the expected value of $X$.
\end{example}

\begin{solution}
$X$ is a simple random variable, so
\[
\expe(X) 
	= \sum_{i=1}^n x_i \prob(A_i) 
	= \left(1\times\frac{1}{8}\right) +\left(2\times\frac{3}{8}\right) +\left(3\times\frac{3}{8}\right) + \left(4\times\frac{1}{8}\right)
	= \frac{5}{2}.
\]
\end{solution}



% definition: positivity & domination
\begin{definition}
Let $X$ and $Y$ be random variables on $\Omega$.
\ben
\it If $X(\omega)\geq 0$ for all $\omega\in\Omega$, we say that $X$ is \emph{non-negative}. This is denoted by $X\geq 0$.
\it If $X(\omega)\leq Y(\omega)$ for all $\omega\in\Omega$, we say that $X$ is \emph{dominated} by $Y$. This is denoted by $X\leq Y$.
\een
\end{definition}

\vspace*{2ex}
% theorem: properties
\begin{theorem}[Properties of expectation for simple random variables]
Let $X,Y:\Omega\to\R$ be simple random variables.% on $(\Omega,\mathcal{F},\prob)$.
\ben
\it \textbf{Positivity}. If $X\geq 0$ then $\expe(X)\geq 0$.
\it \textbf{Linearity}.  For every $a,b\in\R$,\ \ $\expe(aX+bY) = a\expe(X) + b\expe(Y)$.
\it \textbf{Monotonicity}. If $X\leq Y$ then $\expe(X)\leq\expe(Y)$.
\een
\end{theorem}



\begin{proof}
Let $\{x_1,x_2,\ldots,x_n\}$ be the range of $X$, and let $\{y_1,y_2,\ldots,y_m\}$ be the range of $Y$. Then $X$ and $Y$ can be written as
\[
X(\omega) = \sum_{i=1}^n x_i I_{A_i}(\omega) \quad\text{and}\quad Y(\omega) = \sum_{j=1}^m y_j I_{B_j}(\omega),
\]
where $A_i = \{\omega:X(\omega)=x_i\}$ and $B_j = \{\omega:Y(\omega)=y_j\}$
\ben
\it % << (i)
Positivity: \par
If $X(\omega)\geq 0$ for all $\omega$, we must have that each $x_i\geq 0$. Thus $\expe(X) = \sum_{i=1}^n x_i \prob(A_i)$ is a sum of non-negative terms, so $\expe(X)\geq 0$.

\it % << (ii)
Linearity:
\par
The composite variable $aX + bY$ is a simple random variable, because it can be written as
\[
aX(\omega) + bY(\omega) = \sum_{i=1}^n\sum_{j=1}^m (ax_i + by_j) I_{A_i\cap B_j}(\omega).
\]
The expectation of $aX+bY$ is therefore defined to be
\begin{align*}
\expe(aX+bY)
	& = \sum_{i=1}^n\sum_{j=1}^m (ax_i + by_j) \prob(A_i\cap B_j) \\
	& = a\sum_{i=1}^n x_i \sum_{j=1}^m \prob(A_i\cap B_j) + b\sum_{j=1}^m y_j \sum_{i=1}^n \prob(A_i\cap B_j).
\end{align*}
By additivity:
\bit
\it $\{A_i\cap B_j\}_{j=1}^m$ is a partition of $A_i$, so $\sum_{j=1}^m \prob(A_i\cap B_j) = \prob(A_i)$.
\it $\{A_i\cap B_j\}_{i=1}^n$ is a partition of $B_j$, so $\sum_{i=1}^n \prob(A_i\cap B_j) = \prob(B_j)$.
\eit
Thus,
\begin{align*}
\expe(aX+bY)
	& = a\sum_{i=1}^n x_i\prob(A_i) + b\sum_{j=1}^m y_j\prob(B_j) \\
	& = a\expe(X) + b\expe(Y).
\end{align*}
\it % << (iv)
Monotonicity:
\par
If $X\leq Y$ then $Y-X\geq 0$, so
\bit
\it By positivity, $\expe(Y-X)\geq 0$
\it By linearity, $\expe(Y)-\expe(X)\geq 0$, so $\expe(X)\leq\expe(Y)$ as required.
\eit
\een
\end{proof}



% example
\begin{example}
Extending example~\ref{ex:expe}, let $Y:\Omega\to\R$ be the random variable
\[
Y(\omega) = \begin{cases}
2	& \text{if}\ \ \omega\in A'_1=\{TTT,TTH\} \\
3	& \text{if}\ \ \omega\in A'_2=\{THT,THH\} \\
4	& \text{if}\ \ \omega\in A'_3=\{HTT,HTH\} \\
5	& \text{if}\ \ \omega\in A'_4=\{HHT,HHH\}
\end{cases}
\]
Compute the expected value of (i) $Y$ and (ii) $3X+2Y$.
\end{example}

\newpage
\begin{solution}
Note that $X$ and $Y$ are both non-negative, and that $X\leq Y$.% because $X(\omega)\leq Y(\omega)$ for all $\omega$. 
\par
$Y$ is a simple random variable, so its expected value is
\[
\expe(Y) 
	= \sum_{j=1}^4 b_j \prob(B_j) 
	= \left(2\times\frac{1}{4}\right) +\left(3\times\frac{1}{4}\right) +\left(4\times\frac{1}{4}\right) + +\left(5\times\frac{1}{4}\right)
	= \frac{14}{4},
\]
which verifies that $\expe(X)\leq\expe(Y)$. By the linearity of expectation for simple random variables, 
\[
\expe(3X+2Y)
	= 3\expe(X) + 2\expe(Y)
	= \left(3\times\frac{5}{2}\right) + \left(2\times\frac{14}{4}\right)
	= \frac{29}{2}.
\]
\end{solution}

%----------------------------------------------------------------------
\section{Non-negative random variables}\label{sec:expe_non-negative}
%----------------------------------------------------------------------

% theorem: approximating sequences
\begin{theorem}
For every non-negative random variable $X\geq 0$, there exists an increasing sequence of simple non-negative random variables
\[
0\leq X_1\leq X_2\leq \ldots
\]
with the property that $X_n(\omega)\uparrow X(\omega)$ for each $\omega\in\Omega$ as $n\to\infty$.
\end{theorem}

\proofomitted

% definition: expectation of a non-negative r.v.s
\begin{definition}[Expectation of non-negative random variables]\label{def:expe_non-negative}
The \emph{expectation} of a non-negative random variable $X$ is defined to be
\[
\expe(X) = \lim_{n\to\infty}\expe(X_n)
\]
where the $X_n$ are simple non-negative random variables with $X_n\uparrow X$ as $n\to\infty$.
\end{definition}

\begin{remark}
It can be shown that all approximating sequences yield the same value for $\expe(X)$, so the expectation of non-negative random variables is well-defined
\end{remark}

% remark
\begin{remark}[Infinite expectation]
The expectation of non-negative random variables can be infinite:
\ben
\it If $\expe(X)<\infty$ we say that $X$ has \emph{finite} expectation.
\it If $\expe(X)=\infty$ we say that $X$ has \emph{infinite} expectation.
\een
\end{remark}

% theorem: properties
\begin{theorem}[Properties of expectation for non-negative random variables]\label{thm:prop_expe_non-neg}
For non-negative random variables $X,Y\geq 0$,
\ben
\it \textbf{Positivity}. If $X\geq 0$ then $\expe(X)\geq 0$.
\it \textbf{Linearity}.  $\expe(aX+bY) = a\expe(X) + b\expe(Y)$ for every $a,b\in\R_{+}$.
\it \textbf{Monotonicity}. If $X\leq Y$ then $\expe(X)\leq\expe(Y)$.
\it \textbf{Continuity}. If $X_n\to X$ as $n\to\infty$, where the $X_n$ are non-negative, then $\expe(X_n)\to\expe(X)$ as $n\to\infty$.
\een
\end{theorem}

\proofomitted

%
%% proof
%\begin{proof}
%\ben
%\it % << (i)
%$\expe(X)\geq 0$ follows by construction.
%\it % << (ii)
%Let $0\leq X_1\leq X_2\leq \ldots$ and $0\leq Y_1\leq Y_2\leq \ldots$ be sequences of non-negative simple random variables with $X_n\uparrow X$ and $Y_n\uparrow Y$ as $n\to\infty$. Then the random variables $aX_n+bY_n$ are also  non-negative simple random variables, with 
%\[
%aX_n+bY_n\ \uparrow\ aX+bY\quad\text{as}\quad n\to\infty.
%\]
%By definition,
%\begin{align*}
%\expe(aX+bY)
%	& = \lim_{n\to\infty}\expe(aX_n + bY_n) \\
%	& = \lim_{n\to\infty}\big[a\expe(X_n) + b\expe(Y_n)\big] \quad\text{(by linearity of expectation for simple RVs)}\\
%	& = a\lim_{n\to\infty}\expe(X_n) + b\lim_{n\to\infty}\expe(Y_n) \\
%	& = a\expe(X) + b\expe(Y)
%\end{align*}
%\it % << (iii)
%If $X\geq Y$, then $\expe(X)\geq\expe(Y)$ by positivity and linearity.
%\een
%\end{proof}

%Recall: \textbf{Continuity}. If $X_n\to X$ as $n\to\infty$, then $\expe(X_n)\to\expe(X)$ as $n\to\infty$.
%The \emph{monotone convergence theorem} is a fundamental result in measure theory. In the present context, it establishes the continuity of expectation for positive random variables.
%% theorem: monotone convergence theorem
%\begin{theorem}[The monotone convergence theorem]
%Let $0\leq X_1\leq X_2\leq \ldots$ and $X$ be positive random variables with $X_n(\omega)\uparrow X(\omega)$ as $n\to\infty$ for each $\omega\in\Omega$. Then $\expe(X_n)\uparrow\expe(X)$ as $n\to\infty$.
%\end{theorem}
%\proofomitted

%The following lemma will be useful later on.
%% lemma
%\begin{lemma}\label{lem:pos_rv_expe_zero}
%If $X\geq 0$ and $\expe(X)=0$ then $\prob(X=0)=1$.
%\end{lemma}
%
%\begin{proof}
%Let $X\geq 0$ be such that $\expe(X)=0$, and assume that $\prob(X>0)>0$. 
%
%By the right-continuity of distribution functions, there exists $\epsilon>0$ such that $\prob(X>\epsilon)>0$. 
%
%This implies that $X\geq \epsilon\,I(X>\epsilon)$, which in turn implies that $\expe(X)\geq \epsilon\,\prob(X>\epsilon) > 0$. 
%
%This is a contradiction, so we conclude that $\prob(X>0)=0$
%\end{proof}


%----------------------------------------------------------------------
\newpage
\section{Signed random variables}\label{sec:expe_signed}
%----------------------------------------------------------------------

We can extend the definition of expectation to random variables that take both positive and negative values, but only if the random variables are \emph{integrable}:

\begin{definition}[Integrable random variables]
A random variable $X$ is said to be \emph{integrable} if $\expe(|X|) < \infty$.
\end{definition}

\begin{definition}[The positive and negative parts]
%Let $X:\Omega\to\R$ be a random variable.
The positive and negative parts of a random variable $X$, denoted by $X^{+}$ and $X^{-}$ respectively, are defined to be
\[\begin{array}{lll}
X^{+}(\omega) 
	& = \max\{0,X(\omega)\}	
	& = \begin{cases} 
			X(\omega)	& \text{if } X(\omega)\geq 0, \\
	   		0 		 	& \text{if } X(\omega) < 0;
	\end{cases} \\
X^{-}(\omega)	
	& = \max\{0,-X(\omega)\} 
	& = \begin{cases} 
			-X(\omega)	& \text{if } X(\omega)\leq 0, \\
			0 		 	& \text{if } X(\omega) > 0.
		\end{cases}
\end{array}\]
\end{definition}
Note that $X^{+}$ and $X^{-}$ are both non-negative random variables, with $X = X^{+} - X^{-}$.

\begin{definition}[Expectation of signed random variables]
The \emph{expectation} of an integrable random variable $X:\Omega\to\R$ is
\[
\expe(X) = \expe(X^{+}) - \expe(X^{-})
\]
where $X^{+}$ and $X^{-}$ are respectively the positive part and negative part of $X$:
\end{definition}

% remark
\begin{remark}[Undefined expectation]
Because $|X| = X^{+} + X^{-}$, it follows by the linearity of expectation for non-negative random variables that
\[
\expe(|X|) = \expe(X^{+} + X^{-}) = \expe(X^{+}) + \expe(X^{-}).
\]

\ben
\it If $X$ is integrable, then $\expe(X^{+})$ and $\expe(X^{-})$ are both finite.
\it If $X$ is not integrable, then one or both of $\expe(X^{+})$ and $\expe(X^{-})$ must be infinite:
	\bit
	\it if $\expe(X^{+})=\infty$ and $\expe(X^{-})<\infty$, we write $\expe(X) = +\infty$;
	\it if $\expe(X^{+})<\infty$ and $\expe(X^{-})=\infty$, we write $\expe(X)= -\infty$;
	\it if $\expe(X^{+})=\infty$ and $\expe(X^{-})=\infty$, we say that $\expe(X)$ \emph{does not exist}.
	\eit
\een
\end{remark}

% theorem: properties of expectation
\begin{theorem}[Properties of expectation for signed random variables]
Let $X$ and $Y$ be integrable random variables.
\ben
\it \textbf{Monotonicity}. If $X\leq Y$, then $\expe(X)\leq\expe(Y)$.
\it \textbf{Linearity}. $\expe(aX+bY) = a\expe(X) + b\expe(Y)$ for all $a,b\in\R$.
\een
\end{theorem}

\proofomitted

%----------------------------------------------------------------------
\section{Exercises}
% !TEX root = main.tex
%----------------------------------------------------------------------
\begin{exercise}
\begin{questions}
%----------------------------------------
\question
Let $(\Omega,\mathcal{F},\prob)$ be a probability space, and let $0\leq X_1\leq X_2 \leq \ldots$ be an increasing sequence of non-negative random variables over $(\Omega,\mathcal{F})$ such that 
$X_n(\omega) \uparrow X(\omega)$ ans $n\to\infty$ for all $\omega\in\Omega$.
%\[
%X_n(\omega) \uparrow X(\omega)\quad\text{as}\quad n\to\infty\quad\text{for all}\quad\omega\in\Omega.
%\]
Show that $X$ is a random variable on $(\Omega,\mathcal{F})$.
\begin{answer}
Let $x\in\R$. Since the $X_n$ are random variables, we have (by definition) that $\{X_n \leq x\}\in\mathcal{F}$ for every $n\in\N$. Since $\mathcal{F}$ is closed under countable intersections, 
\[
\{X\leq x\} = \bigcap_{n=1}^{\infty} \{X_n \leq x\} \in\mathcal{F}
\]
so $X$ is a random variable.
\end{answer}

%----------------------------------------
\question
Let $X$ be an integrable random variable. Show that $|\expe(X)|\leq \expe(|X|)$.
\begin{answer}
Since $|X|=X^{+}+X^{-}$, by the triangle inequality
\[
|\expe(X)| = |\expe(X^{+})-\expe(X^{-})| \leq \expe(X^{+}) + \expe(X^{-}) = \expe(|X|),
\]
\it If $X\leq Y$ then $X^{+}\leq Y^{+}$ and $X^{-}\geq Y^{-}$ so
\[
\expe(X) = \expe(X^{+})-\expe(X^{-}) \leq \expe(Y^{+})-\expe(Y^{-}) = \expe(Y),
\]
\end{answer}

%==========================================================================
\question
Let $X$ and $Y$ be integrable random variables. Show that $aX+bY$ is integrable.
\begin{answer}
To show that $aX+bY$ is integrable, first we have by the triangle inequality that
\[
|aX+bY| \leq |a||X| + |b||Y|.
\]
By the linearity and monotonicity of expectation for non-negative random variables,
\[
\expe(|aX+bY|) \leq |a|\expe(|X|) + |b|\expe(|Y|)
\]
and since $\expe(|X|)<\infty$ and $\expe(|Y|)<\infty$, it follows that $\expe(|aX+bY|)<\infty$, so $aX+bY$ is integrable.
\end{answer}

%----------------------------------------
\end{questions}
\end{exercise}
%----------------------------------------------------------------------

%======================================================================
\endinput
%======================================================================

%----------------------------------------------------------------------

%======================================================================
\endinput
%======================================================================
