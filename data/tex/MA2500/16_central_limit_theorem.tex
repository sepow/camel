% !TEX root = main.tex
%----------------------------------------------------------------------
\chapter{The Central Limit Theorem}\label{chap:clt}
%----------------------------------------------------------------------
%%----------------------------------------------------------------------
%\section{Preliminaries}
%%----------------------------------------------------------------------
%
%% defn: landau notation
%\begin{definition}[Landau notation]
%\ben
%\it % sequences
%Let $a_1,a_2,\ldots$ and $b_1,b_2,\ldots$ be two sequences of real numbers, with 
%\[
%\displaystyle\frac{a_n}{b_n}\to 0\text{\quad as\quad} n\to\infty.
%\]
%	\bit
%	\it We say that $a_n$ is \emph{asymptotically dominated} by $b_n$ in the limit as $n\to\infty$.
%	\it This is denoted by $a_n = o(b_n)$ as $n\to\infty$.
%	\eit
%\it % functions
%More generally, let $f$ and $g$ be two functions of a real variable, with 
%\[
%\displaystyle\frac{f(x)}{g(x)}\to 0\text{\quad as\quad}x\to x_0.
%\]
%	\bit
%	\it We say that $f(x)$ is \emph{asymptotically dominated} by $g(x)$ in the limit as $x\to x_0$.
%	\it This is denoted by $f(x) = o\big[g(x)\big]$ as $x\to x_0$.
%	\eit
%\een
%\end{definition}
%
%%Examples:
%\begin{example}
%\ben
%\it 
%If $a_n=n$ and $b_n=n^2$, then
%\[
%a_n/b_n \to 0 \text{\quad as\quad} n\to\infty,
%\]
%so $a_n = o(b_n)$ as $n\to\infty$.
%\it 
%If $a_n=n$ and $b_n=2n$, then 
%\[
%a_n/b_n \to 1/2 \text{\quad as\quad} n\to\infty,
%\]
%so $a_n$ is not asymptotically dominated by $b_n$ as $n\to\infty$.
%%\it[]
%\it 
%If $f(x)=x$ and $g(x)=x^2$, then 
%\[
%f(x)/g(x) \to 0 \text{\quad as\quad} x\to\infty,
%\]
%so $x = o(x^2)$ as $x\to\infty$.
%\it 
%If $f(x)=x^2$ and $g(x)=x$, then 
%\[
%f(x)/g(x) \to 0 \text{\quad as\quad} x\to 0,
%\]
%so $x^2 = o(x)$ as $x\to 0$.
%\een
%\end{example}

We will need the following result from elementary analysis:
% lemma
\begin{lemma}\label{lem:standard_identity}
For any constant $c\in\R$,
\[
\left(1+\frac{c}{n}\right)^n\to e^c \quad\text{as}\quad n\to\infty.
\]
\end{lemma}
%\proofomitted
\begin{proof}
By the binomial theorem,
\begin{align*}
\left(1+\frac{c}{n}\right)^n
	& = \sum_{k=1}^n \binom{n}{k}\left(\frac{c}{n}\right)^k \\
	& = \sum_{k=1}^n \frac{n!}{(n-k)!k!}\left(\frac{c^k}{n^k}\right) \\
	& = \sum_{k=1}^n \frac{c^k}{k!}\left(\frac{n(n-1)\ldots(n-k+1)}{n^k}\right) \\
	& = \sum_{k=1}^n \frac{c^k}{k!}\left[1\left(1-\frac{1}{n}\right)\left(1-\frac{2}{n}\right)\cdots\left(1-\frac{k+1}{n}\right)\right] \\
%\Rightarrow \left(1+\frac{c}{n}\right)^n
	& \to \sum_{k=1}^{\infty} \frac{c^k}{k!} = e^c \quad\text{as}\quad n\to\infty.
\end{align*}
%Hence, $\displaystyle \left(1+\frac{c}{n}\right)^n \to \sum_{k=1}^n \frac{c^k}{k!} = e^c \text{\quad as }n\to\infty$.

\end{proof}

%\vspace*{2ex}

\bigskip
We will also need the following analogue of Theorem~\ref{taylor:mgf}, which is a consequence of Taylor's theorem for functions of a complex variable. Here, $o(t^k)$ denotes a quantity with the property that $o(t^k)/t^k\to 0$ in the limit as $t\to 0$, and represents an `error' term that is asymptotically smaller than the other terms of the expression in the limit as $t\to 0$, and which can therefore be neglected when $t$ is sufficiently small. (This is called \emph{Landau notation}.)
% theorem: taylor for functions of complex variable
\begin{theorem}\label{thm:taylor-complex}
If $\expe(|X^k|)<\infty$, then
\[
\phi(t) = \sum_{j=0}^k \frac{\expe(X^j)}{j!}(it)^j + o(t^k) \qquad\text{as}\quad t\to 0,
\]
%In particular, $\phi^{(k)}(0) = i^k\expe(X^k)$.
\end{theorem}
\proofomitted

%\begin{remark}
%In Theorem~\ref{thm:taylor-complex}, the term $o(t^k)$ represents a quantity that tends to zero \emph{faster} than $t^k$ as $t\to 0$. 
%This is known as \emph{Landau notation}: we say that $h(t) = o\big(g(t)\big)$ as $t\to t_0$ whenever $\displaystyle\frac{h(t)}{g(t)}\to 0$ as $t\to t_0$. 
%\end{remark}

%In what follows, we will only use at most the first three terms of the Taylor expansion:
%%\begin{equation}\label{eq:taylor-complex}
%\[
%\phi(t) = 1 + it\mu_1 -\frac{1}{2}t^2\mu_2 + o(t^2) \quad\text{as}\quad t\to 0. 
%\]
%%\end{equation}



%----------------------------------------------------------------------
%\section{The de~Moivre - Laplace theorem}
%----------------------------------------------------------------------
%----------------------------------------------------------------------
\section{Poisson limit theorem}
%----------------------------------------------------------------------
%The following theorem shows that when $n$ is large and $p$ is small, the $\text{Binomial}(n,p)$ distribution can be approximated by the $\text{Poisson}(np)$ distribution.

% theorem
\begin{theorem}[The Poisson limit theorem]
If $X_n\sim\text{Binomial}(n,\lambda/n)$ then the distribution of $X_n$ converges to the $\text{Poisson}(\lambda)$ distribution as $n\to\infty$.
\end{theorem}

% proof
\begin{proof}
By the continuity theorem for characteristic functions, it is enough to show that the characteristic function of $X_n$ converges to the characteristic function of the $\text{Poisson}(\lambda)$ distribution as $n\to\infty$. 

%Recall:
\bit
\it $\text{Binomial}(n,p)$: $$M(t)=(1-p+pe^t)^n \quad\Rightarrow\quad \phi(t)=M(it)=(1-p+pe^{it})^n.$$
\it $\text{Poisson}(\lambda)$: $$M(t)=\exp\big[\lambda(e^t-1)\big] \quad\Rightarrow\quad \phi(t)=M(it)=\exp\big[\lambda(e^{it}-1)\big].$$
\eit
The characteristic function of $X_n\sim\text{Binomial}(n,\lambda/n)$ is 
\[
\phi_n(t) = \expe(e^{itX_n}) 
	= \left(1-\frac{\lambda}{n} + \frac{\lambda}{n}e^{it}\right)^n
	= \left[1 + \frac{\lambda(e^{it}-1)}{n}\right]^n
\]
By Lemma~\ref{lem:standard_identity},
\[
\phi_n(t) \to \exp\big[\lambda(e^{it}-1)\big]\quad\text{as}\quad n\to\infty.
\]
This is the characteristic function of the $\text{Poisson}(\lambda)$ distribution, and the result follows by the continuity theorem for characteristic functions.
\end{proof}

%----------------------------------------------------------------------
\section{Law of large numbers}
%----------------------------------------------------------------------
% theorem
\begin{theorem}
Let $X_1,X_2,\ldots$ be a sequence of i.i.d. random variables with common mean $\mu<\infty$. Then %the sequence of partial sums $S_n = X_1+X_2+\ldots+X_n$ satisfies
\[
\bar{X}_n = \frac{1}{n}\sum_{i=1}^nX_i \to \mu \text{\quad in distribution as\quad} n\to\infty.
\]
\end{theorem}

\bit
\it Unlike Theorem~\ref{thm:wlln}, this result does not require that the $X_i$ have bounded variance.
\it Convergence in distribution is however a weaker property than convergence in probability.
\eit

% proof
\begin{proof}
By the continuity theorem for characteristic functions, it is sufficient to show that the characteristic function of $\bar{X}_n$ converges to the characteristic function of the constant $\mu$ as $n\to\infty$. 

\bit
\it Let $\phi_X$ denote the common characteristic function of the $X_i$.
\it Let $\phi_n$ denote the characteristic function of $\bar{X}_n$.
\eit
By the properties of characteristic functions,
\begin{align*}
\phi_n(t) 
	& = \phi_{\frac{1}{n}(X_1+X_2+\ldots+X_n)}(t) \\
	& = \phi_{(X_1+X_2+\ldots+X_n)}\left(\frac{t}{n}\right)
	= \left[\phi_X\left(\frac{t}{n}\right)\right]^n
\end{align*}
By Theorem~\ref{thm:taylor-complex} (with $k=1$), 
\[
\phi_X(t) = 1 + it\mu + o(t)\quad\text{as}\quad t\to 0,
\]
so by Lemma~\ref{lem:standard_identity}
\[
\phi_n(t) 
	= \left[\phi_X\left(\frac{t}{n}\right)\right]^n
	= \left[1 + \frac{it\mu}{n} + o\left(\frac{t}{n}\right)\right]^n
	\to e^{it\mu} \quad\text{as}\quad n\to\infty.
\]
%Thus by Lemma~\ref{lem:standard_identity},
%\[
%\phi_n(t) 
%	= \left[1 + \frac{it\mu}{n} + o\left(\frac{t}{n}\right)\right]^n \to e^{it\mu} \quad\text{as}\quad n\to\infty.
%\]
This is the characteristic function of the constant $\mu$, and the result follows by the continuity theorem for characteristic functions.
\end{proof}

%----------------------------------------------------------------------
\section{Central limit theorem}
%----------------------------------------------------------------------
Let $X_1,X_2,\ldots$ be i.i.d. random variables, and consider the partial sums
\[
S_n = X_1 + X_2 + \ldots + X_n.
\]
%\bit
%\it 
By independence, $\expe(S_n)=n\mu$ and $\var(S_n)=n\sigma^2$.
%\it
%The law of large numbers says that $S_n$ is approximately equal to its mean $n\mu$ for large $n$.
%\eit

\bigskip
The central limit theorem says that, \emph{irrespective of the distribution of the $X_i$}, the distribution of the standardised variables
\[
S^{*}_n = \frac{S_n-\expe(S_n)}{\sqrt{\var(S_n)}} = \frac{S_n - n\mu}{\sigma\sqrt{n}}
\]
%It turns out that, \emph{irrespective of the distribution of the $X_i$}, the distribution of $S^{*}_n$ 
converges to the standard normal distribution as $n\to\infty$.

% theorem
\begin{theorem}[Central limit theorem]
Let $X_1,X_2,\ldots$ be a sequence of independent and identically distributed with common mean $\mu$ and variance $\sigma^2$. If $\mu$ and $\sigma^2$ are both finite, then the distribution of the normalised sums
\[
S^{*}_n = \frac{S_n - n\mu}{\sigma\sqrt{n}}\qquad\text{where}\qquad S_n=X_1+\ldots+X_n,
\]
converges to the standard normal distribution $\text{N}(0,1)$ as $n\to\infty$.
\end{theorem}

\begin{proof}
Let $\displaystyle Y_i = \frac{X_i-\mu}{\sigma}$. Then $\expe(Y_i)=0$ and $\var(Y_i)=1$, and 
\[
S^{*}_n  = \frac{1}{\sqrt{n}}\sum_{i=1}^n Y_i %= \frac{Y_1+\ldots+Y_1}{\sqrt{n}} %	= \frac{S_n - n\mu}{\sigma\sqrt{n}}
\]

\bit
\it Let $\phi_Y(t)$ denote the common characteristic function of the $Y_i$.
\it Let $\phi_n(t)$ denote the characteristic function of $S^{*}_n$.
\eit
By Taylor's theorem, if $\expe(|Y^k|)<\infty$ we have that 
\[
\phi(t) = \expe(e^{itY}) = \sum_{j=0}^k \frac{\expe(Y^j)}{j!}(it)^j + o(t^k) \qquad\text{as}\quad t\to 0.
\]

Since $\expe(Y^2) = \var(Y) + \expe(Y)^2 = 1$ is finite, we can apply this with $k=2$ to obtain
\[
\phi_Y(t) 
	=  1 - \frac{1}{2}t^2 + o(t^2)
	\quad\text{as}\quad t\to 0
\]
By the properties of characteristic functions, 
\begin{align*}
\phi_n(t)
	& = \phi_{\frac{1}{\sqrt{n}}(Y_1+Y_2+\ldots+Y_n)}(t) \\
	& = \phi_{Y_1+Y_2+\ldots+Y_n}\left(\frac{t}{\sqrt{n}}\right) \\
	& = \left[\phi_Y\left(\frac{t}{\sqrt{n}}\right)\right]^n \\
	& = \left[1 - \frac{t^2}{2n} + o\left(\frac{t^2}{n}\right)\right]^n \\
	& \to e^{-\frac{1}{2}t^2} \quad\text{as}\quad n\to\infty,
\end{align*}
where the last step follows by Lemma~\ref{lem:standard_identity}.
%Finally, using the fact that for any $c\in\R$,
%\[
%\left(1+\frac{c}{n}\right)^n\to e^c \quad\text{as}\quad n\to\infty.
%\]
%we see that %\left(1+\frac{c}{n}\right)^n\to e^c$ as $n\to\infty$ for any $c\in\R$, we see that
%%\begin{align*}
%%\phi_n(t)
%%	& = \left[1 - \frac{t^2}{2n} + o\left(\frac{t^2}{n}\right)\right]^n \quad\text{as}\quad n\to\infty \\
%%	& \to e^{-\frac{1}{2}t^2} \quad\text{as}\quad n\to\infty.
%%\end{align*}
%\[
%\phi_n(t)
%	= \left[1 - \frac{t^2}{2n} + o\left(\frac{t^2}{n}\right)\right]^n 
%	\to e^{-\frac{1}{2}t^2} \quad\text{as}\quad n\to\infty.
%\]
This is the characteristic function of the $\text{N}(0,1)$ distribution, and the result follows by the continuity theorem for characteristic functions.
\end{proof}

% example: (Erlang)
\begin{example}[Erlang Distribution]
The \emph{Erlang distribution} with parameters $k\in\N$ and $\lambda>0$ is defined to be the sum of $k$ independent and identically distributed random variables $X_1,X_2,\ldots,X_k$, where each $X_i$ is exponentially distributed with (rate) parameter $\lambda$. Show that if $Y\sim\text{Erlang}(k,\lambda)$, then the random variable $$Z_k=\displaystyle\frac{\lambda Y-k}{\sqrt{k}}$$ has approximately the standard normal distribution when $k$ is large.
\end{example}

\begin{solution}
Let $Y\sim\text{Erlang}(k,\lambda)$. Then $Y$ can be written as the sum of $k$ independent and identically distributed random variables $X_i$:
\[
Y = X_1 + X_2 + \ldots + X_k\qquad\text{where}\quad X_i\sim\text{Exponential}(\lambda).
\]
Since $X_i\sim\text{Exponential}(\lambda)$ with $\lambda > 0$, we have 
\[
\expe(X_i) = \frac{1}{\lambda} <\infty
\text{\quad and\quad}
\var(X_i) = \frac{1}{\lambda^2} <\infty.
\]

Furthermore, by independence we have
\[
\expe(Y)= \sum_{i=1}^k \expe(X_i) = \frac{k}{\lambda}
\qquad\text{and}\qquad
\var(Y) =  \sum_{i=1}^k \var(X_i) = \frac{k}{\lambda^2}.
\]

Let $Z\sim\text{N}(0,1)$. By the central limit theorem,
\[
\frac{Y - \expe(Y)}{\sqrt{\var(Y)}} \to Z \quad\text{in distribution as $k\to\infty$.}
\]
%\[
%\frac{Y - k\expe(X)}{\sqrt{k\var(X)}} \xrightarrow{D} \text{N}(0,1) \quad\text{as}\quad k\to\infty.
%\]
i.e.\
\[
Z_k = \frac{Y - \expe(Y)}{\sqrt{\var(Y)}} 
	= \frac{Y - k/\lambda}{\sqrt{k/\lambda^2}} 
	= \frac{\lambda Y - k}{\sqrt{k}} 
	\to Z \quad\text{in distribution as $k\to\infty$.}
\]
\end{solution}


%\newpage
%----------------------------------------------------------------------
\section{Exercises}
% !TEX root = main.tex
%----------------------------------------------------------------------
\begin{exercise}
\begin{questions}
%----------------------------------------
%==========================================================================
\question
The continuous uniform distribution on $(a,b)$ has the following PDF:
\[
f(x) = \begin{cases}
	\displaystyle\frac{1}{b-a} 	& a < x < b,  \\[2ex]
	0				& \text{otherwise.}
\end{cases}
\]
Use the central limit theorem to deduce the approximate distribution of the sample mean of $n$ independent observations from this distribution when $n$ is large.
\begin{answer} 
The mean is 
\begin{align*}
\mu 
	& = \int_{a}^{b}\frac{x}{b-a}\,dx = \frac{a+b}{2},
\intertext{and the second moment is}
\mu_{2}
	& = \int_{a}^{b}\frac{x^{2}}{b-a}\,dx = \frac{a^{2}+ab+b^{2}}{3}, 
\intertext{so the variance is}	
\sigma^{2} 
	& = \expe(X^{2}) - \expe(X)^{2} = \frac{(b-a)^{2}}{12}
\end{align*}
By the central limit theorem, if $X$ is a random variable with mean $\mu$ and variance $\sigma^{2}$, the distribution of the sample mean $\bar{X}$ of a random sample of $n$ independent observations is approximately $N(\mu,\frac{\sigma^{2}}{n})$, the approximation being better for larger $n$. In this case, the approximate distribution of $\bar{X}$ is $N\left(\frac{a+b}{2},\frac{(b-a)^{2}}{12n}\right)$. 
\end{answer}


%==========================================================================
\question
The exponential distribution with scale parameter $\theta>0$ has the following PDF:
\[
f(x) = \begin{cases}
	\displaystyle\frac{1}{\theta} e^{-x/\theta} 	& x > 0,  \\[2ex]
	0					& \text{otherwise.}
\end{cases}
\]
Use the central limit theorem to deduce the approximate distribution of the sample mean of $n$ independent observations from this distribution when $n$ is large.
\begin{answer} % <<<
\begin{align*}
\expe(X) 
	& = \frac{1}{\theta}\int_{0}^{\infty} x e^{-x/\theta}\,dx = \theta, \\
\expe(X^2)
	& = \frac{1}{\theta}\int_{0}^{\infty} x^2 e^{-x/\theta}\,dx  = 2\theta^2 \\
\var(Y)
	& = \expe(X^2) - \expe(X)^{2} = \theta^2.
\end{align*}

By the CLT, the distribution of $\bar{X}$ is approximately $\text{N}(\mu,\frac{\sigma^{2}}{n})$, the approximation being better for larger $n$. In this case, the approximate distribution of $\bar{X}$ is $\text{N}\left(\theta,\theta^2/n\right)$. 
\end{answer}


%==========================================================================
\question
Let $X\sim\text{Binomial}(n_1,p_1)$ and $X_2\sim\text{Binomial}(n_2,p_2)$ be independent random variables.
\ben
\it
Use the central limit theorem to find the approximate distribution of $Y = X_1 - X_2$ when $n_1$ and $n_2$ are both large.
\it
Let $Y_1 = X_1/n_1$ and $Y_2 = X_2/n_2$. Show that $Y_1 - Y_2$ is approximately normally distributed with mean $p_1 - p_2$ and variance $\frac{p_1q_1}{n_1} +\frac{p_2q_2}{n_2}$ when $n_1$ and $n_2$ are both large.
\it
Show that when $n_1$ and $n_2$ are both large,
\[
\frac{(Y_1 -Y_2 )-(p_1 -p_2 )}{\sqrt{\frac{p_1 (1-p_1 )}{n_1 } +\frac{p_2 (1-p_2 )}{n_2 } } } 
	 \sim N(0,1) \qquad\text{approx.}
\]
\een

\begin{answer}
\ben
\it
The mean and variance of $X_1$ are respectively $n_1p_1$ and $n_1p_1q_1$ where $q_1=1-p_1$. The mean and variance of $X_2$ are respectively $n_2p_2$ and $n_2p_2q_2$ where $q_2=1-p_2$. Since $Y = X_1 - X_2$ is a linear combination of random variables,
\[
\expe(Y) = \expe(X_1)-\expe(X_2) = n_1p_1- n_2p_2
\]
and since $X_1$ and $X_2$ are independent,
\[
\var(Y) = \var(X_1) + \var(X_2) = n_1p_1q_1 + n_2p_2q_2.
\]
Because both $X_1$ and $X_2$ are the sums of Bernoulli random variables, the CLT applies, so the approximate distribution of $Y$ is
\[ 
Y\sim N(n_1p_1- n_2p_2, n_1p_1q_1 + n_2p_2q_2)
\]
\it
For large $n_1$ and $n_2$, by the CLT the distribution of $X_1$ is approximately $N(n_1p_1,n_1p_1(1 - p_1))$ for $n_1$ large, and the distribution of $X_2$ is approximately $N(n_2p_2,n_2p_2(1 - p_2))$ for $n_2$ large. Thus the distributions of $Y_1$ and $Y_2$ are approximately $N\left(p_1,\frac{p_1q_1}{n_1}\right)$ and $N\left(p_2,\frac{p_2q_2}{n_2}\right)$ respectively, and the distribution of $Y_1 - Y_2$ is therefore approximately $N\left(p_1-p_2, \frac{p_1q_1}{n_1}+\frac{p_2q_2}{n_2}\right)$ for large $n_1$ and $n_2$.
\it
The usual standardization for the normal distribution (subtract the mean and divide by the standard deviation) yields the result. This is used in devising approximate tests and confidence intervals for the difference of proportions.
\een
\end{answer}


%==========================================================================
\question
5\% of items produced by a factory production line are defective. Items are packed into boxes of 2000 items. As part of a quality control exercise, a box is chosen at random and found to contain 120 defective items. Use the central limit theorem to estimate the probability of finding at least this number of defective items when the production line is operating properly.
\begin{answer} % <<<
Let $X$ be the number of defective items in a box. Then $X\sim\text{Binomial}(n,p)$ with $n=2000$ and $p=0.05$. Since $n$ is large, $X$ has approximately normal distribution with mean equal to $np(1-p)=100$, and variance equal to $npq=95$. The standardized variable $Z=(X-100)/\sqrt{95}$ has therefore approximately the standard normal distribution $N(0,1)$.
Thus
\[
\prob(X\geq 120) = \prob\left(Z\geq\frac{120 - 100}{\sqrt{95}}\right) = \prob(Z\geq 2.0520) \approx 0.0202
\]
where the probability $\prob(Z\geq 2.0520)\approx 0.0202$ can be obtained from statistical tables.
\end{answer}


%==========================================================================
\question
Use the central limit theorem to prove the law of large numbers.
\begin{answer} % <<<
Let $X_1,X_2,\ldots$ be a sequence of i.i.d. random variables, and define $S_n=\sum_{i=1}^n X_i$. To prove the (weak) law of large numbers, we need to show that
\[
\prob\left(\left|\frac{S_n}{n}-\mu\right|\geq\epsilon\right) \to 0 \qquad\text{as}\quad n\to\infty
\]
Now,
\[
\prob\left(\left|\frac{S_n}{n}-\mu\right|\geq\epsilon\right)
	= \prob\left(\left|\frac{S_n-n\mu}{\sigma\sqrt{n}}\right|\geq\frac{n\epsilon}{\sigma\sqrt{n}}\right)
	= \prob\left(\left|\frac{S_n-n\mu}{\sigma\sqrt{n}}\right|\geq\frac{\sqrt{n}\epsilon}{\sigma}\right)
\]
By the central limit theorem, $\displaystyle\frac{S_n-n\mu}{\sigma\sqrt{n}}$ is approximately distributed according to $N(0,1)$, so this probability is approximated by the area under the standard normal curve between $\displaystyle\frac{\sqrt{n}\epsilon}{\sigma}$ and infinity, which approaches zero as $n\to\infty$.
\end{answer}


%==========================================================================
\question
We perform a sequence of independent Bernoulli trials, each with probability of success $p$, until a fixed number $r$ of successes is obtained. The total number of failures $Y$ (up to the $r$th succes) has the \emph{negative binomial} distribution with parameters $r$ and $p$, so the PMF of $Y$ is
\[
\prob(Y=k) = \binom{k+r-1}{k} (1-p)^k p^r,\qquad k=0,1,2,\ldots
\]
Using the fact that $Y$ can be written as the sum of $r$ independent geometric random variables, show that this distribution can be approximated by a normal distribution when $r$ is large.

\begin{answer}
If $Y\sim\text{NB}(r,p)$, we can write
\[
Y = X_1 + X_2 + \ldots + X_r\qquad\text{where}\quad X_i\sim\text{Geometric}(p).
\]
Let $X\sim\text{Geometric}(p)$. Since $\var(X)<\infty$, it follows by the central limit theorem that 
\[
\frac{Y - r\expe(X)}{\sqrt{r\var(X)}} \to \text{N}(0,1) \quad\text{in distribution as }r\to\infty.
\]
In fact, since $\expe(X)=(1-p)/p$ and $\var(X)=(1-p)/p^2$, we see that $Y$ can be approximated by the 
$\displaystyle\text{N}\left(\frac{r(1-p)}{p}, \frac{r(1-p)}{p^2}\right)$ distribution as $r\to\infty$.
\end{answer}

%----------------------------------------
\end{questions}
\end{exercise}
%----------------------------------------------------------------------

%======================================================================
\endinput
%======================================================================

%----------------------------------------------------------------------

%======================================================================
\endinput
%======================================================================
