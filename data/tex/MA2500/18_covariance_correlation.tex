% !TEX root = main.tex
%----------------------------------------------------------------------
\chapter{Covariance and Correlation}\label{chap:cov}
%----------------------------------------------------------------------
%----------------------------------------------------------------------
\section{Bivariate Distributions}
%----------------------------------------------------------------------
\begin{definition}
Let $X,Y:\Omega\to\R$ be two random variables defined on the same probability space, let $F_{X,Y}$ denote their joint CDF, and let $g:\R^2\to\R$ be a (Borel) measureable function on $\R^2$. Then% \emph{expected value} of $g(X,Y):\Omega\to\R$ is given by 
\[
\expe\big[g(X,Y)\big] = \iint g(x,y)\,dF(x,y)
\]
whenever this integral exists. In particular,
\ben
\it % << discrete
if $X$ and $Y$ are jointly discrete, with joint PMF $f_{X,Y}(x,y)$, then 
\[
\expe\big[g(X,Y)\big] = \sum_{x,y} g(x,y)f_{X,Y}(x,y)
\]
whenever this sum exists, and
\it % << cts
if $X$ and $Y$ are jointly continuous, with joint PDF $f_{X,Y}(x,y)$, then
\[
\expe\big[g(X,Y)\big] = \iint g(x,y)f_{X,Y}(x,y)\,dx\,dy
\]
whenever this integral exists.
\een
\end{definition}

%----------------------------------------------------------------------
\section{Covariance}
%----------------------------------------------------------------------
% defn: product moment
\begin{definition}
The \emph{product moment} of $X$ and $Y$ is defined to be
\[
\expe(XY) = \iint xy\,dF(x,y)
\]
whenever this integral exists. In particular, 
\ben
\it % << discrete
if $X$ and $Y$ are jointly discrete, with joint PMF $f_{X,Y}(x,y)$, then
\[
\expe(XY) = \sum_{x,y} xy\, f_{X,Y}(x,y)
\]
whenever this sum is absolutely convergent, and
\it % << cts
if $X$ and $Y$ are jointly continuous, with joint PDF $f_{X,Y}(x,y)$, then
\[
\expe(XY) = \iint xy\, f_{X,Y}(x,y)\,dx\,dy
\]
whenever this integral is absolutely convergent.
\een
\end{definition}
%
%\end{definition}
%
%In particular
%\[
%\expe(XY) = 
%\left\{\begin{array}{ll}
%	\displaystyle\sum_{x,y} xy f_{X,Y}(x,y)		& \text{if $X$ and $Y$ are jointly discrete}, \\[4ex]
%	\displaystyle\int xy f_{X,Y}(x,y)\,dxdy		& \text{if $X$ and $Y$ are jointly continuous}.
%\end{array}\right.
%\]



% defn: product moment
\begin{definition}
\ben
\it The \emph{covariance} of $X$ and $Y$ is 
\begin{align*}
\cov(X,Y) 	& = \expe\left[(X-\expe X)(Y-\expe Y)\right] \\
			& = \expe(XY) - \expe(X)\expe(Y)
\end{align*}
\it The \emph{correlation coefficient} of $X$ and $Y$ is 
\[
\rho(X,Y) = \frac{\cov(X,Y)}{\sqrt{\var(X)\cdot\var(Y)}}
\]
\een
\end{definition}

% remark
\begin{remark}
\bit 
\it $\cov(X,Y)$ is the product moment of the \emph{centred} variables $X-\expe(X)$ and $Y-\expe(Y)$.
%\it $\cov(X,Y)$ is the product moment of the \emph{centred} variables $X-\expe X$ and $Y-\expe Y$.
\it $\rho(X,Y)$ is the product moment of the \emph{standardized} variables $\displaystyle\frac{X-\expe(X)}{\sqrt{\var(X)}}$ and $\displaystyle\frac{Y-\expe(Y)}{\sqrt{\var(Y)}}$.
\eit
\end{remark}


\begin{remark}[Variance of sums of random variables]
For any random variables $X_1,X_2,\ldots,X_n$,
\[
\var\left(\sum_{i=1}^n X_i\right) = \sum_{i=1}^n \var(X_i) + 2\sum_{i=1}^{n}\sum_{j=i+1}^n \cov(X_i,X_j).
\]
\end{remark}

%% theorem: bilinearity of covariance
%\begin{theorem}[Bilinearity of covariance]
%\[
%%\cov(a_1X_1+a_2X_2,b_1Y_1+b_2Y_2) = a_1b_1\cov(X_1,Y_1) + a_1b_2\cov(X_1,Y_2) + a_2b_1\cov(X_2,Y_1) + a_2b_2\cov(X_2,Y_2).
%\cov(aX_1+bX_2,cY_1+dY_2) = ac\cov(X_1,Y_1) + ad\cov(X_1,Y_2) + cb\cov(X_2,Y_1) + cd\cov(X_2,Y_2).
%\]
%\end{theorem}
%\begin{proof}
%Exercise.
%\end{proof}
%----------------------------------------------------------------------
\section{Correlation}
%----------------------------------------------------------------------
Correlation quantifies the (linear) dependence between random variables.

% definition: correlation 
\begin{definition} 
Two random variables $X$ and $Y$ are said to be \emph{correlated} if $\expe(XY)\neq\expe(X)\expe(Y)$.%, otherwise they are said to be \emph{uncorrelated}.
\end{definition}

% lemma: independent implies uncorrelated
\begin{lemma}
If $X$ and $Y$ are independent, they are uncorrelated.% (so $\expe(XY)=\expe(X)\expe(Y)$).
\end{lemma}

We prove the lemma only for discrete random variables (the continuous case is similar).
\vspace{0ex}
% proof
\begin{proof}
Since $X$ and $Y$ are independent, $f_{X,Y}(x,y)=f_X(x)f_Y(y)$ for all $x,y\in\R$. Hence
\begin{align*}
\expe(XY) = \sum_{x,y} xy\, f_{X,Y}(x,y)
	& = \sum_{x,y} xy\, f_X(x)f_Y(y) \text{\qquad(by independence),}\\
	& = \left(\sum_x x\, f_X(x)\right)\left(\sum_y y\, f_Y(y)\right)
	= \expe(X)\expe(Y).
\end{align*}
%\vspace*{-2ex}
\end{proof}

% theorem: var_of_sum_is_sum_of_var
\begin{theorem}
If $X$ and $Y$ are uncorrelated, $\var(X+Y) = \var(X) + \var(Y)$.
\end{theorem}

% proof
\begin{proof}
Let $X$ and $Y$ be uncorrelated. Then $\expe(XY)=\expe(X)\expe(Y)$, so
\begin{align*}
\var(X+Y)	& = \expe\left(\big[X+Y - \expe(X+Y)\big]^2\right) \\
			& = \expe\left(\big[X-\expe(X)\big]^2 + 2\big[XY - X\expe(Y) - Y\expe(X) + \expe(X)\expe(Y)\big] + \big[Y-\expe(Y)\big]^2\right) \\
			& = \var(X) + 2\left[\expe(XY)-\expe(X)\expe(Y)\right] + \var(Y) \\
			& = \var(X) + \var(Y).
\end{align*}
\end{proof}



% example: negative binomial
\begin{example}
Let $Y_1,\ldots,Y_r$ be independent and identically distributed random variables, with each $Y_i\sim\text{Geometric}(p)$. Find the mean and variance of their sum $X = \sum_{i=1}^r Y_i$.
\begin{solution}
Since $Y_i\sim\text{Geometric}(p)$, we know that $\expe(Y_i)=\displaystyle\frac{1-p}{p}$ and $\var(Y_i)=\displaystyle\frac{1-p}{p^2}$. Hence by the linearity of expectation,
\[
\expe(X) = \expe(Y_1)+\expe(Y_2)+\ldots+\expe(Y_r) = \frac{r(1-p)}{p}
\]
and because the $Y_i$ are independent,
\[
\var(X) = \var(Y_1)+\var(Y_2)+\ldots+\var(Y_r) = \frac{r(1-p)}{p^2}
\]
\textbf{Remark}: In this case, $X$ has the \emph{negative binomial} distribution, with parameters $r$ and $p$.
\end{solution}
\end{example}

%----------------------------------------------------------------------

\section{The Cauchy-Schwarz Inequality}
%----------------------------------------------------------------------
% lemma
\begin{lemma}\label{lem:pos_rv_expe_zero}
If $X\geq 0$ and $\expe(X)=0$ then $\prob(X=0)=1$.
\end{lemma}

\proofomitted
%\begin{proof}
%Let $X\geq 0$ with $\expe(X)=0$, and assume that $\prob(X>0)>0$. 
%\bit
%\it Because $\prob(X\leq x)$ is right-continuous, there exists $\epsilon>0$ such that $\prob(X>\epsilon)>0$. 
%\it This implies that $X\geq \epsilon I(X>\epsilon)$.
%\it Taking the expected value of both sides, $\expe(X)\geq \epsilon\,\prob(X>\epsilon) > 0$ (by monotonicity). 
%\eit
%This is a contradiction, so we conclude that $\prob(X>0)=0$.
%\end{proof}

% theorem: Cauchy-Schwarz Inequality
\begin{theorem}[Cauchy-Schwarz inequality]
For any two random variables $X$ and $Y$,
\[
\expe(XY)^2 \leq \expe(X^2)\expe(Y^2)
\]
with equality if and only if $\prob(Y=aX)=1$ for some $a\in\R$.
\end{theorem}

\begin{proof}
By Lemma~\ref{lem:pos_rv_expe_zero}, $X^2$ and $Y^2$ are non-negative random variables, so we can assume that $\expe(X^2)>0$ and $\expe(Y^2)>0$ (otherwise both sides of the inequality are identically zero).

%Let $a,b\in\R$ and consider the random variable $aX+bY$. By properties of expectation, 
%\begin{align*}
%(aX+bY)^2\geq 0 
%	& \Rightarrow \expe\left((aX+bY)^2\right) \geq 0 \\
%	& \Rightarrow a^2\expe(X^2) + 2ab\expe(XY) + b^2\expe(Y^2) \geq 0 \\
%\end{align*}
%The left-hand side of the last inequality is a quadratic expression in the variable $a$. Since the expression is always positive, the quadratic has at most one real root. Thus its discriminant $4b^2\expe(XY)-4b^2\expe(X)\expe(Y)$ must be non-positive, so provided $b\neq 0$ we have that
%\[
%\expe(XY)^2 - \expe(X^2)\expe(Y^2) \leq 0
%\]
Let $a\in\R$ consider the random variable $Z=aX-Y$. 
\par
By the properties of expectation,
\begin{align*}
Z^2\geq 0 \text{ for all } a\in\R
	& \Rightarrow \expe(Z^2)\geq 0 \text{ for all } a\in\R \\
	& \Rightarrow \expe(a^2X^2 - 2aXY + Y^2) \geq 0 \text{ for all } a\in\R \\
	& \Rightarrow a^2\expe(X^2) - 2a\expe(XY) + \expe(Y^2) \geq 0 \text{ for all } a\in\R.
\end{align*}

\bit
\it Let $h(a)=a^2\expe(X^2) - 2a\expe(XY) + \expe(Y^2)$. This is a quadratic expression in $a$.
\eit
Since $h(a)\geq 0$ for all $a\in\R$, the roots of the quadratic equation $h(a)=0$, given by
\[
a = \frac{\expe(XY)\pm\sqrt{\expe(XY)^2-\expe(X^2)\expe(Y^2)}}{\expe(X^2)}
\]
are either both complex (discriminant is negative) or co-incide (discriminant is zero). 
\bit
\it Hence $\expe(XY)^2-\expe(X^2)\expe(Y^2) \leq 0$, or equivalently, $\expe(XY)^2\leq \expe(X^2)\expe(Y^2)$.
\eit

Finally, the discriminant is zero if and only if the quadratic has real roots, which occurs if and only if $\expe(Z^2)=0$ for some $a\in\R$, i.e.
\[
\expe\left((aX-Y)^2\right) = 0\qquad\text{for some $a\in\R$.}
\]
Hence, because $(aX-Y)^2$ is a non-negative random variable, we have by Lemma~\ref{lem:pos_rv_expe_zero} that
\begin{align*}
\expe\big[(aX-Y)^2\big]=0
		& \quad\Rightarrow\quad \prob\big[(aX-Y)^2 = 0\big] = 1 
		\quad\Rightarrow\quad \prob\big(Y = aX) = 1. 
\end{align*}		
\end{proof}


% corollary: correlation coefficient
\begin{corollary}
The correlation coefficient satisfies the inequality 
\[
|\rho(X,Y)|\leq 1,
\]
with equality if and only if $\prob(Y=aX+b)=1$ for some $a\in\R$.
\end{corollary}

% proof
\begin{proof}
Apply the Cauchy-Schwarz inequality to $X-\expe X$ and $Y-\expe Y$:
\begin{align*}
\cov(X,Y)^2 
	& =		\expe\big((X-\expe X)(Y-\expe Y)\big) \\
	&\leq 	\expe\big((X-\expe X)^2\big)\expe\big((Y-\expe Y)^2\big) 
	= 	\var(X)\var(Y),
\end{align*}
with equality if and only if there exists $a\in\R$ such that
\[
\prob\big[Y-\expe Y = a(X-\expe X)] = 1.
%\prob\big[Y-\expe Y = a(X-\expe X)] = \prob\big[Y = aX + (\expe Y - a\expe X)\big]= 1.
%\prob(Y = aX + b) = 1 \text{\qquad where\qquad b = \expe Y - a\expe X)\big]= 1.
\]
Hence,
\[	
|\rho(X,Y)| = \left|\frac{\cov(X,Y)}{\sqrt{\var(X)\var(Y)}}\right| \leq 1
\]
with equality if and only if $\prob(Y = aX + b) = 1$,  where $b = \expe Y - a\expe X$.
\end{proof}


%----------------------------------------------------------------------
\section{Exercises}
% !TEX root = main.tex
%----------------------------------------------------------------------
\begin{exercise}
\begin{questions}
%----------------------------------------
%==========================================================================
\question
Let $X$ and $Y$ be two random variables having the same distribution but which are not necessarily independent. Show that
\[
\cov(X+Y,X-Y)=0
\]
provided that their common distribution has finite mean and variance.

\begin{answer}
Perhaps the simplest method is the following: let $U=X+Y$ and $V=X-Y$. Then 
\begin{align*}
\cov(X+Y,X-Y) 
	& = \expe(UV) - \expe(U)\expe(V) \\
	& = \expe\big[(X+Y)(X-Y)\big] - \expe(X+Y)\expe(X-Y) \\
	& = \expe(X^2 - Y^2) - \big[\expe(X)+\expe(Y)\big]\big[\expe(X)-\expe(Y)\big]	\qquad\text{(by the linearity of expectation)} \\
	& = \expe(X^2) - \expe(Y^2) - \expe(X)^2 +\expe(X)\expe(Y) - \expe(Y)\expe(X) + \expe(Y)^2 \quad\text{(by linearity again)} \\
	& = \big[\expe(X^2) - \expe(X)^2\big] - \big[\expe(Y^2) - \expe(Y)^2\big] \\
	& = \var(X) - \var(Y).
\end{align*}
Since $X$ and $Y$ have the same distribution, their variances must be equal, so $\cov(X+Y,X-Y)=0$.
\end{answer}



%==========================================================================
\question
Consider a fair six-sided die whose faces show the numbers $-2,0,0,1,3,4$. The die is independently rolled four times. Let $X$ be the average of the four numbers that appear, and let $Y$ be the product of these four numbers. Compute $\expe(X)$, $\expe(X^2)$, $\expe(Y)$ and $\cov(X,Y)$.
\begin{answer}
Let $X_1,X_2,X_3,X_4$ be independent discrete random variables on the set $\{-2,0,0,1,3,4\}$. Each $X_i$ is identically distributed according to the following PMF:
\[\begin{array}{|c|ccccc|}\hline
k			& -2		& 0		& 1		& 3		& 4 	\\ \hline
\prob(X_i=k)		& 1/6	& 1/3	& 1/6	& 1/6 	& 1/6	\\ \hline
\end{array}\]
Hence for $i=1,2,3,4$,
\begin{align*}
\expe(X_i)		& = \frac{1}{6}(-2+0+0+1+3+4) = 1, \\
\expe(X_i^2)	& = \frac{1}{6}(4+0+0+1+9+16) = \frac{30}{6} = 5, \\
\var(X_i)		& = \expe(X_i^2)-\expe(X_i)^2 = 4.
\end{align*}

Let $X=\frac{1}{4}(X_1+X_2+X_3+X_4)$. Then
\[
\expe(X) 	= \frac{1}{4}\big(\expe(X_1)+\ldots+\expe(X_4)\big) = 1
\]
By independence,
\[
\var(X) 	= \frac{1}{16}\big(\var(X_1)+\ldots+\var(X_4)\big) = 1 
\]
so $\expe(X^2)	= \var(X) + \expe(X)^2 = 2$.

and $Y=X_1 X_2 X_3 X_4$. By independence,
\begin{align*}
\expe(Y) 
	& = \expe(X_1 X_2 X_3 X_4) \\
	& = \expe(X_1)\expe(X_2)\expe(X_3)\expe(X_4) \\
	& = 1 \\
\end{align*}
and because the $X_i$ are identically distributed,
\begin{align*}
\expe(XY) 
	& = \frac{1}{4}\expe\big(\expe(X_1+X_2+X_3+X_4)X_1X_2X_3X_4\big) \\
	& = \expe(X_1^2)\expe(X_2)\expe(X_3)\expe(X_4) \\
	& = \expe(X_1^2) = 5 \\
\end{align*}
so $\cov(XY) = \expe(XY) - \expe(X)\expe(Y) = 4$.
\end{answer} 

%==========================================================================
\question
A fair die is rolled twice. Let $U$ denote the number obtained on the first roll, let $V$ denote the number obtained on the second roll, let $X=U+V$ denote their sum and let $Y=U-V$ denote their difference. Compute the mean and variance of $X$ and $Y$, and compute $\expe(XY)$. Check whether $X$ and $Y$ are uncorrelated. Check whether $X$ and $Y$ are independent.
\begin{answer}
Let $U,V\sim\text{Uniform}\{1,2,3,4,5,6\}$ be independent (and identically distributed) random variables, and define $X=U+V$ and $Y=U-V$. 
\begin{align*}
\expe(X)	& = \expe(U) + \expe(V) = 7 \\
\expe(Y)	& = \expe(U) - \expe(V) = 0 \\
\intertext{By independence,}
\var(X)		& = \var(U) + \var(V) = 35/6 \\
\var(Y)		& = \var(U) + \var(V) = 35/6 \\
\intertext{Because $U$ and $V$ are identically distributed, and}
XY 			& = (U+V)(U-V) = U^2 - V^2 \\
\intertext{it follows that}
\expe(XY)	& = \expe(U^2) - \expe(V^2) = 0 \\
\intertext{$X$ and $Y$ are uncorrelated, since}
\cov(X,Y)	& = \expe(XY) - \expe(X)\expe(Y) =0 \\
\intertext{However $X$ and $Y$ are not independent, because (for example)}
\prob(Y=0)		& \neq \prob(Y=0|X=12)=1
\end{align*}
\end{answer} 



%----------------------------------------
\end{questions}
\end{exercise}
%----------------------------------------------------------------------

%======================================================================
\endinput
%======================================================================

%----------------------------------------------------------------------

%======================================================================
\endinput
%======================================================================
