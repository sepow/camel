% !TEX root = main.tex
%----------------------------------------------------------------------
\begin{exercise}
\begin{questions}
%----------------------------------------
\question
Let $(\Omega,\mathcal{F},\prob)$ be a probability space, and let $0\leq X_1\leq X_2 \leq \ldots$ be an increasing sequence of non-negative random variables over $(\Omega,\mathcal{F})$ such that 
$X_n(\omega) \uparrow X(\omega)$ ans $n\to\infty$ for all $\omega\in\Omega$.
%\[
%X_n(\omega) \uparrow X(\omega)\quad\text{as}\quad n\to\infty\quad\text{for all}\quad\omega\in\Omega.
%\]
Show that $X$ is a random variable on $(\Omega,\mathcal{F})$.
\begin{answer}
Let $x\in\R$. Since the $X_n$ are random variables, we have (by definition) that $\{X_n \leq x\}\in\mathcal{F}$ for every $n\in\N$. Since $\mathcal{F}$ is closed under countable intersections, 
\[
\{X\leq x\} = \bigcap_{n=1}^{\infty} \{X_n \leq x\} \in\mathcal{F}
\]
so $X$ is a random variable.
\end{answer}

%----------------------------------------
\question
Let $X$ be an integrable random variable. Show that $|\expe(X)|\leq \expe(|X|)$.
\begin{answer}
Since $|X|=X^{+}+X^{-}$, by the triangle inequality
\[
|\expe(X)| = |\expe(X^{+})-\expe(X^{-})| \leq \expe(X^{+}) + \expe(X^{-}) = \expe(|X|),
\]
\it If $X\leq Y$ then $X^{+}\leq Y^{+}$ and $X^{-}\geq Y^{-}$ so
\[
\expe(X) = \expe(X^{+})-\expe(X^{-}) \leq \expe(Y^{+})-\expe(Y^{-}) = \expe(Y),
\]
\end{answer}

%==========================================================================
\question
Let $X$ and $Y$ be integrable random variables. Show that $aX+bY$ is integrable.
\begin{answer}
To show that $aX+bY$ is integrable, first we have by the triangle inequality that
\[
|aX+bY| \leq |a||X| + |b||Y|.
\]
By the linearity and monotonicity of expectation for non-negative random variables,
\[
\expe(|aX+bY|) \leq |a|\expe(|X|) + |b|\expe(|Y|)
\]
and since $\expe(|X|)<\infty$ and $\expe(|Y|)<\infty$, it follows that $\expe(|aX+bY|)<\infty$, so $aX+bY$ is integrable.
\end{answer}

%----------------------------------------
\end{questions}
\end{exercise}
%----------------------------------------------------------------------

%======================================================================
\endinput
%======================================================================
