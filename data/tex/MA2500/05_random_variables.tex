% !TEX root = main.tex
%----------------------------------------------------------------------
\chapter{Random Variables}\label{chap:random_variables}
%----------------------------------------------------------------------

%----------------------------------------------------------------------
\section{Random variables}
%----------------------------------------------------------------------
Random variables are functions that transform abstract sample spaces to the real numbers.

\begin{definition}
Let $\Omega$ be the sample space of some random experiment, and let $\mathcal{F}$ be a $\sigma$-field of events over $\Omega$.
A \emph{random variable} on $(\Omega,\mathcal{F})$ is a function 
\[
\begin{array}{rccl}
	X:	& \Omega	& \to		& \R \\[1ex]
		& \omega	& \mapsto	& X(\omega)
\end{array}
\]
with the property that $\{\omega: X(\omega)\in B\}\in\mathcal{F}$ for every $B\in\mathcal{B}$, where $\mathcal{B}$ is the Borel $\sigma$-field over $\R$.
\end{definition}

\begin{remark}
\bit
\it The set $\{\omega: X(\omega)\in B\}$ contains precisely those outcomes that are mapped by $X$ into the set $B$.
\it $X$ is a random variable only if every set of this form is an element of the $\sigma$-field $\mathcal{F}$.
\it This condition means that, for any Borel set $B$, the probability that $X$ takes a value in $B$ is well-defined.
\eit
\end{remark}

Let us define the following notation:
\[
\{X\in B\} = \{\omega: X(\omega)\in B\}
\]


\bit
\it The expression $\{X\in B\}$ should not be taken literally: $X$ is a function, while $B$ is a subset of the real numbers. 
\it Instead, think of $\{X\in B\}$ as the event that $X$ takes a value in $B$.
\it The condition $\{X\in B\}\in\mathcal{F}$ ensures that the probability of this event is well-defined.
\eit

We denote the probability of $\{X\in B\}$ by $\prob(X\in B)$, by which we mean
\[
\prob(X\in B) = \prob\big(\{\omega: X(\omega) \in B\}\big)
\]

\newpage

\begin{proposition}\label{prop:rv_alt}
A function $X:\Omega\to\R$ is a random variable if and only if $\{X\leq x\}\in\mathcal{F}$ for every $x\in\R$.
\end{proposition}

\proofomitted

\begin{remark}
To check whether or not a function $X:\Omega\to\R$ is a random variable, by the proposition we need not verify that $\{X\in B\}\in\mathcal{F}$ for all Borel sets $B\in\mathcal{B}$. Instead, it is enough to verify only that the sets $\{\omega: X(\omega)\leq x\}$ are included in $\mathcal{F}$ (for every $x\in\R$).
\end{remark}

%\begin{proposition}
%A function $X:\Omega\to\R$ is a random variable if 
%\bit
%\it $\{X\leq x\}\in\mathcal{F}$ for every $x\in\R$, or
%\it $\{X\gt  x\}\in\mathcal{F}$ for every $x\in\R$, or
%\it $\{X\lt  x\}\in\mathcal{F}$ for every $x\in\R$, or
%\it $\{X\lt  x\}\in\mathcal{F}$ for every $x\in\R$.
%\eit
%\end{proposition}


%----------------------------------------------------------------------
\section{Indicator variables}
%----------------------------------------------------------------------
The elementary random variable is the \emph{indicator variable} of an event $A$.

% definition: indicator function
\begin{definition}
The \emph{indicator variable} of an event $A$ is the random variable $I_A:\Omega\to\R$ defined by
\[
I_A(\omega) =
  \begin{cases}
   1 & \text{if } \omega\in A, \\
   0 & \text{if } \omega\notin A.
  \end{cases}
\]
\end{definition}

% theorem: properties of indicator variables
\begin{theorem}
Let $A$ and $B$ be any two events. Then
\ben
\it $I_{A^c} = 1 - I_A$
\it $I_{A\cap B} = I_A I_B$
\it $I_{A\cup B} = I_A + I_B - I_{A\cap B}$
\een
\end{theorem}

\begin{proof}
Exercise. Note that for two functions to be equal, they must be equal at every point of their common domain, so for the first part we need to show that $I_{A^c}(\omega) = 1-I_A(\omega)$ for every $\omega\in\Omega$, and similarly for parts (2) and (3).
\end{proof}

%----------------------------------------------------------------------
%\newpage
\section{Simple random variables}
%----------------------------------------------------------------------

% defn
\begin{definition}
A \emph{simple random variable} is one that takes only finitely many values.
\end{definition}

If $X:\Omega\to\R$ is a simple random variable, it can be represented as:
\[
X(\omega)= \sum_{i=1}^n a_i I_{A_i}(\omega)
\]
where
\bit
\it $\{a_1,a_2,\ldots,a_n\}\subset\R$ is the range of $X$, and 
\it $\{A_1,A_2,\ldots,A_n\}$ is a partition of the sample space, $\Omega$.
\eit

%----------------------------------------------------------------------
\section{Probability on $\mathbb{R}$}
%----------------------------------------------------------------------

% definition
\begin{definition}
Let $(\Omega,\mathcal{F},\prob)$ be a probability space and let $X:\Omega\to\R$ be a random variable on $(\Omega,\mathcal{F})$. The function 
\[
\begin{array}{rccl}
	\prob_X:	& \mathcal{B}	& \to 		& [0,1] \\
				& B				& \mapsto 	& \prob(X\in B).
\end{array}
\]
is called the \emph{distribution} of $X$.
\end{definition}

%\newpage
% theorem
\begin{theorem}
$\prob_X$ is a probability measure on $(\R,\mathcal{B})$.
\end{theorem}

% proof
\begin{proof}
First we need to show that $\prob_X(\R) = 1$:
\[
\prob_X(\R) = \prob(X\in\R) = \prob\big(\big\{\omega:X(\omega)\in\R\big\}\big) = 1. 
\]

We also need to show that $\prob_X$ is countably additive. If $B_1,B_2,\ldots$ is a sequence of pairwise disjoint sets in $\mathcal{B}$, then
\begin{align*}
\prob_X\big(\textstyle\bigcup_{i=1}^{\infty} B_i\big)
	& = \prob\big(\big\{\omega : X(\omega)\in \textstyle\bigcup_{i=1}^{\infty} B_i\big\}\big) \\
	& = \prob\big(\textstyle\bigcup_{i=1}^{\infty} \{\omega : X(\omega)\in B_i\}\big) \\
	& = \sum_{i=1}^{\infty} \prob\big(\{\omega : X(\omega)\in B_i\}\big) \quad\text{because the $B_i$ are disjoint,} \\
	& = \sum_{i=1}^{\infty} \prob_X(B_i),
\end{align*}
which concludes the proof.
\end{proof}


\begin{remark}
%\fbox{\begin{minipage}{\linewidth}\centering
A random variable $X$ transforms an abstract probability space $(\Omega,\mathcal{F},\prob)$ into a more tractable probability space $(\R,\mathcal{B},\prob_X)$, where we can apply the methods of \emph{real analysis}.
%\end{minipage}}
\end{remark}


%----------------------------------------------------------------------
\section{Exercises}
% !TEX root = main.tex
%----------------------------------------------------------------------
\begin{exercise}
\begin{questions}
%----------------------------------------
\question
Let $\Omega$ be the sample space of some random experiment, and let $\mathcal{F}$ be a $\sigma$-field over $\Omega$.
\begin{parts}
%
\part
For any $A\in\mathcal{F}$, show that the function $X:\Omega\to\R$, defined by
\[
X(\omega) =
  \begin{cases}
   1 & \text{if } \omega\in A, \\
   0 & \text{if } \omega\notin A,
  \end{cases}
\]
is a random variable on $(\Omega,\mathcal{F})$.
\begin{answer}
For any $B\in\mathcal{B}$,
\bit
\it if $1\in B$, then $\{\omega:X(\omega)\in B\} = A$, which is contained in $\mathcal{F}$;
\it if $1\notin B$, then $\{\omega:X(\omega)\in B\} = \emptyset$, which is also contained in $\mathcal{F}$.
\eit
\end{answer}
%
\part
Let $A_1,A_2,\ldots,A_n\in\mathcal{F}$ be a partition of $\Omega$ and let $a_1,a_2,\ldots,a_n\in\R$. Show that the function $X:\Omega\to\R$, defined by
\[
X(\omega)= \sum_{i=1}^n a_i I_{A_i}(\omega) 
\qquad\text{where}\qquad
I_{A}(\omega) =
  \begin{cases}
   1 & \text{if } \omega\in A, \\
   0 & \text{if } \omega\notin A,
  \end{cases}
\]
is a random variable on $(\Omega,\mathcal{F})$.
\begin{answer}
For any $B\in\mathcal{B}$,
\[
\{\omega:X(\omega)\in B\} = \cup\{A_i:a_i\in B\} \in\mathcal{F},
\]
because $\mathcal{F}$ is closed under finite unions.
\end{answer}
\end{parts}
\end{questions}
\end{exercise}

%======================================================================
\endinput
%======================================================================

%----------------------------------------------------------------------

%======================================================================
\endinput
%======================================================================
