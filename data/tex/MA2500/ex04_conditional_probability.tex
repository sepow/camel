% !TEX root = main.tex
%----------------------------------------------------------------------
% EXERCISE 1: PROOFS
\begin{exercise} [Revision]
\begin{questions}
%----------------------------------------
% GS 1.8.10
\question
Let $\Omega$ be a sample space, and let $A_1,A_2,\ldots$ be a partition of $\Omega$ with the property that $\prob(A_i)>0$ for all $i$.
\begin{parts}
%--------------------
\part Show that $\displaystyle \prob(B) = \sum_{i=1}^{\infty}\prob(B|A_i)\prob(A_i)$.
\begin{answer}
Bookwork: this is the \emph{partition theorem}, also known as the \emph{law of total probability}.
\end{answer}
%--------------------
\part Show that $\displaystyle \prob(A_i|B) = \frac{\prob(B|A_i)\prob(A_i)}{\sum_{j=1}^{\infty}\prob(B|A_j)\prob(A_j)}$.
\begin{answer}
Bookwork: this is \emph{Bayes' formula}.
\end{answer}
%--------------------
\end{parts}
\end{questions}
\end{exercise}

\begin{exercise}
\begin{questions}
%----------------------------------------
% GS 1.8.9
\question
Let $(\Omega,\mathcal{F},\prob)$ be a probability space. Let $B\in\mathcal{F}$ with $\prob(B)>0$, and consider the function $\mathbb{Q}:\mathcal{F}\to[0,1]$ defined by $\mathbb{Q}(A)=\prob(A|B)$.
\begin{parts}
%--------------------
\part Show that $(\Omega,\mathcal{F},\mathbb{Q})$ is a probability space.
\begin{answer}
\bit
\it 
$\mathbb{Q}(\Omega) = \prob(\Omega|B) = 1$. 
\it
Let $\{A_i\}_{i=1}^{\infty}$ be a countable collection of pairwise disjoint events in $\mathcal{F}$. 
\par
Since $\mathcal{F}$ is a $\sigma$-field, $\{A_i\cap B\}_{i=1}^{\infty}$ is also a countable collection of pairwise disjoint events in $\mathcal{F}$. Hence
\small
\[
\mathbb{Q}(\cup_i A_i) 
%	= \prob(\cup_i A_i | B)
	= \frac{\prob\big[(\cup_i A_i)\cap B\big]}{\prob(B)}
	= \frac{\prob\big[\cup_i (A_i\cap B)\big]}{\prob(B)}
	= \frac{\sum_i \prob(A_i\cap B)}{\prob(B)}
	= \sum_i\frac{\prob(A_i\cap B)}{\prob(B)}
%	= \sum_i \prob(A_i|B)
	= \sum_i \mathbb{Q}(A_i).
\]
\normalsize
\eit
\end{answer}
%--------------------
\part If $C\in\mathcal{F}$ and $\mathbb{Q}(C)>0$, show that $\mathbb{Q}(A|C)=\prob(A|B\cap C)$.
\begin{answer}
Since $\mathbb{Q}$ is a probability measure,
\[
\mathbb{Q}(A|C) 
	= \frac{\mathbb{Q}(A\cap C)}{\mathbb{Q}(C)}
	= \frac{\prob(A\cap C|B)}{\prob(C|B)}
	= \frac{\prob(A\cap B\cap C)}{\prob(B\cap C)}
	= \prob(A|B\cap C).
\]
This shows that the order in which we impose the conditions $B$ and $C$ does not matter.
\end{answer}
%--------------------
\end{parts}

%----------------------------------------
% GS 1.8.15
\question A random number $N$ of dice are rolled. Let $A_k$ be the event that $N=k$, and suppose that $\prob(A_k) = 2^{-k}$ for $k\in\{1,2,\ldots\}$ (and zero otherwise). Let $S$ be the sum of the scores shown on the dice. Find the probability that:
\begin{parts}
%--------------------
\part $N=2$ given that $S=4$,
\begin{answer}
\begin{align*}
\prob(N=2|S=4)
	& = \frac{\prob(\{N=2\}\cap\{S=4\})}{\prob(\{S=4\})} \\ 
	& = \frac{\prob(S=4|N=2)\prob(N=2)}{\sum_{k=1}^{n} \prob(S=4|N=k)\prob(N=k)} \\
	& = \frac{1/12\times 1/4}{(1/6\times 1/2) + (1/12\times 1/4) + (3/6^3\times 1/8) + (1/6^4\times 1/16)} \\
	& = 
\end{align*}
\end{answer}
%--------------------
\part $S=4$ given that $N$ is even,
\begin{answer}
\begin{align*}
\prob(S=4|\text{$N$ even})
	& = 	\frac{\prob(S=4|N=2)\times(1/4) + \prob(S=4|N=4)\times(1/16)}{\prob(\text{$N$ even})} \\ 
	& = 	\frac{(1/12\times 1/4) + (1/1296\times 1/16)}	{1/4 + 1/16 + 1/64 + \cdots} \\
%	& = \frac{4^2 3^3 + 1}{4^4 3^3}.
	& = 
\end{align*}
\end{answer}
%--------------------
\part $N=2$ given that $S=4$ and the first die shows $1$,
\begin{answer}
Let $D$ be the score on the first die.
\begin{align*}
\prob(N=2|S=2,D=1)
	& = 	\frac{\prob(N=2,S=4,D=1}{\prob(S=4,D=1)} \\ 
	& = 	\frac{1/6\times 1/6\times 1/4}{(1/6\times 1/6\times 1/4) +(1/6\times 2/36\times 1/8) + (1/6^4\times 1/16)} \\
	& = 
\end{align*}
\end{answer}
%--------------------
\part the largest number shown by any dice is $r$ (where $S$ is unknown).
\begin{answer}
Let $M$ be the maximum number shown on the dice. For $r\in\{1,2,3,4,5,6\}$,
\begin{align*}
\prob(M\leq r)
	& = \sum_{k=1}^{\infty}\prob(M\leq r|N=k)\frac{1}{2^k} \\
	& = \sum_{k=1}^{\infty}\left(\frac{r}{6}\right)^k\frac{1}{2^k} \\
	& = \frac{r}{12}\left(1-\frac{r}{12}\right)^{-1} \\
	& = \frac{r}{12-r}.
\end{align*}
\end{answer}
%--------------------
\end{parts}


%----------------------------------------
% GS 1.5.4
\question
Let $\Omega=\{1,2,\ldots,p\}$ where $p$ is a prime number. Let $\mathcal{F}$ be the power set of $\Omega$, and let $\prob:\mathcal{F}\to [0,1]$ be the probability measure on $(\Omega,\mathcal{F})$ defined by $\prob(A) = |A|/p$, where $|A|$ denotes the cardinality of $A$. Show that if $A$ and $B$ are independent events, then at least one of $A$ and $B$ is either $\emptyset$ or $\Omega$.
\begin{answer}
Let $A$ and $B$ be independent events with $|A|=a$, $|B|=b$ and $|A\cap B|=c$. 
\bit
\it By independence, $\prob(A\cap B)=\prob(A)\prob(B)$. 
\it This means that $(a/p)(b/p)=(c/p)$ and therefore $ab = pc$. 
\it If $ab\neq 0$, then $p$ divides $ab$.
\it Since $p$ is prime, either $p$ divides $a$, or $p$ divides $b$ (by the fundamental theorem of arithmetic).
\it Hence $a=p$ or $b=p$ (or both). 
\it Thus follows that $A=\Omega$ or $B=\Omega$ (or both).
\eit
\end{answer}

\end{questions}
\end{exercise}

%======================================================================
\endinput
%======================================================================
