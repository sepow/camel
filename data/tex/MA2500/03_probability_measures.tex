% !TEX root = main.tex
%----------------------------------------------------------------------
\chapter{Probability Spaces}\label{chap:probability}
%----------------------------------------------------------------------
%----------------------------------------------------------------------
\section{Probability measures}
%----------------------------------------------------------------------

% defn: probability measure/space
\begin{definition}
Let $\Omega$ be a sample space, and let $\mathcal{F}$ be a $\sigma$-field over $\Omega$. A \emph{probability measure} on $(\Omega,\mathcal{F})$ is a function 
\[
\begin{array}{rccl}
	\prob:	& \mathcal{F}	& \to	& [0,1] \\[1ex]
			& A				& \mapsto	& \prob(A)
\end{array}
\]
such that $\prob(\Omega) = 1$, and for any countable collection of pairwise disjoint events $\{A_1,A_2,\ldots\}$,
\[
\prob\left(\bigcup_{i=1}^\infty A_i\right) = \sum_{i=1}^{\infty} \prob(A_i).
\]
The triple $(\Omega,\mathcal{F},\prob)$ is called a \emph{probability space}.
\end{definition}

\begin{remark}
\bit
\it The second property is called \emph{countable additivity}.
\eit
\end{remark}

% remark: measure theory
\begin{remark}
In the more general setting of measure theory:
\bit
\it The elements of $\mathcal{F}$ are called \emph{measurable sets}.	
\it The pair $(\Omega,\mathcal{F})$ is called a \emph{measurable space}.
\it The triple $(\Omega,\mathcal{F},\prob)$ is called a \emph{measure space}.
\eit
\end{remark}

% example
\begin{example}
A fair six-sided die is rolled once. 
\spar
A probability space $(\Omega,\mathcal{F}, \prob)$ for the experiment is given by
\bit
\it $\Omega=\{1,2,3,4,5,6\}$,
\it $\mathcal{F} = \mathcal{P}(\Omega)$, where $\mathcal{P}(\Omega)$ denotes the power set of $\Omega$,
\it $\prob(A)=|A|/|\Omega|$ for every $A\in\mathcal{F}$ (where $|A|$ denotes the cardinality of $A$).
\eit

If we are only interested in odd and even numbers, we can instead take
\bit
\it $\Omega=\{1,2,3,4,5,6\}$,
\it $\mathcal{F} = \big\{\emptyset,\{1,3,5\},\{2,4,6\},\{1,2,3,4,5,6\}\big\}$
\it $\prob(\emptyset)=0$, $\prob(\{1,3,5\})=1/2$, $\prob(\{2,4,6\})=1/2$, $\prob(\{1,2,3,4,5,6\})=1$.
\eit
\end{example}

%----------------------------------------------------------------------
\section{Null and almost-certain events}
%----------------------------------------------------------------------

% definition: null and a.s. events
\begin{definition}
\ben
\it If $\prob(A)=0$, we say that $A$ is a \emph{null event}.
\it If $\prob(A)=1$, we say that $A$ occurs \emph{almost surely} (or ``\emph{with probability 1}'').
\een
\end{definition}

% remark: null and a.s. events
\begin{remark}
\bit
\it A null event is not the same as the impossible event ($\emptyset$). 
\it An event that occurs almost surely is not the same as the certain event ($\Omega$).
\eit
\end{remark}

% example
\begin{example}
A dart is thrown at a dartboard.
\bit
\it The probability that the dart hits a given point of the dartboard is $0$.
\it The probability that the dart does not hit a given point of the dartboard is $1$.
\eit
\end{example}

%----------------------------------------------------------------------
\section{Properties of probability measures}
%----------------------------------------------------------------------

% theorem: properties of probability measures
\begin{theorem}[Properties of probability measures]\label{thm:properties_of_probability_measures}
Let $(\Omega,\mathcal{F},\prob)$ be a probability space, and let $A,B\in\mathcal{F}$. 
\ben
\it Complementarity: $\prob(A^c) = 1 - \prob(A)$.
\it $\prob(\emptyset) = 0$,
\it Monotonicity: if $A\subseteq B$ then $\prob(A)\leq \prob(B)$.
\it Addition rule: $\prob(A\cup B) = \prob(A) + \prob(B) - \prob(A\cap B)$.
\een
\end{theorem}

% proof
\begin{proof}
\ben
\it % complementarity
Since $A\cup A^c=\Omega$ is a disjoint union and $\prob(\Omega)=1$, it follows by additivity that 
\[
1 = \prob(\Omega) = \prob(A\cup A^c) = \prob(A) + \prob(A^c).
\]
\it % emptyset
Since $\emptyset=\Omega^c$ and $\prob(\Omega)=1$, it follows by complemenarity that
\[
\prob(\emptyset) = \prob(\Omega^c) = 1 - \prob(\Omega) = 1 - 1 = 0.
\]
\it % monotonicity
Let $A\subseteq B$ and let us write $B = A\cup (B\setminus A)$. 

Since $A$ and $B\setminus A$ are disjoint sets, it follows by additivity that
\[
\prob(B) = \prob\big[A\cup (B\setminus A)\big] = \prob(A) + \prob(B\setminus A).
\]
Hence, because $\prob(B\setminus A)\geq 0$, it follows that $\prob(B) \geq \prob(A)$.
\it % addition rule
Let us write:
\bit
\it $A\cup B = (A\setminus B) \cup (B\setminus A) \cup (A\cap B)$
\it $A 		 = (A\setminus B) + (A\cap B)$
\it $B 		 = (B\setminus A) + (A\cap B)$
\eit
These are disjoint unions, so by additivity, 
\bit
\it $\prob(A\cup B) = \prob(A\setminus B) + \prob(B\setminus A) + \prob(A\cap B)$
\it $\prob(A) 		= \prob(A\setminus B) + \prob(A\cap B)$
\it $\prob(B)		= \prob(B\setminus A) + \prob(A\cap B)$
\eit
Hence $\prob(A\cup B) = \prob(A) + \prob(B) - \prob(A\cap B)$, as required.
\een
\end{proof}

%----------------------------------------------------------------------
\section{Continuity of probability measures}
%----------------------------------------------------------------------

% theorem: properties of probability measures
\begin{theorem}[Continuity of probability measures]\label{thm:continuity_of_probability_measures}
Let $(\Omega,\mathcal{F},\prob)$ be a probability space.
\ben
\it For an increasing sequence of events $A_1\subseteq A_2\subseteq \ldots$ in $\mathcal{F}$, 
\[
\prob\left(\bigcup_{n=1}^{\infty} A_n\right) = \lim_{n\to\infty}\prob(A_n).
\]
\it For a decreasing sequence of events $B_1\supseteq B_2\supseteq \ldots$ in $\mathcal{F}$, 
\[
\prob\left(\bigcap_{n=1}^{\infty} B_n\right) = \lim_{n\to\infty}\prob(B_n).
\]
\een
\end{theorem}

% proof
\begin{proof}
% continuity from below
To prove the first part, let $A_1\subseteq A_2\subseteq \ldots$ be an increasing sequence of events, and 
\[
A=\bigcup_{i=1}^{\infty} A_i.
\]
We can write $A$ as a disjoint union
\[
A = A_1 \cup (A_2\setminus A_1) \cup (A_3\setminus A_2) \cup \ldots
\]
Since the sets $A_{i+1}\setminus A_i$ are disjoint, by countable additivity we have
\[
\prob(A) = \prob(A_1) + \prob(A_2\setminus A_1) + \prob(A_3\setminus A_2) + \ldots
\]
Furthermore, $A_i\subseteq A_{i+1}$ means that $A_{i+1}=(A_{i+1}\setminus A_i)\cup A_i$ is a disjoint union, so
\[
\prob(A_{i+1}\setminus A_i)=\prob(A_{i+1})-\prob(A_i).
\]
Hence
\begin{align*}
\prob(A) 
	& = \prob(A_1) + \big[\prob(A_2) - \prob(A_1)\big] + \big[\prob(A_3) - \prob(A_2)\big] + \ldots \\
	& = \big[\prob(A_1) - \prob(A_1)\big] + \big[\prob(A_2) - \prob(A_2)\big] + \big[\prob(A_3) - \prob(A_3)\big] + \ldots \\
	& = \lim_{n\to\infty} \prob(A_n).
\end{align*}
\break % << 
% continuity from above
To prove the second part, let $B_1\supseteq B_2\supseteq \ldots$ be a decreasing sequence of events, and 
\[
B=\bigcap_{i=1}^{\infty} B_i.
\]
Let $A_i=B^c_i$ and $A=B^c$. 
\spar
Then $A_1\subseteq A_2\subseteq \ldots$ is an increasing sequence, and 
\[
A=\bigcup_{i=1}^{\infty} A_i.
\]
Hence by the first part of the theorem,
\begin{align*}
\prob(B) 
	& = 1 - \prob(A) \\
	& = 1 - \lim_{n\to\infty} \prob(A_n) \\
	& = \lim_{n\to\infty} (1-\prob(A_n)) \\
	& = \lim_{n\to\infty} \prob(B_n).
\end{align*}
\qed
\end{proof}

%----------------------------------------------------------------------
%%\section{Subadditivity}
%%----------------------------------------------------------------------
%
%\begin{remark}[Subadditivity]
%\ben
%\it
%For any two events $A,B\in\mathcal{F}$ we have $\prob(A\cap B)\geq 0$.
%Hence by the addition rule,
%\[
%\prob(A\cup B) \leq \prob(A) + \prob(B) \text{\quad for all\quad} A,B\in\mathcal{F}.
%\]
%This property is called \emph{subadditivity}.
%\it
%For any countable collection $\{A_1,A_2,\ldots\}$ of events in $\mathcal{F}$, it can be shown that 
%\[
%\prob\left(\bigcup_{i=1}^{\infty}A_i\right)\leq \sum_{i=1}^{\infty} \prob(A_i).
%\]
%This property is called \emph{countable subadditivity}.
%\een
%\end{remark}
%

%----------------------------------------------------------------------
\section{Exercises}
% !TEX root = main.tex
%----------------------------------------------------------------------

% EXERCISE 1: THEORY
\begin{exercise}
\begin{questions}
\question % inclusion/exclusion
Let $(\Omega,\mathcal{F},\prob)$ be a probability space, and let $A,B,C\in\mathcal{F}$. Show that
\[
\prob(A\cup B\cup C) 
	= \prob(A) + \prob(B) + \prob(C) - \prob(A\cap B) - \prob(A\cap C) - \prob(B\cap C) + \prob(A\cap B\cap C)
\]
This is called the \emph{inclusion-exclusion principle}.
\begin{answer}
Set union is an associative operator: $A\cup B\cup C = (A\cup B)\cup C$, so by the addition rule,
\begin{align*}
\prob(A\cup B\cup C) 	
	& = \prob\big((A\cup B)\cup C\big) \\
	& = \prob(A\cup B) + \prob(C) - \prob\big((A\cup B)\cap C\big).
\end{align*}
Set intersection is distributive over set union: $(A\cup B)\cap C = (A\cap C)\cup (B\cap C)$, so by the addition rule,
\begin{align*}	
\prob\big((A\cup B)\cap C\big)
	& = \prob\big((A\cap C)\cup (B\cap C)\big) \\
	& = \prob(A\cap C) + \prob(B\cap C) - \prob\big((A\cap C)\cap (B\cap C)\big) \\
	& = \prob(A\cap C) + \prob(B\cap C) - \prob(A\cap B\cap C).
\end{align*}
\end{answer}

\question % subadditivity
Let $(\Omega,\mathcal{F},\prob)$ be a probability space.
\begin{parts}
\part 
Show that $\prob(A\cup B)\leq\prob(A)+\prob(B)$ for all $A,B\in\mathcal{F}$. This is called \emph{subadditivity}.
\begin{answer}
TODO
\end{answer}
\part 
Show that for any sequence $A_1,A_2,\ldots$ of events in $\mathcal{F}$,
\[
\prob\left(\bigcup_{i=1}^{\infty}A_i\right)\leq \sum_{i=1}^{\infty} \prob(A_i).
\]
This is called \emph{countable subadditivity}.
\begin{answer}
TODO
\end{answer}
\end{parts}
\end{questions}
\end{exercise}


% EXERCISE 2: APPLICATIONS
\begin{exercise}
\begin{questions}
%----------------------------------------
% GS 1.3.1
\question
Let $A$ and $B$ be events with probabilities $\prob(A)=3/4$ and $\prob(B)=1/3$. 
\begin{parts}
%--------------------
\part Show that $\frac{1}{12}\leq\prob(A\cap B)\leq\frac{1}{3}$, and construct examples to show that both extremes are possible.
\begin{answer}
\bit
\it Lower bound: $\prob(A\cup B)\leq 1$ so $\prob(A\cap B) = \prob(A)+\prob(B)-\prob(A\cup B) \geq \prob(A)+\prob(B) -1 = \frac{1}{12}$.
\it Upper bound: $A\cap B\subseteq A$ and $A\cap B\subseteq B$, so $\prob(A\cap B) \leq \min\{\prob(A),\prob(B)\} = \frac{1}{3}$.
\eit
Example: let $\Omega=\{1,2,\ldots,12\}$ with each outcome equally likely, and let $A=\{1,2,\ldots,9\}$.
\bit
\it Let $B=\{9,10,11,12\}$. Then $\prob(A\cap B) = \prob(\{9\}) = \frac{1}{12}$.
\it Let $B=\{1,2,3,4\}$. Then $\prob(A\cap B) = \prob(\{1,2,3,4\}) = \frac{1}{3}$.
\eit
\end{answer}
%--------------------
\part Find corresponding bounds for $\prob(A\cup B)$.
\begin{answer}
\bit
\it Upper bound: $\prob(A\cup B) \leq \min\{\prob(A)+\prob(B), 1\} = 1$.
\it Lower bound: $\prob(A\cup B) \geq \max\{\prob(A),\prob(B)\} = 3/4$.
\eit
These bounds are attained in the above example.
\end{answer}
%--------------------
\end{parts}
%----------------------------------------
% roulette
\question
A roulette wheel consists of 37 slots of equal size. The slots are numbered from 0 to 36, with odd-numbered slots coloured red, even-numbered slots coloured black, and the slot labelled 0 coloured green. The wheel is spun in one direction and a ball is rolled in the opposite direction along a track running around the circumference of the wheel. The ball eventually falls on to the wheel and into one of the 37 slots. A player bets on the event that the ball stops in a red slot, and another player bets on the event that the ball stops in a black slot.
\begin{parts}
%--------------------
\part Define a suitable sample space $\Omega$ for this random experiment, and identify the events of interest.
\begin{answer}
A suitable sample space for the experiment is $\Omega=\{0,1,2,\ldots,36\}$. 
\par
The events of interest are $G=\{0\}$, $R=\{1,3,\ldots,35\}$ and $B=\{2,4,\ldots,36\}$.
\end{answer}
%--------------------
\part Find the smallest field $\mathcal{F}$ over $\Omega$ that contains the events of interest.
\begin{answer}
The smallest field of sets containing the events $G$, $R$ and $B$ is 
\[
\mathcal{F} = \{\emptyset, G, R, B, G\cup R, G\cup B, R\cup B, \Omega\}.
\]
$\mathcal{F}$ is indeed a field of sets, because 
\bit
\it $\Omega\in\mathcal{F}$, 
\it $\mathcal{F}$ is closed under complementation, 
\bit
	\it $\emptyset^c = \Omega\in\mathcal{F}$ and $\Omega^c=\emptyset\in\mathcal{F}$,
	\it $G^c = R\cup B\in\mathcal{F}$, $R^c = B\cup G\in\mathcal{F}$ and $B^c = R\cup G\in\mathcal{F}$,
	\it $(G\cup R)^{c}=B\in\mathcal{F}$, $(G\cup B)^c=R\in\mathcal{F}$ and $(R\cup B)^c = G\in\mathcal{F}$
\eit
\it $\mathcal{F}$ is closed under pairwise unions, for example
	\bit
	\it $R\cup\emptyset = R \in\mathcal{F}$ and $R\cup \Omega = \Omega \in\mathcal{F}$,
	\it $R\cup B \in\mathcal{F}$ and $R\cup G \in\mathcal{F}$, 
	\it $R\cup (R\cup B) = R\cup B \in\mathcal{F}$, 
	\it $R\cup (R\cup G) = R\cup G \in\mathcal{F}$, 
	\it $R\cup (B\cup G) = \Omega \in\mathcal{F}$.
	\eit
and so on.
\eit
\end{answer}
%--------------------
\part Define a suitable probability measure $(\Omega,\mathcal{F})$ to represent the game.
\begin{answer}
A suitable probability measure over $(\Omega,\mathcal{F})$ is given by
\begin{align*}
& \prob(\emptyset)=0, \\
& \prob(R)=18/37,\ \prob(B)=18/37,\ \prob(G)=1/37, \\
& \prob(R\cup B)=36/37,\ \prob(R\cup G)=19/37,\ \prob(B\cup G)=19/37, \\
& \prob(\Omega)=1.
\end{align*}
This is indeed a probability measure, because 
\bit
\it $\prob(\emptyset)=0$, 
\it $\prob(\Omega)=1$, and 
\it $\prob$ is additive over $\mathcal{F}$; for example,
\bit
	\it $\frac{36}{37} = \prob(R\cup B) = \prob(R)+\prob(B) = \frac{18}{37}+\frac{18}{37} = \frac{36}{37}$, 
	\it $\frac{19}{37} = \prob(R\cup G) = \prob(R)+\prob(G) = \frac{18}{37}+\frac{1}{37} = \frac{19}{37}$, 
	\it $\frac{19}{37} = \prob(B\cup G) = \prob(B)+\prob(G) = \frac{18}{37}+\frac{1}{37} = \frac{19}{37}$,
\eit
and so on.
\eit
\end{answer}
\end{parts}
\end{questions}
\end{exercise}


% EXERCISE 2: CONTINUITY
\begin{exercise}
\begin{questions}
%----------------------------------------
% GS 1.8.4(c)
\question A biased coin has probability $p$ of showing heads. The coin is tossed repeatedly until a head occurs. Describe a suitable probability space for this experiment.
\begin{answer}
The sample space is the set of all finite sequences of tails followed by a head, together with the infinite sequence of tails:
\[
\Omega = \{T^nH:n\geq 0\} \cup \{T^{\infty}\}.
\]
The $\sigma$-field can be taken to be the power set of $\Omega$, and the probability measure can be defined on the elementary events by
\begin{align*}
\prob(T^nH) 			& = (1-p)^n p, \\
\prob(T^{\infty})	& = \lim_{n\to\infty}(1-p)^n = 0 \text{ if } p\neq 0.
\end{align*}
\end{answer}
%----------------------------------------
% GS 1.3.2
\question
A fair coin is tossed repeatedly.
\begin{parts}
%--------------------
\part Show that a head eventually occurs with probability one.
\begin{answer}
Let $A_n$ be the event that no heads occur in the first $n$ tosses, and let $A$ be the event that no heads occur at all. Then $A_1,A_2,\ldots$ is a decreasing sequence ($A_{n+1}\subset A_n$), with $A=\cap_{i=1}^{\infty} A_n$. Hence by the continuity property of probability measures,
\[
\prob(A) = \prob\left(\bigcap_{n=1}^{\infty}A_n\right)
	= \lim_{n\to\infty} \prob(A_n)
	= \lim_{n\to\infty} \left(\frac{1}{2}\right)^n = 0,
\]
or alternatively,
\[
\prob(\text{no heads})
	= \lim_{n\to\infty} \prob(\text{no heads in the first $n$ tosses})
	= \lim_{n\to\infty} \left(\frac{1}{2}\right)^n = 0.
\]
Thus we are certain of eventually observing a head.
\end{answer}
%--------------------
\part Show that a sequence of 10 consecutive tails eventually occurs with probability one.
\begin{answer}
Let us think of the first $10n$ tosses as disjoint groups of consecutive outcomes, each group of length $10$. The probability any one of the $n$ groups consists of 10 consecutive tails is $2^{-10}$, independently of the other groups. The event that one of the groups consists of 10 consecutive tails is a subset of the event that a sequence of 10 consecutive tails appears anywhere in the first $10n$ tosses. Hence, using the continuity of probability measures,
\begin{align*}
\prob(\text{$10T$ eventually appears}) 
	& = \lim_{n\to\infty} \prob(\text{$10T$ occurs somewhere in the first $10n$ tosses)} \\
	& \geq \lim_{n\to\infty} \prob(\text{$10T$ occurs as one of the first $n$ groups of $10$}) \\
	& = 1 - \lim_{n\to\infty} \prob(\text{$10T$ does not occur as one of the first $n$ groups of 10}) \\
 	& = 1 - \lim_{n\to\infty} \left(1-\frac{1}{2^{10}}\right)^n = 1.
\end{align*}
Thus we are certain of eventually observing sequence of 10 consecutive tails.
\end{answer}
%--------------------
\part Show that any finite sequence of heads and tails eventually occurs with probability one.
\begin{answer}
Let $s$ be a fixed sequence of length $k$. As in the previous part, we think of the first $kn$ tosses as $n$ distinct groups of length $k$. The event that the one of these groups is exactly equal to $s$ is a subset of the event that first $kn$ tosses contains at least one instance of $s$. Hence
\begin{align*}
\prob(\text{$s$ eventually appears}) 
	& = \lim_{n\to\infty} \prob(\text{$s$ occurs somewhere in the first $kn$ tosses)} \\
	& \geq \lim_{n\to\infty} \prob(\text{$s$ occurs as one of the first $n$ groups of $k$}) \\
	& = 1 - \lim_{n\to\infty} \prob(\text{$s$ does not occur as one of the first $n$ groups of $k$}) \\
 	& = 1 - \lim_{n\to\infty} \left(1-\frac{1}{2^{k}}\right)^n = 1.
\end{align*}
Thus we are certain of eventually observing the sequence $s$.
\bit
\it In an infinite sequence of coin tosses, anything that can happen, does happen!
\eit
\end{answer}
\end{parts}

\end{questions}
\end{exercise}

%======================================================================
\endinput
%======================================================================

%----------------------------------------------------------------------

%======================================================================
\endinput
%======================================================================
