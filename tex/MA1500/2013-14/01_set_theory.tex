\documentclass[lecture]{csm}
%\documentclass[blanks,lecture]{csm}

% set meta information
\modulecode{MA1500}
\moduletitle{Introduction to Probability Theory}
\academicyear{2013/14}
\doctype{Lecture}
\doctitle{Set Theory}
\docnumber{1}

% local
\usepackage{subcaption}
\captionsetup[subfigure]{labelformat=empty}
\captionsetup[figure]{skip=2ex}

\RequirePackage{graphicx}
\graphicspath{{./figures/}}
\DeclareGraphicsExtensions{.pdf,.jpeg,.png,.gif}

\def\it{\item}
\def\bit{\begin{itemize}}
\def\eit{\end{itemize}} 
\def\ben{\begin{enumerate}}
\def\een{\end{enumerate}}

%======================================================================
\begin{document}
\maketitle
\tableofcontents
%======================================================================

%----------------------------------------------------------------------
\section{Sets}
%----------------------------------------------------------------------

A set is a collection of \emph{elements}.
\bit
\it If $a$ is an element of the set $A$, we denote this by $a\in A$.
\it If $a$ is \emph{not} an element of $A$, we denote this by $a\notin A$.
\it The \emph{cardinality} of a set is the number of elements it contains.
\it The \emph{empty set} contains no elements, and is denoted by $\emptyset$.
\it A \emph{universe} is a set containing all objects that are of interest, and will be denoted by $U$.
\eit

%\vspace*{2ex}
\par
Algebra is the study of \emph{operations} and \emph{relations}.
\bit
\it The basic relations of set algebra are \emph{set inclusion} and \emph{set equality}.
\it The basic operations of set algebra are \emph{complementation}, \emph{union} and \emph{intersection}.
\eit

%----------------------------------------------------------------------
\section{Set relations}
%----------------------------------------------------------------------

% definition: set inclusion, set equality
\begin{definition}
Let $A$ and $B$ be sets. 
\ben
\it If every element of $A$ is also an element of $B$, we say that $A$ is a \emph{subset} of $B$.
\par This is denoted by $A\subseteq B$.
\it If every element of $A$ is an element of $B$, and every element of $B$ is an element of $A$, we say that $A$ and $B$ are \emph{equal}. 
\par This is denoted by $A=B$.
\it If $A$ is a subset of $B$, but $A$ is not equal to $B$, we say that $A$ is a \emph{proper subset} of $B$. 
\par This is denoted by $A\subset B$.
\een
\end{definition}

% picture: set inclusion
%\picbox{5cm}

% example
\begin{example}
Let $A=\{a,b\}$, $B=\{a,b\}$ and $C=\{a,b,c\}$. 
\par
Then $A\subseteq B$, $A=B$ and $A\subset C$.
\end{example}
%\begin{example}
%Let $A=\{a,b\}$, $B=\{a,b\}$ and $C=\{a,b,c\}$.
%\bit
%\it $A$ is a subset of $B$: $A\subseteq B$,
%\it $A$ is equal to $B$: $A=B$, and
%\it $A$ is a proper subset of $C$: $A\subset C$.
%\eit
%\end{example}

%----------------------------------------------------------------------
\section{Set operations}
%----------------------------------------------------------------------
\begin{definition}
Let $A$, $B$ and $U$ be sets, and suppose that $A,B\subseteq U$.
\ben
\it The \emph{union} of $A$ and $B$ is the set
$$
A\cup B = \{a\in U: a\in A \text{ or }a\in B\}.
$$
\it The \emph{intersection} of $A$ and $B$ is the set
$$
A\cap B = \{a\in U: a\in A \text{ and }a\in B\}.
$$
\it The \emph{complement} of $A$ is the set 
$$
A^c=\{a\in U:a\notin A\}.
$$
\een
\end{definition}

% example
\begin{example}
Let $A=\{a,b\}$, $B=\{b,c\}$ and $U=\{a,b,c,d\}$.
\par
Then
$A\cup B = \{a,b,c\}$, $A\cap B = \{b\}$ and $A^c = \{c,d\}$.
\end{example}

\break % <<

% table: basic set operations
%\begin{table}[htb]
%\centering
%\end{table}

%\vspace*{7ex}

% figure: basic set operations
\begin{figure}[htb]
\begin{tabular}{ccc}
	\begin{subfigure}{.25\textwidth}
	\resizebox{\linewidth}{!}{\includegraphics{AcupB}}
	\caption{Union}
	\end{subfigure}
&
	\begin{subfigure}{.25\textwidth}
	\resizebox{\linewidth}{!}{\includegraphics{AcapB}}
	\caption{Intersection}
	\end{subfigure}
&
	\begin{subfigure}{.25\textwidth}
	\resizebox{\linewidth}{!}{\includegraphics{Acomp}}
	\caption{Complement}
	\end{subfigure}
\end{tabular}
\end{figure}

\begin{center}
\begin{tabular}{|c|c||c|c|c|} \hline
Set Theory 		& 			& Logic			&		& \\ \hline
Union			& $A\cup B$	& Disjunction 	& OR 	& $\lor$	\\
Intersection		& $A\cap B$	& Conjunction	& AND 	& $\land$\\
Complement		& $A^c$		& Negation		& NOT 	& $\lnot$	\\ \hline
\end{tabular}
\end{center}

%\begin{center}
%\begin{tabular}{|c|c||c|c|c|} \hline
%\multicolumn{2}{|c||}{Set Theory} & \multicolumn{3}{|c|}{Logic} \\ \hline
%Union			& $A\cup B$	& Disjunction 	& OR &	 $\lor$	\\
%Intersection		& $A\cap B$			& Conjunction	& AND & $\land$\\
%Complement		& $A^c$			& Negation		& NOT & $\lnot$	\\ \hline
%\end{tabular}
%\end{center}
%
%\begin{center}
%\begin{tabular}{|c|c||c|c|c|} \hline
%\multicolumn{2}{|c||}{Set Theory}		& Logic			& 		&			\\ \hline
%Union			& $A\cup B$	& Disjunction 	& OR 	& $\lor$		\\
%Intersection		& $A\cap B$	& Conjunction	& AND 	& $\land$	\\
%Complement		& $A^c$		& Negation		& NOT 	& $\lnot$	\\ \hline
%\end{tabular}
%\end{center}
%
%\break % <<
%
%\begin{center}
%\begin{tabular}{cc} 
%\begin{tabular}{|c|c|}\hline
%\multicolumn{2}{|c|}{Set Theory} \\ \hline
%Union			& $A\cup B$ \\
%Intersection		& $A\cap B$ \\
%Complement		& $A^c$ \\ \hline
%\end{tabular}
%& 
%\begin{tabular}{|c|c|c|}\hline
%\multicolumn{3}{|c|}{Logic} \\ \hline
%Disjunction 	& OR &	 $\lor$	\\
%Conjunction	& AND & $\land$\\
%Negation		& NOT & $\lnot$	\\ \hline
%\end{tabular}
%\end{tabular}
%\end{center}

%% figure: basic set operations
%\begin{figure}[htb]
%\centering
%\begin{tabular}{|c|c||c|c|c|} \hline
%\multicolumn{2}{|c||}{Set Theory} & \multicolumn{3}{|c|}{Logic} \\ \hline
%Union			& $A\cup B$	& Disjunction 	& OR &	 $\lor$	\\
%Intersection		& $A\cap B$			& Conjunction	& AND & $\land$\\
%Complement		& $A^c$			& Negation		& NOT & $\lnot$	\\ \hline
%\end{tabular}
%\par
%\vspace*{7ex}
%%\mbox{}\hfill
%\begin{subfigure}{.25\textwidth}
%\resizebox{\linewidth}{!}{\includegraphics{AcupB}}
%\caption{Union}
%\end{subfigure}
%\qquad
%\begin{subfigure}{.25\textwidth}
%\resizebox{\linewidth}{!}{\includegraphics{AcapB}}
%\caption{Intersection}
%\end{subfigure}
%\qquad
%\begin{subfigure}{.25\textwidth}
%\resizebox{\linewidth}{!}{\includegraphics{Acomp}}
%\caption{Complement}
%%\hfill\mbox{}
%%\vspace*{3ex}
%\end{subfigure}
%%\caption{Set operations.}
%\end{figure}
%
\clearpage

%----------------------------------------------------------------------
\section{The fundamental laws of set algebra}
%----------------------------------------------------------------------
\begin{definition}
\ben
\it Commutative property.
\bit 
\it $A\cup B = B\cup A$,
\it $A\cap B = B\cap A$.
\eit
\it Associative property.
\bit 
\it $(A\cup B)\cup C = A\cup (B\cup C)$,
\it $(A\cap B)\cap C = A\cap (B\cap C)$.
\eit
\it Distributive property.
\bit 
\it $A\cup (B\cap C) = (A\cup B)\cap(A\cup C)$,
\it $A\cap (B\cup C) = (A\cap B)\cup(A\cap C)$.
\eit
\een
\end{definition}

\begin{remark}
A statement such as $A\cup B\cap C$ is ambiguous. 
\end{remark}

%----------------------------------------------------------------------
\section{De Morgan's laws}
%----------------------------------------------------------------------
Union and intersection swap roles under complementation.

\begin{theorem}\label{thm:demorgan_simple}
\ben
\it $(A\cup B)^c = A^c\cap B^c$.
\it $(A\cap B)^c = A^c\cup B^c$.
\een
\end{theorem}

\begin{proof}
\ben
\it 
Let $a\in(A\cup B)^c$.Then $a\notin A$ and $a\notin B$, so $a\in A^c\cap B^c$.\par
\quad Hence $(A\cup B)^c\subseteq A^c\cap B^c$.\par
Let $a\in A^c\cap B^c$. Then $a\notin A$ and $a\notin B$, so $a\notin A\cup B$.\par
\quad Hence $A^c\cap B^c\subseteq (A\cup B)^c$.\par
Thus it follows that $(A\cup B)^c = A^c\cap B^c$.
\it 
Apply part (1) to the sets $A^c$ and $B^c$: $(A^c\cup B^c)^c = A\cap B$.\par
Take the complement of both sides: $(A\cap B)^c = A^c\cup B^c$.
\een
\end{proof}



%----------------------------------------------------------------------
\section{Set difference}
%----------------------------------------------------------------------
\begin{definition}
Let $A$, $B$ and $U$ be sets, with $A,B\subseteq U$.
\ben
\it The \emph{set difference} between $A$ and $B$ is the set
$$
A\setminus B = \{a\in U : a\in A \text{ and } a\notin B\}. % = A\cap B^c.
$$
\it The \emph{symmetric difference} between $A$ and $B$ is the set 
$$
A\bigtriangleup B = (A\setminus B) \cup (B\setminus A). %= (A\cap B^c)\cup (B\cap A^c). 
$$
\een
\end{definition}

\bit
\it $A\setminus B$ is the set of points that are in $A$ but not in $B$.
\it $A\bigtriangleup B$ is the set of points that are in either $A$ or $B$, but not both.
\eit

% example
\begin{example}
Let $A=\{a,b\}$ and $B=\{b,c\}$.\par
Then $A\setminus B = \{a\}$ and $A\bigtriangleup B = \{a,c\}$.
\end{example}

\break % <<

% figure
\begin{figure}
\begin{tabular}{cccc}
	
\begin{subfigure}{.15\textwidth}
\resizebox{\linewidth}{!}{\includegraphics{emptyset}}
\caption{$\emptyset$}
\end{subfigure}
&
\begin{subfigure}{.15\textwidth}
\resizebox{\linewidth}{!}{\includegraphics{AcapB}}
\caption{$A\cap B$}
\end{subfigure}
&
\begin{subfigure}{.15\textwidth}
\resizebox{\linewidth}{!}{\includegraphics{BminusA}}
\caption{$B\setminus A$}
\end{subfigure}
&
\begin{subfigure}{.15\textwidth}
\resizebox{\linewidth}{!}{\includegraphics{setB}}
\caption{$B$}
\end{subfigure}

\end{tabular}
\begin{tabular}{cccc}

\begin{subfigure}{.15\textwidth}
\resizebox{\linewidth}{!}{\includegraphics{AcupB_comp}}
\caption{$(A\cup B)^c$}
\end{subfigure}
&
\begin{subfigure}{.15\textwidth}
\resizebox{\linewidth}{!}{\includegraphics{symdiff_comp}}
\caption{$(A\bigtriangleup B)^c$}
\end{subfigure}
&
\begin{subfigure}{.15\textwidth}
\resizebox{\linewidth}{!}{\includegraphics{Acomp}}
\caption{$A^c$}
\end{subfigure}
&
\begin{subfigure}{.15\textwidth}
\resizebox{\linewidth}{!}{\includegraphics{AminusB_comp}}
\caption{$(A\setminus B)^c$}
\end{subfigure}

\end{tabular}
\begin{tabular}{cccc}

\begin{subfigure}{.15\textwidth}
\resizebox{\linewidth}{!}{\includegraphics{AminusB}}
\caption{$A\setminus B$}
\end{subfigure}
&
\begin{subfigure}{.15\textwidth}
\resizebox{\linewidth}{!}{\includegraphics{setA}}
\caption{$A$}
\end{subfigure}
&
\begin{subfigure}{.15\textwidth}
\resizebox{\linewidth}{!}{\includegraphics{symdiff}}
\caption{$A\bigtriangleup B$}
\end{subfigure}
&
\begin{subfigure}{.15\textwidth}
\resizebox{\linewidth}{!}{\includegraphics{AcupB}}
\caption{$A\cup B$}
\end{subfigure}

\end{tabular}
\begin{tabular}{cccc}

\begin{subfigure}{.15\textwidth}
\resizebox{\linewidth}{!}{\includegraphics{Bcomp}}
\caption{$B^c$}
\end{subfigure}
&
\begin{subfigure}{.15\textwidth}
\resizebox{\linewidth}{!}{\includegraphics{BminusA_comp}}
\caption{$(B\setminus A)^c$}
\end{subfigure}
&
\begin{subfigure}{.15\textwidth}
\resizebox{\linewidth}{!}{\includegraphics{AcapB_comp}}
\caption{$(A\cap B)^c$}
\end{subfigure}
&
\begin{subfigure}{.15\textwidth}
\resizebox{\linewidth}{!}{\includegraphics{universe}}
\caption{$\emptyset^c$}
\end{subfigure}

\end{tabular}
\end{figure}

%======================================================================
\end{document}
%======================================================================
