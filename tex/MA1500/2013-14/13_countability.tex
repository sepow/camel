\documentclass[lecture]{csm}
%\documentclass[blanks,lecture]{csm}

% set meta information
\modulecode{MA1500}
\moduletitle{Introduction to Probability Theory}
\academicyear{2013/14}
\doctype{Lecture}
\doctitle{Countable and Uncountable Sets}
\docnumber{13}

% local
\newcommand{\prob}{\mathbb{P}}
\newcommand{\expe}{\mathbb{E}}
\newcommand{\N}{\mathbb{N}}
\newcommand{\Z}{\mathbb{Z}}
\newcommand{\Q}{\mathbb{Q}}
\newcommand{\R}{\mathbb{R}}
\newcommand{\C}{\mathbb{C}}
\def\it{\item}
\def\bit{\begin{itemize}}
\def\eit{\end{itemize}} 
\def\ben{\begin{enumerate}}
\def\een{\end{enumerate}}

%======================================================================
\begin{document}
\maketitle
\tableofcontents
%======================================================================

%----------------------------------------------------------------------
\section{Infinite sample spaces}
%----------------------------------------------------------------------
So far, we have considered only \emph{finite sample spaces} $\Omega = \{1,2,3,\ldots,n\}$.

%\bigskip
%The set of all subsets of $\Omega$ is known as its \emph{power set}, and denoted by $\mathcal{P}(\Omega)$.
%%\[
%%\mathcal{P}(\Omega) = \big\{\emptyset, \{1\},\{2\},\ldots,\{n\},\{1,2\},\{1,3\},\ldots\{n-1,n\},\ldots,\{2,3,\ldots,n\},\Omega\big\}
%%\]
%
%\bit
%\it The \emph{cardinality} of a set is the number of elements it contains.
%\it If $\Omega = \{1,2,3,\ldots,n\}$, then the cardinality of $\mathcal{P}(\Omega)$ is $2^n$ (a finite number).
%\eit
%
%What is a set of \emph{infinite cardinality?}

\newpage

\begin{example}
Consider a random experiment in which a coin is tossed repeatedly until the first head occurs. There is an infinite number of possible sequences:
\[
H,\quad TH,\quad TTH,\quad TTTH,\quad TTTTH,\quad\ldots
\]
Let the outcome of the experiment be the number of times the coin is tossed:
\[
\Omega = \{1,2,3,4,5,\ldots\}
\]
% pmf
Let $0<p<1$ be the probability that the coin shows heads. 
\par
If we assume that the trials are independent, the probability mass function for this experiment is
\[
p(k) = (1-p)^{k-1}p\text{\qquad for\quad} k\in\{1,2,3,\ldots\}.
\]
The sum of these probabilities is equal to one:
\[
\sum_{k=1}^{\infty} p(k) = p\sum_{k=1}^{\infty} (1-p)^{k-1}  = \frac{p}{1-(1-p)} = 1.
\]
Here, we have used the formula for the sum of a geometric progression:% with $a = p$ and $r = (1-p)$:
\[
a + ar + ar^2 + \ldots = \frac{a}{1-r} \text{\qquad provided $|r|<1$},
\]
with $a = p$ and $r = (1-p)$.

\newpage

Let $A$ be the event that the coin is tossed an even number of times:
\[
A = \{2,4,6,\ldots \}.
\]
This event corresponds to the sequences $TH, TTTH, TTTTTH,\ldots$.

% recall
Recall that a \emph{probability mass function} is a function on the sample space,
\[
\begin{array}{rcl}
p:	\Omega & \to & [0,1] \\
	\omega & \mapsto & p(\omega),
\end{array}
\] 
and the associated \emph{probability measure} is a function on \emph{subsets} of the sample space,
\[
\begin{array}{rccl}
\prob:	& \mathcal{P}(\Omega)	& \to & [0,1] \\
		& A					& \mapsto & \displaystyle\sum_{\omega\in A} p(\omega).
\end{array}
\] 
The probability that the coin is tossed an even number of times is 
\[
\prob(A) 
	= \sum_{j=1}^{\infty} p(2j) 
	= \sum_{j=0}^{\infty} p(1-p)^{2j+1}
	= p(1-p)\sum_{j=0}^{\infty} (1-p)^{2j}
	= \frac{p(1-p)}{1-(1-p)^2}
	= \frac{1-p}{2-p}.
\]
%Here, we have again used the formula for the sum of a geometric progression, this time with $a = p(1-p)$ and $r = (1-p)^2$.
%\bit
%\it Note that if $p=1/2$ then $\prob(A) = 1/3$.
%\eit
\end{example}

\newpage

Recall that the set of all subsets of $\Omega$ is known as its \emph{power set}, denoted by $\mathcal{P}(\Omega)$.
\bit
\it The \emph{cardinality} of a set is the number of elements it contains.
\it If $\Omega = \{1,2,3,\ldots,n\}$, then the cardinality of $\mathcal{P}(\Omega)$ is $2^n$ (a finite number).
\eit

\textbf{Problem}:
\par\bigskip
\fbox{\begin{minipage}{\linewidth}
\bigskip
\bit
\it If $\Omega$ is an infinite set, its power set is \emph{uncountable}.% (i.e. it has an uncountable number of subsets).
\it If $\Omega$ is an uncountable set, its power set is (very) \emph{uncountable}.
\eit
\bigskip
\end{minipage}}
\bigskip

\textbf{Questions}:
\par\bigskip
\fbox{\begin{minipage}{\linewidth}
\bigskip
\bit
\it How can we assign probabilities to the elements of uncountable sets?
\it Today we look at what it means for a set to be finite, countably infinite or uncountable.
\eit
\bigskip
\end{minipage}}
\bigskip

%----------------------------------------------------------------------
\newpage
\section{Number systems}
%----------------------------------------------------------------------
%Georg Cantor (1845--1918) is known as the father of set theory.
%\begin{quote}
%By a \textit{set} we understand any collection $M$ into a whole of de�nite, well-distinguished objects of our intuition or our thought (which will be called the \textit{elements} of $M$).
%\par\mbox{}\hfill Cantor (1932)
%\end{quote}
%
%\begin{quote}
%No one shall expel us from the Paradise that Cantor has created.
%\par\mbox{}\hfill David Hilbert (1932)
%\end{quote}

%--------------------------------------------------
\subsection{The natural numbers}
%--------------------------------------------------
The set of natural numbers is denoted by 
\[
\N = \{1,2,3,\ldots\}.
\]

Addition and multiplication are \emph{binary operations} on the natural numbers.
\[
\begin{array}{ccccc}
+ 	& :	& \N\times\N	& \to 		& \N \\
	& 	& (a,b) 		& \mapsto 	& a+b \\[3ex]
\times	& :	& \N\times\N	& \to 		& \N \\
		&	& (a,b) 		& \mapsto 	& a\cdot b.
\end{array}
\]

If $a,b\in\N$ then $a+b\in\N$ and $a\cdot b\in\N$.
\bit
\it 
We say that $\N$ is \emph{closed under addition and multiplication}.
\eit

\newpage

The set of natural numbers, along with $+$ and $\times$, has the following properties:
\ben
\it \textbf{Commutative property}:
	\bit 
	\it $a + b = b + a$,
	\it $a\cdot b = b\cdot a$.
	\eit
\it \textbf{Associative property}:
	\bit 
	\it $(a + b) + c = a + (b + c)$,
	\it $(a\cdot b)\cdot c = a \cdot (b\cdot c)$,
	\eit
\it \textbf{Distributive property}:
	\bit 
	\it $a\cdot(b + c) = (a\cdot b) + (a\cdot c)$,% (left distribuivity),
	\it $(a + b)\cdot c = (a\cdot c) + (b\cdot c)$.% (right distributivity).
	\eit
\it Existence of a \textbf{multiplicative identity} element:
	\bit
	\it There exists an element called $1\in\Z$ such that $a \cdot 1 = a$ for all $a\in\Z$.
	\eit
\een

%--------------------------------------------------
\newpage
\subsection{The integers}
%--------------------------------------------------
The set of integers is denoted by
\[
\Z = \{\ldots,-3,-2,-1,0,1,2,3,\ldots\}.
\]

The set of integers, along with $+$ and $\times$, has the following additional properties:
\ben
\setcounter{enumi}{4}
\it Existence of an \textbf{additive identity} element (called the zero element):
	\bit
	\it There exists an element called $0\in\Z$ such that $a + 0 = a$ for all $a\in\Z$.
	\eit
\it Existence of \textbf{additive inverse} elements:
	\bit
	\it For each $a\in\Z$, there exists an element $-a\in\Z$, called the (additive) \emph{inverse} of $a$, such that $a + (-a) = 0$.
	\eit
\een

\begin{remark}
Mathematical objects with the properties (1) --(6) are called \emph{rings}.
\end{remark}

%--------------------------------------------------
\newpage
\subsection{The rational numbers}
%--------------------------------------------------
The set of rational numbers (a.k.a. fractions) is denoted by
\[
\Q = \{a/b:a\in\Z,b\in\N\}.
\]

The set of rational numbers, along with $+$ and $\times$, has the following additional property:
\ben
\setcounter{enumi}{6}
\it Existence of \textbf{multiplicative inverse} elements:
	\bit
	\it For each $q\in\Q$, there exists an element $q^{-1}\in\Q$ such that $q\cdot q^{-1} = 1$.
	\eit
\een
\begin{remark}
Mathematical objects with the properties (1) --(7) are called \emph{fields}.
\end{remark}

%--------------------------------------------------
\newpage
\subsection{The real numbers}
%--------------------------------------------------
The set of real numbers is denoted by $\mathbb{R}$, and can (vaguely) be defined as
\[
\R = \text{the set of infinite decimal expansions.}
\]
\bit
\it We define (for example) $0.999999\ldots$ and $1.000000$ to be equal.
\it Any number base can be used (e.g. binary expansions, ternary expansions, etc).
\eit

\begin{remark}[Cauchy sequences converge]
The set of real numbers $\R$ has the property that, for any sequence $x_1,x_2,\ldots$ of real numbers with the property that $|x_m-x_n|\to 0$ as $m,n\to\infty$, there exists some $x\in\R$ such that $|x_n-x|\to 0$ as $n\to\infty$.
\bit
\it In this sense, we can say that $\R$ is \emph{topologically complete}.
\eit
\end{remark}

%--------------------------------------------------
\newpage
\subsection{The complex numbers}
%--------------------------------------------------
The set of complex numbers is denoted by $\mathbb{C}$:
\[
\C = \{x+iy : x,y\in\R, i=\sqrt{-1}\}.
\]

\begin{remark}
The set of complex numbers has the property that the roots of any polynomial with coefficients in $\C$ are themselves elements of $\C$.
\bit
\it In this sense, we can say that $\C$ is \emph{algebraically complete}.
\eit
\end{remark}

%----------------------------------------------------------------------
\newpage
\section{One-to-one correspondence}
%----------------------------------------------------------------------

Functions are also known as \emph{mappings}, \emph{transformations}, \emph{correspondences} and \emph{operators}.
\[
\begin{array}{rlcl}
f: 	& A & \to		& B \\
	& a & \mapsto	& f(a).
\end{array}
\]

%\bit
%\it The set $A$ is called the \emph{domain} of $f$.
%\it The set $B$ is called the \emph{co-domain} of $f$.
%\it The set $C=\{f(a):a\in A\}\subseteq B$ is called the \emph{range} of $f$.
%\eit

\bit
\it $f$ is called \emph{injective} if for all $a_1,a_2\in A$, $f(a_1)=f(a_2) \Rightarrow a_1=a_2$.
\it $f$ is called \emph{surjective} if for all $b\in B$, there exists $a\in A$ such that $f(a)=b$.
\it $f$ is called \emph{bijective} if it is both injective and surjective.
\eit

Bijective functions represent a \emph{one-to-one correspondence} between the elements of $A$ and $B$.

%----------------------------------------------------------------------
\newpage
\section{Countable and uncountable sets}
%----------------------------------------------------------------------
\begin{definition}
\ben
\it
A set $A$ is called \emph{finite} if there is a bijection $\phi:A\to\{1,2,\ldots,n\}$; otherwise $A$ is called \emph{infinite}.
\it
A set $A$ is called \emph{countable} if it is finite, or there is a bijection $\phi:A\to\N$; otherwise $A$ is called \emph{uncountable}.
\een
\end{definition}

% thm: Z countable
\begin{theorem}
$\Z$ is countable.
\end{theorem}
\begin{proof}
\vspace*{20ex}
\end{proof}

\newpage

% thm: Q countable
\begin{theorem}
$\Q$ is countable.
\end{theorem}
\begin{proof}
\vspace*{45ex}
\end{proof}

\newpage

% thm: R uncountable
\begin{theorem}
$\R$ is uncountable.
\end{theorem}
\begin{proof} We use the method of \emph{proof by contradiction}.
\par
Suppose that $\R$ is a countable set. By definition, $\R$ can be defined as the set of all infinite binary expansions (as well as the set of all decimal expansions), so by hypothesis, we can list them all:
\footnotesize
\[
\begin{array}{ccccccccccc}
\mathbf{1} & 1 & 0 & 1 & 0 & 0 & 0 & 1 & 1 & 1 & \ldots \\
1 & \mathbf{0} & 1 & 0 & 1 & 0 & 1 & 1 & 1 & 0 & \ldots  \\
0 & 1 & \mathbf{1} & 1 & 0 & 0 & 1 & 0 & 1 & 0 & \ldots  \\
1 & 0 & 0 & \mathbf{0} & 1 & 1 & 1 & 1 & 1 & 1 & \ldots  \\
0 & 1 & 1 & 1 & \mathbf{0} & 0 & 0 & 0 & 0 & 1 & \ldots  \\
1 & 0 & 1 & 0 & 0 & \mathbf{1} & 0 & 1 & 1 & 0 & \ldots  \\
0 & 0 & 0 & 1 & 0 & 0 & \mathbf{1} & 1 & 0 & 1 & \ldots  \\
0 & 1 & 0 & 0 & 1 & 1 & 0 & \mathbf{1} & 0 & 0 & \ldots  \\
0 & 0 & 0 & 1 & 0 & 0 & 1 & 0 & \mathbf{1} & 1 & \ldots  \\
1 & 1 & 0 & 0 & 1 & 1 & 1 & 0 & 1 & \mathbf{0} & \ldots  \\
\vdots & \vdots & \vdots & \vdots & \vdots & \vdots & \vdots & \vdots & \vdots & \vdots & \ddots 
\end{array}
\]
\normalsize

\newpage

\bit
\it Consider the sequence along the diagonal: $1,0,1,0,0,1,1,1,1,0,\ldots$
\it Suppose we reverse each digit in the sequence: $0,1,0,1,1,0,0,0,0,1,\ldots$
\it This sequence differs from the first sequence in the first place, differs from the second sequence in the second place, differs from the third sequence in the third place, etc.
\it This sequence cannot therefore be among those listed.
\it This contradicts our original assumption that $\R$ is countable.
\eit
Our original assumption must therefore have been false, so $\R$ is uncountable
\end{proof}

% thm: power set uncountable
\begin{theorem}
The power set of $\N$ (or any countably infinite set) is uncountable.
\end{theorem}
\begin{proof}
As above, with subsets of $\N$ represented as infinite binary strings.
\end{proof}


%----------------------------------------------------------------------
\newpage
\section{The continuum hypothesis}
%----------------------------------------------------------------------

\begin{definition}
\ben
\it
Two sets $A$ and $B$ are said to have the \emph{same cardinality} if there exists a bijection $\phi:A\to B$.
\it
If there exists a bijection $\phi:A\to\N$, we say that $A$ has the \emph{cardinality of the integers}.
\it
If there exists a bijection $\phi:A\to\R$, we say that $A$ has the \emph{cardinality of the real numbers}.
\een
\end{definition}

\begin{hypothesis}[The continuum hypothesis]
There is no set whose cardinality is strictly between the cardinality of the integers and the cardinality of the real numbers.
\end{hypothesis}

In 1940, Kurt G\"{o}del showed that the continuum hypothesis cannot be proved or disproved within the standard framework of set theory.

\newpage

%----------------------------------------------------------------------
\section*{How do uncountable sets affect me?}
%----------------------------------------------------------------------
\bigskip


\centering
\begin{tabular}{|l|l|l|}\hline
Finite sets				& Finite sums			& $\expe(X) = \displaystyle\sum_{i=1}^n x_i\prob(X_i=x_i)$ \\ % \hline
Countable sets			& Infinite sums	 		& $\expe(X) = \displaystyle\sum_{i=1}^{\infty} x_i\prob(X_i=x_i)$ \\ % \hline
Uncountable sets	\qquad	& Integrals (calculus)	& $\expe(X) = \displaystyle\int_{-\infty}^{\infty} x\,f(x)\,dx$ (later) \\ \hline
\end{tabular}

%======================================================================
\end{document}
%======================================================================
