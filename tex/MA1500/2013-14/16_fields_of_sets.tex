\documentclass[lecture]{csm}

% set meta information
\modulecode{MA1500}
\moduletitle{Introduction to Probability Theory}
\academicyear{2013/14}
\doctype{Lecture}
\doctitle{Fields of Sets}
\docnumber{16}

% local
\newcommand{\prob}{\mathbb{P}}
\newcommand{\expe}{\mathbb{E}}
\newcommand{\N}{\mathbb{N}}
\newcommand{\R}{\mathbb{R}}
\def\it{\item}
\def\bit{\begin{itemize}}
\def\eit{\end{itemize}} 
\def\ben{\begin{enumerate}}
\def\een{\end{enumerate}}

%======================================================================
\begin{document}
\maketitle
\tableofcontents
%======================================================================

%----------------------------------------------------------------------
\section{Finite collections of sets}
%----------------------------------------------------------------------
Let $\Omega$ be the sample space of some random experiment.
\bit
\it Suppose that we are interested in the events $A$ and $B$.
\it Then we are also interested in the events $A\cup B$, $A\cap B$ and $A^c$.
\eit

% definition: collection of sets
\vspace*{2ex}
Recall that the set of all subsets of $\Omega$ is called its \emph{power set}, 
$\mathcal{P}(\Omega) = \{A:A\subseteq\Omega\}$.

\begin{definition}
Any subset of $\mathcal{P}(\Omega) = \{A:A\subseteq\Omega\}$ is called a \emph{collection of sets} over $\Omega$. 
\end{definition}

% definition: field of sets
\begin{definition}
Let $\Omega$ be any set. A collection of sets $\mathcal{F}$ is called a \emph{field of sets} over $\Omega$ if 
\ben
\it $\Omega\in\mathcal{F}$,
\it $\mathcal{F}$ is closed under complementation: if $A\in\mathcal{F}$ then $A^c\in\mathcal{F}$, and
\it $\mathcal{F}$ is closed under pairwise unions: if $A,B\in\mathcal{F}$ then $A\cup B\in\mathcal{F}$.
\een
\end{definition}

% remark
\begin{remark}
A field of sets over $\Omega$ is also called an \emph{algebra} of sets over $\Omega$.
\end{remark}

% example
\begin{example}
Let $\Omega=\{a,b\}$. The following collections are both fields of sets over $\Omega$:
\begin{hidebox}
\bit
\it $\mathcal{F} = \{\emptyset,\Omega\}$.
\it $\mathcal{F} = \big\{\emptyset,\{a\},\{b\},\,\Omega\big\}$.
\eit
\end{hidebox}
\end{example}

% example
\begin{example}
Let $\Omega=\{a,b,c\}$. The following collections are all fields of sets over $\Omega$:
\begin{hidebox}
\bit
\it $\mathcal{F} = \{\emptyset,\,\Omega\}$.
\it $\mathcal{F} = \big\{\emptyset,\{a\},\{b,c\},\,\Omega\big\}$.
\it $\mathcal{F} = \big\{\emptyset,\{a\},\{b\},\{c\},\{a,b\},\{a,c\},\{b,c\},\,\Omega\big\}$.
\eit
\end{hidebox}
\end{example}

% example
\begin{example}\label{ex:sigma_fields}
A fair six-sided die is rolled once. Let $\Omega=\{1,2,3,4,5,6\}$ denote the sample space.
\bit
\it The power set $\mathcal{P}(\Omega)$ is a field of sets over $\Omega$.
\eit
\vspace*{2ex}
If we are only interested in whether the outcome is an odd or even number, we need only consider 
\bit
\it the events $A=\{1,3,5\}$ and $B=\{2,4,6\}$, and
\it the field ${\mathcal{F}} = \{\emptyset, A, B, \Omega\}$.
\eit
\end{example}

% properties of fields
\begin{theorem}\label{prop:properties_of_fields}
Let $\mathcal{F}$ be a field of sets over $\Omega$. Then
\ben
\it $\emptyset\in\mathcal{F}$,
\it $\mathcal{F}$ is closed under pairwise intersections: if $A,B\in\mathcal{F}$ then $A\cap B\in\mathcal{F}$,
\it $\mathcal{F}$ is closed under set differences: if $A,B\in\mathcal{F}$ then $A\setminus B\in\mathcal{F}$,
\it $\mathcal{F}$ is closed under finite unions: if $A_1,\ldots,A_n\in\mathcal{F}$ then \ $\bigcup_{i=1}^{n}A_i\in\mathcal{F}$, 
\it $\mathcal{F}$ is closed under finite intersections: if $A_1,\ldots,A_n\in\mathcal{F}$ then \ $\bigcap_{i=1}^{n}A_i\in\mathcal{F}$.
\een
\end{theorem}

\newpage

% proof
\begin{proof}
\ben
\it 
$\emptyset = \Omega^c$ where $\Omega\in\mathcal{F}$. Since $\mathcal{F}$ is closed under complementation, $\emptyset\in\mathcal{F}$.
\it
Let $A,B\in\mathcal{F}$. Then $A\cap B = (A^c\cup B^c)^c$ (De Morgan's laws). Since $\mathcal{F}$ is closed under complementation and pairwise unions, $A\cap B\in\mathcal{F}$.
\it
Let $A,B\in\mathcal{F}$. Then $A\setminus B = A\cap B^c = (A^c\cup B)^c$ (De Morgan's laws). Since $\mathcal{F}$ is closed under complementation and pairwise unions, $A\setminus B\in\mathcal{F}$.
\it
Proof by induction. Suppose that $\mathcal{F}$ is closed under unions of $n$ sets (where $n\geq 2$). Let $A_1,A_2,\ldots,A_{n+1}\in\mathcal{F}$. By the inductive hypothesis, $\bigcup_{i=1}^nA_i\in\mathcal{F}$, so $\bigcup_{i=1}^{n+1} A_i = \big(\bigcup_{i=1}^{n} A_i\big) \cup A_{n+1} \in\mathcal{F}$, because $\mathcal{F}$ is closed under pairwise unions.
\it
Let $A_1,A_2,\ldots,A_n\in\mathcal{F}$. Then $\bigcap_{i=1}^n A_i = \big(\bigcup_{i=1}^n A_i^c\big)^c$ (De Morgan's laws). Since $\mathcal{F}$ is closed under complementation and finite unions, $\bigcap_{i=1}^n A_i\in\mathcal{F}$.
\een
\end{proof}

%----------------------------------------------------------------------
\newpage
\section{Countable collections of sets}
%----------------------------------------------------------------------

\bit
\it A field of sets $\mathcal{F}$ over $\Omega$ is closed under finite unions. 
\it We extend this idea to collections that are closed under \emph{countable} unions.
\eit

\begin{definition}
Let $\{A_1,A_2,\ldots\}$ be a countable collection of sets over $\Omega$.
\ben
\it The \emph{union} of the sets $A_1,A_2,\ldots$ is the set
\[
\bigcup_{i=1}^\infty A_i = \{\omega:\omega\in A_i \text{ for some }A_i\}.
\]
\it The \emph{intersection} of the sets $A_1,A_2,\ldots$  is the set
\[
\displaystyle\bigcap_{i=1}^\infty A_i = \{\omega:\omega\in A_i \text{ for all } A_i\}.
\]
\een
\end{definition}

\break % <<

% theorem: De Morgan
\begin{theorem}[De Morgan's laws]
For a countable collection of sets $\{A_1,A_2,\ldots\}$,
\ben
\it $\big(\bigcup_{i=1}^{\infty} A_i\big)^c = \bigcap_{i=1}^{\infty} A_i^c$,
\it $\big(\bigcap_{i=1}^{\infty} A_i\big)^c = \bigcup_{i=1}^{\infty} A_i^c$.
\een
\end{theorem}

% proof
\begin{proof}
\ben
\it 	\bit
	\it Let $a\in\big(\bigcup_{i=1}^\infty A_i\big)^c$. Then $a\notin\bigcup_{i=1}^\infty A_i$, so $a\in A_i^c$ for all $A_i$, and therefore
		\par $\big(\bigcup_{i=1}^\infty A_i\big)^c \subseteq \bigcap_{i=1}^\infty A_i^c$. 
	\it Let $a\in\bigcap_{i=1}^\infty A_i^c$. Then $a\notin A_i$ for all $A_i$, so $a\notin\bigcup_{i=1}^\infty A_i$, and therefore
		\par $\bigcap_{i=1}^\infty A_i^c \subseteq \big(\bigcup_{i=1}^\infty A_i\big)^c$.
	\it Thus it follows that $\big(\bigcup_{i=1}^{\infty} A_i\big)^c = \bigcap_{i=1}^{\infty} A_i^c$, as required.
	\eit
\it \bit
	\it Applying part (1) to the collection of sets $\{A_1^c,A_2^c,\ldots\}$, 
		\par $\big(\bigcup_{i=1}^\infty A_i^c\big)^c = \bigcap_{i=1}^\infty \big(A_i^c\big)^c = \bigcap_{i=1}^\infty A_i$.
	\it Taking the complement of both sides, we obtain
	 	\par $\big(\bigcap_{i=1}^\infty A_i\big)^c = \bigcap_{i=1}^\infty A_i^c$.
	\eit
\een
\end{proof}

% definition: sigma-field
\begin{definition}
Let $\Omega$ be any set. A collection of sets $\mathcal{F}$ is called a \emph{$\sigma$-field} over $\Omega$ if 
\ben
\it $\Omega\in\mathcal{F}$,
\it $\mathcal{F}$ is closed under complementation: if $A\in\mathcal{F}$ then $A^c\in\mathcal{F}$, and
\it $\mathcal{F}$ is closed under countable unions: if $A_1,A_2,\ldots\in\mathcal{F}$ then $\bigcup_{i=1}^{\infty}A_i \in\mathcal{F}$.
\een
\end{definition}


% theorem: properties of sigma fields
\begin{theorem}\label{thm:properties_of_sigma_fields}
Let $\mathcal{F}$ be a $\sigma$-field over $\Omega$. Then
\ben
%\it $\emptyset\in\mathcal{F}$,
%\it $\mathcal{F}$ is closed under set differences: if $A,B\in\mathcal{F}$ then $A\setminus B\in\mathcal{F}$,
\it $\mathcal{F}$ is closed under finite unions: if $A_1,\ldots,A_n\in\mathcal{F}$ then \ $\bigcup_{i=1}^{n}A_i\in\mathcal{F}$,
\it $\mathcal{F}$ is closed under finite intersections: if $A_1,\ldots,A_n\in\mathcal{F}$ then \ $\bigcap_{i=1}^{n}A_i\in\mathcal{F}$,
\it $\mathcal{F}$ is closed under countable intersections: if $A_1,A_2,\ldots\in\mathcal{F}$ then \ $\bigcap_{i=1}^{\infty}A_i \in\mathcal{F}$.
\een
\end{theorem}

% proof
\begin{proof}
\ben
\it 
Let $A_1,A_2,\ldots,A_n\in\mathcal{F}$. Since $\mathcal{F}$ is closed under countable unions and $\emptyset\in\mathcal{F}$, 
\[
\bigcup_{i=1}^n A_i = A_1\cup A_2\cup\ldots\cup A_n\cup\emptyset\cup\emptyset\ldots \in\mathcal{F}.
\]
\it
Let $A_1,A_2,\ldots,A_n\in\mathcal{F}$. Since $\mathcal{F}$ is closed under complementation and finite unions,
\[
\bigcap_{i=1}^n A_i = A_1\cap\ldots\cap A_n = (A^c_1\cup\ldots\cup A^c_n)^c \in\mathcal{F}.
\]
\it
Let $A_1,A_2,\ldots\in\mathcal{F}$. Since $\mathcal{F}$ is closed under complementation and countable unions,
\[
\bigcap_{n=1}^{\infty} A_n = \left(\bigcup_{n=1}^{\infty} A^c_n\right)^c \in\mathcal{F}.
\] 
\een
\end{proof}


%======================================================================
\end{document}
%======================================================================
